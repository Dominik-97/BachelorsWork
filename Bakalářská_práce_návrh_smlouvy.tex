\documentclass[parskip=half]{scrreprt}

\usepackage[margin=2cm]{geometry}
\usepackage[utf8]{inputenc}
\usepackage[T1]{fontenc}
\usepackage[czech]{babel}
\usepackage{lmodern}
\usepackage[juratotoc]{scrjura}
\usepackage{color}

\makeatletter 
\renewcommand*{\parformat}{% 
  \global\hangindent 2em 
  \makebox[2em][l]{(\thepar)\hfill}%
}  
\makeatother
\renewcommand*{\parformatseparation}{}

\begin{document}
\addchap{Příloha č. 1 Vzor smlouvy o založení svěřenského fondu a statutu svěřenského fondu}

Mezi

\textcolor{blue}{<Jméno, Adresa>}, dále jen «Zakladatel»,

a

\textcolor{blue}{<Jméno, Adresa>}, dále jen «Svěřenský správce».
/ \textcolor{blue}{<Jméno, Adresa>}, a \textcolor{blue}{<Jméno, Adresa>} dále jen «Svěřenští správci».



\begin{contract}
\Clause{title={Vytvoření svěřenského fondu}}%Prohlášení smluvních stran}}

\SubClause{title={Prohlášení smluvních stran}}

Zakladatel prohlašuje, že je ke dni podpisu této smlouvy mimo jiné majitelem následujícího majetku, který tvoří jeho výlučné vlastnictví: \textcolor{blue}{<Přesný výčet, určení a popis majetku, který je ve výlučném vlastnictví zakladatele, a který si přeje vyčlenit do svěřenského fondu>}.

\parnumberfalse
[Příklad.: Zakladatel do svěřenského fondu vyčlení rodinný dům, chalupu a finanční prostředky uložené na zakladatelově běžném účtu u poskytovatele finančních služeb. Popřípadě je majetek (jmění) určeno v příloze \textcolor{blue}{<Číslo>}.]
\parnumbertrue

\SubClause{title={Vyčlenění majetku}}

Zakladatel na základě projevu své svobodné vůle touto smlouvou (právním jednáním) vyčleňuje ze svého výlučného vlastnictví majetek (jmění), které je přesně určen v § 1a tohoto článku (této části) smlouvy a v souladu s tímto zároveň svěřuje takto vyčleněný majetek ke správě správci ke dni určeném v § 1c.

Předmětný majetek má být předaný správci ke správě v rámci svěřenského fondu bezplatně bez protiplnění. Náklady spojené se založením, vznikem a převodem vyčleněného předmětu (majetku) smlouvy budou uhrazeny určeným správcem, který má právo po vzniku svěřenského fondu na vyplacení těchto nezbytných nákladů z majetku svěřenského fondu. O těchto nákladech a jejich proplacení je správce povinen pořídit záznam a seznámit s ním obmyšlené svěřenského fondu.

%Pokud to bude nutné, popsat podobný paragraf ve vztahu k obmyšleným, ti vstupují do svěřenského fondu bez jakéhokoliv protiplnění.

% Toto doplnit, ani zakladatel, ani správce nenesou odpovědnost za určený majetek, který nebyl vyčleněn bez zavinění jakékoliv strany smlouvy, například zásahem vyšší moci. Der Verkäufer trägt die Gefahr des zufälligen Untergangs oder der zufälligen Verschlechterung des Kaufgegenstands, bis er am Erfüllungsort dem Käufer übergeben wird oder dieser mit der Annahme in Verzug ist.

%Nemovitosti (bez příslušenství jsou určeny přesně a je k nim omezeno dispoziční právo zakladatele. Prostředky na finančním účtu budou do fondu vloženy ve výši v jaké na účtě jsou po uhrazení nezbytných výloh v minimální výši XY, kterou je zakladatel povinen na účtu držet a zajistit.

Zakladatel se zavazuje dle § 1761 Občanského zákoníku v platném znění, či obdobného paragrafu v případě novelizace Občanského zákoníku k řádné správě vyčleněného nemovitého majetku včetně jeho příslušenství a smluvní strany si mezi sebou ujednávají zákaz zatížení nebo zcizení vyčleněného nemovitého majetku. Tento zákaz bude bez zbytečného odkladu zapsán do katastru nemovitostí jako věcné právo s účinností ode dne podpisu této smlouvy až do vzniku svěřenského fondu ke dni určeném v § 1c, který tvoří rozvazovací podmínku.

% V případě úmyslného snížení hodnoty vyčleňovaného nemovitého a movitého majetku je, kterákoliv ze stran smlouvy povina nahradit takto způsobenou škodu v plné výši, popřípadě obnovit vyčleňovaný majetek do původního stavu tak, aby byl vnesen ve stejné hodnotě.

%Prodávající nese riziko náhodné ztráty nebo náhodného zhoršení předmětu koupě, dokud nebude předán dopravní společnosti v místě plnění nebo dokud nebude v prodlení.

Správce přijímá pověření ke správě jmění vyčleněného do svěřenského fondu a správě svěřenského fondu. Správce přijímá zakladatelem vyčleněný a svěřený majetek do své správy a tímto se zavazuje tento majetek držet a spravovat s péčí řádného hospodáře ke dni určeném v § 1c.

Pokud by z jakýchkoliv důvodů nebyl svěřenský správce schopen ujmou se správy, určí nového svěřenského správce všichni obmyšlení na společném hlasování, kde každý obmyšlený má právě tolik hlasů, kolik mu je let. V případě neshod a neurčení nového správce ze strany obmyšlených určí nového správce na návrh kteréhokoliv z obmyšlených soud.

\parnumberfalse
[Bsp.: Der Kaufgegenstand ist dem Transportunternehmen \textcolor{blue}{<Name>} an \textcolor{blue}{<Adresse, Ort>} zu übergeben.

Die Verkäufer trägt die Gefahr des zufälligen Untergangs oder der zufälligen Verschlechterung des Kaufgegenstands, bis dieser am Erfüllungsort dem Transportunternehmen übergeben wird oder dieses mit der Annahme in Verzug ist.]
\parnumbertrue

\SubClause{title={Okamžik vzniku svěřenského fondu}}

Svěřenský fond vzniká okamžikem, kdy budou kumulativně splněny obě podmínky níže určené.
1b První podmínkou je přijetí pověření ke správě ze strany správce.\\
2b Drouhou pomínku je splnění odkládací podmínky spočívající v úmrtí zakladatele svěřenského fondu.

%Finalní podmínku tvoří zápis svěřenského fondu do evidence svěřenských fondů kde dni, ve kterém došlo ke splnění obou kumulativních podmínek.

Je li svěřenských správců výce, postačí ke splnění podmínky vymezené výše v bodě 1b, aby takovouto správu přijal alespoň jeden z nich.

\Clause{title={Správa a výkonvlastnických práv}}

\SubClause{title={Kaufpreis und Sicherheit}}

Svěřenskému správci náleži plná správa k vyčleněnému majetku.

Svěřenský správce vykonává vlastnická práva k majetku, každý svým vlastním jménem na účet svěřenského fondu.

Der Käufer hat den Kaufpreis von CHF \textcolor{blue}{<Zahl>} bis zum \textcolor{blue}{<Datum>} auf das Konto Nr.~\textcolor{blue}{<Zahl>} der Z-Bank in \textcolor{blue}{<Postleitzahl, Ort>} zu überweisen.

Das Eigentum am gelieferten Kaufgegenstand verbleibt bis zur Bezahlung des Kaufpreises beim Verkäufer. Kommt der Käufer mit der Zahlung des Kaufpreises in Verzug, so ist der Verkäufer berechtigt, den Eigentumsvorbehalt auf Kosten des Käufers im Eigentumsvorbehaltsregister eintragen zu lassen.

Der Käufer hat bis spätestens am \textcolor{blue}{<Datum>} eine bedingungslose und unbefristete Bankgarantie über die Bezahlung von CHF \textcolor{blue}{<Zahl>} einer schweizerischen Grossbank oder einer Kantonalbank beizubringen \textcolor{blue}{<evt. andere Sicherheiten, z.B. Bürgschaft usw.>}.

\parnumberfalse[Bsp.:
Der Käufer hat den Kaufpreis von CHF \textcolor{blue}{<Zahl>} dem Verkäufer Zug um Zug gegen Übergabe des Kaufgegenstandes am Erfüllungsort in bar zu bezahlen.]
\parnumbertrue

Ist der Käufer mit einer Zahlungspflicht in Verzug, so hat er Verzugszinsen von \textcolor{blue}{<Zahl>}\% und Schadenersatz zu bezahlen.

Ist der Käufer mit der Zahlung des Kaufpreises oder der Leistung einer Sicherheit in Verzug, so kann der Verkäufer dem Käufer eine Nachfrist von \textcolor{blue}{<Zahl>} Tagen setzen und nach deren unbenutztem Ablauf entweder innert \textcolor{blue}{<Zahl>} Tagen den Rücktritt vom Vertrag erklären und Schadenersatz (positives oder negatives Vertragsinteresse) oder weiterhin die Zahlung des Kaufpreises oder die Leistung der Sicherheit verlangen.

\Clause{title={Statut svěřenského fondu}}

\SubClause{title={Označení}}

\SubClause{title={Označení majetku}}

\SubClause{title={Vymezení účelu svěřenského fondu}}

\SubClause{title={Podmínky pro plnění ze svěřenského fondu}}

\SubClause{title={Trvání svěřenského fondu}}

\SubClause{title={Určení osob, kterýmžto má být ze svěřenského fondu plněno}}

\SubClause{title={Počet svěřenských správců a způsob jejich jednání}}

Der Käufer hat den Kaufgegenstand am Liefertag/während der Lieferfrist am Erfüllungsort anzunehmen/abzuholen.

Ist der Käufer mit der Annahme des Kaufgegenstandes in Verzug, so kann der Verkäufer dem Käufer eine Nachfrist von \textcolor{blue}{<Zahl>} Tagen setzen und nach deren unbenutztem Ablauf entweder innert \textcolor{blue}{<Zahl>} Tagen den Rücktritt vom Vertrag erklären und Schadenersatz (positives oder negatives Vertragsinteresse) oder weiterhin die Annahme verlangen.

\parnumberfalse[Bsp.:
Der Käufer hat den Kaufgegenstand spätestens bis zum \textcolor{blue}{<Datum>} am Erfüllungsort anzunehmen/abzuholen. Unterlässt er dies, so kann der Verkäufer entweder innert \textcolor{blue}{<Zahl>} Tagen den Rücktritt vom Vertrag erklären und Schadenersatz (positives oder negatives Vertragsinteresse) oder weiterhin die Annahme verlangen.]
\parnumbertrue



\Clause{title={Haftung des Verkäufers für Vertragswidrigkeiten der Lieferung}}

\SubClause{title={Ansprüche des Käufers}}

Hat der Verkäufer vertragswidrige Ware geliefert, so kann der Käufer vom Verkäufer zunächst nur verlangen, dass er die Vertragswidrigkeit nach seiner Wahl durch Nachbesserung oder Ersatzlieferung kostenlos und ohne unverhältnismässige Unannehmlichkeiten für den Käufer behebt. Der Käufer kann dem Verkäufer dazu eine angemessene Frist von mindestens \textcolor{blue}{<Zahl>} Tagen setzen. Hat der Verkäufer die Vertragswidrigkeit innert dieser Frist nicht behoben, so kann der Käufer Minderung oder, wenn ihm das Behalten der Lieferung unzumutbar ist, Wandelung verlangen.

\parnumberfalse
[Bsp. 1: Hat der Verkäufer vertragswidrige Ware geliefert, so kann der Käufer nach seiner Wahl die kostenlose Nacherfüllung durch Ersatzlieferung oder Nachbesserung am Ort verlangen, wo sich die Lieferung gewöhnlich befindet. Die vom Käufer verlangte Nacherfüllung kann der Verkäufer verweigern, wenn er nachweist, dass sie unmöglich oder unverhältnismässig ist. Der Käufer kann dem Verkäufer zur Nacherfüllung eine angemessene Frist von mindestens \textcolor{blue}{<Zahl>} Tagen setzen. Wird die Nacherfüllung innert dieser Frist nicht erbracht, so kann der Käufer nach seiner Wahl zudem Minderung oder Wandelung verlangen. Die Wandelung ist jedoch bei unerheblichen Mängeln ausgeschlossen.]

[Bsp. 2: Die Haftung des Verkäufers für Vertragswidrigkeiten der Lieferung wird ausgeschlossen. Dieser Ausschluss gilt nicht, soweit der Verkäufer dem Käufer gewisse Eigenschaften des Kaufgegenstands ausdrücklich zugesichert oder arglistig verschwiegen hat.]
\parnumbertrue

\SubClause{title={Verwirkung}}

Ansprüche aus bei einer übungsgemässen Untersuchung erkennbaren Vertragswidrigkeiten sind verwirkt, wenn diese der Käufer dem Verkäufer nicht innert \textcolor{blue}{<Zahl>} Tagen nach Übergabe der Lieferung am Erfüllungsort \textcolor{blue}{<ev. Bestimmungsort>} angezeigt hat.

Ansprüche aus anderen Vertragswidrigkeiten sind verwirkt, wenn diese der Käufer dem Verkäufer nicht \textcolor{blue}{<Zahl>} Tage nach ihrer Entdeckung, spätestens jedoch \textcolor{blue}{<Zahl>} Jahre nach Übergabe der Lieferung am Erfüllungsort angezeigt hat.

\parnumberfalse
[Bsp. 1: Ansprüche aus Vertragswidrigkeiten sind verwirkt, wenn diese der Käufer dem Verkäufer nicht \textcolor{blue}{<Zahl>} Tage nach ihrer Entdeckung, spätestens jedoch \textcolor{blue}{<Zahl>} Monate/Jahre nach der Übergabe der Lieferung am Erfüllungsort angezeigt hat.]

[Bsp. 2: Ansprüche aus Vertragswidrigkeiten sind verwirkt, wenn diese der Käufer dem Verkäufer nicht spätestens \textcolor{blue}{<Zahl>} Monate/Jahre nach der Übergabe der Lieferung am Erfüllungsort schriftlich angezeigt hat.]
\parnumbertrue

\SubClause{title={Verjährung}}

Die Gewährleistungsansprüche verjähren \textcolor{blue}{<Zahl>} Jahr(e) nach Übergabe der Lieferung am Erfüllungsort.

\parnumberfalse
[Bsp.: Gewährleistungsansprüche verjähren \textcolor{blue}{<Zahl>} Monate seit Eintreffen der Mängelanzeige.]
\parnumbertrue



\Clause{title={Schadenersatzpflicht des Verkäufers und Versicherung}}

Schadenersatzansprüche des Käufers werden auf insgesamt CHF \textcolor{blue}{<Zahl>} beschränkt.

\parnumberfalse
[Bsp. 1: Die Haftung des Verkäufers für indirekte Schäden oder Folgeschäden, wie entgangener Gewinn, wird im gesetzlich zulässigen Umfang abbedungen.]

[Bsp. 2: Schadenersatzansprüche des Käufers werden im gesetzlich zulässigen Umfang abbedungen.]
\parnumbertrue

Der Verkäufer verpflichtet sich, eine (Betriebs-)Haftpflichtversicherung abzuschliessen und während \textcolor{blue}{<Zahl>} Jahren nach Vertragsschluss aufrechtzuerhalten, die sämtliche Personen- und Sachschäden (und Folgeschäden daraus) von mindestens CHF \textcolor{blue}{<Zahl>} pro Schadensfall und übrige Schäden (reine Vermögensschäden) von mindestens CHF \textcolor{blue}{<Zahl>} pro Schadensfall deckt. Der Verkäufer verpflichtet sich, dem Käufer auf dessen Verlangen einen entsprechenden Versicherungsnachweis zu übergeben.



\Clause{title={Schriftform}}

Dieser Vertrag tritt mit der Unterzeichnung durch beide Parteien in Kraft. Er umfasst den Vertragstext und die darin erwähnten Anhänge. Vertragsänderungen und -ergänzungen sind nur in Schriftform und bei Unterzeichnung durch beide Vertragsparteien gültig.

Mitteilungen, die sich auf diesen Vertrag und seine Abwicklung beziehen, sind in deutscher Sprache zu verfassen und schriftlich oder in einer Form zu übermitteln, welche den Nachweis durch Text ermöglicht, wie namentlich Telex, Telefax und E-Mail.



\Clause{title={Salvatorische Klausel}}

Sollten Bestimmungen dieses Vertrages ganz oder teilweise rechtsunwirksam sein oder werden, so wird dadurch die Gültigkeit der übrigen Bestimmungen nicht berührt. Die Parteien verpflichten sich, unwirksame Bestimmungen so zu ersetzen, dass ihr wirtschaftlicher Zweck soweit zulässig gewahrt wird.



\Clause{title={Anwendbares Recht}}

Dieser Vertrag untersteht schweizerischem Recht, unter Ausschluss des Übereinkommens der Vereinten Nationen über Verträge über den Internationalen Warenkauf.



\Clause{title={Mediationsklausel und Gerichtsstand}}

Sollte es im Zusammenhang mit diesem Vertrag oder dessen Gültigkeit zu Streitigkeiten kommen, beabsichtigen die Parteien, zunächst eine Mediation nach den jeweils gültigen Regeln der \textcolor{blue}{<Name der Organisation>} einzuleiten, und ordentliche Klagen erst zu erheben, wenn in der Mediation keine gütliche Einigung gefunden werden konnte.

Ausschließlicher Gerichtsstand ist \textcolor{blue}{<Ortsangabe>}.

\end{contract}

\vspace{50pt} 
\noindent\rule{7cm}{.4pt}\hfill\rule{7cm}{.4pt}\par 
\noindent Datum, Ort \hfill Unterschrift Verkäufer

\vspace{50pt} 
\noindent\rule{7cm}{.4pt}\hfill\rule{7cm}{.4pt}\par 
\noindent Datum, Ort \hfill Unterschrift Käufer

\end{document}