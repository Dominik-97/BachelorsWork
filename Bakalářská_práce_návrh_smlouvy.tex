\documentclass[parskip=half]{scrreprt}

\usepackage[margin=2cm]{geometry}
\usepackage[utf8]{inputenc}
\usepackage[T1]{fontenc}
\usepackage[czech]{babel}
\usepackage{lmodern}
\usepackage[juratotoc]{scrjura}
\usepackage{color}

\makeatletter 
\renewcommand*{\parformat}{% 
  \global\hangindent 2em 
  \makebox[2em][l]{(\thepar)\hfill}%
}  
\makeatother
\renewcommand*{\parformatseparation}{}

\begin{document}
\addchap{Příloha č. 1 Vzor smlouvy o založení svěřenského fondu a statutu svěřenského fondu}

Mezi

\textcolor{blue}{<Jméno, Adresa>}, dále jen «Zakladatel»,

a

\textcolor{blue}{<Jméno, Adresa>}, dále jen «Svěřenský správce».
/ \textcolor{blue}{<Jméno, Adresa>}, a \textcolor{blue}{<Jméno, Adresa>} dále jen «Svěřenští správci».



\begin{contract}
\Clause{title={Vytvoření svěřenského fondu}}%Prohlášení smluvních stran}}

\SubClause{title={Prohlášení smluvních stran}}

Zakladatel prohlašuje, že je ke dni podpisu této smlouvy mimo jiné majitelem následujícího majetku, který tvoří jeho výlučné vlastnictví: \textcolor{blue}{<Přesný výčet, určení a popis majetku, který je ve výlučném vlastnictví zakladatele, a který si přeje vyčlenit do svěřenského fondu>}.

\parnumberfalse
[Příklad.: Zakladatel do svěřenského fondu vyčlení rodinný dům, chalupu a finanční prostředky uložené na zakladatelově běžném účtu u poskytovatele finančních služeb. Popřípadě je majetek (jmění) určeno v příloze \textcolor{blue}{<Číslo>}.]
\parnumbertrue

\SubClause{title={Vyčlenění majetku}}

Zakladatel na základě projevu své svobodné vůle touto smlouvou (právním jednáním) vyčleňuje ze svého výlučného vlastnictví majetek (jmění), které je přesně určen v § 1a tohoto článku (této části) smlouvy a v souladu s tímto zároveň svěřuje takto vyčleněný majetek ke správě správci ke dni určeném v § 1c.

Předmětný majetek má být předaný správci ke správě v rámci svěřenského fondu bezplatně bez protiplnění. Náklady spojené se založením, vznikem a převodem vyčleněného předmětu (majetku) smlouvy budou uhrazeny určeným správcem, který má právo po vzniku svěřenského fondu na vyplacení těchto nezbytných nákladů z majetku svěřenského fondu. O těchto nákladech a jejich proplacení je správce povinen pořídit záznam a seznámit s ním obmyšlené svěřenského fondu.

%Pokud to bude nutné, popsat podobný paragraf ve vztahu k obmyšleným, ti vstupují do svěřenského fondu bez jakéhokoliv protiplnění.

% Toto doplnit, ani zakladatel, ani správce nenesou odpovědnost za určený majetek, který nebyl vyčleněn bez zavinění jakékoliv strany smlouvy, například zásahem vyšší moci. Der Verkäufer trägt die Gefahr des zufälligen Untergangs oder der zufälligen Verschlechterung des Kaufgegenstands, bis er am Erfüllungsort dem Käufer übergeben wird oder dieser mit der Annahme in Verzug ist.

%Nemovitosti (bez příslušenství jsou určeny přesně a je k nim omezeno dispoziční právo zakladatele. Prostředky na finančním účtu budou do fondu vloženy ve výši v jaké na účtě jsou po uhrazení nezbytných výloh v minimální výši XY, kterou je zakladatel povinen na účtu držet a zajistit.

Zakladatel se zavazuje dle § 1761 Občanského zákoníku v platném znění, či obdobného paragrafu v případě novelizace Občanského zákoníku k řádné správě vyčleněného nemovitého majetku včetně jeho příslušenství a smluvní strany si mezi sebou ujednávají zákaz zatížení nebo zcizení vyčleněného nemovitého majetku. Tento zákaz bude bez zbytečného odkladu zapsán do katastru nemovitostí jako věcné právo s účinností ode dne podpisu této smlouvy až do vzniku svěřenského fondu ke dni určeném v § 1c, který tvoří rozvazovací podmínku.

% V případě úmyslného snížení hodnoty vyčleňovaného nemovitého a movitého majetku je, kterákoliv ze stran smlouvy povina nahradit takto způsobenou škodu v plné výši, popřípadě obnovit vyčleňovaný majetek do původního stavu tak, aby byl vnesen ve stejné hodnotě.

%Prodávající nese riziko náhodné ztráty nebo náhodného zhoršení předmětu koupě, dokud nebude předán dopravní společnosti v místě plnění nebo dokud nebude v prodlení.

Správce přijímá pověření ke správě jmění vyčleněného do svěřenského fondu a správě svěřenského fondu. Správce přijímá zakladatelem vyčleněný a svěřený majetek do své správy a tímto se zavazuje tento majetek držet a spravovat s péčí řádného hospodáře ke dni určeném v § 1c.

Pokud by z jakýchkoliv důvodů nebyl svěřenský správce schopen ujmou se správy, určí nového svěřenského správce všichni obmyšlení, kteří jsou zletilí a plně svéprávní, na společném hlasování, kde každý obmyšlený má právě tolik hlasů, kolik mu je let. V případě neshod a neurčení nového správce ze strany obmyšlených určí nového správce na návrh kteréhokoliv z obmyšlených soud.

%\parnumberfalse
%[Bsp.: Der Kaufgegenstand ist dem Transportunternehmen \textcolor{blue}{<Name>} an \textcolor{blue}{<Adresse, Ort>} zu übergeben.

%Die Verkäufer trägt die Gefahr des zufälligen Untergangs oder der zufälligen Verschlechterung des Kaufgegenstands, bis dieser am Erfüllungsort dem Transportunternehmen übergeben wird oder dieses mit der Annahme in Verzug ist.]
%\parnumbertrue

\SubClause{title={Okamžik zřízení svěřenského fondu}}

Svěřenský fond vzniká okamžikem, kdy budou kumulativně splněny obě podmínky níže určené.
1b První podmínkou je přijetí pověření ke správě ze strany správce.\\
2b Drouhou pomínku je splnění odkládací podmínky spočívající v úmrtí zakladatele svěřenského fondu.

%Finalní podmínku tvoří zápis svěřenského fondu do evidence svěřenských fondů kde dni, ve kterém došlo ke splnění obou kumulativních podmínek.

Je li svěřenských správců výce, postačí ke splnění podmínky vymezené výše v bodě 1b, aby takovouto správu přijal alespoň jeden z nich.

%Popsat také, že zde již nemusí být uvedené nic jiné, neboť všechny náležitosti týkající se převodu majetku popřípadě dědického řízení jsou již jasné z právní úpravy institutu svěřenského fondu a úpravy dědického řízení.

\SubClause{title={Okamžik vzniku svěřenského fondu}}

Svěřenský fond vzniká dnem zápisu do evidence svěřenských fondů.

\Clause{title={Správa a výkonvlastnických práv}}

%\SubClause{title={Kaufpreis und Sicherheit}}

\SubClause{title={Vlastnická práva k vyčleněnému majetku a správa o vyčleněný majetek}}

Svěřenskému správci náleži plná správa k vyčleněnému majetku.

Svěřenský správce vykonává vlastnická práva k majetku, každý svým vlastním jménem na účet svěřenského fondu.

%Der Käufer hat den Kaufpreis von CHF \textcolor{blue}{<Zahl>} bis zum \textcolor{blue}{<Datum>} auf das Konto Nr.~\textcolor{blue}{<Zahl>} der Z-Bank in \textcolor{blue}{<Postleitzahl, Ort>} zu überweisen.

%Das Eigentum am gelieferten Kaufgegenstand verbleibt bis zur Bezahlung des Kaufpreises beim Verkäufer. Kommt der Käufer mit der Zahlung des Kaufpreises in Verzug, so ist der Verkäufer berechtigt, den Eigentumsvorbehalt auf Kosten des Käufers im Eigentumsvorbehaltsregister eintragen zu lassen.

%Der Käufer hat bis spätestens am \textcolor{blue}{<Datum>} eine bedingungslose und unbefristete Bankgarantie über die Bezahlung von CHF \textcolor{blue}{<Zahl>} einer schweizerischen Grossbank oder einer Kantonalbank beizubringen \textcolor{blue}{<evt. andere Sicherheiten, z.B. Bürgschaft usw.>}.

%\parnumberfalse[Bsp.:
%Der Käufer hat den Kaufpreis von CHF \textcolor{blue}{<Zahl>} dem Verkäufer Zug um Zug gegen Übergabe des Kaufgegenstandes am Erfüllungsort in bar zu bezahlen.]
%\parnumbertrue

%Ist der Käufer mit einer Zahlungspflicht in Verzug, so hat er Verzugszinsen von \textcolor{blue}{<Zahl>}\% und Schadenersatz zu bezahlen.

%Ist der Käufer mit der Zahlung des Kaufpreises oder der Leistung einer Sicherheit in Verzug, so kann der Verkäufer dem Käufer eine Nachfrist von \textcolor{blue}{<Zahl>} Tagen setzen und nach deren unbenutztem Ablauf entweder innert \textcolor{blue}{<Zahl>} Tagen den Rücktritt vom Vertrag erklären und Schadenersatz (positives oder negatives Vertragsinteresse) oder weiterhin die Zahlung des Kaufpreises oder die Leistung der Sicherheit verlangen.

\Clause{title={Statut svěřenského fondu \textcolor{blue}{<Název svěřenského fondu>}}}

\SubClause{title={Označení}}

Označení svěřenského fondu je \textcolor{blue}{<Označení>}, \textbf{svěřenský fond} (dále jen "svěřenský fond").

\SubClause{title={Označení majetku}}

Svěřenský fond je tvořen \textcolor{blue}{<Tímto>} majetkem. / Majetek je označen ve smlouvě o založení svěřenského fondu.

\parnumberfalse
[Příklad z Knihy Svěřenské fondy, Příležitosti a rizika od Autorů Svejkovský, Kovář a kolektiv.: \\

Nemovitý majetek zakladatele: \\
%(1a) pozemek parc. č. .................... jehož součástí je budova ..................., vše zapsáno na listu vlastnictví č. ................... v kat. území ..................., obec ................... Nemovitosti jsou v územním obvodu, ve kterém vykonává státní správu Katastrální úřad pro ..................., katastrálnípracoviště...................\\
(1a) pozemek parc. č. ........ jehož součástí je budova ........, vše zapsáno na listu vlastnictví č. ........ v kat. území ........, obec ........ Nemovitosti jsou v územním obvodu, ve kterém vykonává státní správu Katastrální úřad pro ........, katastrální pracoviště ........\\
%(2a) podíl jedné ideální šestiny k pozemku parc.č...................., zapsaný na listu vlastnictví č. .................... v obci..................., kat. území.................... Nemovitosti jsou v územním obvodu, ve kterém vykonává státní správu Katastrální úřad pro .................... kraj, katastrální pracoviště....................\\
(2a) podíl jedné ideální šestiny k pozemku parc. č. ........, zapsaný na listu vlastnictví č. ........ v obci ........, kat. území ........ Nemovitosti jsou v územním obvodu, ve kterém vykonává státní správu Katastrální úřad pro ........ kraj, katastrální pracoviště ........\\
%(3a) podíl jedné ideální šestiny k pozemkům parc. č. ........, parc. č........, parc.č........, parc.č........, vše zapsáno na listu vlastnictví č. ........ v obci ........, kat. území ........ Nemovitosti jsou v územním obvodu, ve kterém vykonává státní správu Katastrální úřad pro ........, katastrální pracoviště........
(3a) podíl jedné ideální šestiny k pozemkům parc. č. ........, parc. č. ........, parc. č. ........, parc. č. ........, vše zapsáno na listu vlastnictví č. ........ v obci ........, kat. území ........ Nemovitosti jsou v územním obvodu, ve kterém vykonává státní správu Katastrální úřad pro ........, katastrální pracoviště ........

Dále z majetku zakladatele, který je v SJM:

(1b) \textcolor{blue}{<Obdobný výpis>},\\
(2b) \textcolor{blue}{<Obdobný výpis>}.

Zakladatel tímto vyčleňuje ze svého vlastnictví majetek vypsaný v bodech 1a, 2a, 3a, 1b, 2b se všemi součástmi a příslušenstvím, se všemi omezeními váznoucími na nemovitostech a to zejména: \\
- \textcolor{blue}{<Obdobný výpis>},\\
- \textcolor{blue}{<Obdobný výpis>},\\
- \textcolor{blue}{<Podíly ve společnosti, identifikace společnosti>},\\
- \textcolor{blue}{<Akcie ve společnosti, identifikace společnosti>},\\
- peněžitý majetek ve výši 5 000 000 Kč (slovy: pět milionů korun českých)
]
\parnumbertrue


\SubClause{title={Vymezení účelu svěřenského fondu}}

Účelem svěřenského fondu je \textcolor{blue}{<Účel>}.

\parnumberfalse
[Příklad.: Svěřenský fond je zřízen za soukromým účelem a to za účelem finanční a materiální podpory rodiny \textcolor{blue}{<Příjmení>} / Zakladatelem určených obmyšlených. / Účelem svěřenského fondu je podpora obmyšlených určených v § 3f. / Účelem svěřenského fondu je podpora \textcolor{blue}{<Jména obmyšlených>}. / Účelem svěřenského fondu je převod podpora \textcolor{blue}{<Jména obmyšlených>} a následný převod majetku do jejich vlastnictví / do vlastnictví \textcolor{blue}{<Jméno a Příjmení>}.]
\parnumbertrue

\SubClause{title={Podmínky pro plnění ze svěřenského fondu}}

Plnění ze svěřenského fondu je možné poskytnou pouze obmyšlenému svěřenského fondu.

O vyplacení plnění ze svěřenského fondu, včetně plodů a užitků k němu náležejícím, popřípadě o převodu majetku, rozhoduje správce dle plnění podmínek níže ustanovených ze strany obmyšlených.

Obdobně o vyplacení výnosů vzešlých z funkce svěřenského fondu, včetně majetku, plodů a užitků k nim náležejícím rozhoduje svěřenský správce.

Podmínky, které musí obmyšlení svěřenského fondu splňovat, aby jim mohlo být plněno z prostředků svěřenského fondu,(aby jim mohl být převeden majetek do jejich vlastnictví, nebo spoluvlastnictví) jsou následující:\\
1. \textcolor{blue}{<Podmínka>},\\
2. \textcolor{blue}{<Podmínka>},\\
3. \textcolor{blue}{<Podmínka>}.

\parnumberfalse
[Příklad.: 
Podmínky, které musí obmyšlení svěřenského fondu splňovat, aby jim mohlo být plněno z prostředků svěřenského fondu,(aby jim mohl být převeden majetek do jejich vlastnictví, nebo spoluvlastnictví) jsou následující:\\
1. \textcolor{blue}{<Vždy uplynutí celého kalendářního měsíce, přičemž vplate se provádí vždy prvního dne měsíce následující po měsíci, ve kterém vznikl nárok na plnění>},\\
%Zjistit jak se říká mateřská dovolená (pro výpočet považuje za)
2. \textcolor{blue}{<Trvající pracovní poměr obmyšleného, výkon samostatné výdělečné činnosti, studium na vysoké škole po standardní dobu studia, nebo stav, který se pro účely sociálního pojištění považuje za výkon práce dle zákoníku práce, mateřká a rodičovská dovolená, péče o dítě, či jinou osobu závyslou na péči obmyšleného, popřípadě pokud je obmyšlený účasten na důchodovém pojištění z titulu náhradní doby sociálního pojištění.>},\\
3. \textcolor{blue}{<Tížívá životní situaci, či situace, která se považuje za přiměřenou k důvodu výplaty prostředků ze svěřenského fondu podle uvážení Svěřenského správce.>}.\\

Právo na výplatu plnění obmyšlenému ve výši stanovené zakladatelem/určené správcem/určené v části \textcolor{blue}{<Část>} této smlouvy/statutu/určeného v příloze k této smlouvě/statutu vzniká splněním jakékoliv z podmínek stanovených v § 3d(4) statutu svěřenského fondu a to okamžikem, kdy došlo ke splnění této podmínky. Je možné žádat o vyplacení i zpětně, maximálně však do 2 let od vzniku nároku na plnění ze svěřenského fondu.

Výplata se provádí vždy k rukoum obmyšleného na jeho účet vedený u poskytovatele bankovních služem v případě peněžních prostředků. V případě převodu jiné věci se Svěřenský správce a obmyšlený řídí odpovídajícím právním předpisem.

V případě kumulativního splnění několika podmínek stanovených výše je obmyšlenému vyplácena stejná částka/několik plnění současně, Svěřenský správce však může dle svého uvážení v případě nutnosti navýšit částku k vyplacení dle svého uvážení (maximálně do x násobku standardního plnění.

S ohledem na situaci svěřenského fondu, nebo osobní situaci obmyšleného může ten, kdo je oprávněn k určení obmyšleného snížit, zvýšit, nebo zrušit plnění ze svěřenského fondu a obdobně změnit, nebo zrušit své předcházející rozhodnutí.]
\parnumbertrue

\SubClause{title={Trvání svěřenského fondu}}

Svěřenský fond je dle dohody smluvní stran uzavřen na dobu \textcolor{blue}{<Doba>}. / Svěřenský fond je uzavřen do doby splění rozvazovací podmínky \textcolor{blue}{<Podmínka>}.

Svěřenský fond končí a jeho správa končí v případě, kdy dojde k vyplacení veškerého plnění ze svěřenského fondu, tedy veškerého majetku, včetně jeho plodů a užitků vyjma případů, kdy se obmyšlení, či jiná osoba přistupující ke svěřenskému fondu dohodnou na doplnění majetku do svěřenského fondu.

Svěřenský fond a jeho správa také končí v případě, kdy se všichni obmyšlení vzdají práva na plnění ze svěřenského fondu. V takovém případě bude stávající majetek ve vlastnictví svěřenského fondu rozdělen všem obmyšleným podle přání zakladatele svěřenského fondu.

%Doplnit že toto plnění bude vyplaceno všem obmyšleným rovným dílem.

\SubClause{title={Určení osob, kterýmžto má být ze svěřenského fondu plněno}}

Obmyšlenými svěřenského fondu jsou \textcolor{blue}{<Jméno, Příjmení>}. / Obmyšlenými svěřenského fondu jsou ti, kteří tak určí svěřenský správce v souladu s přáním zůstavitele. Toto přání musí být v listiné formě a obsahovat při nejmenším ověřený podpis zakladatele.

Ve smyslu paragrafu 1464 Občanského zákoníku je obmyšlený svěřenského fondu i ten člen rodiny \textcolor{blue}{<Příjmení>}, který v den vzniku svěřenského fondu ještě není.

%Zde se podívat, proč uvedl Tomáš Střeleček, že toto je dle ustanovení paragrafu 700 Občanského zákoníku.

Svěřenský správce může určit dalšího obmyšleného svěřenského fondu, pokud je příbuzný kteréhokoliv stávajícího obmyšleného svěřenského fondu na žádost daného obmyšleného, či pokud měl být dle účelu svěřenského fondu určen jako obmyšlený, ale zakladatel jej opomenul zmínit.

\SubClause{title={Počet svěřenských správců, jejich jmenování, odvolání a způsob jejich jednání}}

Svěřenský fond má jednoho/více správce/ů.\\
1b Zakladatel jmenuje tyto správce:\\
1. \textcolor{blue}{<Jméno, Příjmení, Adresa, Datum narození>},\\
2. \textcolor{blue}{<Jméno, Příjmení, Adresa, Datum narození>}.

Svěřenský správce jedná ve všech záležitostech svěřenského fondu, přičemž má právo konat cokoliv, co považuje za nezbytné k udržení a rozšíření majetku do svěřenského fondu vloženého.

Svěřenský správce jedná s náležitou péčí a s péčí řádného hospodáře.

V případě škody se postupuje dle odpovídajících ustanovení Občanského zákoníku v platném znění.

Dokud není správce pověřen správou svěřenského fondu a tuto správu nepřijme, nelze rozhodovat o jakémkoliv plnění ze svěřenského fondu do doby nápravy této situace.

Průběh určení náhradního svěřenského správce je dále upraven ve smlouvě o založení svěřenského fondu.

\parnumberfalse
[Příklad z Knihy Svěřenské fondy, Příležitosti a rizika od Autorů Svejkovský, Kovář a kolektiv, na kterém lze dobře doložit, že je možné ze strany zakladatele stanovit mnoho dalších požadavků na fungování správy.: \\
Správci budou vždy dva, a to vždy jedna osoba z řad obmyšlených nebo zakladatelů a jedna osoba odlišná od zakladatelů i obmyšlených.

Správci budou svěřenský fond spravovat společně. Správci budou jednat společně.

Činnost správce skončí:\\
- odstoupením správce,\\
- odvoláním zakladatele,\\
- omezením svéprávnosti osoby dosud svéprávné,\\
- nebo osvědčením o úpadku správce.

V případě skončení činnosti správce jmenují zakladatelé nového správce rozhodnutím přijatým zakladateli/zakladateli a obmyšlenými/obmyšlenými.

Nebude-li jmenován svěřenský správce postupem uvedeným shora do jednoho měsíce od skončení činnosti správce, jmenuje svěřenského správce tak, aby vždy byli jmenováni dva správci, soud podle § 1455 odst. 2 ObčZ. a to tak, aby byli vždy jmenováni dva správci.
.]
\parnumbertrue

\SubClause{title={Odměna správce}}

Správci náleží za správu svěřenského fondu \textcolor{blue}{<Období>} odměna ve výši \textcolor{blue}{<Výše>} měsíčně/za vykonaný úkon/za započatou hodinu provádění správy s připočtením příslušné DPH.

%Der Käufer hat den Kaufgegenstand am Liefertag/während der Lieferfrist am Erfüllungsort anzunehmen/abzuholen.

%Ist der Käufer mit der Annahme des Kaufgegenstandes in Verzug, so kann der Verkäufer dem Käufer eine Nachfrist von \textcolor{blue}{<Zahl>} Tagen setzen und nach deren unbenutztem Ablauf entweder innert \textcolor{blue}{<Zahl>} Tagen den Rücktritt vom Vertrag erklären und Schadenersatz (positives oder negatives Vertragsinteresse) oder weiterhin die Annahme verlangen.

%\parnumberfalse[Bsp.:
%Der Käufer hat den Kaufgegenstand spätestens bis zum \textcolor{blue}{<Datum>} am Erfüllungsort anzunehmen/abzuholen. Unterlässt er dies, so kann der Verkäufer entweder innert \textcolor{blue}{<Zahl>} Tagen den Rücktritt vom Vertrag erklären und Schadenersatz (positives oder negatives Vertragsinteresse) oder weiterhin die Annahme verlangen.]
%\parnumbertrue



%\Clause{title={Haftung des Verkäufers für Vertragswidrigkeiten der Lieferung}}

%\SubClause{title={Ansprüche des Käufers}}

%Hat der Verkäufer vertragswidrige Ware geliefert, so kann der Käufer vom Verkäufer zunächst nur verlangen, dass er die Vertragswidrigkeit nach seiner Wahl durch Nachbesserung oder Ersatzlieferung kostenlos und ohne unverhältnismässige Unannehmlichkeiten für den Käufer behebt. Der Käufer kann dem Verkäufer dazu eine angemessene Frist von mindestens \textcolor{blue}{<Zahl>} Tagen setzen. Hat der Verkäufer die Vertragswidrigkeit innert dieser Frist nicht behoben, so kann der Käufer Minderung oder, wenn ihm das Behalten der Lieferung unzumutbar ist, Wandelung verlangen.

%\parnumberfalse
%[Bsp. 1: Hat der Verkäufer vertragswidrige Ware geliefert, so kann der Käufer nach seiner Wahl die kostenlose Nacherfüllung durch Ersatzlieferung oder Nachbesserung am Ort verlangen, wo sich die Lieferung gewöhnlich befindet. Die vom Käufer verlangte Nacherfüllung kann der Verkäufer verweigern, wenn er nachweist, dass sie unmöglich oder unverhältnismässig ist. Der Käufer kann dem Verkäufer zur Nacherfüllung eine angemessene Frist von mindestens \textcolor{blue}{<Zahl>} Tagen setzen. Wird die Nacherfüllung innert dieser Frist nicht erbracht, so kann der Käufer nach seiner Wahl zudem Minderung oder Wandelung verlangen. Die Wandelung ist jedoch bei unerheblichen Mängeln ausgeschlossen.]

%[Bsp. 2: Die Haftung des Verkäufers für Vertragswidrigkeiten der Lieferung wird ausgeschlossen. Dieser Ausschluss gilt nicht, soweit der Verkäufer dem Käufer gewisse Eigenschaften des Kaufgegenstands ausdrücklich zugesichert oder arglistig verschwiegen hat.]
%\parnumbertrue

%\SubClause{title={Verwirkung}}

%Ansprüche aus bei einer übungsgemässen Untersuchung erkennbaren Vertragswidrigkeiten sind verwirkt, wenn diese der Käufer dem Verkäufer nicht innert \textcolor{blue}{<Zahl>} Tagen nach Übergabe der Lieferung am Erfüllungsort \textcolor{blue}{<ev. Bestimmungsort>} angezeigt hat.

%Ansprüche aus anderen Vertragswidrigkeiten sind verwirkt, wenn diese der Käufer dem Verkäufer nicht \textcolor{blue}{<Zahl>} Tage nach ihrer Entdeckung, spätestens jedoch \textcolor{blue}{<Zahl>} Jahre nach Übergabe der Lieferung am Erfüllungsort angezeigt hat.

%\parnumberfalse
%[Bsp. 1: Ansprüche aus Vertragswidrigkeiten sind verwirkt, wenn diese der Käufer dem Verkäufer nicht \textcolor{blue}{<Zahl>} Tage nach ihrer Entdeckung, spätestens jedoch \textcolor{blue}{<Zahl>} Monate/Jahre nach der Übergabe der Lieferung am Erfüllungsort angezeigt hat.]

%[Bsp. 2: Ansprüche aus Vertragswidrigkeiten sind verwirkt, wenn diese der Käufer dem Verkäufer nicht spätestens \textcolor{blue}{<Zahl>} Monate/Jahre nach der Übergabe der Lieferung am Erfüllungsort schriftlich angezeigt hat.]
%\parnumbertrue

%\SubClause{title={Verjährung}}

%Die Gewährleistungsansprüche verjähren \textcolor{blue}{<Zahl>} Jahr(e) nach Übergabe der Lieferung am Erfüllungsort.

%\parnumberfalse
%[Bsp.: Gewährleistungsansprüche verjähren \textcolor{blue}{<Zahl>} Monate seit Eintreffen der Mängelanzeige.]
%\parnumbertrue



%\Clause{title={Schadenersatzpflicht des Verkäufers und Versicherung}}

%Schadenersatzansprüche des Käufers werden auf insgesamt CHF \textcolor{blue}{<Zahl>} beschränkt.

%\parnumberfalse
%[Bsp. 1: Die Haftung des Verkäufers für indirekte Schäden oder Folgeschäden, wie entgangener Gewinn, wird im gesetzlich zulässigen Umfang abbedungen.]

%[Bsp. 2: Schadenersatzansprüche des Käufers werden im gesetzlich zulässigen Umfang abbedungen.]
%\parnumbertrue

%Der Verkäufer verpflichtet sich, eine (Betriebs-)Haftpflichtversicherung abzuschliessen und während \textcolor{blue}{<Zahl>} Jahren nach Vertragsschluss aufrechtzuerhalten, die sämtliche Personen- und Sachschäden (und Folgeschäden daraus) von mindestens CHF \textcolor{blue}{<Zahl>} pro Schadensfall und übrige Schäden (reine Vermögensschäden) von mindestens CHF \textcolor{blue}{<Zahl>} pro Schadensfall deckt. Der Verkäufer verpflichtet sich, dem Käufer auf dessen Verlangen einen entsprechenden Versicherungsnachweis zu übergeben.



%\Clause{title={Schriftform}}

%Dieser Vertrag tritt mit der Unterzeichnung durch beide Parteien in Kraft. Er umfasst den Vertragstext und die darin erwähnten Anhänge. Vertragsänderungen und -ergänzungen sind nur in Schriftform und bei Unterzeichnung durch beide Vertragsparteien gültig.

%Mitteilungen, die sich auf diesen Vertrag und seine Abwicklung beziehen, sind in deutscher Sprache zu verfassen und schriftlich oder in einer Form zu übermitteln, welche den Nachweis durch Text ermöglicht, wie namentlich Telex, Telefax und E-Mail.



%\Clause{title={Salvatorische Klausel}}

%Sollten Bestimmungen dieses Vertrages ganz oder teilweise rechtsunwirksam sein oder werden, so wird dadurch die Gültigkeit der übrigen Bestimmungen nicht berührt. Die Parteien verpflichten sich, unwirksame Bestimmungen so zu ersetzen, dass ihr wirtschaftlicher Zweck soweit zulässig gewahrt wird.



%\Clause{title={Anwendbares Recht}}

%Dieser Vertrag untersteht schweizerischem Recht, unter Ausschluss des Übereinkommens der Vereinten Nationen über Verträge über den Internationalen Warenkauf.



%\Clause{title={Mediationsklausel und Gerichtsstand}}

%Sollte es im Zusammenhang mit diesem Vertrag oder dessen Gültigkeit zu Streitigkeiten kommen, beabsichtigen die Parteien, zunächst eine Mediation nach den jeweils gültigen Regeln der \textcolor{blue}{<Name der Organisation>} einzuleiten, und ordentliche Klagen erst zu erheben, wenn in der Mediation keine gütliche Einigung gefunden werden konnte.

%Ausschließlicher Gerichtsstand ist \textcolor{blue}{<Ortsangabe>}.

\end{contract}
\newpage

\vspace{50pt} 
\noindent\rule{7cm}{.4pt}\hfill\rule{7cm}{.4pt}\par 
\noindent Datum, Místo \hfill Podpis zakladatele

\vspace{50pt} 
\noindent\rule{7cm}{.4pt}\hfill\rule{7cm}{.4pt}\par 
\noindent Datum, Místo \hfill Podpis správce / správců

\addchap{Příloha č. 2 Vzor listiny přání zakladatele o určení obmyšlených svěřenského fondu}.

\begin{contract}

\Clause{title={Určení obmyšlených a výplaty plnění v případě zániku svěřenského fondu}}%Prohlášení smluvních stran}}

\SubClause{title={Prohlášení zakladatele}}

Pro případy uvedené ve smlouvě o založení svěřenského fondu nebo případy uvedené ve statutu svěřenského fondu \textcolor{blue}{<Označení svěřenského fondu>}, odkazují na tuto listinu, zakladatel svěřenského fondu dále pořizuje tuto listinu přání.

\SubClause{title={Určení obmyšlených}}

Obmyšlenými svěřenského fondu určuji:
1. \textcolor{blue}{<Jméno, Příjmení, Datum narození, Adresa>},
2. \textcolor{blue}{<Jméno, Příjemní, Datum narození, Adresa>}.

Dále si zakladatel přeje, aby se obmyšleným svěřenského fondu stal dle paragrafu 1464 Občanského zákoníku i člen rodiny \textcolor{blue}{<Příjmení>}, který ke dni vzniku svěřenského fondu ještě není.

\SubClause{title={Správce svěřenského fondu}}

Správcem svěřenského fondu určiji:
1. \textcolor{blue}{<Jméno, Příjmení, Datum narození, Adresa>},
2. \textcolor{blue}{<Jméno, Příjemní, Datum narození, Adresa>}.

\SubClause{title={Způsob vyplacení majetku svěřenského fondu po jeho zániku}}

Vyplacení majetku náležejícího do svěřenského fondu ke dni jeho zániku nechť je rovnoměrně rozdělen mezi všechny obmyšlené svěřenského fondu.

%Napsat, že zde je možné určit ze strany zakladatele cokoliv, zjistit jak je to například se založením dalšího svěřenského fondu, popřípadě popsat nějaké další jiné možnosti, aby nedošlo ke štěpení majetku svěřeného do svěřenského fondu. Inspirovat se tím co psala Barbora Bednařáková na konci svého díla Svěřenský fond jako institut mezigeneračního převodu majetku.

\SubClause{title={Závěrem}}

Pokud je možno, Svěřenský fond není vázán na jakoukoliv jurisdikci. Fond se podřídí právní úpravě jiné jurisdikce na základě rozhodnutí správce.

Výše uvedené slouží pouze jako příklad pro správu Svěřenského fondu a správci se mohou dle svého uvážení, vyjma základního určení obmyšlených svěřenského fondu, odchýlit, zejména při placení daní nebo z pragmatických důvodů.
	
\end{contract}
\newpage

\vspace{50pt} 
\noindent\rule{7cm}{.4pt}\hfill\rule{7cm}{.4pt}\par 
\noindent Datum, Místo \hfill Podpis zakladatele

\vspace{50pt} 
\noindent\rule{7cm}{.4pt}\hfill\rule{7cm}{.4pt}\par 
\noindent Datum, Místo \hfill Podpis správce / správců

\end{document}