\documentclass{article}

% --------------------------------------------------------------------------
% Language definition
% --------------------------------------------------------------------------

\usepackage[utf8]{inputenc}
%\usepackage[czech]{babel}

% --------------------------------------------------------------------------
% Package definition and folder with images
% --------------------------------------------------------------------------

\usepackage{graphicx}
\graphicspath{ {./Obrazky/} }

% --------------------------------------------------------------------------
% Package difinition
% --------------------------------------------------------------------------

\usepackage{geometry}
 \geometry{
 a4paper,
 total={170mm,257mm},
 left=20mm,
 top=20mm,
 }
\usepackage{fancyhdr}
\usepackage{xcolor}
\usepackage{titling}
\usepackage{array}
\usepackage{todonotes}
\usepackage{tabularx} % Pro vytvoření tabulky, která bude mít full page width, musím zkontrolovat, zda tento package někde v práci něco nerozhodil
\usepackage{enumitem}
\usepackage{pdfpages}

\usepackage{fontspec}
\setmainfont{Times New Roman}

% --------------------------------------------------------------------------
% Base header definition
% --------------------------------------------------------------------------

\setlength\headheight{26pt}
\fancyhf{}
\lhead{\includegraphics[width=3cm,height=\dimexpr \headheight-\dp\strutbox]{newcevro}}
\rhead{\small{\leftmark}}
\rfoot{\thepage}

% --------------------------------------------------------------------------
% Bibliography definition
% --------------------------------------------------------------------------

\usepackage[backend=biber,defernumbers=true]{biblatex}
\addbibresource{Zdroje.bib}

% --------------------------------------------------------------------------
% Přidán backend biber pro sorting, nakonec jsem se ale rozhodl skrýt čísla v bibliografii pomocí defernumbers a omitnumbers
% Deklarace, že před bibliografií nic nebude, od kraje nebude žádný prostor a mezi bibligorafickými záznami bude větší mezera
% --------------------------------------------------------------------------

\DeclareFieldFormat{labelnumberwidth}{}
\setlength{\biblabelsep}{0pt}
\setlength\bibitemsep{1.5\itemsep}

% --------------------------------------------------------------------------
% Footnote package definition
% --------------------------------------------------------------------------

\usepackage{footnote} %Package použitý k tomu, aby byli citace uvnitř float elementů pod čarou
\makesavenoteenv{figure} %Nastavení toho, aby byli citace uvnitř figure pod čarou

% --------------------------------------------------------------------------
% Package definition
% --------------------------------------------------------------------------

\usepackage{xcolor}

% --------------------------------------------------------------------------
% Different language for contents and figures
% --------------------------------------------------------------------------

\renewcommand\contentsname{Obsah}
\renewcommand\listfigurename{Seznam obrázků}
\renewcommand\figurename{Obrázek}

% --------------------------------------------------------------------------
% For printing roman numerals
% --------------------------------------------------------------------------

\makeatletter
\newcommand*{\rom}[1]{\expandafter\@slowromancap\romannumeral #1@}
\makeatother

% --------------------------------------------------------------------------
% For hiding part of the work if needed
% Usage
% \ifcomment
% Part I want to hide
% \fi
% Then
% |           Command          | Result |
% | -------------------------- | ------ |
% | \commenttrue               | Shown  |
% | <percentsight>\commenttrue | Hidden |
% --------------------------------------------------------------------------

\newif\ifcomment
%\commenttrue

% --------------------------------------------------------------------------
% Special header style definition for sections with long text in header
% This is however useful only if page width would be too small to fit the whole section name on one line
% --------------------------------------------------------------------------

\fancypagestyle{smallertextinheader}{ % Dokončit styl s menším textem v hlavičce, již je dokončeno
   \fancyhf{}
   \fancyhead[L]{\includegraphics[width=3cm, height=\dimexpr \headheight-\dp\strutbox]{newcevro}}
   \fancyhead[R]{%
   \parbox[b]{\dimexpr \textwidth-3cm-\columnsep}%
   {\small\uppercase\leftmark}}%
   \fancyfoot[R]{\thepage}
}

% --------------------------------------------------------------------------
% Special header for contents only
% --------------------------------------------------------------------------

\fancypagestyle{Contents}{ % Dokončit styl s menším textem v hlavičce, již je dokončeno
   \fancyhf{}
   \fancyhead[LE,LO]{\includegraphics[width=3cm, height=\dimexpr \headheight-\dp\strutbox]{newcevro}}
   \fancyhead[RE,RO]{\small{\uppercase{\rightmark}}}
}

% --------------------------------------------------------------------------
% Basic page style definition
% --------------------------------------------------------------------------

\pagestyle{fancy}

% --------------------------------------------------------------------------
% Pretitle definition
% --------------------------------------------------------------------------

\pretitle{
	\begin{center}
	\LARGE
	\includegraphics[width=10cm,height=3cm,keepaspectratio]{newcevro}\\
}
\posttitle{\end{center}}

% --------------------------------------------------------------------------
% Document body --------------
% --------------------------------------------------------------------------

\begin{document}

\setcounter{section}{5}
\section{Svěřenský fond jako mezigenerační nástroj transferu majetku}

\setcounter{footnote}{344}

\newpage

% --------------------------------------------------------------------------
%zda na svěřenský fond pohlížím optikou zakladatele, nebo obmyšleného, je zřejmé, že

%\begin{table}[]
% --------------------------------------------------------------------------

\noindent\begin{tabularx}{\textwidth}{l|X|X|X|X|}
\cline{2-5}
                               & \multicolumn{2}{c|}{Svěřenský fond} & \multicolumn{2}{c|}{Dědictví} \\ \hline
\multicolumn{1}{|l|}{\textbf{Faktor}}   & \textbf{Výhody}           & \textbf{Problémy}          & \textbf{Výhody}        & \textbf{Problémy}       \\ \hline
\multicolumn{1}{|l|}{Faktor 1} &

% --------------------------------------------------------------------------
%\begin{itemize}
%\item Nejasná regulace a chybějící judikatura
%\item Rozdílné pojetí vlastnictví
%\item Zákonné nejasnosti změn svěřenských fondů před a za jejich trvání
% --------------------------------------------------------------------------

Bohužel, k tomuto datu žádná. Útěchou však může být, že institut projde nanejvýše novelizací. Domnívám se, že zákonodárce nemá v úmyslu zrušit možnost zakládání svěřenských fondů.

% --------------------------------------------------------------------------
%\end{itemize}
% --------------------------------------------------------------------------

&      Strohá právní úprava, neexistence soudní praxe v této oblasti.
Problematika ve vztahu k započtení na povinný díl a k nepominutelným dědicům\footnote{Tamtéž.}. Není zcela iluzorní očekávat, že se právní úprava svěřenského fondu může změnit, neboť se stále jedná o poměrně nový institut.

% --------------------------------------------------------------------------
%Ochrana dědictví\footnote{Kapitola 6.7.1, Ochrana dědictví před rozmařilostí, nezkušeností a nezodpovědností ze strany dědiců, aneb bod 1.}
% --------------------------------------------------------------------------

&      Dlouhodobá, stálá a rigidní úprava, poskytuje veskrze všechny potenciální aspekty.          &       Není vyřešena kolize se svěřenským fondem, tento problém je ale možné rovněž přičítat svěřenskému fondu.       \\ \hline
\multicolumn{1}{|l|}{Faktor 2} &         Zde se dostáváme k dominanci svěřenského fondu, možnosti účelů a nastavení statutu svěřenského fondu, stejně tak jako vymezení postavení obmyšleného, jsou opravdu velké, to vše hraje společně s insolvenčním efektem důležitou roli pro určité specifické převody majetku, například v případě obchodního závodu, popřípadě třeba i za filantropickým účelům\footnote{Tamtéž.}.          &        Žádné ve smyslu autonomie vůle zakladatele. Avšak negativum lze spatřovat v tom, že po vzniku svěřenského fondu zaniká právo zakladatele a nevzniká ani právo obmyšleného jakožto vlastnické právo k vyčleněnému majetku, možnost s ním tak disponovat neexistuje. Naopak v případě konečného převodu pomocí instrumentů dědického práva získá dědic daný majetek do výlučného vlastnictví a vlastnická triáda je tak plně uplatnitelná. Rovněž vztahy uvnitř svěřenského fondu jsou pouze obligační, což může sehrát negativní roli například v případě, že by svěřenský správce s majetkem neoprávněně naložil. Jako nevýhodu lze rovněž považovat nemožnost změny statutu\footnote{Tamtéž.}.      &       Také velké možnosti autonomie ze strany zůstavitele, například ve formě dovětku, rovněž placení veškerých pohledávek u předluženého dědictví je možné se vyhnout za pomoci výhrady soupisu. Převedení majetku je finální, nový vlastník má jak věcná práva, tak práva vlastnická. Může tak například majetek použít jako zástavu.       &      Autonomie vůle není tak vysoká jako u svěřenského fondu. Možnost nakládání s převedeným majetek je však mnohonásobně vyšší, u svěřenského fondu lze přeci čistě v teoretické rovině hovořit o tom, že obmyšlený je konečný vlastník majetku, toto ale nemá jakoukoliv oporu v zákoně, obmyšlený tak nemá k majetku jakékoliv právo, má pouze právo na plnění a na vydání při zániku svěřenského fondu, určí-li tak statut, a na kontrole činnosti svěřenského správce\footnote{\textit{Stricto sensum} lze tak práva obmyšleného na samotný majetek za doby trvání svěřenského fondu připodobnit k právům člověka k bytové jednotce jeho souseda, nemá tedy žádná věcné nebo vlastnická práva, jedná se o zcela oddělený majetek. Rozdíl je samozřejmě v plnění, finálním transferu majetku a oprávněním obmyšleného/zakladatele vůči správci a tedy v jistých ohledech správě svěřenského fondu.}.        \\ \hline
\end{tabularx}

\newpage

\noindent\begin{tabularx}{\textwidth}{l|X|X|X|X|}
\cline{2-5}
                               & \multicolumn{2}{c|}{Svěřenský fond} & \multicolumn{2}{c|}{Dědictví} \\ \hline
\multicolumn{1}{|l|}{Faktor 3} &         Svěřenský fond poskytuje i přes povinnou evidenci značnou míru anonymity, neboť jsou zveřejněny pouze základní informace o svěřenském fondu a svěřenském správci, listiny obsahující neveřejné údaje se rovněž nezveřejňují\footfullcite[§ 65e odst. 2]{noauthor_zakon_vr}. Princip materiální publicity ve vztahu ke zveřejněným údajům pak rovněž uspokojuje veřejnou důvěru.          &    Samotnou existenci evidence lze však považovat za zásah do soukromí uživatelů svěřenského fondu, osobně se však domnívám, že se o značný zásah do soukromí nejedná.             &        Převáděný majetek nemusí být předmětem žádné evidence, v jistých případech tak může být anonymita větší.        &     V evidencích je zapsaný samotný vlastník, například v katastru nemovitostí je tak možné přímo dohledat jméno majitele, v případě svěřenského fondu je možné dohledat pouze osobu svěřenského správce. V určitých případech tak může být pro majitele vhodnější, aby nebyl zapsán a jeho jméno tak nebylo veřejně dohledatelné.         \\ \hline
\multicolumn{1}{|l|}{Faktor 4} &         Je možné stanovit poměrně široký způsob dohledu a to dokonce aktivního.          &        Zakladatel ani obmyšlený nemají právo udělovat svěřenskému správci přímo pokyny, jejich přímý vliv na majetek je tak tímto omezen. Z povahy svěřenského fondu je to logické, neboť osamostatnění majetku a zvýšená anonymita společně s insolvenčním efektem z logiky věci musí omezit možnosti dispozic s majetkem původním vlastníkem. Ronovská rovněž zmiňuje nedostatečnou regulaci postavení a pravomocí svěřenských správců\footfullcite[Vydání 7-8 2016]{ronovska_bulletin_vyber_formy_spravy_majetku}, osobně se však domnívám, že s ohledem na možnost stanovit ve statutu poměrně extenzivní možnosti dohledu toto nepředstavuje zásadní problém. Další nevýhodu pak spatřuji v nemožnosti ustanovení správcem právnickou osobu specializující se na správu cizího majetku tak, jako je možné učinit v rámci jiných zahraničních právních řádů.        &        Majetek nabyde přímo další vlastník, kontrola nad majetkem je tak absolutní.        &    Kontrola nad specifickým majetkem může být velice složitá, například v případě obchodního závodu, v případě svěřenského fondu je možné docílit toho, že majetek bude spravovat profesionální správce lépe, než dědic nebo obmyšlený.          \\ \hline
\end{tabularx}

\newpage

\noindent\begin{tabularx}{\textwidth}{l|X|X|X|X|}
\cline{2-5}
                               & \multicolumn{2}{c|}{Svěřenský fond} & \multicolumn{2}{c|}{Dědictví} \\ \hline
\multicolumn{1}{|l|}{Faktor 5} &         Současná občanskoprávní úprava svěřenského fondu jej činí dostupný pro každého, kdo by o ně měl zájem a vlastní majetek, který by do něj mohl vyčlenit. U svěřenského fondu není před potenciálního zakladatele kladená žádná překážka, a to ať již institucionální, mocenská a dokonce ani finanční, neboť k jejich zřízení a vzniku nejsou kladeny, oproti jiným právnickým osobám, žádné nároky na minimální vložený majetek. Jediné co tak potenciální zakladatel musí mít na paměti jsou základní správní poplatky, které je nutné uhradit, ty však nejsou nikterak závratné výše, a náklady spojené se správou svěřenského fondu, ty budou zahrnovat zejména odměnu správce. Naopak při správném nastavení svěřenského fondu a výběru správce se majetek může kvůli vhodné správě zvyšovat.         & Svěřenský fond je levným institutem, který se hodí i pro správu menších majetků, žádné zásadní výdaje nelze v souvislosti s jeho správou očekávat, právě naopak transakční náklady jsou například oproti právnickým osobám nižší\footfullcite[Strana 312]{hayton_d_j_extending_2002}.                &        Finanční náklady v souvislosti s projednáním dědictví nejsou rovněž nikterak vysoké, bude se jednat zejména o odměnu notáři, jakožto soudnímu komisaři. Naopak následná správa majetku může být levnější než u svěřenského fondu, neboť jí vykonává sám vlastník.        &       Autor si není vědom zásadnějších nevýhod spojených s finanční nákladností projednání dědictví.       \\ \hline
\multicolumn{1}{|l|}{Faktor 6} &         Obecně řečeno jsou svěřenské fondy při plnění z majetku postaveny na roveň dědění, v případě osoby se vztahem k zakladateli\footnote{Viz kapitola 6.11.1, bod 4.}, nebo při zřízení \textit{mortis causa} jsou totiž od daně osvobozeny.          &  Avšak v případě plnění ze zisku je již uplatňována srážková daň společně se zdaněním zisku svěřenského fondu. Může tak docházek k dvojímu zdanění. Jedná se tak v podstatě o stejný režim, který se uplatňuje u jiných právnických osob\footnote{Za předpokladu, že se nejedná o veřejně prospěšného poplatníka.}.              &        Dědictví je rovněž osvobozeno od daně jakožto bezúplatné nabytí\footfullcite[§ 4a]{noauthor_zakon_zdp}, dědicové tak platí pouze odměnu notáři. Navíc je zamezeno dvojímu zdanění v případech, kdy k němu dochází u svěřenských fondů.       &       Ve smyslu zdanění dědictví, nebo zdanění po nabytí majetku žádné nespatřuji.       \\ \hline
\end{tabularx}

\newpage

\end{document}
