\documentclass{article}
\usepackage[utf8]{inputenc}
\usepackage{graphicx}
\graphicspath{ {./Obrazky/} }
\usepackage{fancyhdr}
\usepackage{xcolor}
\usepackage{titling}
\usepackage{array}
\setlength\headheight{26pt}
\lhead{\includegraphics[width=3cm]{newcevro}}

\pretitle{
	\begin{center}
	\LARGE
	\includegraphics[width=10cm,height=3cm,keepaspectratio]{newcevro}
}
\posttitle{\end{center}}

\begin{document}

\title{Svěřenský fond jako nástroj transferu majetku mezi generacemi}
\author{Dominik Bálint}
\date{01.12.2019}

\pagenumbering{gobble}
  \includegraphics[width=10cm,height=3cm,keepaspectratio]{newcevro} \\
  \vspace{15mm}
  \begin{center}
  {\Large Vysoká škola Cevro Institut} \\
  \vspace{15mm}
  {\Large \textbf{Svěřenský fond jako mezigenerační nástroj převodu majetku}} \\
  \vspace{15mm}
  {\Large Dominik Bálint} \\
  \vspace{15mm}
  {\Large Bakalářská práce} \\
  \vspace{49mm}
  {\Large \textbf{Praha 2019}} \\
  \end{center}
  
\newpage
  \maketitle
  \begin{center}
  {\Large Katedra veřejného práva} \\	
  \vspace{15mm}
  {\Large \textbf{Studijní program:} Veřejné právo} \\
  {\Large \textbf{Studijní obor:} Právo v obchodních vztazích} \\
  {\Large \textbf{Jméno vedoucího diplomové práce:} } \\
  {\Large Mgr. Ing.  Střeleček  Tomáš LL.M.} \\
  \end{center}
  
\newpage
  \thispagestyle{empty}
  \vspace*{\fill}

\noindent \textbf{Čestné prohlášení} \\

Prohlašuji, že jsem předkládanou práci zpracovala samostatně, uvedla
v ní všechny použité prameny a zdroje, které jsou uvedeny v seznamu použité
literatury, a v textu řádně vyznačila jejich použití. \\

\noindent V Praze dne \today \\
\vspace{10mm} \\
\begin{tabular}{p{6cm}c}
& ................................................. \\
& Dominik Bálint
\end{tabular}

\newpage

\vspace*{\fill}
\noindent \textbf{Poděkování} \\

	Na tomto místě bych rád poděkoval svému vedoucímu práce panu Mgr. Ing. 
Střeleček  Tomáš LL.M. za podporu, trpělivost, spolupráci, podněty ke zlepšení a 
také za čas, který mi věnoval při vedení práce.

\vspace*{\fill}
 
\newpage
  \pagestyle{fancy}
  \tableofcontents
  
\newpage
  \pagestyle{fancy}
  \pagenumbering{arabic}
  
\section{Úvod}

Práce si klade za cíl v historickém a právním kontextu představit institut svěřenského fondu a převzetí jeho právní úpravy z právního řádu provincie Quebec.
\linebreak

Dále se práce bude zabývat představením momentální právní úpravy svěřenského fondu, jeho možnostmi, ukázanými na příkladu praktického využití – jako nástroj, který může substituovat klasické dědění, i potenciálními možnostmi zneužití a jak (a zda) jsou odstraněny obavy o zneužití svěřenských fondů v souvislosti s novelizací občanského zákoníku v roce 2018. 
\linebreak

Dalším bodem práce bude faktické porovnání právní úpravy dědického práva s právní úpravou svěřenského fondu. Těžištěm této časti je snaha potvrdit, nebo vyvrátit tvrzení, že právní úprava dědického práva a svěřenského fondu je vyvážená a svěřenským fondem se nedají obcházet práva dědiců.
\linebreak

V tomto rámci se budu zabývat zejména zhodnocením právní úpravy svěřenských fondů a její analýzou v souvislosti s ochranou dědiců, vyložení potenciálních problémů momentální právní úpravy v rámci českého právního řádu, tzn. potenciální zneužití, v souvislosti s právem nepominutelných dědiců, které můžou vzniknout v případě souběhu dědického řízení a založení svěřenského fondu.
\linebreak

V závěru práce budou vyloženy systémové nedostatky v případě, že je naleznu a následovat bude návrh řešení nalezených problémů.

\newpage

\section{Svěřenské fondy}

\subsection{Historie svěřenských fondů u nás a ve světě}
	
\end{document}