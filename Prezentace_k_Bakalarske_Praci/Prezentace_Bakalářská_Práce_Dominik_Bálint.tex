\documentclass{beamer}

% +==============================+
% | Vydefinování maker a packagů |
% +==============================+

\usepackage[utf8]{inputenc}
\usepackage[czech]{babel}
\usepackage{graphicx}
\usepackage{tikz}
\usepackage{MnSymbol,wasysym}
\usepackage{multirow}

\newcommand{\cevroLogo}{
	\begin{tikzpicture}[remember picture,overlay]  
		\node [xshift=-1.6cm,yshift=-0.5cm] at (current page.north east)
		{\includegraphics[width=3cm,height=2cm,keepaspectratio]{images/newcevro}};
	\end{tikzpicture}
}

% +=======================================+
% | Základní informace o bakalářské práci |
% +=======================================+

\title{Svěřenský fond jako nástroj mezigeneračního převodu majetku}
%\subtitle[krátký název prezentace]{dlouhý název prezentace}
%\author{Dominik Bálint}
\author[My name]{Dominik Bálint\\ \footnotesize Vedoucí práce: Mgr. Ing. Tomáš Střeleček, LL.M., Ph.D.\\ \footnotesize Oponent práce: JUDr. Miroslav Sedláček, Ph.D., LL.M.}
\institute{CEVRO Institut}
\date{\today}

% +=============+
% | První sekce |
% +=============+

\section{Prezentace k obhajobě}

% +===============================================+
% | Použitý theme v prezentaci k bakalářské práci |
% +===============================================+

\usetheme{Singapore}

\begin{document}

% +===============+
% | Titulní slide |
% +===============+

\maketitle

% +====================+
% | Začátek prezentace |
% +====================+

\begin{frame}
	\frametitle{Obsah}
	\cevroLogo

\begin{itemize}
	\item Cíl práce%\pause
	\item Metodika%\pause
	\item Výsledky, doporučení%\pause
	\item Závěr%\pause
	\item Zodpovězení otázek
\end{itemize}
	
\end{frame}

% +=============+
% | Druhý slide |
% +=============+

\begin{frame}
	\frametitle{Cíl práce}
	\cevroLogo

Hello everybody

\end{frame}

% +=============+
% | Třetí slide |
% +=============+

\begin{frame}
	\frametitle{Metodika}
	\cevroLogo

\begin{columns}
	\column{0.5\textwidth}
		Hello everybody
	\column{0.5\textwidth}
		Hello everybody
\end{columns}

\end{frame}

% +==============+
% | Čtvrtý slide |
% +==============+

\begin{frame}
	\frametitle{Výsledky, doporučení a závěry}
	\cevroLogo

\begin{columns}
	\column{0.5\textwidth}
		Hello everybody
	\column{0.5\textwidth}
		Hello everybody
\end{columns}

\end{frame}

% +============+
% | Pátý slide |
% +============+

\begin{frame}
	\frametitle{Výsledky, doporučení a závěry}
	\cevroLogo

\begin{columns}
	\column{0.5\textwidth}
		Hello everybody
	\column{0.5\textwidth}
		Hello everybody
\end{columns}

\end{frame}

% +=============+
% | Šestý slide |
% +=============+

\begin{frame}
	\frametitle{Zodpovězení otázek}
	\cevroLogo

\begin{table}[]
\begin{tabular}{|l|}
\hline
Otázka 1 Vedoucí práce                                                                                                                                                                                                                       \\ \hline
\begin{tabular}[c]{@{}l@{}}Které otázky problematického vztahu právní úpravy svěřenských\\ fondů a dědického práva si podle autora žádají nejnaléhavěji\\ řešení přijetím novely?\\ Jak by taková novela mohla, podle autora, znít?\end{tabular} \\ \hline
Nejzásadnější změna by měla nastat ve smyslu kolize dědického\\ práva s právem svěřenských fondů.\\
---\\
Další problémy vytyčené v práci (započtení, daňová problematika)\\ jsou oproti této problematice absolutně marginálního charakteru.\\
---\\
Mělo by dojít primárně k řešení problematiky nepominutelných\\ dědiců.                                                                                                                                                                                                                             \\ \hline
\end{tabular}
\end{table}

\end{frame}

% +============+
% | Sedmý slide |
% +============+

\begin{frame}
	\frametitle{Zodpovězení otázek}
	\cevroLogo

\begin{columns}
	\column{0.5\textwidth}
		Hello everybody
	\column{0.5\textwidth}
		Hello everybody
\end{columns}

\end{frame}

% +============+
% | Osmý slide |
% +============+

\begin{frame}
	\frametitle{Zodpovězení otázek}
	\cevroLogo

\begin{table}[]
\begin{tabular}{|l|}
\hline
Otázka 2 Vedoucí práce                                                                                                                                                                                                                       \\ \hline
\begin{tabular}[c]{@{}l@{}}Lze změnit statut svěřenského fondu zřízeného mortis causa,\\ příp. za jakých podmínek? Kdo ho může změnit?\end{tabular} \\ \hline
Zřízení jako založení vs. vznik.\\
---\\
Zřízení (§ 1451) pro případ smrti - založen, zakladatel je živ.\\
---\\
Svěřenský fond neexistuje a proto je možné jej měnit stejně,\\ jako je možné měnit pořízení pro případ smrti.\\
---\\
Dědická dohoda - pouze po dohodě stran - závazek, například\\ novace.                                                                                                                                                                                                                              \\ \hline
\end{tabular}
\end{table}

\end{frame}

% +==============+
% | Devátý slide |
% +==============+

\begin{frame}
	\frametitle{Zodpovězení otázek}
	\cevroLogo

\begin{table}[]
\begin{tabular}{|l|}
\hline
Otázka 1 Oponent práce                                                                                                                                                                                                                       \\ \hline
\begin{tabular}[c]{@{}l@{}}Nechť autor v rámci ústní obhajoby shrne základní podobu\\ konceptu majetkové struktury svěřenského fondu.\end{tabular} \\ \hline
Hello everybody\\
Hello everybody\\
Hello everybody\\
Hello everybody\\
Hello everybody\\
Hello everybody\\
Hello everybody\\
Hello everybody\\
Hello everybody\\
Hello everybody                                                                                                                                                                                                                              \\ \hline
\end{tabular}
\end{table}

\end{frame}

% +==============+
% | Desátý slide |
% +==============+

\begin{frame}
	\frametitle{Zodpovězení otázek}
	\cevroLogo

\begin{table}[]
\begin{tabular}{|l|}
\hline
Otázka 2 Oponent práce                                                                                                                                                                                                                      \\ \hline
\begin{tabular}[c]{@{}l@{}}Autor by mohl dále shrnout některé dílčí odlišnosti québecké\\ a české úpravy svěřenských fondů a tím učinit závěr v dané otázce\\ na str. 20 až 23.\end{tabular} \\ \hline
Hello everybody\\
Hello everybody\\
Hello everybody\\
Hello everybody\\
Hello everybody\\
Hello everybody\\
Hello everybody\\
Hello everybody\\
Hello everybody\\
Hello everybody                                                                                                                                                                                                                              \\ \hline
\end{tabular}
\end{table}

\end{frame}

% +=================+
% | Jedenáctý slide |
% +=================+

\begin{frame}

\begin{center}
\Large{Děkuji za pozornost	\smiley{}}
\end{center}

\end{frame}
	
\end{document}
