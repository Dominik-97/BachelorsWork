\documentclass{article}
\usepackage[utf8]{inputenc}
\usepackage{graphicx}
\graphicspath{ {./Obrazky/} }
\usepackage{fancyhdr}
\usepackage{xcolor}
\usepackage{titling}
\usepackage{array}
\usepackage{todonotes}
\setlength\headheight{26pt}
\lhead{\includegraphics[width=3cm,height=\dimexpr \headheight-\dp\strutbox]{newcevro}}
\rhead{\small{\leftmark}}

\usepackage{biblatex}
\addbibresource{Zdroje.bib}

\renewcommand\contentsname{Obsah}
\renewcommand\listfigurename{Seznam obrázků}

\fancypagestyle{smallertextinheader}{ % Dokončit styl s menším textem v hlavičce
   \fancyhf{}
   \fancyhead[LE,LO]{\includegraphics[width=3cm, height=\dimexpr \headheight-\dp\strutbox]{newcevro}}
   \fancyhead[RE,RO]{%
   \parbox[b]{\dimexpr \textwidth-3cm-\columnsep}%
   {\small\uppercase\leftmark}}%
}

\fancypagestyle{Contents}{ % Dokončit styl s menším textem v hlavičce
   \fancyhf{}
   \fancyhead[LE,LO]{\includegraphics[width=3cm, height=\dimexpr \headheight-\dp\strutbox]{newcevro}}
   \fancyhead[RE,RO]{\small{\uppercase{\rightmark}}}
}

\pagestyle{fancy}

\pretitle{
	\begin{center}
	\LARGE
	\includegraphics[width=10cm,height=3cm,keepaspectratio]{newcevro}
}
\posttitle{\end{center}}

\begin{document}

\title{Svěřenský fond jako nástroj transferu majetku mezi generacemi}
\author{Dominik Bálint}
\date{01.12.2019}

\pagenumbering{gobble}
  \thispagestyle{empty}
  \begin{center}
  \includegraphics[width=10cm,height=3cm,keepaspectratio]{newcevro} \\
  \end{center}
  \vspace{15mm}
  \begin{center}
  {\Large Vysoká škola Cevro Institut} \\
  \vspace{15mm}
  {\Large \textbf{Svěřenský fond jako mezigenerační nástroj převodu majetku}} \\
  \vspace{15mm}
  {\Large Dominik Bálint} \\
  \vspace{15mm}
  {\Large Bakalářská práce} \\
  \vspace{49mm}
  {\Large \textbf{Praha 2019}} \\
  \end{center}
  
\newpage
  \thispagestyle{empty}
  \maketitle
  \begin{center}
  {\Large Katedra veřejného práva} \\	
  \vspace{15mm}
  {\Large \textbf{Studijní program:} Veřejné právo} \\
  {\Large \textbf{Studijní obor:} Právo v obchodních vztazích} \\
  {\Large \textbf{Jméno vedoucího diplomové práce:} } \\
  {\Large Mgr. Ing.  Střeleček  Tomáš LL.M.} \\
  \end{center}
  
\newpage
  \thispagestyle{empty}
  \vspace*{\fill}

\noindent \textbf{Čestné prohlášení} \\

Prohlašuji, že jsem předkládanou práci zpracoval samostatně, uvedl
v ní všechny použité prameny a zdroje, které jsou uvedeny v seznamu použité
literatury, a v textu řádně vyznačil jejich použití. \\

\noindent V Praze dne \today \\
\vspace{10mm} \\
\begin{tabular}{p{6cm}c}
& ................................................. \\
& Dominik Bálint
\end{tabular}

\newpage

\thispagestyle{empty}

\vspace*{\fill}
\noindent \textbf{Poděkování} \\

	Na tomto místě bych rád poděkoval svému vedoucímu práce panu Mgr. Ing. 
Střeleček  Tomáš LL.M. za podporu, trpělivost, spolupráci, podněty ke zlepšení a 
také za čas, který mi věnoval při vedení práce.

\vspace*{\fill}
 
\newpage
  \thispagestyle{Contents}
  \tableofcontents
  \listoffigures
  
\newpage
  \pagenumbering{arabic}
  
\section{Úvod}

Práce si klade za cíl v historickém a právním kontextu představit institut svěřenského fondu a převzetí jeho právní úpravy z právního řádu provincie Quebec.
\linebreak

\indent Dále se práce bude zabývat představením momentální právní úpravy svěřens\-kého fondu, jeho možnostmi, ukázanými na příkladu praktického využití – jako nástroj, který může substituovat klasické dědění, i potenciálními možnostmi zneužití a jak (a zda) jsou odstraněny obavy o zneužití svěřenských fondů v souvislosti s novelizací občanského zákoníku v roce 2018. 
\linebreak

\indent Dalším bodem práce bude faktické porovnání právní úpravy dědického práva s právní úpravou svěřenského fondu. Těžištěm této časti je snaha potvrdit, nebo vyvrátit tvrzení, že právní úprava dědického práva a svěřenského fondu je vyvážená a svěřenským fondem se nedají obcházet práva dědiců.
\linebreak

\indent V tomto rámci se budu zabývat zejména zhodnocením právní úpravy svěřens\-kých fondů a její analýzou v souvislosti s ochranou dědiců, vyložení potenciálních problémů momentální právní úpravy v rámci českého právního řádu, tzn. potenciální zneužití, v souvislosti s právem nepominutelných dědiců, které můžou vzniknout v případě souběhu dědického řízení a založení svěřenského fondu.
\linebreak

\indent V závěru práce budou vyloženy systémové nedostatky v případě, že je naleznu a následovat bude návrh řešení nalezených problémů.

\newpage

\section{Správa cizího majetku - historie}

\subsection{Úvod do historických počátků svěřenství a svěřenských fondů}

\indent Dne 22.3.2012 vstoupil v platnost nový občanský zákoník, který nabyl účinnosti prvním dnem roku 2014. Nový občanský zákoník s číslem 89/2012 Sb. rekodifikoval civilní právo a zavedl v Českém právu několik nových institutů. Rekodifikace civilního práva se dotka i právní úpravy správy cizího majetku, v jejichž mezích bylo zavedeno několik nových právních institutů. Mezi tyto instituty můžeme zařadit také institut svěřenství respektive svěřenského nástupnictví, který předchozí občanský zákoník, až na jeden paragraf\footnote{V §859 je upraven zánik všech omezení vyplívajících ze svěřenského nástupnictví}, neupravoval a také svěřenský fond, na který se práce bude především zaměřovat. \\

\indent Nově zavedená podoba institutu svěřenského fondu a obecné správy cizího majetku je pro nás sice nová, nicméně jedná se o instrument, který v českém právu v minulosti v určitých formách existoval až do 1.1.1951 do zrušení svěřens\-kého nástupnictví, zvané též jako fideikomisární substituce\footfullcite[§565]{noauthor_zakon_nodate}. Omezení vyplívající z institutu svěřenského nástupnictví nicméně trvala až do roku 1964, kdy vstoupil v platnost a zanedlouho i v účinnost nový federální občanský zákoník pod číslem 40/1960 Sb., který úplně zrušil svěřenské nástupnictví\footnote{Viz citace 4}. Mezi roky 1964 a 2014 lze tedy hovořit o faktickém vakuu právní úpravy svěřenských institutů v Československu a potažmo v České republice, respektive zákazu používání těchto institutů. Následující kapitoly ve zkratce představí institut svěřenství od, respektive z, dob Římské říše, následovat bude popis vývoje svěřenství během období monarchie a po vzniku samostatného Československého státu až po zrušení svěřenství v roce 1964. V poslední části bude popsáno znovuzavedení institutu svěřenství a svěřenského fondu novým občanským zákoníkem v roce 2014 s důrazem na převzetí právní úpravy svěřenského fondu z právního řádu provincie Quebec, kterému bude, pro lepší pochopení České právní úpravy, předcházet popis a historie trustu, s důrazem na momentální úpravu v právním řádu provincie Quebec tato kapitola bude vycházet především z Komentovaného znění části nového občanského zákoníku upravující problematiku správy cizího majetku od Jaroslava Svejkovského a kolektivu. V návaznosti na znovuzavedení institutu svěřenství bude následovat základní popis novelizace právní úpravy svěřenských fondů z roku 2018, který bude dále podrobněji objasněn v druhé části práce. \\

\newpage

\subsection{Pojem svěřenství}

\indent Aby bylo možné dále hovořit o institutu svěřenství, respektive svěřenského fondu, je třeba si nejprve vymezit co institut svěřenství znamená, představuje a k jakému účelu slouží. Barbora Bednaříková institut svěřenstí definuje jako: "...takový vztah, kdy v principu jedna osoba svěří svůj určitý majetek druhé osobě ve prospěch osoby třetí."\footfullcite[Kapitola Úvod, str. IX]{bednarikova_barbora_sverenske_2014} Účel takovéhoto svěření, jak je výše popsáno, je tedy zajištění určitého prospěchu pro třetí osobu/y. V historickém kontextu tento prospěch spočíval především v poskytování užitků z vyčleněného majetku a/nebo jeho převedení ve prospěch třetí osoby, nicméně v rámci momentální právní úpravy lze hovořit o více účelech svěřenství. \\

\indent V kontinentálním právu se instituty, které by se svým obsahem shodovaly s výše popsaným vymezením, začaly používat již v antickém Římě ve formě takzvaného fideikomisu (latinsky \textit{fideicommissum}, pochází ze slov \textit{fīdē} v překladu "důvěra" a \textit{commissum} v překladu "svěřené", dohromady též jako "tvé důvěře svěřuji"\footfullcite{noauthor_fideicommissum_nodate}) a také v právu anglosaském ve formě trustů (anglicky \textit{trust}, v překladu "důvěra"). Ve většině zemí západního světa, používajících obou právních systémů je tato forma správy cizího majetku dlouhodobě používána k uchování, převodu, správě, zachování celistvosti, nebo použití k dalším soukromým, nebo veřejným účelům\footfullcite[Kapitola Úvod, str. IX]{bednarikova_barbora_sverenske_2014}, v různých formách. Některé země tradičně využívají institutu trustu, například Irsko, jiné země právní úpravu trustu známou především ze zemí Common Law\footnote{Země používající anglosaský systém práva} přizpůsobili kontinentálnímu právnímu myšlení, mezi tyto země lze zařadit Lichtenštejnsko (Treuhandverhältnis) a Lucembursko (Fiducie), další země se rozhodly funkci trustu nahradit jiným institutem - Německo (Treuhand) a Rakousko (Privatstftungsgesetz) \footfullcite{trust_2014}. \\

\indent V momentální právní úpravě, tedy v občanském zákoníku, jsou v paragrafech 1448 a 1449 fakticky vymezeny účely svěřenského fondu, těmito účely je účel soukromý a účel veřejný\footfullcite{noauthor_zakon_2012}. Soukromý svěřenský fond slouží ku prospěchu určité osoby, nebo na její památku, z tohoto ustanovení je možné dovodit, že svěřenský fond založený za soukromým účelem slouží k ochraně, uchování, rozmnožení, správě a převodu vyčleněného majetku. Co se veřejného účelu týče, hlavním účelem nemůže být dosahování zisku nebo provozování závodu, svěřenský fond založený za veřejným účelem tedy slouží k veřejnému, především socioekonomickému, prospěchu.\\

\newpage

\subsection{Antický řím}

Počátky svěřenských institutů je možné pozorovat již za dob Antického Říma, k pochopení co vedlo ke vzniku těchto institutů bude následující kapitola obsahovat popis systému práva a dědického práva Antického Říma, následovat bude představení svěřenských institutů vzniklých v této době a následně bude popsán jejich vývoj s ohledem na změny právního systému Říma.

\subsubsection{Praetorské právo a Ius Civile}

K pochopení dvoukolejnosti římského dědického práva je v první řadě třeba vymezit rozdíly mezi \textit{ius civile} a \textit{ius honorarium}, které se v rámci dědického práva lišily způsoby stanovování dědické posloupnosti a přístupu k dědickému řízení. Pro celkové pochopení struktury Římského práva soukromého je vhodné zmínit a popsat i další části, kterými jsou \textit{ius gentium} a \textit{ius naturale}. \\

Soukromé právo se sestávalo ze 4 částí, pro jejich vysvětlení je vhodné zkombinovat práci dvou významných Římských právníků Ulpiána a Gaiuse. Můžeme přitom vycházet z \textit{Corpus Juris Civilis}, jakožto souhrného občanského zákonníku vydaného ve Východořímské říši za vlády císaře Justiniána a specificky ze sbírky děl římských právnků s názvem \textit{Digesta} vydaného roku 530 našeho letopočtu. \\

Ulpián píše následující: \textit{Ius naturale est, quod natura omnia animalia docuit: nam ius istud non humani generis proprium, sed omnium animalium, quae in terra, quae in mari nascuntur, avium quoque commune est}, což lze volně přeložit jako: "Přírodní řád je řádem, kterůmu příroda učí všem živým tvorům, tento řád nejen že není člověku cizí, ale ovlivňuje všechny tvory ať již pochází ze země, moře, či jsou ptáky."\footnote{Dig. 1.1.1.3, Ulpianus 1 inst.} \\

Ulpián dále říká: \textit{privatum ius tripertitum est: collectum etenim est ex naturalibus praeceptis aut gentium aut civilibus}, což lze volně přeložit jako: "Soukromé právo skládá se ze tří částí, neboť odvozeno jest buď z příkazů přírodních, příkazů národů, nebo těch vycházejíc z práva civilního."\footnote{Dig. 1.1.1.2, Ulpianus 1 inst.} \\

Gaius v knize první (Dig. 1.1.9., Gaius 1 inst.) soukromé právo rozděluje obdobě na právo civilní a právo národů: \textit{Omnes populi, qui legibus et moribus reguntur, partim suo proprio, partim communi omnium hominum.}, což lze volně přeložit jako: "Všechny národy, které se řídí zvyky a právem částečně používají práva svého a práva národů", tedy \textit{ius civile} a \textit{ius gentium}. \\

\newpage

Poslední součást Římského soukromého práva tvoří tvz. \textit{ius honorarium}, tedy právo tvořené Praetory. \\

Dle informací výše uvedených lze Římské soukromé právo rozdělit následovně:

\vspace{5 mm}

\begin{itemize}
\item \textit{ius civile},
\item \textit{ius honorarium},
\item \textit{ius gentium},
\item \textit{ius naturale}.
\end{itemize}

\vspace{5 mm}

Struktura římského soukromého práva, s ohledem na popis v této kapitole, by se dala vizuálně vyzobrazit takto, pro popis fungování dědického práva je důležité především \textit{ius civile} a \textit{ius honorarium}:

\begin{figure}[h]
\centering
\includegraphics[width=15cm,height=5cm,keepaspectratio]{rimskepravostruktura.jpeg}
\caption{Struktura soukromého římského práva}
\label{fig:struktura}
\end{figure}

\newpage

\underline{\textbf{\textit{Ius civile}}} je právo, jehož subjekty jsou Římští občané, původně vycházelo z obyčejů. Nejstarší \textit{ius scriptum}\footnote{psané právo}, vykládající ius civile, který je nám znám je \textbf{Zákon \MakeUppercase{\romannumeral 12} desek}. Je zajímavé zmínit, že dle několika dochovaných informací potenciálně existovala psaná úprava již na samém počátku republiky ve formě sbírky \textit{leges regiae}\footnote{zákony vydávané Římskými krály}, která byla kodifikována do takzvaného \textit{ius civile Papirianum}\footfullcite[str.31]{mousourakis_roman_2014}. \textit{Ius civile bylo} tvořeno pontifikálními interpretacemi\footnote{zákony tvořeny Pontifiky, kteří vykládali právo} a komiciálními zákony\footnote{zákony tvořeny lidovým shromážděním} a v době císařství císařskými konstitucemi\footnote{zákony tvořeny císařem}. Problém \textit{ius civile} však spočíval v jeho nepružnosti, která se projevovala špatným přizpůsobováním změnám ve společnosti a vývoji doby, a jednoduchého a striktního určení některých zásad\footfullcite[kapitola 1, str.3]{bednarikova_barbora_sverenske_2014}. \\

\underline{\textbf{\textit{Ius honorario}}}, neboli praetorské právo, bylo tvořeno vysokým Římským úředníkem zvaným \textit{Praetor}. Úřad praetora byl vytvořen roku 367 před naším letopočtem Liciniovým zákonem\footfullcite[str.35]{blaho_peter_haramia_ivan_a_zidlicka_michaela_zaklady_1997}. Praetor řídil soudní procesy a za pomoci \textit{aequitas}\footnote{spravedlnost a slušnost} a \textit{bona fide}\footnote{dobrá víra, dobrý úmysl} mohl ovlivňovat a řídit soudní řízení a tím i vytvářet, respektive nalézat, \textit{ius honorarium}, jež rozvíjelo neobratné a zastaralé principy civilního práva, ze kterého vycházel a jehož principy vztahoval na skutkové vztahy, které původní \textit{ius civile} neupravovalo. V situacích, kdy se pro svůj nedostatek formy a obsahu z \textit{ius civile} vycházet nedalo, mohl praetor vytvořit hypotézu skutkové podstaty, která když je naplněna, má být něco vykonáno, například má být žalovaný odsouzen. Praetor zároveň v řízení účastníkům umožnil podávat námitku (\textit{exceptio}), na kterou dle civilního práva nebyl brán zřetel, nicméně pokud se v rámci \textit{ius honorario} daná námitka prokázala, brala se v úvahu. \\

Praetor zároveň vydával na začátku svého jednoročního funkčního období takzvaný edikt, který fakticky sloužil jako soupis honorárního práva. 

Doplnit další informace. \\

\newpage

\underline{\textbf{\textit{Ius gentium}}} bylo. \\

\underline{\textbf{\textit{Ius naturale}}} bylo. \\

\newpage

\subsubsection{Dědické právo v dobách antického Říma}

Po vyjasnění struktury Římského práva je tedy možné přesunout se přímo k popisu způsobu fungování Římského dědického práva. Tato část je důležitá pro historické pochopení vzniku svěřenských institutů ve starověkém Říme a pokračování dalšího vývoje těchto institutů, které mají svůj prvopočátek právě v Římské říši. \\

Dědické právo v Říme, vycházelo z principu univerzální sukcese\footnote{Do dědictví spadá jmění, tedy jak majetek, tak i závazky}\textsuperscript{,} \footfullcite{blaho_peter_haramia_ivan_a_zidlicka_michaela_zaklady_1997}, což sehrálo spolu s pasivní dědickou legitimací významnou roli při tvorbě alternativních způsobů odkazu majetku zůstavitele. \\

Po zůstavitelově smrti šlo dědit na základě zákona, nebo na základě testamentu, toto zároveň určovalo i dědickou posloupnost. \\

\vspace{5 mm}

Římské právo tedy rozlišovalo dvě posloupnosti:
\begin{enumerate}
\item \textit{hereditas testamentaria} - dědická posloupnost na základě testamentu,
\item \textit{hereditas legitima} - dědická posloupnost na základě zákonu.
\end{enumerate}

\vspace{5 mm}

Je třeba poznamenat, že posloupnost zakládající se na testamentu, tedy jednostranného právního dokumentu ustanovujícího zůstavitelovi dědice, měla přednost před posloupností zákonnou. Nevyčerpání celého dědictví ze strany hereditas testamentaria, tedy nemohlo založit právní důvod pro delaci\footnote{povolání} intestátních dědiců\footnote{dědicové dle hereditas legitima}, v takovém případě dědil i zbytek pozůstalosti dědic povolaný na základě testamentu. \\

Přes fakt, že dědění na základě testamentu mělo přednost, znalo Římské právo institut nepominutelného dědice, které se ve svých prvopočátcích vymáhalo takzvanou kverelou, tedy žalobou před soudem na zrušení testamentu. Až v pozdější době císařské měli nepominutelní dědici při jejich opomenutí zůstavitelem v závěti právo na dorovnání svého zákonného podílu bez nutnosti rušit testament\footfullcite[3. vydání z roku 2012, str.230]{marek_karel_redakcni_rada_casopis_2012}. Povinnost zohlednit nepominutelného dědice se mohl zůstavitel zprostit vyděděním daného dědice. Z počátku nemusel zůstavitel uvádět důvod, toto bylo změněno vydáním Justiniánských reforem, které taxativně vymezili důvod pro vydědění nepominutelného dědice \footfullcite[str.147]{blaho_peter_haramia_ivan_a_zidlicka_michaela_zaklady_1997} . \\

\newpage

Římské právo dále rozlišovalo právní postavení dědiců ve smylu povinnosti dědit na:

\vspace{5 mm}

Římské právo tedy rozlišovalo dvě posloupnosti \footfullcite[str.141]{blaho_peter_haramia_ivan_a_zidlicka_michaela_zaklady_1997}:
\begin{enumerate}
\item \textit{heredes voluntarii} - dobrovolní dědicové, jsou povoláni k dědictví projevem vůle nebo zachováním se ve smyslu přijetí - tedy například uhrazení pohledávky zůstavitele,
\item \textit{heredes necessarii} - nutní dědicové, jsou povoláni k dědictví smrtí zůstavitele bez možnosti odmítnutí dědictví, tato skupina se dále dělí na další dvě podskupiny:
\begin{itemize}
\item \textit{heredes sui et necessarii} - dědici nutní a vlastní, římská rodina byla silně patriarchálního charakteru a vztahy se rozlišovaly na takzvané agnátské a kognátské. Paterfamilias, tedy otec rodiny měl moc na členy jeho rodiny. Tito členové, kteří v době zůstavitelovi smrti spadali pod jeho moc se stali členy této podskupiny. Do vzniku \textit{beneficium abstinendi} v rámci praetorského práva něměli možnost dědictví odmítnout.
\item \textit{heredes necessarii} - jedná se o otroky, kteří byli v testamentu zároveň osvobozeni, neměli možnost dědictví odmítnout.
\end{itemize}
\end{enumerate}

\vspace{5 mm}

Pokud byl v testamentu jako dědic označen otrok v moci jiného římského občana, stal se dědicem onen římský občan. \todo{Dohledat zdroj a najít vhodné místo k doplnění v textu} \\

V rámci praetorského práva vyhradil dědicům praetor takzvané beneficium abstinenti, které jim umožňovalo vzdát se dědictví a přesto, že stále byli dědici dle ius civile, zabraňovalo také možnosti podávání žaloby zůstavitelových věřitelů proti těmto dědicům.

\newpage

V rámci Římského práva a zvyklostí bylo možně přenechat určitou část majetku někomu, kdo nebyl dědicem formou takzvaného legata (v překladu odkazu). Tento způsob byl nicméně zatížen formálními požadavky na jeho formu, z tohoto důvodu se začaly vyčleňovat i jiné postupy převodu majetku, které nebyly zatíženy takovou formálností\footfullcite{noauthor_fideicommissum_nodate}. Jeden z těchto postupů byl již výše zmíněný fideikomis, jehož splnění bylo ve svých prvopočátcích svěřeno dědici čistě na základě důvěry. Faktem však nadále zůstávalo, že fideikomis nebyl právně vymahatelný a tím pádem nebylo možné splnění zůstavitelova přání požadovat formou žaloby.

\subsubsection{Fideikomis v Antickém Římě}

Jak již bylo výše poznámenáno, v důsledku faktických omezení kladených na \textit{legatum} se začal vyvíjet i jiný institut k účelu převodu majetku po smrti zůstavitele. Tímto institutem byl takzvaný \textit{fideikomis}. Podstata \textit{fideikomisu} spočívala v tom, že zůstavitel žádal jím určenou osobu (především univerzálního dědice), aby po jeho smrti určitou naplnil určité zůstavitelovo přání, toto přání spočívalo nejčastěji v převodu části zůstavitelova majetku na jinou osobu, než na určené dědice. Jak již je poznamenáno v úvodu, slovo \textit{fideicomissum} se dá volně přeložit jako "tvé důvěře svěřuji", toto označení bylo výstižné, protože tato žádast se opravdu zakládala čistě na důvěře mezi zůstavitelem a osobou jím určenou. Fideikomisář (rozuměj osoba v jejíž prospěch má být plněno) se tedy nemohl žalobou domáhat svého práva. Tato žádost se nejčastěji připojovala ke kodicilu. \\

Kodicil tvořil v této době společně s legatem a fideikomisem 

\newpage

\subsection{Historie svěřenských fondů u nás a ve světě}

Co se historie svěřenský fondů týče, tak \footnote[4]{Zde bude nějaká informace}
Doplnit úpravu za dob království/monarchie a první republiky a vývoj svěřenských institutů jinde ve světě, držet se struktury popsané v druhém paragrafu v kapitole druhé, to znamená popsat i vznik trustu.
Dále popsat zrušení svěřenských institutů v období první republiky a úplné zrušení po druhé světové válce.

\subsection{Recepce právní úpravy Svěřenského fondu z právního řáduprovincie Quebec}

 Následně popsat znovuzavedení těchto institutů v roce 2014, převzetí z Quebecké právní úpravy a popsat i novelu z roku 2018 - na základě stížnosti ministerstev a vrchního státního zastupitelství, odkážu poté na další kapitolu věnující se této novelizaci.


\newpage
\thispagestyle{smallertextinheader}

\section{Momentální právní úprava svěřenských fondů}

Popsat jak je v momentální právní úpravě vyložen institut svěřenského fondu a jaké možnosti poskytuje.

\subsection{Potenciální možnosti použití v rámci převodu majetku}

Ukázat příklady, ke kterým se hodí svěřenský fond, v souvislosti s tématem práce by zde bylo vhodné popsat institut svěřenského fondu jako nástroj k mezigeneračnímu převodu majetku.

\subsection{Novelizace právní úpravy - zápis svěřenských fondů do rejstříku}

Rozepsaná novelizace svěřenský fondů z roku 2018, zhodnotit, zda přispělo ke zmírnění možností využití svěřenských fondů jako prostředků k legalizaci výnosu z trestných činností.

\newpage

\thispagestyle{smallertextinheader}

\section{Komparace právní úpravy svěřenského fondu s právní úpravou dědictví}

Zde proběhne komparace, musím zde napsat, že neexistuje právní úpravy, která by upravovala situaci, ve které bude založen svěřenský fond mortis causa a zároveň bude probíhat dědické řízení, ve kterém budou figurovat nepominutelní dědici, na které v souvislosti se založením svěřenského fondu nevyjde jejich zákonný podíl. Konstatovat, že těmto problémům se dá vyhnout tím, že se svěřenský fond vyčlení už za života, dokázat toto tvrzení na základě jiných institutů převodu majetku za zůstavitelova života a poukázat na to, že tento majetek následně není předmětem dědického řízení a tudíž ani nemůže připadnout jakýmkoliv dědicům, neboť pokud budeme mít na časové ose bod x, ve kterém zůstavitel zemře, a svěřenský fond vznikne v bodě x-y, tak v době zůstavitelova úmrtí již není součástí jmění a tudíž nemůže být součástí dědického řízení. Pokud by nicméně zůstavitel zemřel v bodě x a svěřenský fond by také vznikl v bodě x, tak lze polemizovat o tom, zda by dědici nemohli mít právo na snahu obsáhnout i tento vyčleněný majetek v soupisu dědického řízení.\\

Dále se pokusit obhájit můj názor, že pokud budou obmyšlenými svěřenského fondu i nepominutelní dědicové, tak plněním ze svěřenského fondu právě tímto dědicům dojde k neformálnímu doplnění jejich povinného podílu ze strany zůstavitele - pokusit se najít nějakou právní zásadu, která by něco takového říkala. Pokud by však z fondu mělo být plněno nikomu jinému, tak se pokusit obhájit fakt, že tato situace není zákonně obhájena a není na ní ani žádná judikatura či ustálená soudní praxe, a potenciálně by se tedy dědicové mohli bránit soudní cestou. Pokusit se stanovit, jak by takovýto spor mohl v rámci našich existujících právních norem dopadnout, pokud to půjde tak se zkusit podívat, zda toto neni řešeno na mezinárodní úrovni například v rámci mezinárodního práva soukromého.

\newpage

\section{Systémové nedostatky a návrhy na řešení}

Zhodnocení faktu, že by taková právní úprava měla existovat, poznámka, že tento problém zatím u nás není řešen soudní praxí. Pokusit se navrhnout legislativní změny, vedoucí ke zlepšení postavení nepominutelných dědiců v rámci dědického řízení za předpokladu, že zároveň vznikne i svěřenský fond mortis causa, do kterého se vyčlení majetek, který by jinak byl součástí dědického řízení. Zkusit navrhnout řešení na základě inspirace jinými právními systémy, ve kterých existuje jak institut trustu, tak i institut nepominutelného dědice, mohlo by se jednat například o provincii Quebec, nebo stát Louisiana, pokusit se tyto změny navrhnout na základě studia zákonů a dále se pokusit najít soudní rozhodnutí, v Angličtině se institut nepominutelného dědice nazývá forced heir. \\

Jak již je zmíněno v kapitole výše, v rámci české právní úpravy bohužel chybí explicitní úprava funkce svěřenského fondu pokud se nachází ve střetu s institutem nepominutelného dědice. Podobné stanovisko zaujímá například i Kryštof Horn ve svém příspěvku do Časopisu Ad Notam. Ve svém příspěvku říká následující: "Příkladem takové nejasnosti může být vztah fondu k institutu nepominutelného dědice, který je neznámý jak v zemích common law, kde má své kořeny trust, tak bohužel i v Québecu, odkud zákonodárce přejal úpravu svěřenských fondů. Výslovná ustanovení o vztahu svěřenského fondu mortis causa a dědického práva totiž chybí." \footfullcite{sro_httpwwwpraguebestcz_prakticke_nodate-1} Tento příspěvek se zakládá na pravdě a potvrzuje fakt, že takováto právní úprava v českém právu chybí a bylo by vhodné jí prostřednictvím novelizace doplnit. S výrokem nicméně nemohu souhlasit jako s celkem, protože přesto, že je prava, že institut nepominutelného dědice jako takový v Quebeckém právu neexistuje a Quebecká právní úprava se dále vyznačuje téměr úplnou testamentální volností, tak Quebec Civil Code připouští určitou ochranu dědiců před opomenutím v závěti. Abychom byli přesní, tak Quebecký právní řád upravuje tři případy ochrany dědiců, které by se daly přirovnat k institutu nepominutelného dědice v našem právním řádu. Je tedy možné říci, že, přestože tak zákonodárce neučinil, mohl se u Quebecké právní úpravy inspirovat i v rámci ochrany nepominutelných dědiců, protože Quebecké právní úprava, narozdíl od té České, upravuje i případy, ve kterých by tyto nároky nemohly být splněny kvůli právnímu jednání učiněnému zůstavitelem, mezi toto právní jednání je právě možné zařadit i založení trustu.\footfullcite{leclercq_inheritance_2014} \\

\newpage

\subsection{Typy ochrany dědických nároků v rámci Quebec Civil Code}

Jak již jsem vyložil výše, v Quebeckém právní řádu existují tři instituty, které by se svým účelem a formou daly přirovnat k českému institutu nepominutelného dědice. Ve své pdostatě se jedná o instituty, které mají chránit osobu, které je zůstavitel povinnen poskytnou nějaké plnění před opomenutím této osoby ze strany zůstavitele v rámci závěti, nebo opomenutí této osoby pokud se dědí na základě intestátní posloupnosti. Zákonodárce zároveň upravil vztah těchto institutů k právnímu jednání zůstavitele, které mění výši zůstavitelova majetku již v obodobí před jeho smrtí a s účinností k datu jeho smrti. \\

\vspace{5 mm}

Tři ochrany, které jsou upravené v Quebeckém právním řádu:

\begin{enumerate}
\item Survival of the Obligation to Provide Support
\item Family Patrimony
\item Compensatory Allowance	
\end{enumerate}

\vspace{5 mm}

\underline{\textbf{Survival of The Obligation to Provide Support}}

Prvním způsobem ochrany jistých dědických nároků vůči majetku zachované\-mu zůstavitelem spočívá v Institutu zvaném \textit{The Survival of the Obligation to Provide Support}, který zaručuje příjemci určité finanční podpory vyplácené danému příjemci za zůstavitelova života tuto podporu i po smrti zůstavitele. \\

Důležitou roli pro dokázání argumentu, že Quebecká právní úprava obsahuje v jisté formě ochranu dědiců a upravuje i případy zmenšení zůstavitelova majetku v období před smrtí popřípadě učinné ke dni smrti zůstavitele je odstavec 689, který stanovuje následující: "Where the assets of the succession are insufficient to make full payment of the contributions due to the spouse or to a descendant, as a result of liberalities made by acts inter vivos during the three years preceding the death or having the death as a term, the court may order the liberalities reduced. \\

Liberalities to which the spouse or descendant consented may not be reduced, however, and those he has received shall be imputed to his claim."\footfullcite{noauthor__nodate} Tento odstavec ve zkratce zaručuje osobám s nárokem vůči zůstavitelovi ochranu tohoto nároku v případě nedostatečného jmění v rámci dědictví, které bylo jistým právním jednáním zůstavitele sníženo na tuto nedostatečnou hranici. Soud může v tomto případě omezit právní jednání učiněné zůstavitelem dokonce i za jeho života, co je ale důležité pro založení trustu mortis cause je fakt, že může omezit i právní jednání s účinností ke dni zůstavitelovi smrti. Tímto způsobem tedy může omezit i vyčlenění majetku do svěřenského fondu, respektive Trustu. \\

Dále je potřebné a důležité poznamenat i to, že odstavec 689 ve své druhé části říká, že v případě, že z těchto právních jednání bude mít prospěch právě osoba, která se domáhá svého podílu na zůstavitelově majetku, tak tento prospěch bude započten v rámci potenciálního omezení právního jednání zůstavitele. Z této části je tedy možné dovodit, že uspokojit tento právní nárok je možné i založením trustu, který bude daný prospěch oprávněným osobám poskytovat místo likvidátora dědictví\footnote{V Quebeckém právu je likvidátor osobou, která}.

\underline{\textbf{Family Patrinomy}}

\underline{\textbf{Compensatory Allowance}}

\subsection{Ochrana zájmů nepominutelných dědiců zakladatele trustu v Itálii}

Napsat o situaci v Itálii.\footfullcite{lubos_tichy_sverensky_2017}

\subsection{Návrhy na změny České právní úpravy}

\newpage

\section{Závěr}

Shrnutí toho, co práce obsahovala. Následovat bude shrnutí praktické části práce, tedy že v tato úprava chybí a zároveň není ustálená rozhodovací praxe soudů. Zhodnotit návrhy řešení a nastínit, zda by mohli pomoci, nebo ne.

\newpage
\thispagestyle{Contents}
\section*{Zdroje}\markright{ZDROJE}
\newcounter{SecZdroje}
\setcounter{SecZdroje}{\thesection}
\addtocounter{SecZdroje}{1}
\addcontentsline{toc}{section}{\theSecZdroje \hspace{1,7 mm} Zdroje}
\printbibliography[type=misc,heading=subbibliography,title={Online zdroje}]
\printbibliography[type=book,heading=subbibliography,title={Další zdroje}]
	
\end{document}