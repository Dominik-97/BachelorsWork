\documentclass{article}
\usepackage[utf8]{inputenc}
\usepackage{graphicx}
\graphicspath{ {./Obrazky/} }
\usepackage{fancyhdr}
\usepackage{xcolor}
\usepackage{titling}
\usepackage{array}
\usepackage{todonotes}
\setlength\headheight{26pt}
\lhead{\includegraphics[width=3cm,height=\dimexpr \headheight-\dp\strutbox]{newcevro}}

\usepackage{biblatex}
\addbibresource{Zdroje.bib}

\fancypagestyle{smallertextinheader}{ % Dokončit styl s menším textem v hlavičce
   \fancyhf{}
   \fancyhead[LE,LO]{\includegraphics[width=3cm, height=\dimexpr \headheight-\dp\strutbox]{newcevro}}
   \fancyhead[RE,RO]{%
   \parbox[b]{\dimexpr \textwidth-3cm-\columnsep}%
   {\small\uppercase\leftmark}}%
}

\pagestyle{fancy}

\pretitle{
	\begin{center}
	\LARGE
	\includegraphics[width=10cm,height=3cm,keepaspectratio]{newcevro}
}
\posttitle{\end{center}}

\begin{document}

\title{Svěřenský fond jako nástroj transferu majetku mezi generacemi}
\author{Dominik Bálint}
\date{01.12.2019}

\pagenumbering{gobble}
  \thispagestyle{empty}
  \begin{center}
  \includegraphics[width=10cm,height=3cm,keepaspectratio]{newcevro} \\
  \end{center}
  \vspace{15mm}
  \begin{center}
  {\Large Vysoká škola Cevro Institut} \\
  \vspace{15mm}
  {\Large \textbf{Svěřenský fond jako mezigenerační nástroj převodu majetku}} \\
  \vspace{15mm}
  {\Large Dominik Bálint} \\
  \vspace{15mm}
  {\Large Bakalářská práce} \\
  \vspace{49mm}
  {\Large \textbf{Praha 2019}} \\
  \end{center}
  
\newpage
  \thispagestyle{empty}
  \maketitle
  \begin{center}
  {\Large Katedra veřejného práva} \\	
  \vspace{15mm}
  {\Large \textbf{Studijní program:} Veřejné právo} \\
  {\Large \textbf{Studijní obor:} Právo v obchodních vztazích} \\
  {\Large \textbf{Jméno vedoucího diplomové práce:} } \\
  {\Large Mgr. Ing.  Střeleček  Tomáš LL.M.} \\
  \end{center}
  
\newpage
  \thispagestyle{empty}
  \vspace*{\fill}

\noindent \textbf{Čestné prohlášení} \\

Prohlašuji, že jsem předkládanou práci zpracoval samostatně, uvedl
v ní všechny použité prameny a zdroje, které jsou uvedeny v seznamu použité
literatury, a v textu řádně vyznačil jejich použití. \\

\noindent V Praze dne \today \\
\vspace{10mm} \\
\begin{tabular}{p{6cm}c}
& ................................................. \\
& Dominik Bálint
\end{tabular}

\newpage

\thispagestyle{empty}

\vspace*{\fill}
\noindent \textbf{Poděkování} \\

	Na tomto místě bych rád poděkoval svému vedoucímu práce panu Mgr. Ing. 
Střeleček  Tomáš LL.M. za podporu, trpělivost, spolupráci, podněty ke zlepšení a 
také za čas, který mi věnoval při vedení práce.

\vspace*{\fill}
 
\newpage
  \tableofcontents
  \listoffigures
  
\newpage
  \pagenumbering{arabic}
  
\section{Úvod}

Práce si klade za cíl v historickém a právním kontextu představit institut svěřenského fondu a převzetí jeho právní úpravy z právního řádu provincie Quebec.
\linebreak

\indent Dále se práce bude zabývat představením momentální právní úpravy svěřens\-kého fondu, jeho možnostmi, ukázanými na příkladu praktického využití – jako nástroj, který může substituovat klasické dědění, i potenciálními možnostmi zneužití a jak (a zda) jsou odstraněny obavy o zneužití svěřenských fondů v souvislosti s novelizací občanského zákoníku v roce 2018. 
\linebreak

\indent Dalším bodem práce bude faktické porovnání právní úpravy dědického práva s právní úpravou svěřenského fondu. Těžištěm této časti je snaha potvrdit, nebo vyvrátit tvrzení, že právní úprava dědického práva a svěřenského fondu je vyvážená a svěřenským fondem se nedají obcházet práva dědiců.
\linebreak

\indent V tomto rámci se budu zabývat zejména zhodnocením právní úpravy svěřens\-kých fondů a její analýzou v souvislosti s ochranou dědiců, vyložení potenciálních problémů momentální právní úpravy v rámci českého právního řádu, tzn. potenciální zneužití, v souvislosti s právem nepominutelných dědiců, které můžou vzniknout v případě souběhu dědického řízení a založení svěřenského fondu.
\linebreak

\indent V závěru práce budou vyloženy systémové nedostatky v případě, že je naleznu a následovat bude návrh řešení nalezených problémů.

\newpage

\section{Správa cizího majetku - historie}

\subsection{Úvod do historických počátků svěřenství a svěřenských fondů}

\indent Dne 22.3.2012 vstoupil v platnost nový občanský zákoník, který nabyl účinnosti prvním dnem roku 2014. Nový občanský zákoník s číslem 89/2012 Sb. rekodifikoval civilní právo a zavedl v Českém právu několik nových institutů. Rekodifikace civilního práva se dotka i právní úpravy správy cizího majetku, v jejichž mezích bylo zavedeno několik nových právních institutů. Mezi tyto instituty můžeme zařadit také institut svěřenství respektive svěřenského nástupnictví, který předchozí občanský zákoník, až na jeden paragraf\footnote{V §859 je upraven zánik všech omezení vyplívajících ze svěřenského nástupnictví}, neupravoval a také svěřenský fond, na který se práce bude především zaměřovat. \\

\indent Nově zavedená podoba institutu svěřenského fondu a obecné správy cizího majetku je pro nás sice nová, nicméně jedná se o instrument, který v českém právu v minulosti v určitých formách existoval až do 1.1.1951 do zrušení svěřens\-kého nástupnictví, zvané též jako fideikomisární substituce\footfullcite[§565]{noauthor_zakon_nodate}. Omezení vyplívající z institutu svěřenského nástupnictví nicméně trvala až do roku 1964, kdy vstoupil v platnost a zanedlouho i v účinnost nový federální občanský zákoník pod číslem 40/1960 Sb., který úplně zrušil svěřenské nástupnictví\footnote{Viz citace 4}. Mezi roky 1964 a 2014 lze tedy hovořit o faktickém vakuu právní úpravy svěřenských institutů v Československu a potažmo v České republice, respektive zákazu používání těchto institutů. Následující kapitoly ve zkratce představí institut svěřenství od, respektive z, dob Římské říše, následovat bude popis vývoje svěřenství během období monarchie a po vzniku samostatného Československého státu až po zrušení svěřenství v roce 1964. V poslední části bude popsáno znovuzavedení institutu svěřenství a svěřenského fondu novým občanským zákoníkem v roce 2014 s důrazem na převzetí právní úpravy svěřenského fondu z právního řádu provincie Quebec, kterému bude, pro lepší pochopení České právní úpravy, předcházet popis a historie trustu, s důrazem na momentální úpravu v právním řádu provincie Quebec tato kapitola bude vycházet především z Komentovaného znění části nového občanského zákoníku upravující problematiku správy cizího majetku od Jaroslava Svejkovského a kolektivu. V návaznosti na znovuzavedení institutu svěřenství bude následovat základní popis novelizace právní úpravy svěřenských fondů z roku 2018, který bude dále podrobněji objasněn v druhé části práce. \\

\newpage

\subsection{Pojem svěřenství}

\indent Aby bylo možné dále hovořit o institutu svěřenství, respektive svěřenského fondu, je třeba si nejprve vymezit co institut svěřenství znamená, představuje a k jakému účelu slouží. Barbora Bednaříková institut svěřenstí definuje jako: "...takový vztah, kdy v principu jedna osoba svěří svůj určitý majetek druhé osobě ve prospěch osoby třetí."\footfullcite[Kapitola Úvod, str. IX]{bednarikova_barbora_sverenske_2014} Účel takovéhoto svěření, jak je výše popsáno, je tedy zajištění určitého prospěchu pro třetí osobu/y. V historickém kontextu tento prospěch spočíval především v poskytování užitků z vyčleněného majetku a/nebo jeho převedení ve prospěch třetí osoby, nicméně v rámci momentální právní úpravy lze hovořit o více účelech svěřenství. \\

\indent V kontinentálním právu se instituty, které by se svým obsahem shodovaly s výše popsaným vymezením, začaly používat již v antickém Římě ve formě takzvaného fideikomisu (latinsky \textit{fideicommissum}, pochází ze slov \textit{fīdē} v překladu "důvěra" a \textit{commissum} v překladu "svěřené", dohromady též jako "tvé důvěře svěřuji"\footfullcite{noauthor_fideicommissum_nodate}) a také v právu anglosaském ve formě trustů (anglicky \textit{trust}, v překladu "důvěra"). Ve většině zemí západního světa, používajících obou právních systémů je tato forma správy cizího majetku dlouhodobě používána k uchování, převodu, správě, zachování celistvosti, nebo použití k dalším soukromým, nebo veřejným účelům\footfullcite[Kapitola Úvod, str. IX]{bednarikova_barbora_sverenske_2014}, v různých formách. Některé země tradičně využívají institutu trustu, například Irsko, jiné země právní úpravu trustu známou především ze zemí Common Law\footnote{Země používající anglosaský systém práva} přizpůsobili kontinentálnímu právnímu myšlení, mezi tyto země lze zařadit Lichtenštejnsko (Treuhandverhältnis) a Lucembursko (Fiducie), další země se rozhodly funkci trustu nahradit jiným institutem - Německo (Treuhand) a Rakousko (Privatstftungsgesetz) \footfullcite{trust_2014}. \\

\indent V momentální právní úpravě, tedy v občanském zákoníku, jsou v paragrafech 1448 a 1449 fakticky vymezeny účely svěřenského fondu, těmito účely je účel soukromý a účel veřejný\footfullcite{noauthor_zakon_2012}. Soukromý svěřenský fond slouží ku prospěchu určité osoby, nebo na její památku, z tohoto ustanovení je možné dovodit, že svěřenský fond založený za soukromým účelem slouží k ochraně, uchování, rozmnožení, správě a převodu vyčleněného majetku. Co se veřejného účelu týče, hlavním účelem nemůže být dosahování zisku nebo provozování závodu, svěřenský fond založený za veřejným účelem tedy slouží k veřejnému, především socioekonomickému, prospěchu.\\

\newpage

\subsection{Antický řím}

Počátky svěřenských institutů je možné pozorovat již za dob Antického Říma, k pochopení co vedlo ke vzniku těchto institutů bude následující kapitola obsahovat popis systému práva a dědického práva Antického Říma, následovat bude představení svěřenských institutů vzniklých v této době a následně bude popsán jejich vývoj s ohledem na změny právního systému Říma.

\subsubsection{Praetorské právo a Ius Civille}

K pochopení dvoukolejnosti římského dědického práva je v první řadě třeba vymezit rozdíly mezi \textit{ius civile} a \textit{ius honorarium}. \\

\textit{Ius civile} je právo, jehož subjekty jsou Římští občané, původně vycházelo z obyčejů. Nejstarší \textit{ius scriptum}\footnote{psané právo}, vykládající ius civile, který je nám znám je \textbf{Zákon \MakeUppercase{\romannumeral 12} desek}, později upraveno v \textit{ius civile Papirianum}. \textit{Ius civile bylo} tvořeno pontifikálními interpretacemi\footnote{zákony tvořeny Pontifiky, kteří vykládali právo} a komiciálními zákony\footnote{zákony tvořeny lidovým shromážděním} a v době císařství císařskými konstitucemi\footnote{zákony tvořeny císařem}. Problém \textit{ius civile} však spočíval v jeho nepružnosti, která se projevovala špatným přizpůsobováním změnám ve společnosti a vývoji doby, a jednoduchého a striktního určení některých zásad.

\begin{figure}[h]
\centering
\includegraphics[width=15cm,height=5cm,keepaspectratio]{rimskepravostruktura.jpeg}
\caption{Struktura soukromého římského práva}
\label{fig:struktura}
\end{figure}

\newpage

\subsubsection{Dědické právo v dobách antického Říma}

Pro historické pochopení vzniku svěřenských institutů ve starověkém Říme je nejprve třeba přiblížit způsob fungování dědického práva v Římě, vycházejícího z principu univerzální sukcese\footnote{Do dědictví spadá jmění, tedy jak majetek, tak i závazky}\textsuperscript{,} \footfullcite{blaho_peter_haramia_ivan_a_zidlicka_michaela_zaklady_1997}. Po smrti zůstavitele šlo dědit na základě zákona, nebo na základě testamentu, toto zároveň určovalo i dědickou posloupnost. \\

\vspace{5 mm}

Římské právo tedy rozlišovalo dvě posloupnosti:
\begin{enumerate}
\item \textit{hereditas testamentaria} - dědická posloupnost na základě testamentu,
\item \textit{hereditas legitima} - dědická posloupnost na základě zákonu.
\end{enumerate}

\vspace{5 mm}

Je třeba poznamenat, že posloupnost zakládající se na testamentu, tedy jednostranného právního dokumentu ustanovujícího zůstavitelovi dědice, měla přednost před posloupností zákonnou. Nevyčerpání celého dědictví ze strany hereditas testamentaria, tedy nemohlo založit právní důvod pro delaci\footnote{povolání} intestátních dědiců\footnote{dědicové dle hereditas legitima}, v takovém případě dědil i zbytek pozůstalosti dědic povolaný na základě testamentu. \\

Přes fakt, že dědění na základě testamentu mělo přednost, znalo Římské právo institut nepominutelného dědice, které se ve svých prvopočátcích vymáhalo takzvanou kverelou, tedy žalobou před soudem na zrušení testamentu. Až v pozdější době císařské měli nepominutelní dědici při jejich opomenutí zůstavitelem v závěti právo na dorovnání svého zákonného podílu bez nutnosti rušit testament\footfullcite[3. vydání z roku 2012, str.230]{marek_karel_redakcni_rada_casopis_2012}. Povinnost zohlednit nepominutelného dědice se mohl zůstavitel zprostit vyděděním daného dědice. Z počátku nemusel zůstavitel uvádět důvod, toto bylo změněno vydáním Justiniánských reforem, které taxativně vymezili důvod pro vydědění nepominutelného dědice \footfullcite[str.147]{blaho_peter_haramia_ivan_a_zidlicka_michaela_zaklady_1997} . \\

\newpage

Římské právo dále rozlišovalo právní postavení dědiců ve smylu povinnosti dědit na:

\vspace{5 mm}

Římské právo tedy rozlišovalo dvě posloupnosti \footfullcite[str.141]{blaho_peter_haramia_ivan_a_zidlicka_michaela_zaklady_1997}:
\begin{enumerate}
\item \textit{heredes voluntarii} - dobrovolní dědicové, jsou povoláni k dědictví projevem vůle nebo zachováním se ve smyslu přijetí - tedy například uhrazení pohledávky zůstavitele,
\item \textit{heredes necessarii} - nutní dědicové, jsou povoláni k dědictví smrtí zůstavitele bez možnosti odmítnutí dědictví, tato skupina se dále dělí na další dvě podskupiny:
\begin{itemize}
\item \textit{heredes sui et necessarii} - dědici nutní a vlastní, římská rodina byla silně patriarchálního charakteru a vztahy se rozlišovaly na takzvané agnátské a kognátské. Paterfamilias, tedy otec rodiny měl moc na členy jeho rodiny. Tito členové, kteří v době zůstavitelovi smrti spadali pod jeho moc se stali členy této podskupiny. Do vzniku \textit{beneficium abstinendi} v rámci praetorského práva něměli možnost dědictví odmítnout.
\item \textit{heredes necessarii} - jedná se o otroky, kteří byli v testamentu zároveň osvobozeni, neměli možnost dědictví odmítnout.
\end{itemize}
\end{enumerate}

\vspace{5 mm}

Pokud byl v testamentu jako dědic označen otrok v moci jiného římského občana, stal se dědicem onen římský občan. \todo{Dohledat zdroj a najít vhodné místo k doplnění v textu} \\

V rámci praetorského práva vyhradil dědicům praetor takzvané beneficium abstinenti, které jim umožňovalo vzdát se dědictví a přesto, že stále byli dědici dle ius civile, zabraňovalo také možnosti podávání žaloby zůstavitelových věřitelů proti těmto dědicům.

\newpage

V rámci Římského práva a zvyklostí bylo možně přenechat určitou část majetku někomu, kdo nebyl dědicem formou takzvaného legata (v překladu odkazu). Tento způsob byl nicméně zatížen formálními požadavky na jeho formu, z tohoto důvodu se začaly vyčleňovat i jiné postupy převodu majetku, které nebyly zatíženy takovou formálností\footfullcite{noauthor_fideicommissum_nodate}. Jeden z těchto postupů byl již výše zmíněný fideikomis, jehož splnění bylo ve svých prvopočátcích svěřeno dědici čistě na základě důvěry. Faktem však nadále zůstávalo, že fideikomis nebyl právně vymahatelný a tím pádem nebylo možné splnění zůstavitelova přání požadovat formou žaloby.

\subsection{Historie svěřenských fondů u nás a ve světě}

Co se historie svěřenský fondů týče, tak \footnote[4]{Zde bude nějaká informace}

\subsection{Recepce právní úpravy Svěřenského fondu z právního řáduprovincie Quebec}

\newpage
\thispagestyle{smallertextinheader}

\section{Momentální právní úprava svěřenských fondů}

\subsection{Potenciální možnosti použití v rámci převodu majetku}

\subsection{Novelizace právní úpravy - zápis svěřenských fondů do rejstříku}

\newpage

\thispagestyle{smallertextinheader}

\section{Komparace právní úpravy svěřenského fondu s právní úpravou dědictví}

\newpage

\section{Systémové nedostatky a návrhy na řešení}

\newpage

\section{Závěr}

\newpage
\printbibheading
\printbibliography[type=misc,heading=subbibliography,title={Online sources}]
\printbibliography[type=book,heading=subbibliography,title={Other sources}]
	
\end{document}