\documentclass{article}

% Language definition
\usepackage[utf8]{inputenc}
%\usepackage[czech]{babel}

% Package definition and folder with images
\usepackage{graphicx}
\graphicspath{ {./Obrazky/} }

% Package difinition
\usepackage{fancyhdr}
\usepackage{xcolor}
\usepackage{titling}
\usepackage{array}
\usepackage{todonotes}

% Base header definition
\setlength\headheight{26pt}
\lhead{\includegraphics[width=3cm,height=\dimexpr \headheight-\dp\strutbox]{newcevro}}
\rhead{\small{\leftmark}}

% Bibliography definition
\usepackage{biblatex}
\addbibresource{Zdroje.bib}

% Footnote package definition
\usepackage{footnote} %Package použitý k tomu, aby byli citace uvnitř float elementů pod čarou
\makesavenoteenv{figure} %Nastavení toho, aby byli citace uvnitř figure pod čarou

% Package definition
\usepackage{xcolor}

% Different language for contents and figures
\renewcommand\contentsname{Obsah}
\renewcommand\listfigurename{Seznam obrázků}
\renewcommand\figurename{Obrázek}

% Special header style definition for sections with long text in header
\fancypagestyle{smallertextinheader}{ % Dokončit styl s menším textem v hlavičce, již je dokončeno
   \fancyhf{}
   \fancyhead[L]{\includegraphics[width=3cm, height=\dimexpr \headheight-\dp\strutbox]{newcevro}}
   \fancyhead[R]{%
   \parbox[b]{\dimexpr \textwidth-3cm-\columnsep}%
   {\small\uppercase\leftmark}}%
   \fancyfoot[C]{\thepage}
}

% Special header for contents only
\fancypagestyle{Contents}{ % Dokončit styl s menším textem v hlavičce, již je dokončeno
   \fancyhf{}
   \fancyhead[LE,LO]{\includegraphics[width=3cm, height=\dimexpr \headheight-\dp\strutbox]{newcevro}}
   \fancyhead[RE,RO]{\small{\uppercase{\rightmark}}}
}

% Basic page style definition
\pagestyle{fancy}

% Pretitle definition
\pretitle{
	\begin{center}
	\LARGE
	\includegraphics[width=10cm,height=3cm,keepaspectratio]{newcevro}
}
\posttitle{\end{center}}

% Document body --------------
\begin{document}

% Title definition for second page
\title{Svěřenský fond jako nástroj transferu majetku mezi generacemi}
\author{Dominik Bálint}
\date{01.12.2019}

% First page of my bachelors work
\pagenumbering{gobble}
  \thispagestyle{empty}
  \begin{center}
  \includegraphics[width=10cm,height=3cm,keepaspectratio]{newcevro} \\
  \end{center}
  \vspace{15mm}
  \begin{center}
  {\Large Vysoká škola Cevro Institut} \\
  \vspace{15mm}
  {\Large \textbf{Svěřenský fond jako mezigenerační nástroj převodu majetku}} \\
  \vspace{15mm}
  {\Large Dominik Bálint} \\
  \vspace{15mm}
  {\Large Bakalářská práce} \\
  \vspace{49mm}
  {\Large \textbf{Praha 2019}} \\
  \end{center}
  
 % Second page information 
\newpage
  \thispagestyle{empty}
  \maketitle
  \begin{center}
  {\Large Katedra veřejného práva} \\	
  \vspace{15mm}
  {\Large \textbf{Studijní program:} Veřejné právo} \\
  {\Large \textbf{Studijní obor:} Právo v obchodních vztazích} \\
  {\Large \textbf{Jméno vedoucího diplomové práce:} } \\
  {\Large Mgr. Ing.  Střeleček  Tomáš LL.M.} \\
  \end{center}
  
% Čestné prohlášení  
\newpage
  \thispagestyle{empty}
  \vspace*{\fill}

\noindent \textbf{Čestné prohlášení} \\

Prohlašuji, že jsem předkládanou práci zpracoval samostatně, uvedl
v ní všechny použité prameny a zdroje, které jsou uvedeny v seznamu použité
literatury, a v textu řádně vyznačil jejich použití. \\

\noindent V Praze dne \today\\%{\selectlanguage{czech}\today} \\ %Změněno na české datum, možná budu muset odebrat, babel package způsobuje z nějakého důvodu error
\vspace{10mm} \\
\begin{tabular}{p{6cm}c}
& ................................................. \\
& Dominik Bálint
\end{tabular}

% Poděkování
\newpage

\thispagestyle{empty}

\vspace*{\fill}
\noindent \textbf{Poděkování} \\

	Na tomto místě bych rád poděkoval svému vedoucímu práce panu Mgr. Ing. 
Tomáši Střelečkovi LL.M. za podporu, trpělivost, spolupráci, podněty ke zlepšení a 
také za čas, který mi věnoval při vedení práce.

\vspace*{\fill}

% Page with contents 
\newpage
  \thispagestyle{Contents}
  \tableofcontents
  \listoffigures

% Strana se seznamem použitých zkratek  
\newpage
 \pagenumbering{arabic}
 
\begin{center}
\section{Seznam použitých zkratek}
\end{center}

\vspace{5 mm}

\textbf{Právní předpisy}

\vspace{5 mm}

\begin{tabular}{p{3cm}p{8cm}}
\textbf{OZ} & zákon č. 89/2012 Sb., občanský zákoník	
\end{tabular}

% Samotný text práce začíná zde  
\newpage
  
\section{Úvod}

Práce si klade za cíl v historickém a právním kontextu představit institut svěřenského fondu a převzetí jeho právní úpravy z právního řádu provincie Quebec.
\linebreak

\indent Dále se práce bude zabývat představením současné právní úpravy svěřens\-kého fondu, jeho možnostmi a výhodami, ukázanými na příkladu praktického využití – jako nástroj, který může substituovat, nebo být komplementem klasickému dědění a dalším způsobům převodů majetku a jeho porovnání s těmito způsoby. Dále se budu zabývat i potenciálními možnostmi zneužití a jak (a zda) jsou odstraněny obavy o zneužití svěřenských fondů v souvislosti s novelizací občanského zákoníku z roku 2018.
\linebreak

\indent Dalším bodem práce bude faktické porovnání právní úpravy dědického práva s právní úpravou svěřenského fondu. Těžištěm této časti je snaha potvrdit, nebo vyvrátit tvrzení, že právní úprava dědického práva a svěřenského fondu je vyvážená a svěřenským fondem se nedají obcházet práva dědiců. Pokud bych tedy hypotézu refrázoval jako otázku, zněla by následovně: "Dá se svěřenský fond použít k obcházení práv nepominutelných dědiců?"
\linebreak

\indent V tomto rámci se budu zabývat zejména zhodnocením právní úpravy svěřens\-kých fondů a její analýzou v souvislosti s ochranou dědiců, vyložení potenciálních problémů současné právní úpravy v rámci českého právního řádu, tzn. potenciální zneužití, v souvislosti s právem nepominutelných dědiců, které mohou vzniknout v případě souběhu dědického řízení a založení svěřenského fondu.
\linebreak

\indent Tato část je nesmírně důležitá, neboť věřím, že s ohledem na nastupující novou generaci, která se již narodila do svobodného kapitalistického tržněeko\-nomického systému, čeká svěřenský fond období, ve kterém o něj bude velký zájem a začne se hromadně využívat s čím souvisí i potenciální možnost rozšíření důvodů a účelů založení svěřenského fondu a jeho postupné přibližování angloamerické právní úpravě. Je tedy nezbytné ujasnit si určité otázky vyplývající z recepce\ tohoto právního institutu. \\
%\linebreak

\indent V závěru práce budou vyloženy systémové nedostatky, založené na historickém kontextu, praktické části této práce a komparaci s právní úpravou tohoto institutu v jiných zemích, v případě, že je naleznu, a následovat bude návrh řešení nalezených problémů.

%{\color{green}\noindent===================Comment===================} \\
%Přidat hypotézu.\\
%Dále by bylo vhodné poznamenat do úvodu, že věřím, že s ohledem na nastupující novou generaci, která již se narodila do svobodného kapitalistického tržněekonomického systému, čeká svěřenský fond období, ve kterém o něj bude velký zájem a začně se hromadně využívat. Je tedy třeba ujasnit si určité otázky vyplývající z recepce tohoto právního institutu. Dále si i myslím, že by se mohly rozšířit důvody a účely založení trustu respektive svěřenského fondu s ohledem na to, jak o něj poroste zájem.\\
%Nelze převzít institut ale právní úpravu, tedy Komparace České právní úpravy s úpravou ostatních zemí
%Je možné komparovat čskou a quebeckou právní úpravu
%Zjistit úpravu zadání co se týče porovnání anglického Trustu s českou právní úpravou
%Musím komparovat právní úpravu ne institut

%V komparaci již představím právní úpravy, tedy nemusím dále představit momentální právní úpravu svěřenského fondu

%Porovnání dědického práva.

%V tomato rámci se budu zabývat zejména tím, že porovnám právní úpravu s právní úpravou svěřenského fondu, budu se zabývat její analýzou v souvisloti s ochranou dědiců

%Hypotéza: Pokud není jako otázka - ANO nebo NE, pokud bych dal jako otázku, tak bych dopředu nepresumoval jak to je, ale dospěl bych k tomu

%V závěru může krátce popsat rozdííy mezi Anglickou právní úpravou a Českou právní úpravou

%V závěru bych mohl komparovat pro případy návrhů jak zamezit zneužití
%Ještě se podívat na poznámku ohledně bakalářské práce v Notes
%{\color{green}\noindent============================================} \\

\newpage

\section{Svěřenský fond - obecně}

\subsection{Pojem svěřenství, svěřenský fond a struktura práce}
%Zamyslet se nad tím, zda jinak nepojmenovat tuto část. Přejmenováno, zkonzultovat, zda to takto dává smysl.

\indent Dne 22.3.2012 vstoupil v platnost nový občanský zákoník, který nabyl účinnosti prvním dnem roku 2014. Nový občanský zákoník s číslem 89/2012 Sb. rekodifikoval civilní právo a zavedl v Českém právu několik nových institutů. Rekodifikace civilního práva se dotka i právní úpravy správy cizího majetku, v jejichž mezích bylo zavedeno několik nových právních institutů. Mezi tyto instituty můžeme zařadit také institut svěřenství respektive svěřenského nástupnictví a svěřenského náhradnictví, který předchozí občanský zákoník, až na jeden paragraf\footfullcite[V 40/1964, §859 je upraven zánik všech omezení vyplývajících ze svěřenského náhradnictví]{noauthor_zakon_1964}, neupravoval a také svěřenský fond, na který se práce bude především zaměřovat. \\

\indent Nově zavedená podoba institutu svěřenského fondu a obecné správy cizího majetku je pro nás sice nová, nicméně jedná se o podobný instrument, který v českém právu v minulosti v určitých formách existoval až do 1.1.1951 do zrušení svěřens\-kého náhradnictví, zvané též jako fideikomisární substituce\footfullcite[§565]{noauthor_zakon_nodate}. Omezení vyplývající z institutu svěřenského náhradnictví nicméně trvala až do roku 1964, kdy vstoupil v platnost a zanedlouho i v účinnost nový federální občanský zákoník pod číslem 40/1960 Sb., který úplně zrušil svěřenské náhradnictví\footnote{Viz citace 4}. Mezi roky 1964 a 2014 lze tedy hovořit o faktickém vakuu právní úpravy svěřen\-ských institutů v Československu,České a Slovenské republice a následně v České republice, respektive zákazu používání těchto institutů.\\

\indent Práce je zaměřena jak na toto období a dále i na širší historii a vznik těchto institutů, tak i na momentální právní úpravu a možnosti z ní vycházející. Z tohoto důvodu se tedy sestává ze tří dílčích částí, z nichž každá saturuje informační potřebu nezbytnou pro další část práce. První část práce je čistě teoretická, pojednává o historii a pojmových znacích svěřenského fondu a jemu předcházejících institutů a obecnému úvodu do dědického práva, které je nezbytné pro zodpovězení stanovené hypotézy. Druhá část práce je zaměřena na praktické zkoumání možností svěřenského fondu zejména s ohledem na možnosti mezigeneračního převodu majetku s ním spojeného a zodpovězení nejasností z této možnosti vyplývajících, především s ohledem na institut nepominutelného dědice. Ve třetí a tedy konečné části práce bude následovat můj pohled na tyto problémy a mnou navrhnuté potenciální řešení těchto problémů. Považuji za vhodné tyto jednotlivé části více popsat a činím proto tak na další straně.\\

\newpage

{\Large Teoretická část}\\

S ohledem na výše uvedený fakt, je v první řadě nezbytné obecné představení tohoto nově zavedeného institutu, v teoretické rovině, které je detailněji popsáno v následující podkapitole a bude sloužit jako solidní odrazový můstek pro vylož\-ení možností, které nám svěřenský fond přinesl stejně tak jako jeho potenciálních problémů a nejasností. \\

Kapitola a jednotlivé podkapitoly následující rovněž ve zkratce představí institut svěřenství od, respektive z dob Římské říše, následovat bude popis vývoje svěřenství během období monarchie a po vzniku samostatného Československého státu až po zrušení svěřenství v roce 1964. V poslední části bude popsáno znovuzavedení institutu svěřenství a svěřenského fondu novým občanským zá\-koníkem v roce 2014 s důrazem na převzetí právní úpravy svěřenského fondu z právního řádu provincie Quebec, kterému bude, pro lepší pochopení České právní úpravy, předcházet popis a historie trustu, s důrazem na momentální úpravu v právním řádu provincie Quebec. Tato kapitola bude vycházet především z Komentovaného znění části nového občanského zákoníku upravující problematiku správy cizího majetku od Jaroslava Svejkovského a kolektivu. Takto vyložené informace jsou, s ohledem na jejich přímou návaznost na institut svěřenského fondu, zásadní pro pochopení historie a funkčnosti svěřenských fondů, které je možné používat pro velké množství účelů. Prioritně bude práce klást důraz na svěřenský fond jako na institut, který je možné použít k mezigeneračnímu převedení majetku. V návaznosti na znovuzavedení institutu svěřenství bude následovat základní popis novelizace právní úpravy svěřenských fondů z roku 2018, který bude dále podrobněji objasněn v druhé, tedy praktické, části práce. V neposlední řadě bude součástí této části i stručný úvod do českého dědického práva, který je důležitý pro pochopení kolizí, které mohou mezi právní úpravou dědictví a právní úpravou svěřenských fondů vzniknout.\\

{\Large Praktická část}\\

Po historickém úvodu do problematiky bude následovat plynulý přechod k momentální právní úpravě svěřenských fondů a bližšímu vysvětlení novelizace Občanského zákoníku z roku 2018, který mimo jiné přinesl povinnou evidenci svěřenských fondů. Poté se posunu k praktickému využítí instututu svěřenského fondu pro výše zmíněné účely mezigeneračního převod majetku. V této části také zhodnotím výhody této možnosti, respektive cesty, oproti jiným možnostem převodu majetků za stejným účelem. V tomto rámci se také naplno projeví a bude možno pozorovat jakousi nedotaženost právní úpravy svěřenského fondu v našem právní řádu pramenící v první řadě z faktu, že svěřenský fond je obdobou institutu trustu, který je především používaný v systémech common law, ve kterých neexistuje mnoho právních institutů vlastní našemu systému civil law. Přes fakt, že institut svěřenského fondu byl přejat z Quebeckého občanského zákoníku, který si zachoval charakter systému civil law, tak i tato norma neupravuje mnoho institutů nám známým stejně tak i na mnoho institutů nahlíží jinak. Z tohoto vyplývá, že i po šesti letech účinnosti nového občanského zákonníku, a tedy i možnosti zakládat svěřenské fondy, přetrvávají nejasnosti ohledně kolize úpravy svěřenského fondu s jinými právními předpisy. Jednu z těchto kolizí představuje i nejasnost mezi založením svěřenského fondu pro případ smrti a právem nepominutelných dědiců. Na tento rozkol existuje mnoho zásadně odlišných názorů a k dnešnímu datu neexistuje dokonce ani žádné precedenční soudní rozhodnutí, které by tuto otázku zodpovědělo. Na toto a další nejasnosti, které vyplývají z recepce institutu vlastního systémům common law do našeho právního řádu, bude klást důraz část následující po části popisující současnou právní úpravu - Systémové nedostatky a návrhy na řešení. Nejasnosti se pokusím, v historickém kontextu a s ohledem na cizí právní úpravu a zhodnocení odborníků, vyložit a nalézt na ně vhodnout odpověď nebo řešení, které bude představeno v následující, tedy závěrečné, části. \\

{\Large Třetí část - Návrhy na změny České právní úpravy}\\

Po analýze a nalezení nedostatků v rámci české právní úpravy je dle mého názoru vhodné se nad těmito problémy hlouběji zamyslet a navrhnout jistá řešení. Daná řešení bych chtěl obsáhnout právě v této části práce. Tedy po zanalyzování a popsání jednotlivých nedostatků a vyložení jednotlivých názorových proudů na tyto nedostatky z předchozí části.

\newpage

\subsection{Úvod do historických počátků svěřenství a svěřenských fondů}

\indent Aby bylo možné dále hovořit o institutu svěřenství, respektive svěřenského fondu a jeho předlohy ve formě trustu v CCQ, je třeba si nejprve vymezit co institut svěřenství znamená, představuje a k jakému účelu slouží, tedy definovat pojmové znaky a jeho konstrukci.\\

Samotnou definici českým zákonodárcem přijatého trustu dobře shrnuje Barbora Bednaříková, ta institut svěřenstí definuje jako: "...takový vztah, kdy v principu jedna osoba svěří svůj určitý majetek druhé osobě ve prospěch osoby třetí."\footfullcite[Kapitola Úvod, str. IX]{bednarikova_barbora_sverenske_2014}\\

Tato definice se do značné míry ztotožňuje s definicí čistě common law trustu z knihy Underhill and Hayton, Law of Trusts and Trustees, což je s ohledem na fakt, že trust sloužil jako inspirace logické. V daném díle je definován následovně: \\

\textit{"A trust is an equitable obligation, binding a person (called a trustee) to deal with property (called trust property) owned by him as a separate fund, distinct from his own private property, for the benefit of persons (called beneficiaries or, in old cases, cestuis que trust), of whom he may himself be one, and any one of whom may enforce the obligations."}\footfullcite{underhill_a_hayton_d_law_2010} \\
%Doplnit zdroj a pod čáru dát poznámku, že jsem daný zdroj objevil v jiné Diplomové práci, překlad udělám sám, protože nevím koho bych měl ozdrojovat.

\textit{"Trust je spravedlivý závazek, zavazující osobu (trustee - obdoba svěřenského správce), aby se starala o majetek, kterýžto vlastní jako o oddělené jmění, odlišné od jmění jím vlastněné, za účelem prospěchu osob (beneficient), kteroužto může rovněž býti i on sám, nebo kdokoliv další kdo má právo dané závazky vymáhat."}\footnote{Inspirováno překladem z práce Martina Skuhrovce, zdroj překladu není patrný} \textsuperscript{,}  \footfullcite{skuhrovec_michal_sverensky_2018} \\

Z litery zákona, definice České autorky i zahraničních autorů jasně vyplývá to, co stanovuje i důvodová zpráva, a to jasná inspirace trustem, tedy trustem upraveným v rámci CCQ. \\

Účel takovéhoto svěření, jak je výše popsáno, ať již se bavíme o první, či druhé definici, je tedy zajištění určitého prospěchu pro třetí osobu/y. V historickém kontextu tento prospěch spočíval především v poskytování užitků z vyčleněného majetku a/nebo jeho převedení ve prospěch třetí osoby, nicméně v rámci současné právní úpravy lze hovořit o mnoho dalších účelech svěřenství, ke kterým se v průběhu práce také blížeji dostanu. Jedná se jak o svěřenské fondy založené za veřejně prospěšným, tak i za soukromým účelem.\\

\indent V kontinentálním právu se instituty, které by se svým obsahem shodovaly s výše popsaným vymezením, začaly používat již v antickém Římě ve formě takzvaného fideikomisu (latinsky \textit{fideicommissum}, pochází ze slov \textit{fīdē} v překladu "důvěra" a \textit{commissum} v překladu "svěřené", dohromady též jako "tvé důvěře svěřuji"\footfullcite{noauthor_fideicommissum_nodate}). V právu anglosaském, které se vyvýjelo odděleně od dob konce Římské nadvlády a především od dob heptarchie v 7. století našeho letopočtu\footfullcite{jan_kuklik_dejiny_2011}, poté ve formě trustů (anglicky \textit{trust}, v překladu "důvěra", již zde lze i z čistě lingivstického hlediska pozorovat podobnost mezi trustem a fideicomissem, významy těchto slov jsou stejné: "důvěře svěřuj" a "důvěra"), které jsou ale obdobně ve své podstatě potomky římských fideicomissů, nicméně ke vzniku trustu jako takového došlo v 16.století jako důsledek Zákona o závětích\footnote{Dějiny angloamerického práva - strana 86}. Ve většině zemí západního světa, používajících obou právních sys\-témů, je tato forma správy cizího majetku dlouhodobě používána k uchování, převodu, správě, zachování celistvosti, nebo použití k dalším soukromým, nebo veřejným účelům\footfullcite[Kapitola Úvod, str. IX]{bednarikova_barbora_sverenske_2014}, v různých formách. Některé země tradičně využívají institutu trustu, například Irsko, jiné země právní úpravu trustu známou především ze zemí Common Law\footnote{Země používající anglosaský systém práva} přizpůsobili kontinentálnímu právnímu myšlení. Mezi tyto země lze zařadit Lichtenštejnsko (Treuhandverhältnis) a Lucembursko (Fiducie), další země se rozhodly funkci trustu nahradit jiným institutem - Německo (Treuhand) a Rakousko (Privatstftungsgesetz) \footfullcite{trust_2014}. \\

\indent V současné právní úpravě, tedy v občanském zákoníku, jsou v paragrafech 1448 a 1449 fakticky vymezeny účely svěřenského fondu, těmito účely je účel soukromý a účel veřejný\footfullcite{noauthor_zakon_2012}. Soukromý svěřenský fond slouží ku prospěchu určité osoby, nebo na její památku, z tohoto ustanovení je možné dovodit, že svěřenský fond založený za soukromým účelem slouží k ochraně, uchování, rozmnožení, správě a převodu vyčleněného majetku, práce tedy bude primárně zaměřena na svěřenské fondy založené za tímto účelem, protože právě tyto svěřenské fondy se dají použít k mezigeneračnímu převodu majetku. Co se veřejného účelu týče, hlavním účelem nemůže být dosahování zisku nebo provozování závodu, svěřenský fond založený za veřejným účelem tedy slouží k veřejnému, především socioekonomickému, prospěchu, do kterého důvodová správa k § 1448 - 1452 řadí účely kulturní, vzdělávací, vědecké, náboženské a další podobné účely.\footfullcite{jaroslav_svejkovsky_sprava_2015} \\

Stejně jako o základních účelech svěřenských fondů, tedy veřejném a soukromém, lze hovořit i o jednotlivých typech svěřenských fondů používaných v praxi. Pro lepší pochopení je vhodné začít s trustem v common law. Trust jakožto velice flexibilní právní institut by měl mít systém, který dovoluje jisté setřídění individuálních trustů dle jejich účelu a dle potřeb, ke kterým jsou tyto jednotlivé skupiny využívané. Tento systém umožňuje rozdílný přístup k individuálním skupinám lidí jak ze strany úřadů, tak i v rámci samotné funkčnosti trustu. Toto rozdělení trustů je také vhodné pro účely zakladatele (settlor), správce (trustee) a na konec i obmyšleného (beneficiary), neboť jim poskytuje jistou strukturu, které zjednodušuje rozhodovací proces související s volbou vhodného trustu.\\

Inspiraci je vhodné čerpat například v Anglii, kde trusty mají sofistikovaný systém, jehož jednotlivé části jsou vhodné k různým potřebám a kde slouží jako jeden z pěti možných způsobů jak držet majetek\footfullcite{andrew_burrows_english_2013}.\\

\begin{figure}[h]
\centering
\includegraphics[width=15cm,height=5cm,keepaspectratio]{English_UK_Structure.png}
\caption{Struktura trust systému v Anglii}
\label{fig:struktura}
\end{figure}

Krom výše načrtnuté struktury je možné trusty dělit dle dalších kategorií, living, nebo testamentary, revocable, či irrevocable, funded, či unfunded a dále taky je možné rozdělení dle typů trustů, které jsou používány v praxi. Zde se může jednat například o Insurance Trust, A Spendthrift Trust, Blind Trust a podobně. Vzhledem k tématu práce není třeba jít v tomto směru do detailu, představení tohoto rozdělení v rámci Anglického trustu je důležité k představení typů svěřenských fondů v českém právu.\\

Obdobné členění trustů jako v CCQ lze pozorovat i v české praxi.

Následující kapitoly možná budou působit vzhledem k názvu a zaměření práce poněkud extenzivně, ale považuji za důležité zaměřit se v mé práci i na historii svěřenských institutů a správy cizího majetku, neboť k pochopení, vysvětlení a popsání mnoha sporných otázek, které si dnes mnozí právníci pokládají, s ohledem na svěřenské fondy, je nutné znát historické souvislosti a kontext vývoje právní vědy na našem území, neboť současná právní úprava se těmito tradičními právními předpisy v značné míře inspirovala\footfullcite[str. XIX-XXI]{svestka_j_obcansky_2014}, proto je třeba se k zodpovězení těchto otázek občas vydat až k samotným kořenům tohoto institutu. Toto bude důležité především v částech zaměřujících se na rozdíly v pojetí vlastnictví a v otázkách dědického práva. \\

\newpage

\section{Správa cizího majetku - historie}
\subsection{Antický řím}

Počátky svěřenských institutů, které dnes známe v podobě trustů, svěřenských fondů, či jiných obdobných institutů\footnote{Především instituty jiných evropských zemí, které pojali institut trustu odlišným způsobem}, je možné pozorovat již za dob Antického Říma. K pochopení, co vedlo ke vzniku těchto institutů a s nimi spojených problémů, jejichž následky můžeme pozorovat i v dnešní době, bude následující kapitola obsahovat popis systému práva a dědického práva Antického Říma, následovat bude představení svěřenských institutů, respektive fiduciárních smluv, vzniklých v této době a následně bude popsán jejich vývoj s ohledem na změny právního systému Říma.

\subsubsection{Praetorské právo a Ius Civile}

K pochopení dvoukolejnosti římského dědického práva je v první řadě třeba vymezit rozdíly mezi \textit{ius civile} a \textit{ius honorarium}, které se v rámci dědického práva lišily způsoby stanovování dědické posloupnosti a přístupu k dědickému řízení. Pro celkové pochopení struktury Římského práva soukromého je vhodné zmínit a popsat i další části, kterými jsou \textit{ius gentium} a \textit{ius naturale}. \\

Soukromé právo se sestávalo ze 4 částí, pro jejich vysvětlení je vhodné zkombinovat práci dvou významných Římských právníků Ulpiána a Gaiuse. Můžeme přitom vycházet z \textit{Corpus Juris Civilis}, jakožto souhrného občanského zákonníku vydaného ve Východořímské říši za vlády císaře Justiniána a specificky ze sbírky děl římských právnků s názvem \textit{Digesta} vydaného roku 530 našeho letopočtu. \\

Ulpián píše následující: \textit{Ius naturale est, quod natura omnia animalia docuit: nam ius istud non humani generis proprium, sed omnium animalium, quae in terra, quae in mari nascuntur, avium quoque commune est}, což lze volně přeložit jako: "Přírodní řád je řádem, kterůmu příroda učí všem živým tvorům, tento řád nejen že není člověku cizí, ale ovlivňuje všechny tvory ať již pochází ze země, moře, či jsou ptáky."\footfullcite[Dig. 1.1.1.3, Ulpianus 1 inst., vlastní překlad]{noauthor_digest_nodate}\textsuperscript{,}\footfullcite[Použito k překladu]{noauthor_digest_book1} \\

Ulpián dále říká: \textit{privatum ius tripertitum est: collectum etenim est ex naturalibus praeceptis aut gentium aut civilibus}, což lze volně přeložit jako: "Soukromé právo skládá se ze tří částí, neboť odvozeno jest buď z příkazů přírodních, příkazů národů, nebo těch vycházejíc z práva civilního."\footfullcite[Dig. 1.1.1.2, Ulpianus 1 inst., vlastní překlad]{noauthor_digest_nodate}\textsuperscript{,}\footfullcite[Použito k překladu]{noauthor_digest_nodate} \\

Gaius v knize první (Dig. 1.1.9., Gaius 1 inst.) soukromé právo rozděluje obdobě na právo civilní a právo národů: \textit{Omnes populi, qui legibus et moribus reguntur, partim suo proprio, partim communi omnium hominum.}, což lze volně přeložit jako: "Všechny národy, které se řídí zvyky a právem částečně používají práva svého a práva národů", tedy \textit{ius civile} a \textit{ius gentium}. \\

\newpage

Poslední součást Římského soukromého práva tvoří tvz. \textit{ius honorarium}, tedy právo tvořené Praetory. \\

Dle informací výše uvedených lze Římské soukromé právo rozdělit následovně:

\vspace{5 mm}

\begin{itemize}
\item \textit{ius civile},
\item \textit{ius honorarium},
\item \textit{ius gentium},
\item \textit{ius naturale}.
\end{itemize}

\vspace{5 mm}

Struktura římského soukromého práva, s ohledem na popis v této kapitole, by se dala vizuálně vyzobrazit takto, pro popis fungování dědického práva je důležité především \textit{ius civile} a \textit{ius honorarium}:

\begin{figure}[h]
\centering
\includegraphics[width=15cm,height=5cm,keepaspectratio]{rimskepravostruktura.jpeg}
\caption{Struktura soukromého římského práva}
\label{fig:struktura}
\end{figure}

\newpage

\underline{\textbf{\textit{Ius civile}}} je právo, jehož subjekty jsou Římští občané, původně vycházelo z obyčejů. Nejstarší \textit{ius scriptum}\footnote{psané právo}, vykládající ius civile, který je nám znám je \textbf{Zákon \MakeUppercase{\romannumeral 12} desek}. Dle několika dochovaných informací potenciálně existovala psaná úprava již na samém počátku republiky ve formě sbírky \textit{leges regiae}\footnote{zákony vydávané Římskými krály}, která byla kodifikována do takzvaného \textit{ius civile Papirianum}\footfullcite[str.31]{mousourakis_roman_2014}. \textit{Ius civile bylo} tvořeno pontifikálními interpretacemi\footnote{zákony tvořeny Pontifiky, kteří vykládali právo} a komiciálními zákony\footnote{zákony tvořeny lidovým shromážděním} a v době císařství císařskými konstitucemi\footnote{zákony tvořeny císařem}. Problém \textit{ius civile} však spočíval v jeho nepružnosti, která se projevovala špatným přizpůsobováním změnám ve společnosti a vývoji doby, a jednoduchého a striktního určení některých zásad\footfullcite[kapitola 1, str.3]{bednarikova_barbora_sverenske_2014}. \\

%Dle několika informací, doplnit zdroj

\underline{\textbf{\textit{Ius honorario}}}, neboli praetorské právo, bylo tvořeno vysokým Římským úředníkem zvaným \textit{Praetor}. Úřad praetora byl vytvořen roku 367 před naším letopočtem Liciniovým zákonem\footfullcite[str.35]{blaho_peter_haramia_ivan_a_zidlicka_michaela_zaklady_1997}. Praetor řídil soudní procesy a za pomoci \textit{aequitas}\footnote{spravedlnost a slušnost} a \textit{bona fide}\footnote{dobrá víra, dobrý úmysl} mohl ovlivňovat a řídit soudní řízení a tím i vytvářet, respektive nalézat, \textit{ius honorarium}, jež rozvíjelo neobratné a zastaralé principy civilního práva, ze kterého vycházel a jehož principy vztahoval na skutkové vztahy, které původní \textit{ius civile} neupravovalo. V situacích, kdy se pro svůj nedostatek formy a obsahu z \textit{ius civile} vycházet nedalo, mohl praetor vytvořit hypotézu skutkové podstaty, která když je naplněna, má být něco vykonáno, například má být žalovaný odsouzen. Praetor zároveň v řízení účastníkům umožnil podávat námitku (\textit{exceptio}), na kterou dle civilního práva nebyl brán zřetel, nicméně pokud se v rámci \textit{ius honorario} daná námitka prokázala, brala se v úvahu. \\

Praetor zároveň vydával na začátku svého jednoročního funkčního období takzvaný edikt, který fakticky sloužil jako soupis honorárního práva. 

Doplnit další informace. \\

\newpage

\underline{\textbf{\textit{Ius gentium}}} bylo. \\

\underline{\textbf{\textit{Ius naturale}}} bylo. \\

\newpage

\subsubsection{Dědické právo v dobách antického Říma}

Po vyjasnění struktury Římského práva je tedy možné přesunout se přímo k popisu způsobu fungování Římského dědického práva. Tato část je důležitá pro historické pochopení vzniku svěřenských institutů ve starověkém Říme a pokračování dalšího vývoje těchto institutů, které mají svůj prvopočátek právě v Římské říši. \\

Dědické právo v Říme, vycházelo z principu univerzální sukcese\footnote{Do dědictví spadá jmění, tedy jak majetek, tak i závazky}\textsuperscript{,} \footfullcite{blaho_peter_haramia_ivan_a_zidlicka_michaela_zaklady_1997}, což sehrálo spolu s pasivní dědickou legitimací významnou roli při tvorbě alternativních způsobů odkazu majetku zůstavitele. \\

Po zůstavitelově smrti šlo dědit na základě zákona, nebo na základě testamentu, toto zároveň určovalo i dědickou posloupnost. \\

\vspace{5 mm}

Římské právo tedy rozlišovalo dvě posloupnosti:
\begin{enumerate}
\item \textit{hereditas testamentaria} - dědická posloupnost na základě testamentu,
\item \textit{hereditas legitima} - dědická posloupnost na základě zákonu.
\end{enumerate}

\vspace{5 mm}

Posloupnost zakládající se na testamentu, tedy jednostranného právního dokumentu ustanovujícího zůstavitelovi dědice, měla přednost před posloupností zákonnou. Nevyčerpání celého dědictví ze strany hereditas testamentaria, tedy nemohlo založit právní důvod pro delaci\footnote{povolání} intestátních dědiců\footnote{dědicové dle hereditas legitima}, v takovém případě dědil i zbytek pozůstalosti dědic povolaný na základě testamentu. \\

Přes fakt, že dědění na základě testamentu mělo přednost, znalo Římské právo i institut nepominutelného dědice, který se ve svých prvopočátcích vymáhal takzvanou kverelou, tedy žalobou před soudem na zrušení testamentu. Až v pozdější době císařské měli nepominutelní dědici při jejich opomenutí zůstavitelem v závěti právo na dorovnání svého zákonného podílu bez nutnosti rušit testament\footfullcite[3. vydání z roku 2012, str.230]{marek_karel_redakcni_rada_casopis_2012}. Povinnost zohlednit nepominutelného dědice se mohl zůstavitel zprostit vyděděním daného dědice. Z počátku nemusel zůstavitel uvádět důvod, toto bylo změněno vydáním Justiniánských reforem, které taxativně vymezili důvod pro vydědění nepominutelného dědice \footfullcite[str.147]{blaho_peter_haramia_ivan_a_zidlicka_michaela_zaklady_1997} . \\

\newpage

Římské právo dále rozlišovalo právní postavení dědiců ve smylu povinnosti dědit na dvě posloupnosti \footfullcite[str.141]{blaho_peter_haramia_ivan_a_zidlicka_michaela_zaklady_1997}:

\vspace{5 mm}

%Římské právo tedy rozlišovalo dvě posloupnosti \footfullcite[str.141]{blaho_peter_haramia_ivan_a_zidlicka_michaela_zaklady_1997}:
\begin{enumerate}
\item \textit{heredes voluntarii} - dobrovolní dědicové, jsou povoláni k dědictví projevem vůle nebo zachováním se ve smyslu přijetí - tedy například uhrazení pohledávky zůstavitele,
\item \textit{heredes necessarii} - nutní dědicové, jsou povoláni k dědictví smrtí zůstavitele bez možnosti odmítnutí dědictví, tato skupina se dále dělí na další dvě podskupiny:
\begin{itemize}
\item \textit{heredes sui et necessarii} - dědici nutní a vlastní, římská rodina byla silně patriarchálního charakteru a vztahy se rozlišovaly na takzvané agnátské a kognátské. Paterfamilias, tedy otec rodiny měl moc na členy jeho rodiny. Tito členové, kteří v době zůstavitelovi smrti spadali pod jeho moc se stali členy této podskupiny. Do vzniku \textit{beneficium abstinendi} v rámci praetorského práva něměli možnost dědictví odmítnout.
\item \textit{heredes necessarii} - jedná se o otroky, kteří byli v testamentu zároveň osvobozeni, neměli možnost dědictví odmítnout.
\end{itemize}
\end{enumerate}

\vspace{5 mm}

Pokud byl v testamentu jako dědic označen otrok v moci jiného římského občana, stal se dědicem onen římský občan. \todo{Dohledat zdroj a najít vhodné místo k doplnění v textu} \\

V rámci praetorského práva vyhradil dědicům praetor takzvané beneficium abstinenti, které jim umožňovalo vzdát se dědictví a přesto, že stále byli dědici dle ius civile, zabraňovalo také možnosti podávání žaloby zůstavitelových věřitelů proti těmto dědicům.

\newpage

V rámci Římského práva a zvyklostí bylo možné přenechat určitou část majetku někomu, kdo nebyl dědicem, formou takzvaného legata (v překladu odkazu). Tento způsob byl nicméně zatížen formálními požadavky na jeho formu, z tohoto důvodu se začaly vyčleňovat i jiné postupy převodu majetku, které nebyly zatíženy takovou formálností\footfullcite{noauthor_fideicommissum_nodate}. Jeden z těchto postupů byl již výše zmíněný fideikomis, jehož splnění bylo ve svých prvopočátcích svěřeno dědici čistě na základě důvěry. Faktem však nadále zůstávalo, že fideikomis nebyl právně vymahatelný a tím pádem nebylo možné splnění zůstavitelova přání požadovat formou žaloby.

\subsubsection{Fideikomis v Antickém Římě}

V důsledku faktických omezení kladených na \textit{legatum} se tedy, již za dob Římské republiky, začal vyvíjet i jiný institut k účelu převodu majetku po smrti zůstavitele, který byl potřebný pro specifické situace v rámci dědění. Tímto institutem byl takzvaný \textit{fideikomis}, neboli fiduciární smlouvy. Podstata \textit{fideikomisu}, na který lze nahlížet jako otce současně používaného trustu a obdobných fiduciárních institutů používaných jako suplementy nebo substituty klasického dědění, respektive převodu majetku, spočívala v tom, že zůstavitel žádal jím určenou osobu (především univerzálního dědice), aby po jeho smrti naplnil určité zůstavitelovo přání. Toto přání spočívalo nejčastěji v převodu části zůstavitelova majetku na jinou osobu, než na určené dědice. Jak již je poznamenáno v úvodu, slovo \textit{fideicomissum} se dá volně přeložit jako "tvé důvěře svěřuji", toto označení bylo výstižné, protože tato žádast se opravdu zakládala čistě na důvěře mezi zůstavitelem a osobou jím určenou. Fideikomisář (rozuměj osoba v jejíž prospěch má být plněno) se tedy nemohl žalobou domáhat svého práva. Tato žádost se nejčastěji připojovala ke kodicilu. \\

Kodicil tvořil v této době společně s legatem a fideikomisem 

\newpage

\subsection{Historie svěřenských fondů u nás a ve světě}

Co se historie svěřenský fondů týče, tak \footnote[4]{Zde bude nějaká informace}\\
---Koment\\
Doplnit úpravu za dob království/monarchie a první republiky a vývoj svěřenských institutů jinde ve světě, držet se struktury popsané v druhém paragrafu v kapitole druhé, to znamená popsat i vznik trustu.
Dále popsat zrušení svěřenských institutů v období první republiky a úplné zrušení po druhé světové válce.\\
---\\
\subsection{Recepce právní úpravy Svěřenského fondu z právního řádu provincie Quebec}

---Koment\\
 Následně popsat znovuzavedení těchto institutů v roce 2014, převzetí z Quebecké právní úpravy a popsat i novelu z roku 2018 - na základě stížnosti ministerstev a vrchního státního zastupitelství, odkážu poté na další kapitolu věnující se této novelizaci.
 
 Tyto instituty v mnohém navazují na starší instituty používané již za dob antického říma a později v Česku, Československu, jejich tradice a dále na to dobré, na co se ohlížíme.
 
 Existují i právní úpravy, respektive země, které kontinuálně využívali tohoto institutu.\\
 
 Pro další pochopení takto flexibilního institutu a možnosti z něho vycházejících pro převody rodiného majetku je nejprve nutné vysvětlit dvě věci. První z těchto věcí je samotná úprava svěřenského fondu, kterou obsahuje kapitola následující, věcí druhou je pak samotný základ institutu správy cizího majetku a trustu, tedy jeho struktura a základní premisy a historické souvislosti vedoucí až k podobě tohoto institutu tak, jak jí známe dnes, ať již se jedná o historické souvislosti sahající až k dobám Římské říše, nebo znovupoložení si otázek, které si kladli slavní právníci 20. století a snahu zodpovědět je.\\
 ---\\
 %Tento odstavec podle mě nedává smysl, bude nutné ho mírně upravit, chci v něm říct to, že jsem v předchozích kapitolách vyložil historické souvisloti a v další kapitole popíšu právní úpravu. S touto přípravou je možné zaměřit se na možnosti svěřenského fondu v rámci mezigeneračního majetku a i na mojí výzkumnou otázku.

%\newpage
%\thispagestyle{smallertextinheader}

%\section{Pojem svěřenský fond, úvod do problematiky}

%Text.

\newpage
\section{Dědické právo v České republice}

Pro komparaci právní úpravy svěřenského fondu s právní úpravou dědického práva je v první řadě důležité popsat jakým způsobem v České republice funguje dědické právo. Dědické právo je obecně upraveno převážně v paragrafech 1475 až 1720, přičemž důležitou součást tvoří Díl 5, který upravuje institut nepominutelného dědice. \\

V rámci české právní úpravy vzniká dědické právo smrtí zůstavitele, ten kdo zemře společně se zůstavitelem, nebo před ním, není způsobilý dědit\footfullcite[§ 1479]{noauthor_zakon_2012}. S ohledem na toto je vhodné poukázat na formální požadavky právní úpravy na zůstavitele a dědice. Nejobecnějším formálním požadavkem kladeným na zůstavitele, je pořizovací způsobilost, respektive nezpůsobilost. Obecně zákon neklade žádné překážky tomu, aby bylo jmění zůstavitele součástí dědického řízení, a aby tak mohlo přejít na jeho dědice. Je nicméně omezen způsob, který může zůstavitel použít k vyjádření své vůle o tom, jakým způsobem má toto převedení proběhnout. To znamená, že zůstavitel je omezen ve způsobech, kterými může pořizovat o svém majetku pokud byl nezletilcem, který nedovršil věku patnácti let, popřípadě pokud by byl omezen ve svéprávnosti\footfullcite[§ 1525 až § 1528]{noauthor_zakon_2012}. Tyto ustanovení dopadají pouze na pořízení pro případ smrti, dědění na základě dědického titulu ze zákona tímto není dotčeno. \\

Množina požadavků kladených na jednotlivé dědice, jinak řečeno tedy náležitosti, které dědicové musí splňovat, aby mohli býti dědicky způsobilí, je z rozhodnutí zákonodárce širší. Ve chvíli, kdy tyto požadavky nejsou naplněny nebo je porušena nějaká povinnost kladená na dědice, není umožněno danému dědici nabýt celou pozůstalost, nebo její část. \\

Z tohoto vyplývá, že aby byl dědic způsobilý dědit, musí splňovat jak pasivní, tak i aktivní podmínky na něj zákonem kladené. Aktivní podmínky, které musí dědic splňovat jsou vymezeny v §1481-1482 Občanského zákoníku (Dědická nezpůsobilost), aby dědici svědčila dědická způsobilost, nesmí se tedy dopustit ani jednoho z vymezených jednání. Pokud se ho dopustí, může být ještě způsobilý dědit, pokud mu zůstavitel dané jednání výslovně odpustí. A podmínky pasivní, které ve své podstatě spočívají pouze v tom, že dědic musí být buď právnická, nebo fyzická osoba. Tato pasivní podmínka je nesmírně důležitá ve vztahu ke svěřenským fondům, neboď svěřenský fond není považován ani za jednu z těchto osob.

\newpage

Jako jistou podskupinu k aktivním podmínkám bych dále zařadil i ostatní skutečnosti vylučující dědice z dědického práva, které mohou nastat na straně dědice, tedy již uvedená dědická nezpůsobilost, či další jednání na straně dědice, tak i na straně zůstavitele. Co se tedy tohoto vyloučení dědice z dědického práva v rámci aktivní podskupiny týče, Občanský zákoník taxativně vymezuje následující důvody k němu vedoucí:

\begin{figure}[h]
\centering
\includegraphics[width=12cm,height=4cm,keepaspectratio]{Dedicka_nezpusobilost.png}
\caption[Vyloučení dědice z dědického práva]{Vyloučení dědice z dědického práva\footfullcite{noauthor_informacni_nodate}}
\label{fig:komparace}
\end{figure}

Vyloučení z dědického práva se obecně dá rozdělit na důvody závislé na jednání dědice a nezávislé na jednání dědice. Do první skupiny patří ve své podstatě dědická nezpůsobilost, zřeknutí, odmutnutí, vzdání se a zcizení dědictví. Do druhé skupiny patří vydědění. \\

\noindent\textbf{Jednotlivé důvody vylučující dědice z dědického práva:} \\

Dědická nezpůsobilost je upravena v § 1481 až § 1483, jedná o případy, ve kterých se dědic dopustil úmyslného trestného činu proti zůstavitelovi, nebo jeho blízkým, popřípadě pokud se dopustil zavrženíhodného činu proti zůstavitelově poslední vůli, spočívající v lstivém svedení k projevu poslední vůle, popřípadě donucení k tomuto projevu, překažení projevu, zatajení, falšování a podvrhnutí nebo zničení pořízení. Dědic se i přes toto stává způsobilým, pokud mu zůstavitel výslovně tento čin prominul. \\

Zřeknutí se dědického práva je upraveno v § 1484 a stanovuje, že se dědic může předem zříci dědického práva, nebo práva na povinný díl smlouvou se zůstavitelem. Tato smlouva může působit jak proti samotnému dědici, tak i proti jeho potomkům, jedná se o dispozitivní ustanovení občanského zákona a záleží tedy na vůli smluvních stran, aby si mezi sebou toto upravili. Smlouva musí být uzavřena formou veřejné listiny. \\

\newpage
\thispagestyle{smallertextinheader}

\section{Svěřenský fond jako nástroj mezigeneračního transferu majetku}

--Koment\\
Popsat jak je v momentální právní úpravě vyložen institut svěřenského fondu a jaké možnosti poskytuje. \\
--\\

Česká právní úprava zakládá mnoho možností, které lze použít k převedení majetku. Těmito mohou býti klasické aktivní způsoby běžně využívané na denní bázi všemi z nás, zahrnující například koupi, darování či výměnu. Lze nicméně použít i méně častých možností jako vydržení, nebo i zakládání právnických osob sloužících za účelem převodu majetku jako je například nadace. Troufám si však tvrdit, že nejčastější jsou způsoby upravené v rámci dědického práva počínajících od nabytí majetku na základě dědického titulu vyplývajícího ze zákona, tedy v případech kdy zůstavitel zůstal za svého života nečinný a neupravil způsob, jakým má být s jeho jměním naloženo po jeho smrti, nebo na základě dědických titulů vycházejících z pořízení pro případ smrti, tedy z právního jednání aktivně učiněného za života zůstavitele. \\

Způsoby, kterými lze dosáhnout převedení majetku jsou tedy dva, aktivní jednání zůstavitele, nebo ponechání převodu majetku na zákoně a na tom jak jsou dědicové schopni se mezi sebou dohodnout, tedy určitá pasivnost zůstavitele. Dá se říci, že s ohledem na převáděné jmění a na budoucí nakládání s takto převedeným jměním, je více než vhodné, aby zůstavitel upravil tyto záležitosti již za svého života. Může tím totiž i nadále v uvozovkách vykonávat správu nad svým jměním, respektive může určit jak s jeho jměním má být po jeho smrti naloženo, takovéto jednání je vhodné zejména s ohledem na mezilidské vtahy v rodině, ochránění jmění před neuvážlivým utrácením ze strany dědiců, ochraně před rozdrobením majetku a mnoha dalšími potenciálními nepříznivými dopady převedení majetku. S ohledem na toto primárně přichází v úvahu způsoby, které jsou upravené v rámci dědického práva, v očích běžného občana tímto bude především závěť, v očích těch znalejších to může být i dědická smlouva, dovětek, nebo odkaz. Tyto způsoby však mají i svá specifická omezení a ve svém důsledku spočívají v konečném převedení jmění na jednotlivé dědice, nebo odkazovníky. Zůstavitel také nemá tak širokou možnost stanovit svým dědicům růzce podmínky a to i přes existenci dovětku. Z tohoto důvodu nemusí být vhodné jejich použití ve všech případech. \\

%Porovnat a popsat možnosti, které má zůstavitel v dovětku a ve svěřenském fondu.

\newpage
\thispagestyle{smallertextinheader}

Jako jistá alternativa k těmto klasickým způsobům byla do české právní úpravy transplantována úprava trustu z Quebeckého občanského zákoníku, tento poskytuje zůstavitelům další možnost, jak naložit se svým jměním již za svého života (založení svěřenského fondu \textit{inter vivos}), popřípadě po své smrti (založení svěřernského fondu \textit{mortis causa}). Institut svěřenského fondu přináší mnoho možností, které zůstavitel před jeho zavedením neměl, ale zároveň do českého práva vnáší i spoustu otázek, které je potřeba pro jeho správné používání a ochranu zůčastněných stran zodpovědět. V následujících kapitolách proto představím způsoby, kterými lze svěřenský fond použít pro mezigenerační převod majetku, představím jeho přínosy a benefity, ale zároveň se budu snažit zodpovědět otázky týkající se nejistého výkladu právní úpravy svěřenského fondu, jeho potenciálních problémů a možností zneužití. \\

\newpage

\thispagestyle{smallertextinheader}

\subsection{Svěřenský fond v České republice}

--- Toto asi odstraním\\
Pro další pochopení svěřenského fondu a jeho možností, problémů, výhod, nevýhod a nejasností je v samém úvodu této kapitoly nutné čtenáře obeznámit s právní úpravou svěřenského fondu v rámci českého práva. \\
---\\

Definovat správně svěřenský fond, jakožto velice flexibilní právní institut, je poměrně složité a až doposud, tedy i relativně dlouhou dobu po jeho zavedení, existuje spousta protichůdných názorů (například v rámci lingvistického výkladu jednotlivých ustanovení, definicí majetku nebo koexistenci právního institutu přejatého z právního systému Common Law s instituty vlastní kontinentálním právním systémům, jako je například institut nepominutelného dědice) na tento institut, který v době svého přijetí do naší, tedy kontinentální, právní úpravy vyvolal poměrně mnoho emocí, od mnohých právních expertů se dočkal dílem chladného přijetí a od mnohých dílem přímé kritiky. Ať již nahlížíme na svěřenský fond jakkoliv, je zřejmé, že pro jednotlivé subjekty práva a pro jednotlivé účely, ke kterým je možné jej užít, skýtá i velké výhody. S ohledem na název bakalářské práce bude hlavní těžiště této práce v rámci praktické části spočívat především v představení těchto nových možností a výhod v procesu převádění rodinného majetku na další generace a jeho uchování jako alternativní, nebo rozšiřující prostředky ke stávajícím možnostem mezigeneračního převodu majetku, nebo převodu majetku obecně. \\
% Dopsat někde de lege lata
% Dopsat alterace a modifikace testamentu
% Nevystupuje jako subjekt ale spíše jako objekt právních vztahů
% Jelikož nelze svěřenský fond zařadit mezi právnické osoby, není možné na něj aplikovat ustanovení o právnických osobách v procesních, daňových, trestních a dalších předpisech. Zákonodárce proto musel tyto právní předpisy novelizovat s ohledem na úpravu svěřenských fondů. Důsledky zařazení svěřenského fondu do českého právního řádu tak lze pozorovat v právních předpisech napříč celým právním systémem


V rámci současné právní úpravy je svěřenský fond upravený především v části třetí OZ, hlavě \MakeUppercase{{\romannumeral 2}} - věcná práva v § 1448 až § 1474. \\

%Doplnit pojem svěřenský fond, inspirace z práce Lukáš Kolenský Obchodní korporace v.svěřenský fondz pohledu věřitele

Věci ve svěřenském fondu náležejí svému účelu\footfullcite{svejkovsky_jaroslav_sverenske_2018}.\\

Úprava institutu svěřenského fondu a obecná úprava správy cizího majetku vstoupila v účinnost 1. ledna 2014 jako součást Nového Občanského zákoníku. Svěřenský fond je speciálně upraven v rámci ustanovení paragrafů 1448 až 1474, které jsou zařazeny do oddílu 4, jež je součástí dílu 6, Správa cizího majetku, spadající do části třetí, hlavy druhé, Věcná práva. \\

Úprava svěřenského fondu jest úpravou speciální, která doplňuje úpravu generální obsaženou v ustanoveních 1400 až 1447, Správa cizího majetku. Český zákonodárce se inspiropval, jak již je vysvětleno v kapitole Historie \footnote{Správa cizího majetku - historie; Recepce právní úpravy svěřenských fondů}, Občanským zákoníkem z Quebecu ("CCQ"), z kterého ve své podstatě transplantoval právní úpravu svěřenského fondu, která, přes určité výkladové změny \footnote{například v pojmech majetek a jmění, viz Svěřenský fond a trust - Jejich fungování v mezinárodním srovnání od Luboše Tichého}, do značné míry zachovala původní znění úpravy trustu tak, jak jej známe z CCQ, jak dokládá Svejkovský, Marek a kol. v díle Správa cizího majetku v novém občanském zákoníku v části Srovnání právní úpravy ObčZ a CCQ se stručným popisem rozdílů. 

\newpage
\thispagestyle{smallertextinheader}

Důvod, pro který si Český zákonodárce vybral jako zdroj inspirace pro zavedení trustu do našeho právního řádu práve CCQ je ten, že Quebecký Občanský zákoník si zachoval výrazný charakter kontinentálního práva\footfullcite{Duvodovazprava}, ačkoli je obklopena státy s právním systémem Common law\footnote{Výjimku představují například DOPLNIT}.\\

Výhodou je i vysoká flexibilita svěřenského fondu. Ve statutu lze do značné míry podle potřeby upravit vnitřní fungování svěřenského fondu podle představ zakladatele - např. počet svěřenských správců a obmyšlených, a způsob jejich jmenování, pravidla výplaty plnění obmyšleným, pravidla správy apod.

%Tento paragraf bude ještě potřeba přepsat, aby pasoval do této kapitoly.

\subsubsection{Svěřenský fond a právní subjektivita}

Důležitou součástí definice svěřenského fondu je fakt, že svěřenský fond je koncipován jako oddělený majetek, který je svěřen svému účelu, jež nemá vlastníka a nemá ani právní subjektivitu. Přes jasnou definici, kterou obsahuje Občanský zákoník, důvodová zpráva a komentář k občanskému zákoníku a i samotný zdroj inspirace právní úpravy svěřenského fondu, tedy CCQ, není však naškodu podrobit svěřenský fond jisté komparaci s právnickou osobou. Není totiž tajemstvím, že určité právní normy na svěřenský fond pohlížejí jako na právnickou osobu\footfullcite[§17]{noauthor_zakon_zdp}\textsuperscript{,}\footfullcite[§4b]{noauthor_zakon_zdph} a právě tato tendence k subjektivizaci svěřenského fondu by se mohla projevit i v jiných aspektech svěřenského fondu. DOPLNIT

Brilantní náhled na toto téma již vyslovil Michal Skuhrovec ve své diplomové práci, která z tohoto hlediska poskytuje vhodný náhled na tuto otázku a také patřičný úvod do této problematiky.

%Doplnit do začátku práce, že jsem se touto prací hodně inspiroval, ale myslím si, že neposkytuje odpovědi na všechny otázky a v určitých částech není aktuální, nebo odpovídající, doplnil bych jí tedy o svůj pohled, nicméně některé pohledy autora jsou výtečné a já se s nimi jen ztotožňuji, stejně tak jako se s nimi ztotožňují i pohledy právních expertů. Je nicméně potřeba jeho pohled rozšířit i o nové poznatky v souvislosti s institutem svěřenského fondu

%Na tomto místě je vhodné uvést dvě základní premisy související se svěřenským fondem, které budou v bližším detailu rozvedeny dále. Povaha svěřenského fondu předně vylučuje, aby měl svěřenský fond právní osobnost (svěřenský fond tedy není právnickou osobou) a stejně tak vylučuje, aby byl svěřenský fond věcí v právním slova smyslu. Uvedenému přirozeně neodporují závěry některých autorů, kteří nevylučují, aby měl majetek vložený do svěřenského fondu povahu věci hromadné. 2 HORN, K.: Podrobněji k svěřenskému fondu. Ad notam, č. 6, ročník 2014, s. 1.

%Lze nicméně uvést, že svěřenský fond představuje naprosto ojedinělou a zvláštní entitu, která se skládá z (bývalého) majetku zakladatele a v souvislosti s kterou je prostřednictvím činnosti správce svěřenského fondu naplňován zakladatelem určený cíl. Přitom základním cílem společným pro všechny svěřenské fondy je nepochybně ochrana majetku zejména před nežádoucími dispozicemi 260 260 AKONTinfo, Trust jako nejúčinnější nástroj ochrany majetku. [online]. 2010. On-line zdroj: http://www.akont.cz/cz/375.trust-jako-nejucinnejsi-nastroj-ochrany-majetku.

\subsection{Novelizace právní úpravy - zápis svěřenských fondů do rejstříku}

%Zákon č. 33/2020 Sb. novela, se svěřenských fondů téměř vůbec nedotýká, pro tuto práci není důležitá, neboť se měni jen určité ustanovení ve vztahu k notářskému zápisu.

Rozepsaná novelizace svěřenský fondů z roku 2018, zhodnotit, zda přispělo ke zmírnění možností využití svěřenských fondů jako prostředků k legalizaci výnosu z trestných činností.\\

Poměrně zásaní změny v souvislosti se zakládáním svěřenských fondů přinesla novelizace č.z. 460/2006, která s sebou přinesla nutnost zápisu svěřenského fondu do evidence s tím, že tento zápis je konstitutivní. Svěřenský fond tedy vzniká tímto zápisem. Toto se nicméně dotklo pouze svěřenského fondu založeného mezi živými. Pro svěřenský fond založený pro případ smrti zákonodárce stanovil v § 1451 odstavci 3 následující:\\

\textit{"Byl-li však svěřesnký fond zřízen pořízením pro případ smrti, vznikne smrtí zůstavitele. Do evidence svěřenských fondů se zapíše po svém vzniku."}\\

Důvod, pro který zákonodárce dopřál toto, do jisté míry zvýhodňující, postavení způsobu založení svěřenského fondu \textit{mortis causa}, je v důvodové zprávě objasněno následovně:\\

\textit{"Odlišný princip deklaratorního zápisu se má uplatnit u svěřenských fondů, které byly zřízeny pořízením pro případ smrti. Tento princip lépe vyhovuje fundamentu testovací volnosti a respektu k vůli zůstavitele."}

\newpage
\thispagestyle{smallertextinheader}

\subsection{Založení svěřenského fondu}

S ohledem na téma práce, tedy mezigenerační převod majetku, je otázka založení svěřenského fondu, ať již mezi živými, tedy \textit{inter vivos}, či pořízením pro případ smrti \textit{mortis causa}, nebo velice specifickou možností založení svěřenského fondu \textit{inter vivos} s odkládací podmínkou smrti zakladatele a dalších naležitostí s tímto spojených, integrální součástí této práce a je nezbytné k dalšímu zkoumání jednotlivých možností, které nám s přijetím tohoto institutu Český zákonodárce vložil do rukou a obecně i celé problematiky spojené se svěřenským fondem. \\

Ať již se budeme bavit o jakémkoliv způsobu založení svěřenského fondu, tak zákon klade na zakladatele v paragrafech 1448 - 1474 Občanského zákoníku následující čtyři povinnosti:

\begin{enumerate}
\item Označení a vyčlenění majetku k určitému účelu,
\item statut,
\item přijetí správy svěřenského fondu ze strany svěřenského správce,
\item zápis svěřenského fondu do evidence.
\end{enumerate}

Obecné vysvětlení k pojmu a založení svěřenského, tedy k paragrafům 1448 až 1452 Občanského zákoníku, fondu je obsaženo již v důvodové zprávě, respektive v její části vtahující se ke svěřenskýn fondům.\\

\textit{"Svěřenský fond může být zřízen jednak bezúplatně rozhodnutím zakladatele věnovat část svého majetku určitému účelu, zřízení svěřenského fondu však může být sjednáno i za úplatu nebo jiné protiplnění (např. od osoby, která se má stát obmyšleným, ale i od osoby jiné). Podle toho se pak odvíjí i právní postavení obmyšleného, jak to reflektují následující ustanovení osnovy. Osnova rozlišuje svěřenské fondy podle účelového určení na veřejně prospěšné (zřízené k naplňování účelů kulturních, vzdělávacích, vědeckých, náboženských nebo podobných) a soukromé. Důvodem vzniku svěřenského fondu může být zákon, pravidelně však smlouva nebo ustanovení závěti. Protože svěřenskému fondu musí být ustaven svěřenský správce, vyžaduje § 1448 souhlas osoby označené za svěřenského správce se svým ustavením. I to je tedy právní podmínka vzniku svěřenského fondu (§ 1451). Svěřenský fond se spravuje statutem obsaženým ve smlouvě, závěti či vydaným samostatně."}\footfullcite{prof_dr_judr_karel_elias_a_kolektiv_novy_2012}\\

S touto částí důvodové zprávy pak přímo souvisí §1448, ve kterém zákonodárce uvádí následující:\\

\textit{"§ 1448
(1) Svěřenský fond se vytváří vyčleněním majetku z vlastnictví zakladatele tak, že ten svěří správci majetek k určitému účelu smlouvou nebo pořízením pro případ smrti a svěřenský správce se zaváže tento majetek držet a spravovat.\\
(2) Vznikem svěřenského fondu vzniká oddělené a nezávislé vlastnictví vyčleněného majetku a svěřenský správce je povinen ujmout se tohoto majetku a jeho správy.\\
(3) Vlastnická práva k majetku ve svěřenském fondu vykonává vlastním jménem na účet fondu svěřenský správce; majetek ve svěřenském fondu však není ani vlastnictvím správce, ani vlastnictvím zakladatele, ani vlastnictvím osoby, které má být ze svěřenského fondu plněno."}

V uvedeném paragrafu tedy zákonodárce umožňuje pouze explicitní založení svěřenského fondu dvěma způsoby, to znamená, že svěřenský fond nemůže být založen rozhodnutím soudu tak, jako tomu je v jiných právních řádech. Jedním z těchto právních řádů je i CCQ, který umožňuje i takzvané implicitní založení svěřenského fondu, tedy založení na základě rozhodnutí soudu, nebo i ze zákona, pokud existuje zákonné zmocnění, například advokátní a notářské úschovy\footfullcite{claxton_john_b_studies_2005}. Další možností, která není explicitně uvedena v tomto paragrafu, ale je možné ji využít k založení svěřenského fondu, je založení svěřenského fondu mezi živými s odkládací podmínkou. 

Účelem se rozumí veřejný nebo soukromý.

\subsubsection{Vznik svěřenského fondu \textit{inter vivos}}

S ohledem na historické konotace svěřenských fondů popsané výše a velkou flexibilitu tohoto institutu skutečně není divu, že Český zákonodárce umožnil založení svěřenského fondu jak mezi živými, tak i pro případ smrti. Začnu s popisem založení svěřenského fondu mezi živými, neboť tento způsob založení je méně problematický než založení svěřenského fondu pro případ smrti a přináší obdobné možnosti mezigeneračního předání majetku.\\

V tomto rámci je podstatné zmínit, že se rozlišuje mezi vytvořením a vznikem svěřenského fondu, celkový proces konstituování svěřenského fondu, tedy od okamžiku samotného rozhodnutí vytvořit svěřenský fond, až k samotnému vzniku, má tedy více fází. Vytvořením se rozumí jednání jednoho, nebo i více zakladatelů\footfullcite{spacil_j_a_kolektiv_obcansky_2013}\textsuperscript{,}\footnote{S ohledem na fakt, že samotná právní úprava svěřenského fondu v §1452 odkazuje na založení nadace, je vhodné i jiné ustanovení, která se týkají nadací, využít analogicky ve vztahu ke svěřenským fondům. Jedním z těchto paragrafů je paragraf  309 odstavec 3, který stanoví, že pokud na straně zakladatelů stojí více osob, považují se obdobně za jednoho zakladatele.}, kteří vyčlení část svého majetku, popřípadě celý svůj majetek a tento poté převedou do jiného jmění, tedy svěřenského fondu, který je zřízen za určitým účelem. Toto jednání zahrnuje v první řadě řádnou nabídku, tedy ofertu, vůči jiné osobě, která by se měla stát potenciálním správcem\footfullcite{jaroslav_svejkovsky_sprava_2015}.\\

%Doplnit nějaké zdroje k založení svěřenského fondu ve více fázích

Jak již je z podstaty svěřenského fondu založeného za života zakladatele zřejmé, jedná se o minimálně o dvoustranné právní jednání, přičemž samotná akceptace této oferty\footnote{Návrh na uzavření smlouvy je dále upraven v §1731 a dalších v Občanském zákoníku} vede dle §1448 a souvisejícího §1451 Občanského zákoníku jednáním vedoucím ke zřízení, tedy vytvoření, svěřenského fondu na základě takto uzavřené smlouvy, kterou je možné uzavřít jak s, tak i bez protiplnění. Tato oferta a akceptace zahrnující uzavření smlouvy o správě by se dala považovat za první fázi založení svěřenského fondu. Druhá fáze spočívá v samotném zápisu svěřenského fondu do evidece dle §1451 Občanského zákoníku a §65a Zákona o veřejných resjtřících právnických a fyzických osob, který je konstitutivní, svěřenský fond tedy formálně vzniká až tímto zápisem.\\

Je zřejmé, že v případě založení svěřenského fondu \textit{inter vivos} je v zásadě bezpředmětné bavit se o možnosti ochrany nepominutelných dědiců. V rámci tohoto způsobu založení svěřenského fondu je nutné poznamenat, že ačkoli je možné, aby zakladatel vyčlenil svůj majetek do svěřenského fondu ještě během svého života s tím, že by tímto jednáním sledoval zájem poměrně ponížit o vyčleněný majetek hodnotu pozůstalosti a tím snížit hodnotu vyplacenou jednotlivým nepominutelným dědicům v rámci jejich povinného dědického podílu, není možné se tomuto jednání ze strany jednotlivých nepominutelných dědiců bránit ve smyslu zkrácení povinného podílu. Toto tvrzení lze doložit hned několika ustanoveními, předně Čl. 11 Listiny základních práv a svobod, která stanoví, že \textit{"Každý má právo vlastnit majetek...."}\footfullcite[čl. 11]{noauthor_zakon_lzps} a v návaznosti na toto poté Občanský zákoník v §996 \textit{"Poctiví držitel smí v mezích právního řádu věc držet a užívat ji, ba ji i zničit nebo s ní jinak nakládat..."} a v §1012 \textit{"Vlastník má právo se svým vlastnictvím v mezích právního řádu libovolně nakládat a jiné osoby z toho vyloučit...."}, z tohoto vyplývá, že vlastník může za svého života se svým majetkem jakkoliv nakládat, tedy má možnost ho zcizit. Ustanovení § 1475, které stanoví, že pozůstalost tvoří celé jmění zůstavitele už dále jen potvrzuje, že takto zcizené věci nejsou a nemohou být součástí pozůstalosti. S ohledem na fakt, že dle definice svěřenského fondu v § 1448 vzniká oddělené a nezávislé vlastnictví vyčleněného majetku a dále již tento vyčleněný majetek není ve vlastnictví zůstavitele a dále vzhledem k tomu, že toto právní jednání bylo učiněno ještě za života zakladatele, lze důvodně předpokládat, že tento majetek již nebude \textit{apriori} předmětem pozůstalosti a dědického řízení, tím pádem nemůže být ani připočten k jmění zakladatele v době smrti a tím pádem z tohoto majetku nemůže být počítát povinný díl nepominutelných dědiců.\\

%Dopsat ještě ústavu, že je chráněno právo na to, aby si majitel se svým majetkem nakládal podle své vůle.

%Doplnit konstitutivní a deklaratorní

Jako jednoduchá analogie k založení svěřenského \textit{inter vivos} a dopadu tohoto jednání na dědický podíl se může jevit jiná forma zcizení majetku, například darování, koupě, směna a další smluvní či mocenské formy zcizení věci, které působí na snížení, či zvýšení majetku zůstavitele. \\

---Koment\\
Doplnit další informace.\\
---\\
Je nicméně vhodné říci, že toto neznamená, že by dědici neměli možnost brát se o své právo jinou formou.\\
---Koment\\
Doplnit možnosti žaloby ve smyslu zneplatnění právního jednání pokud by bylo učiněno například pod tlakem, relativní a absolutní neúčinnost.\\
---\\

\newpage
\thispagestyle{smallertextinheader}

\subsubsection{Vznik svěřenského fondu \textit{mortis causa}}

Na rozdíl od svěřenského fondu založeného mezi živými, dochází ke vzniku v svěřenského fondu založeného pro případ smrti již samotnou smrtí zakladatele, nikoli tedy zápisem do evidence. Tento fakt je důležitý k pozdějšímu zkoumání, zda tento majetek spadá do pozůstalosti, nebo nikoliv.\\

Další možnost pro převod majetku představuje založení svěřenského fondu \textit{morti causa}, která vedle dědického řízení slouží jako další prostředek, který zůstavitel může, krom pořízení pro případ smrti, užít ke kontrole způsobu nakládání se svým jměním po jeho smrti. Potenciální problém se založením svěřenského fondu \textit{mortis causa} může nicméně nastat, pokud by toto založení mělo směřovat k úmyslnému zmenšení jednotlivých povinných dílů náležejících nepominutelným dědicům. V aktuální právní úpravě totiž není jasně stanoveno jakým způsobem se majetek při smrti zůstavitele, který svěřenský fond zřídil pořízením pro případ smrti do tohoto fondu vyčlení. Nejasný výklad ustanovení svěřenského fondu ve vztahu k dědickému právu představuje právní nejistotu, nejen pro dědice, která není žádoucí. \\

S ohledem na popis dědického práva vyložený v předchozí kapitole je třeba s ohledem na založení svěřenského fondu \textit{mortis causa} nejdříve vyřešit jakým způsobem probíhá jeho založení pořízením pro případ smrti. S ohledem na § 1452, odstavec 1 je po výkladové stránce zřejmé, že zákonodráce zamýšlel způsob vyčlenění majetku do svěřenského fondu ve smyslu § 311, který upravuje vyčlenění majetku do nadace, a který zároveň satnovuje v ostavci 1 následující: "(1) Při založení nadace pořízením pro případ smrti se do nadace vnáší vklad povoláním nadace za dědice nebo nařízením odkazu. V takovém případě nabývá založení nadace účinnosti smrtí zůstavitele."\footfullcite[§ 311, odst. 1]{noauthor_zakon_2012} Z tohoto ustanovení, které je možné analogicky použít na svěřenský fond, je možné dovodit úmysl zákonodárce pohlížet na majetek, který má být vyčleněn do svěřenského fondu, jako na součást pozůstalosti. Toto tvrzení svým komentářem potvrzuje i Vlastimil Pihera, který ve svém komentáři k § 1448 dovozuje, že při zřízení svěřenského fondu pořízením proo případ smrti vyčlení zakladatel (rozuměj zůstavitel) majetek ve prospěch svěřenského fondu tak, že v jeho prospěch vyhradí dědické právo, nebo nařídí odkaz\footfullcite[komentář k § 1448]{spacil_j_a_kolektiv_obcansky_2013}. \\

%Na svěřenský fond zřízený \textit{mortis causa} je v rámci občanského zákoníku kladený velký důraz, je nicméně vhodné poznamenat, ať již se jedná o svěřenský fond založený za veřejným a nebo soukromým účelem.

V rámci pohledu na právní úpravy trustů ze zahraničí je možné se setkat s dalšími druhy trustů a mnohými jinými možnosti, za kterýchžto účelem může být svěřenský fond založený, tyto se však Český zákonodárce dále rozhodl v našem občanském zákoníku neupravovat.

\newpage
\thispagestyle{smallertextinheader}

Tento majetek je dle Vlastimila Pihery a i dle mého výkladu a pochopení OZ tedy součástí pozůstalosti, tím pádem, dle § ...., který stanovuje, že pozůstalost tvoří veškeré jmění v držení zůstavitele ke dni jeho smrti, se z výše tohoto jmění spolu se započtením výše jmění, které není vyčleněno do svěřenského fondu počítají jednotlivé dědické podíly, které je správce pozůstalosti povinen převést jednotlivým nepominutelným dědicům. S ohledem na to, že další úprava upravující tetnto souběh v OZ chybí a zároveň k tomuto problému neexistuje žádná judikatura a není k němu ustálená soudní praxe.....

a výklad komentáře.

Popsat, že existují dva různé pohledy, já se ztotožňuji s druhým pohledem.

\subsubsection[Vznik svěřenského fondu \textit{inter vivos} s odkládacé podmínkou]{Vznik svěřenského fondu \textit{inter vivos} s odkládací\\ podmínkou}

\newpage
\thispagestyle{smallertextinheader}

\subsection{Potenciální možnosti použití v rámci převodu majetku a jeho přínos oproti klasickým způsobům pořízení pro případ smrti a jiným způsobům převodu práv}

%Možná toto pojmu s ohledem na jakékoliv jiné formy převodu majetku, ne jenom dědění, částečně to tak již mam, ale mohl bych to i rozšířit. Musel bych tedy tuto část rozšířit i o porovnání s těmito způsoby převodu majetku.

Nový občanský zákoník, který vstoupil v účinnost prvního ledna roku 2014 okruh nástrojů sloužících pro převody a správu majetku nadále rozšířil zavedením obecné úpravy správy cizího majetku a možnosti zřizování svěřenských fondů. S ohledem na téma práce je důležité zmínit, že tato nová právní úprava občanského práva přispěla k rozšíření možností, které lze využít při mezigeneračním převodu majetku, tedy především v případech, které byli dříve řešeny výhradně v rámci dědického řízení popřípadě pomocí darování.\\

Tyto mezigenerační převody majetku spočívají nejčastěji v situacích, kdy takovéto nakládání s majetkem je činěno ve vztahu k mladší generaci v jedné rodině, tedy především k potomkům a prapotomkům. %Lingvistický výklad slova mezigenerační v sobě nicméně nezahrnuje pouze pokrevně či právně zpřízněny generace, ale i různé demografické ročníky. Je tak klidně možné tento převod učinit 
Tento převod majetku samozřejmě nemusí být jenom mezigenerační, jedná se i o případy, kdy byl majetek převáděn manželovi, partnerovi, rodinému příslušníkovi, či kterémukoliv dalšímu příbuznému. Není nicméně vyloučeno, aby takovéto jednání spočívalo v převedení majetku jiným blízkým osobám, jako jsou například přátelé, nebo klidně i osobám zůstaviteli naprosto cizím.\\

%či k jiným blízkým osobám - přátelé, není nicméně vyloučeno, aby takovéto jednání spočívalo ve snaze převést majetek na osobu, která je zůstavitelovi zcela cizí. \\

%Toto budu asi muset ještě upravit a lépe navázat na další odstavec.

Převod majetku z účelem jeho mezigeneračního předání lze samozřejmě dosáhnout i jinými způsoby. Tyto způsoby zahrnují klasické způsoby převodu ve smyslu dědění ale i jiné způsoby nabytí vlastnických práv. Způsobů nabytí vlastnických práv upravuje Občanský zákoník v části třetí, Absolutní majetková práva, hlavě II., věcná práva, dílu 3, oddílu 2 hned několik. V rámci mezigeneračního převodu je však použitelný výhradně pododdíl 6, převod vlastnického práva, na kterém stojí i všechny odlišné způsoby mezigeneračního převodu majetku.

%Doplnit všechny odlišené způsoby převodu majetku a okomentovat je v souvislosti s výhodami mezigeneračního převodu majetku za použití svěřenského fondu

S ohledem na to, že institut svěřenského fondu představuje v české právní úpravě stále poměrně nový přírůstek, tak nejsou zásadněji představeny, natož používány, možnosti, kterých lze prostřednictvím založení svěřenského fondu dosáhnout v rámci mezigeneračního převodu majetku. S ohledem na výše zmíněné a na, i díky událostem poslední doby, větší zájem široké veřejnosti o toto téma, se tato kapitola bude soustředit na možnosti a výhody svěřenského fondu právě v oblasti převodu majetku mezi generacemi a zároveň s ohledem na tyto body, které do značné míry vychází z pojetí svěřenského fondu z pohledu Kocího, načne tedy i potenciální problémy, především s ohledem na kogentní úpravu dědického práva a absenci právní úpravy svěřenských fondů, která by upravovala kolizi mezi kogentní úpravou dědického práva a úpravou svěřenských fondů, které mohou vzniknout při zvolení této cesty společně s nevýhodami, které může tato forma převodu majetku pro všechny zůčastněné strany mít. Část textu se bude věnovat i čistě jursiprudenčnímu pohledu na recepci právní úpravy trustu, spočívající především v pohledu na jednotlivé otázky vyvstávající s přijetím institutu běžnému právní úpravě common law do systému civil law, například ve smyslu rozdílného pojetí vlastnického práva, právní osobnosti.
%Poslední větu bude asi nejspíše třeba upravit. V rámci právní osobnosti myslím právnickou osobu. V případě potřeby upravit tento odstavec.
%Hlavně bude tuto poslední část potřeba přesunout do vhodné sekce v rámci mé bakalářské práce.

\subsubsection{Výhody svěřenského fondu jako alternativního nástroje dědické sukcese}

%Doplnit porovnání i s ostatními prostředky převodu majetku mimo dědictví, to je v další kapitole, doplnit tedy tak například, darování, založení nadace, či jiné právnické osoby a tak dále.

%S ohledem na hlavní téma této práce, tedy svěřenský fond jako mezigenerační nástroj převodu majetku, představují způsoby založení svěřenského fondu a jejich specifika vhodný startovní bod pro tuto část práce a obecně pro celou zkoumanou problematiku. \\

%Základním faktem týkajícím se založení svěřenského fondu je, i s ohledem na popsanou flexibilut instutut trustu, ze kterého český zákonodárce vycházel, je nepřekvapivě možnost založení jak mezi živými (\textit{inter vivos}), tak i pro případ smrti (\textit{mortis causa}).

Pokud se podíváme na internet a položíme jednomu z mnoha internetových vyhledávačů otázku "Jaké výhody má založení svěřenského fondu", tak nás ohromí nepřeberné množstí odkazů na jednotlivé stránky, které jen překypují popsanými benefity. Při mé první exkurzi do světa svěřenských fondů po novelizaci Občanského zákoníku mě toto nepřeberné množství výhod opravdu překvapilo. Od té doby jsem samozřejmě vystřízlivěl a na jednotlivé body již nahlížím mnohem kritičtějším pohledem. Jedna věc se ovšem musí svěřenským fondům nechat, možnosti, které jejich založení přináší zakladateli ve smyslu možností, jak ovlivňovat nakládání se svým majetkem i po své smrti jsou opravdu praktické.\\

Přes toto vše není založení svěřenského fondu pro všechny, neboť má i jisté nevýhody a vyplácí se jen v určitých situacích. Dále nepovažuji zakotvení této právní úpravy v Občanském zákoníku za konečné a troufám si tvrdit, že s ohledem na různé výkladové nejasnosti, je velice pravděpodobné, že dojde ke změnám právní úpravy. Tyto změny, jejichž příklady uvádím v poslední kapitole, jsou nutné k tomu, aby tyto alternativní nástroje mohli zcela plnit svůj účel jako alternativy, nebo suplementy ke klasickým způsobům převodu majetku zejména v oblasti dědického práva. Z tohoto hlediska je možné v tomto směru pochybovat o právní jistotě, neboť se dle mého názoru dočkáme jak hmotněprávních, tak i procesněprávních změn.\\

 Problematické části svěřenských fondů budou nicméně zhodnoceny níže. Tato část je věnována, jak již název napovídá, výhodám tohoto alternativního způsobu předání majetku. 
 
 Pokud se budeme bavit o mezigeneračním převodu majetku, tak se nám jistě primárně vybaví rodinný svěřenský fond. Jak již jsem výše poznamenal, obmyšlený může být samozřejmě kdokoliv, i rodině naprosto cizí člověk. Věřím však, že svěřenský fond jako nástroj mezigeneračního převodu majetku bude sloužit především převodu v rámci rodiny, a z tohoto důvodu se budu dále soustředit výhradně na tento typ svěřenského fondu a na ukázku příkladů k jednotlivým situacím, které je výhodné řešit právě pomocí založení svěřenského fondu.\\
 
 %Popsat i ostatní typy svěřenských fondů
 
 Rodina dozajista představuje největší radost v životě a dá se považovat i za jistý cíl naší existence. Je proto logické, že pro staší generaci představuje rodina motivaci, kvůli které se snaží finančně zajistit jak sebe, tak i svoji rodinu a společně pak pro všechny členy najít bezpečí po všech směrech a ve smyslu všech lidských potřeb. Finanční zajištění saturuje mnoho těchto potřeb a pro mnoho rodičů tedy představuje zachování a předání jistého odkazu ve formě dědictví pro další generace téměř životní poslání. Předání tohoto odkazu může ale mít i potenciální negativní konsekvence, jak říká i známe české přísloví \textit{"Rychle nabyl, rychle pozbyl"}, velkou neznámou tedy může představovat fakt, zda by takto rychle nabyté dědictví mladší generace zvládla. Řešení nejen tohoto problému může představovat právě založení svěřenského fondu, ať už za života zakladatele, či pro případ jeh smrti.\\
 
 Věřím, že nejlepší pojetí této kapitoly představuje kompilace mnoha různých výhod svěřenského fondu a k nim uvedeným příkladům, proto jsem se rozhodl zvolit obdobnou formu jako Barbora Bednaříková ve svém díle Institut pro uchování a převody rodinného majetku. Začal bych tedy, z mého pohledu, největší výhodou svěřenského fondu. Zde se bez jakýchkoliv výhrad ztotožňuji s pohledem Lucie Joskové a Lukáše Pěsny, kteří tvrdí, že základní výhodou svěřenského fondu je samostatné vyčlenění části majetku\footfullcite{lucie_joskova_sprava_2017}. Tato výhoda tedy spočívá ve faktu, že daný majetek je v podstatě majetkem \textit{sui generis}, který nepatří nikomu, ale pouze je spravován svěřenským správcem. Je tedy chráněn před počínáním obmyšlených a jedná se mimo jiné o jistou formu ochrany dědictví. Jedná se zde tedy nejen o základní výhodu samotného svěřenského fondu, která svojí použitelností přesahuje i do mnoho dalších právních odvětví, ale specificky ve vztahu k dědickému právu z této obecné výhody vyplývá i, dle mého názoru, nejzásadnější výhoda a důvod, pro které je vhodné využít svěřenský fond k mezigeneračnímu předání majetku. Touto výhodou je ochrana dědictví jako takového.\\
 
 %Tato ochrana je dle m=ho názoru základní výhodou svěřenského fondu, pokud ho speciálně vztáhneme na tuto funkic. 
 
 %mezigeneračního převodu majetku za pomoci svěřenského fondu, kterou jsem již nakousl výše, tedy jako jistou ochranu dědictví.
 
 %Další výhoda, možnost dát si podmínku, aby se o zakladatele obmyšlení starali v rámci statutu svěřenského fondu, samozřejmě pro svěřenské fondy inter vivos, porovnat s dovětkem
 
 %Ochrana nejen před věřiteli, ale i před třetími osobami, které by mohli chtít obmyšlené, nebo zakladatele využít
 
 %Hodně těchto benefitů lze dosáhnou i klasickými smlouvami, ale všechny jednotlivé benefity svěřenských fondů působí společně s těmi dvoumi nebo třemi hlavními benefity a těmi jsou oddělené vlastnictví, ochrana před věřiteli a anonimizace vlastnictví, pokud by byla tedy zvolena cesta pouze smluvní, tyto benefity by vedle těchto smluvně ujednaných náležitostí, které poskytují tyto benefity dále nepůsobily.\\
 
 \begin{enumerate}
 {\Large\item Ochrana dědictví před rozmařilostí, nezkušeností a nezodpovědností ze strany dědiců}
 \item[1.] Mezigenerační transfer a uchování majetku
 \end{enumerate}
 
 Přes zákonné nejasnosti\footnote{ V první řadě je důležité zmínit, že následující výklad nemusí být kvůli nedostatečnostem jak lingvistickým, tak i věcným v právní úpravě svěřenských fondů absolutně přesný a je možné, že díky potenciálním problémům vytyčeným níže se může zákonná úprava změnit a s tím společně i tento bod. S ohledem na pohled právních expertů na problematiku svěřenských fondů a mého názoru k datu publikace této práce je vhodné tento bod zařadit v podobě, která se s ohledem na názorovou disputaci a pohledem na procesněprávní náležitosti svěřenského fondu kloní spíše k názoru Miloše Kocího. Tedy, že majetek vyčleněný do svěřenského fondu není součástí pozůstalosti, další informace o tomto problému jsou popsány v další kapitole.} v úpravě svěřenských fondů, se možnost založení svěřenského fondu za účelem ochrany dědictví jeví jako vhodná v případech, kdy se jedná o větší majetek.\\
 
 Nestává se zřídka, že rodiče se strachují, zda jejich potomci bezpečně zvládnou po nich nabyté dědictví. Založení svěřenského fondu tak v tomto případě může pomoci nejen s ohledem na fakt, že potomci nedostanou jednorázově velký finanční obnos, ale i v mnoha dalších ohledech.\\
 
 Nezvládnutí takto nově nabytého dědictví se může projevovat mnoha způsoby, které zahrnují; neuvážlivé a rozmařilé utrácení peněz, ztrátu motivace k vlastnímu výdělku, zanedbávání povinností, špatná finanční rozhodnutí a nerozvíjení finanční gramotnosti, nepoznání hodnoty peněz, práce a času. Tyto rizika lze mitigovat právě založením svěřenského fondu, díky kterému nejen, že dědicové nedostanou jednorázově všechno jmění\footnote{Srovnání další bod, ochrana před věřitely}, ale v rámci statutu svěřenského fondu jim lze stanovit podmínky, které musí pro plnění ze svěřenského fondu splňovat a plnit\footfullcite[§1452, odst. 2, bod d)]{noauthor_zakon_2012}. Těmito mohou být například požadavek studia, podmínka dosáhnutí určitého věku k vyplácení vyšší části a podobně.\\
 
 Dále lze jako součást procesu založení svěřenského fondu definovat i jisté rodinné poslání, které může sloužit jako opora při stanovení společných hodnot a cílů. Založení svěřenského fondu tak ve svém důsledku nemusí chránit jednotlivce před majetkem a mejetek před jedinci, ale může sloužit jako ochrana pro celou rodinu. Díky svěřenskému fondu lze sledovat společné cíle všech členů rodiny a vytvořit tak díky tomuto společný mezigenerační plán. Správa rodinného majetku je tak svěřena všem jejím členům, kteří se na ní podílejí a sdílí mezi sebou odpovědnost za rozhodnutí učiněná při péči o vyčleněn\-ý majetek. Tento proces pozitivně formuje potomky a v konečném důsledku je připraví na zodpovědnou správu rodinného jmění.\\
 
 \begin{enumerate}
 \item[2.] Ochrana dědictví před jinými dědici
 \end{enumerate}
 
 Přes obdobné zákonné nejasnosti\footnote{Při tomto bodu lze také namítat nejasnost právní úpravy svěřenských fondů, srovnej DOPLNIT ODKAZ NA PROBLÉM VZTAHUJÍCÍ SE K TOMUTO BODU} lze jako na potenciální benefit použití svěřen\-ského fondu k předání majetku nahlížet i na možnost zvolení beneficienta dle uvážení zakladatele. Přes kogentní ustanovení dědického práva je tedy možné stanovit obmyšlené, kteří nespadají do kategorie nepominutelných dědiců, či dokonce obmyšlených, kteří nejsou v žádném příbuzenském vztahu se zakladatelem, a tímto způsobem tak omezit právo těchto dědiců na podíl z pozůstalosti, bez jejich faktického vydědění\footnote{Srovnej s kapitolou Dědické právo v České republice, důvody k vydědění}\textsuperscript{,}\footnote{Toto lze samozřejmě považovat i za nevýhodu, pokud se na tuto možnost koukáme z pohledu dědice, srovnej DOPLNIT ODKAZ}. Tento způsob se jeví při současném výkladu právní úpravy svěřenského fondu, s porovnáním s právní úpravou dědického práva, jaho vhodnější oproti vydědění. Dědické právo nenabízí mnoho důvodů vydědění a není výjimkou, že i když takovéto důvody existují, je složité je prokázat. Tato možnost je vhodná v případech, kdy chce zakladatel zamezit dědickému nároku osob, u kterých neshledává, že by měli mít na část pozůstalosti nárok. Na majetek vložený do svěřenského fondu by tedy tyto dědicové neměli již nárok, neboť takto vylčeněný majetek přestává být ve vlastnictví zakldadatele a stává se výlučným vlastnictvím svěřenského fondu a plody a užitky z tohoto majetku náleží výhradně obmyšlenému.
 
 %Hodnota peněz a práce a času
 
 \newpage
 \thispagestyle{smallertextinheader}
  \begin{enumerate}
 {\Large\item[2.] Ochrana před věřitely}
 \end{enumerate}
 
 Další důležitou výhodu uvádí ve svém díle Správa cizího majetku (dále jen Správa cizího majetku) Lucie Joková a Lukáš Pěšna. Tento bod úzce souvisí s výhodou, kterou jsem již nastínil v předchozím bodu, tedy s převedením majetku na jiný kvazi subjekt a tím subsekventní oddělení daného majetku od majetku zakladatele, svěřenského správce i obmyšleného. Věřitelé koholiv z výše vymezených osob, v rámci tohoto bodu je jako tato osoba přednostně myšlena osoba zakladatele svěřenského fondu, a tedy nemohou v rámci svých pohledávek za jakoukoliv z těchto osob uspokojovat z majetku, který byl vyčleněn do svěřenského fondu, toto platí i v případě úpadku těchto osob. Je nicméně důležité stanovit, že plnění z tohoto fondu ve prospěch obmyšlených může sloužit k uspokojení věřitelů\footfullcite{lucie_joskova_sprava_2017} dle exekučního řádu\footfullcite[§59]{noauthor_zakon_2001}.\\
 
 Výborný příklad je uveden v tomtéž díle: \textit{"Zakladatel se zymýšlí pustit do rizikového podnikání. Vyčlení proto část svého majetku do svěřenského fondu a jako obmyšlené určí sebe a svou rodinu. Tím majetek vyjme z dosahu svých potenciálních budoucích věřitelů. V případě podnikatelského neúspěchu se věřitelé budou moci uspokojit z majetku zakladatele, nikoli však z majetku vyčleěného do svěřenského fondu. To nevylučuje právo věřitelů domáhat se uspokojení z jednotlivých nároků, které zakladateli ze svěřenskému fondu plynou jako obmyšlenému."}\\
 
 Jak dále uvádí autoři, je nicméně nutné podotknout, že si pod založením svěřenského fondu nelze představovat jakýsi nástroj ochrany zakladatelova majetku pro každou situaci. Obecně tak například nelze do svěřenského fondu vyčlenit majetek po vzniku pohledávky, neboť takto vyčleněný majetek by mohl zhoršit dobyvatelnost pohledávky a toto vyčlenění by tak bylo zatíženo relativní neúčinností dle § 589 až 599 obč. zák., popřípadě by mohl být věřiteli odporován dle paragrafů insolvenčního práva, §235 až §243 insolvenčního zákona.\\
 
 Tomuto přisvědčuje nejen Kocí v Komentáři k Občanskému zákoníku jak uvádí správně autoři, ale i Svejkovský a Marek ve svém díle Správa cizího majetku v novém občanském zákoníku v komentáři k paragrafu 1467\footfullcite{jaroslav_svejkovsky_sprava_2015}.\\
 
 \newpage
 \thispagestyle{smallertextinheader}
 
  \begin{enumerate}
 {\Large\item[3.] Anonimizace či diskrétnost}
 \end{enumerate}
 
 V případech snahy o zachování jisté diskrétnosti, pro jednoduchost uveďmě, převodce a nabyvatele určitých práv, poskytuje svěřenský fond v tomto ohledu značnou výhodu oproti klasickým způsobům převodu, kdy je ve veřejně přístupných seznamech možné vidět obě osoby ať již přímo, či ve sbírce listin. Novela Občanského zákoníku a Zákona o veřejných rejstřících právnických a fyzických osob z roku 2018 tento bod poměrně zkomplikovala povinným zápisem svěřenských fondů do evidence. Měla ovšem i pozitivní přínost v jiných ohledech, zejména pokud se jednalo o oprávněné obavy, že by svěřenský fond mohl být použit k legalizaci výnosu z trestné činnosti, či dalších nelegálních aktivit. V mnohém tedy tato novela zbourala představu absolutní anonimizace zakladatelé a obmyšlených, kteří jsou teď povinně zapisováni do evidence. Není tomu ovšem tak, je sice pravda, že osoba zakladatele a jednotliví obmyšlení jsou zapsání do evidence, ale data z této evidence jsou poskytnuty jiné osobě pouze pokud prokáže právní zájem\footfullcite[§65e odst. 3]{noauthor_zakon_vr}. Na první pohled je ve všech relevantních seznamech uvedena pouze osoba svěřenského správce.\\
 
 Tato dodatečná bariéra, mezi daty a subjekty, které by dané informace chtěli získat slouží jako další ochrana před věřiteli, či jinými osobami, kteřé by tyto informace mohli využít ke svému prospěchu a poškození zakladatele či obmyšlených, slouží tedy jako firewall, který zajišťuje zakladateli a obmyšleným větší bezpečí a pohodu. Kvůli nečitelnosti vlastnictví je tedy ztíženo zjistit informace týkající se převodu majetku, distribuci výnosů z majetku ať již vně rodiného kruhu, či mimo rodiný kruh.\\
 
 % \begin{enumerate}
 %{\Large\item[4.] Mezigenerační transfer a uchování majetku}
 %\end{enumerate}
 
 %Vypuštěno, neboť si myslím, že jsem tento bod zahrnul pod prvním bodem
 
 \newpage
 \thispagestyle{smallertextinheader}
 
  \begin{enumerate}
 {\Large\item[4.] Udržení celistvosti a ochrana majetku}
 \end{enumerate}
 
 Založení rodinného svěřenského fondu má také pozitivní vliv na udržení celistvosti a tedy i na ochranu majetku. Toto je důležité zejména v případech, kdy existuje více dědiců a pozůstalost by mělo tvořit jmění, u kterého si lze jen ztěží představit možnost jeho rozdělení mezi dědice, popřípadě by takové rozdělení představovalo například značnou konkurenční nevýhodu, popřípadě přenesení odpovědnosti na všechny dědice, aby se zabránilo nezodpovědnému počínání jen některých z nich, ovšem bez nutnosti ponižovat podíly náležející těmto obmyšleným. Takovéto jmění může představovat například rodinný závod, nemovitý majetek či různé investiční fondy a podobně. Je samozřejmě možné tento potenciální problém vyřešit dědickou smlouvou, závětí, či darováním. S ohledem na bod 1 si však myslím, že vložení tohoto majetku do svěřenského fondu mu zajistí ochranu a v souvilosti s bodem 1 může obmyšlené i v uvozovkách vzdělat v uvědomělé správě majetku. Vyčlenění majetku do svěřenského fondu také může sloužit k zajištění řádné správy tohoto majetku v případě, kdy by větší problém, než potenciální rozdrobení majetku mohla představovat například potenciální budoucí nemožnost nebo další neochota ze strany zakladatele se o majetek dále starat nebo odůvodněné obavy, že právní nástupci nejsou, ať už z důvodu nedostatku zkušeností nebo nízkého věku, připraveni na povinnosti spojené se zodpovědnou správou majetku.\\
 
 %Podívat se jak probíha dědění rodinného závodu a jak rodinný závod funguje ve svěřenském fondu.
 
 Příklad uvedený v díle Správa cizího majetku: \textit{"Zakladatel může  do svěřenského fondu vyčlenit závod na výrobu automobilů a potomkům přiznat právo podílet se rovným dílem na zisku dosaženým tímto závodem."} Tento závod může býti přitom spravován profesionálem, potřeba jisté odbornosti dědiců ve směru řízení podniku tak nemusí být nikterak vysoká, je jim pouze přiznáno právo na plnění, aby byly zajištěny jejich potřeby. A dále uvádím pro druhý dílčí přínos zvolení předání majetku ve formě svěřenského fondu tento příklad:\textit{"Otec vyčlení část majetku - akcií a určí jako obmyšlené své nezletilé děti. Správou majetku pověří profesionála zběhlého ve správě cenných papírů a děti budou až do dosažení vysokoškolského vzdělání oprávněni pouze k výnosu z akcií."} V tomto příkladu bude zabráněno rozdrobení majetku vlivem rozdílný názorů a ekonomických zájmů jednotlivých dědiců, či jak již je výše zmíněno nepřipravenosti\footfullcite{lucie_joskova_sprava_2017}.\\
 
 \newpage
 \thispagestyle{smallertextinheader}
 
 Svěřenský fond také dále může být praktickým nástrojem pro správu velkého majetku, který lze rozdělit a toto rozdělení by potenciálně ani nepředstavovalo vážnou konkurenční nevýhodu, ale je praktičtější majetek držet pohromadě než ho drobit mezi dědice. Dobrým příkladem tak může být syndikovaný úvěr\footnote{Syndikovaný úvěr je takový úvěr, kdy je úvěr poskytnut syndikátem, tedy nějakou skupinou věřitelů, tyto úvěry jsou poskytovány v případech, kdy je třeba vysokého objemu zapůjčených finančních prostředků, například při fúzi, akvizici nebo expanzi korporace, refinancování stávajících závazků, nebo realizaci projektového financování jak správně uvádí na svých stránkách například Komerční banka,https://www.kb.cz/cs/firmy-a-instituce/produkty/uvery-a-financovani/strukturovane-financovani/syndikovany-uver}, kdy majetek slouží ku prospěchu velkého počtu osob, jež jsou spojeny společným zájmem, v těchto případech je se správou práv a povinností spojena velká administrativní zátěž, která může být vytvořením svěřenského fondu zmírněna jak pro věřitele, tak i pro dlužníky\footfullcite{lucie_joskova_sprava_2017}. Další příklad uvádí Barbora Bednaříková u rodiny Rockefellerů. Jedná se o velice rozsáhlou rodinu, která svým počtem převyšuje 100 členů. I přes fakt, že mnoho majetku bylo rozdrobeno do menších fondů, je obecně majetek rodiny držen v celku místo osobního vlastnictví jednotlivých členů. Společně s takto značným majetkem, který je centralizovaně spravován, a správnou investiční strategií, dosahují výnosy z takovýchto fondů nemalé částky, z nichž jednotliví členové mohou pohodlně žít. I přes fakt, že by tento majetek tedy potenciálně mohl být rozdělen mezi členy rodiny, tak pokud je soustředěn v celku, výnosy z něho budou mnohem větší a mohou sloužit rodiným členům k jejich potřebám lépe, než kdyby se majetek rozdělil\footfullcite{bednarikova_barbora_sverenske_2014}.
 
 %v případě, že zakladatel již nechce nebo nemůže řádně majetek spravovat, stejně tak pokud by měl pochybnosti o tom, zdaby jeho právní nástupci například pro nedostatek zkušeností, či nízký věk tuto zprávu zvládli, m, 
 
 \newpage
 \thispagestyle{smallertextinheader}
 
  \begin{enumerate}
 {\Large\item[5.] Péče o členy rodiny}
 \end{enumerate}
 
 Další výhodu svěřenského fondu spatřuji v možnosti jeho nastavení tak, aby se dal využít pro péči o jednotlivé členy rodiny, kteří danou péči potřebují, ať již díky fyzickému nebo mentálnímu postižení nebo kvůli pokročilému, nebo naopak nízkému věku, a tedy zajištění jejich hmotného zabezpečení. Těmito mohou býti jak zakladatelé, tak jiní členové rodiny, či dokonce osoby mimo rodinný kruh. Tato výhoda je velice posílena flexibilitou svěřenského fondu, kterou vložil zákonodárce do rukou jeho zakladateli a ten prostřednictvím statutu do rukou svěřenského správce, který poté může vykonávat profesionální správu svěřenského fondu bez toho, aby se jakýkoliv člen rodiny musel angažovat. Díky flexibilitě svěřenského fondu, je možné přesně popsat a vymezit situace a podmínky, za kterých mají jednotliví členové rodiny právo na plnění ze svěřenského fondu. Vhodným nastavením je možné docílit hmotného zabezpečení a péči o osoby, které to nejvíce potřebují po celý jejich život.\\
 
 Toto využití je tak například vhodné v případech, kdy je zakladatel stižen vážnou chorobou, která v budoucnu zapříčiní znemožnění vlastního racionálního úsudku. Založením svěřenského fondu tak může zakladatel docílit toho, že potomci budou mít motivaci se o něj starat a prostředky z majetku takto vyčleněného budou distribuovány výhradně osobám, které o zakladatele projevují zájem a starají se o něj v jakékoliv nelehké situaci a podíl na majetku nebudou mít osoby, které si takovéto plnění nezaslouží\footnote{Tamtéž strana 113}\textsuperscript{,}\footfullcite[Kocí M. Strana 1208]{svestka_j_obcansky_2014-1}. Vedle tohoto benefitu také samozřejmě stojí i výhody popsané v bodech 1, 4 a 6. Toto myslím tak, že například s ohledem na výhody uvedené v bodě 1 se může zakladatel takto rozhodnout, aby uchránil majetek před dědici, či jinými osobami, které by chtěli využít jeho ztížené situace, kdy by dle platných předpisů nebyla ještě učiněna žádná podpůrná opatření při narušení schopnosti jednat\footnote{§38-65 Občanského zákoníku} a byl by tak plně způsobilý jednat po právní stránce, nicméně po stránce mentální by tohoto již nemusel být schopen.\\
 
 Obdobně je možné svěřenského fondu využít jako zajištění pro postižené dítě pro smrti rodičů spolčně s přiznámím odměny za tuto péči jednotlivým osobám.
 
 %Je praktické vyčlenit určitou část do svěřenského fondu a jako obmyšlené určit potomky. Tento způsob\\
 
 %Dědic je v rámci postavení k ostatním dědicům v postavení věřitele, je možné se bránit jako dědic jednání zůstavitele za jeho života, v podstatě to nejde, jen pokud by takovéto jednání bylo učiněno pod nátlakem, nebo pokud bychom se bavili o kolaci, tzn. započtení na dědický podíl díl?
 
 %Pokud by byli dva dědicové a zůstavitel by založil svěřenský fond jen pro jednoho dědice, mohl by druhý dědic požadovat, aby se tento majetek započetl na dědický podíl?
 
 %V první řadě je důležité zmínit, že následující výklad nemusí být kvůli nedostatečnostem jak lingvistickým, tak i věcným v právní úpravě svěřenských fondů absolutně přesný a je možné, že díky potenciálním problémům vytyčeným níže se může zákonná úprava změnit a s tím společně i tento bod. S ohledem na pohled právních expertů na problematiku svěřenských fondů a mého názoru k datu publikace této práce je vhodné tento bod zařadit v podobě, která se s ohledem na názorovou disputaci a pohledem na procesněprávní náležitosti svěřenského fondu kloní spíše k názoru Miloše Kocího. Tedy, že majetek vyčleněný do svěřenského fondu není součástí pozůstalosti, další informace o tomto problému jsou popsány v další kapitole.\\ 
 %DOPLNIT

%Ukázat příklady, ke kterým se hodí svěřenský fond, v souvislosti s tématem práce by zde bylo vhodné popsat institut svěřenského fondu jako nástroj k mezigeneračnímu převodu majetku.

%S ohledem na hlavní téma této práce, tedy svěřenský fond jako mezigenerační nástroj převodu majetku, představují způsoby založení svěřenského fondu a jejich specifika vhodný startovní bod pro tuto část práce a obecně pro celou zkoumanou problematiku. \\

%Základním faktem týkajícím se založení svěřenského fondu je, i s ohledem na popsanou flexibilut instutut trustu, ze kterého český zákonodárce vycházel, je nepřekvapivě možnost založení jak mezi živými (\textit{inter vivos}), tak i pro případ smrti (\textit{mortis causa}).

  \begin{enumerate}
 {\Large\item[6.] Přiznání různých práv oprávněným osobám}
 \end{enumerate}
 
 S ohledem na široké možnosti účelu svěřenského fondu a podmínky plnění, které je možné stanovit ve statutu, dále svěřenský fond umožňuje efektivní využívání majetku. Toto efektivní využívání majetku spočívá ve své podstatě v zahrnutí možností poskytnutých dědickým právem v §1507-1524 Občanského zákoníku, tedy možností svěřenského náhradnictví a svěřenského nástupnictví. V rámci statutu je možné, aby zakladatel přiznal jednotlivým obmyšleným různá práva jednorázově i postupně\footfullcite{lucie_joskova_sprava_2017}. Toto se dá využít například tak, že zakladatel přizná určité právo na věci vložené do svěřenského fondu jedné osobě po dobu jejího života a po její smrti či naplnění jiného zakladatelem stanoveného kritéria přejde toto právo na osobu jinou. Pokud by se tak například zůstavitel nemohl spolehnout na svého potomka, kvůli jakémokoliv důvodu, že se nebude schopen o dědictví vhodně postarat, ale chtěl by aby dané dědictví následně přešlo na potomka tohoto potomka, přičemž by ale chtěl zabezpečit i svého potomka, tak použití institutu svěřenského nástupnictví není vhodné ani v případech, kdy by podle §1521 nebo §1522 Občanského zákoníku nebylo svěřerno dědici s dědictvím volně nakládat. V takovéto obdobné situaci je nicméně zřízení svěřenského fondu ve prospěch potomka se stanovením podmínky vyplacení celé podstaty svěřenského fondu po jeho smrti jeho potomkovi\footnote{Toto je plně v souladu s §1460 Občanského zákoníku} velice praktické, neboť majetek vložený do svěřenského fondu může spravovat profesionální svěřenský správce a v souladu s bodem 1 - Ochrana majetku, bude majetek zachován pro další generace. V souladu s důrazem na pořizovací volnost tedy takto bude zajištěno, že zakladatel svěřenského fondu bude mít nepřímý vliv na svůj majetek i po své smrti a o tento majetek bude postaráno dle jeho vůle.\\
 
 Vhodný příklad opět uvádí autoři v díle Správa cizího majetku, pro ilustraci výše napsaného si však dovolím tuto ukázku trochu pozměnit: \textit{"Po smrti zakladatele má vyčleněný majetek (bytovou jednotku) využívat jeho syn s jeho vnukem. Po smrti jeho syna má majetek získat vnuk zakladatele."}
 
   \begin{enumerate}
 {\Large\item[6.] Možnost převedení majetku na jinak nezpůsobilé dědice}
 \end{enumerate}

\newpage

\thispagestyle{smallertextinheader}
Data.
\newpage

\section{Systémové nedostatky a návrhy na řešení}

%Zneužití svěřenského fondu viz Babiš, nebo k obcházení věřitelů, dále rozdílné pojetí vlastnictví, započtení, obcházení nepominutelného dědice

%Právní jednání za svěřenský fond, úvěr, závazky, vložení závoud do svěřenského fondu, změna statutu za trvání svěřenského fondu

%Je možné převést do svěřenských fondů například majetek, na kterém vázne zástavní právo? Je tedy možné převést jmění? Tedy nejanom aktiva ale i pasiva

%Pokud je zakladatel i správce, tak je neovlivnitelnost druhého správce, kterého musí povinně mít velkou otázkou.

%Daně nejsou nevýhodou, v rámci příbuzenských vztahů je plnění bezúplatné, problém by byl při plnění ze zisku, nebo pokud by obmyšlenou byla cizí osoba. Při zániku svěřenského fondu a převedení majetku na obmyšleného by se žádná daň platit neměla, tohle ale musím ještě ověřit. Dle článku(portál pohoda), který jsem našel by stejné principy, které platí na plnění ze svěřenského fondu měli platit i na zánik a tedy převedení majetku ze svěřenského fondu obmyšlenému, nebo zakladateli. Jedná se o bezúplatný příjem na základě Zákona o dani z příjmu.

%Co se daní týče tak zde je to hezky shrnuto v oblasti daní z pozemků a nemovitostí

%Daňové aspekty

%Daňová problematika týkající se svěřenských fondů je rozsáhlá a vzhledem k absenci relevantní judikatury či ustálené praxe ne vždy jednoznačná. S ohledem na zaměření tohoto článku se proto omezíme na daňové povinnosti svěřenského fondu u daně z pozemků a ze staveb a daně z nabytí nemovitých věcí.

%Svěřenský fond je poplatníkem daně z nabytí nemovitosti dle ust. § 1 odst. 1 a 2 zákonného opatření Senátu č. 340/2013 Sb. o dani z nabytí nemovitých věcí („zákon o dani z nabytí nemovitosti“) a poplatníkem daně z pozemků a staveb dle ust. § 3 odst. 2 písm. b) a § 8 odst. 2 písm. b) zákona č. 338/1992 Sb., o dani z nemovitých věcí.

%Co se týče daně z pozemků a staveb je svěřenský fond standardním poplatníkem a platí pro něj srovnatelná pravidla jako pro ostatní daňové poplatníky.

%Často řešenou otázkou je však vklad nemovitosti do svěřenského fondu z pohledu daně z nabytí nemovité věci. Jak je uvedeno výše, svěřenský fond je poplatníkem této daně. Předmětem daně je úplatné nabytí vlastnického práva k nemovitosti.

%Vzhledem k tomu, že se při vyčlenění nemovitosti z majetku zakladatele či zvýšení majetku svěřenského fondu o nemovitost dle ust. § 1468 občanského zákoníku nejedná o úplatné nabytí vlastnického práva, není tento převod ani zatížen daní z nemovitých věcí. Nelze zde analogicky uplatňovat úpravu pro nepeněžité vklady do obchodních korporací, neboť zakladatel/vkladatel nezískává podíl na svěřenském fondu ani obdobné protiplnění.[18] K tomuto závěru dochází i odborná literatura.[19]

%Závěrem

%Využití svěřenského fondu pro řešení rozličných situací již nevyvolává údiv (ani úlek), nýbrž je akceptovaným a efektivním východiskem mnohých situací. Výše nastíněný scénář má za cíl ukázat, že i z praktického hlediska se nejedná o nic složitého.

%[19] SVEJKOVSKÝ, Jaroslav a Ivan KOVÁŘ. Svěřenské fondy: příležitosti a rizika. V Praze: C. H. Beck, 2018. Právní praxe. ISBN 978-80-7400-726-2, str. 56.

%Další nejasnost, jak naložit s majetkem ve chvíli, kdy všechni obmyšlení zemřeli? Vrátí se následně do dědictví a dědicům bude náležet povinný díl? Popsaáno v práci Ivany Vladyková, ochrana nepominutelných dědiců a jejich vydědění

---Koment\\
Zhodnocení faktu, že by taková právní úprava měla existovat, poznámka, že tento problém zatím u nás není řešen soudní praxí. Pokusit se navrhnout legislativní změny, vedoucí ke zlepšení postavení nepominutelných dědiců v rámci dědického řízení za předpokladu, že zároveň vznikne i svěřenský fond mortis causa, do kterého se vyčlení majetek, který by jinak byl součástí dědického řízení. Zkusit navrhnout řešení na základě inspirace jinými právními systémy, ve kterých existuje jak institut trustu, tak i ins  titut nepominutelného dědice, mohlo by se jednat například o provincii Quebec, nebo stát Louisiana, pokusit se tyto změny navrhnout na základě studia zákonů a dále se pokusit najít soudní rozhodnutí, v Angličtině se institut nepominutelného dědice nazývá forced heir. \\
---\\

Je nepochybé, že s přijetím institutu svěřenského fondu vzniklo mnoho otázek, na které nám zákonodárce neposkytl patřičné odpovědi. Zajímavé ovšem je, že zákonodárci v rámci procesu přejímání právní úpravy svěřenského fondu, a její implementací do českého právního řádu, muselo, u mnoha z níže popsaných problémů, být jasné, že takovéto vzniknou. Otázka tedy je, proč byl svěřenský fond takto upraven a proč jeho přijetí nevedlo k odpovídajícím změnám v rámci českého práva, které by takovéto problémy řešily. Věřím, že pomocí analýzy těchto problémů a zodpovězením otázek z nich plynoucích, budu schopen v závěrečné kapitole navrhnou patřičná opatření a změny, které by mohly vést k vyjasnění, či přímému odstranění těchto problémů české právní úpravy.\\
\newpage

%Ověřit si, že jsem nezapomněl na žádné nedostatky, nebo nejasnosti

\subsection{Nejasná regulace a chybějící judikatura}

\subsection{Systémové nedostatky vztahující se k dědickému právu}

\subsection{Započtení na dědický podíl}
\subsection{Nejasný výklad ochrany nepominutelných dědiců}

Jak již je zmíněno v kapitole výše, v rámci české právní úpravy bohužel chybí explicitní úprava funkce svěřenského fondu, pokud se nachází ve střetu s institutem nepominutelného dědice. Podobné stanovisko zaujímá například i Kryštof Horn ve svém příspěvku do Časopisu Ad Notam. Ve svém příspěvku říká následující: "Příkladem takové nejasnosti může být vztah fondu k institutu nepominutelného dědice, který je neznámý jak v zemích common law, kde má své kořeny trust, tak bohužel i v Québecu, odkud zákonodárce přejal úpravu svěřenských fondů. Výslovná ustanovení o vztahu svěřenského fondu mortis causa a dědického práva totiž chybí." \footfullcite{sro_httpwwwpraguebestcz_prakticke_nodate-1} Tento příspěvek se zakládá na pravdě a potvrzuje fakt, že takováto právní úprava v českém právu chybí a bylo by vhodné jí prostřednictvím novelizace doplnit. S výrokem nicméně nemohu souhlasit jako s celkem, protože přesto, že je pravda, že institut nepominutelného dědice jako takový v Quebeckém právu neexistuje a Quebecká právní úprava se dále vyznačuje téměr úplnou testamentální volností, tak Quebec Civil Code připouští určitou ochranu dědiců před opomenutím v závěti. Abych byl přesný, tak Quebecký právní řád upravuje tři případy ochrany dědiců, které by se daly přirovnat k institutu nepominutelného dědice v našem právním řádu. Je tedy možné říci, že, přestože tak zákonodárce neučinil, mohl se u Quebecká právní úpravy inspirovat i v rámci ochrany nepominutelných dědiců, protože Quebecké právní úprava, narozdíl od té České, upravuje i případy, ve kterých by tyto nároky nemohly být splněny kvůli právnímu jednání učiněnému zůstavitelem, mezi toto právní jednání je právě možné zařadit i založení trustu.\footfullcite{leclercq_inheritance_2014} \\

Obdobně je také možné hledat inspiraci pro vyřešení této nejastnosti ve vztahu k dědickému právu i v případech z Itálie, či v právní úpravě institutu obdobného svěřenskému fondu v Lichtenštejnsku, tento fakt a jeho možné aplikace na českou právní úpravu jsou zhodnoceny v kapitolách. \\

\subsubsection{Komparace právní úpravy svěřenského fondu s právní úpravou dědictví}

---Koment\\
Zde proběhne komparace, musím zde napsat, že neexistuje právní úpravy, která by upravovala situaci, ve které bude založen svěřenský fond mortis causa a zároveň bude probíhat dědické řízení, ve kterém budou figurovat nepominutelní dědici, na které v souvislosti se založením svěřenského fondu nevyjde jejich zákonný podíl. Konstatovat, že těmto problémům se dá vyhnout tím, že se svěřenský fond vyčlení už za života, dokázat toto tvrzení na základě jiných institutů převodu majetku za zůstavitelova života a poukázat na to, že tento majetek následně není předmětem dědického řízení a tudíž ani nemůže připadnout jakýmkoliv dědicům, neboť pokud budeme mít na časové ose bod x, ve kterém zůstavitel zemře, a svěřenský fond vznikne v bodě x-y, tak v době zůstavitelova úmrtí již není součástí jmění a tudíž nemůže být součástí dědického řízení. Pokud by nicméně zůstavitel zemřel v bodě x a svěřenský fond by také vznikl v bodě x, tak lze polemizovat o tom, zda by dědici nemohli mít právo na snahu obsáhnout i tento vyčleněný majetek v soupisu dědického řízení.\\
---\\

---Koment\\
Dále se pokusit obhájit můj názor, že pokud budou obmyšlenými svěřenského fondu i nepominutelní dědicové, tak plněním ze svěřenského fondu právě tímto dědicům dojde k neformálnímu doplnění jejich povinného podílu ze strany zůstavitele - pokusit se najít nějakou právní zásadu, která by něco takového říkala. Pokud by však z fondu mělo být plněno nikomu jinému, tak se pokusit obhájit fakt, že tato situace není zákonně obhájena a není na ní ani žádná judikatura či ustálená soudní praxe, a potenciálně by se tedy dědicové mohli bránit soudní cestou. Pokusit se stanovit, jak by takovýto spor mohl v rámci našich existujících právních norem dopadnout, pokud to půjde tak se zkusit podívat, zda toto neni řešeno na mezinárodní úrovni například v rámci mezinárodního práva soukromého. \\
---\\

Pro tuto kapitolu, a následné návrhy na změny, je v první řadě podstatné vymezit,  v jakých fázích vzniká svěřenský fond, a tyto následně porovnat s úpravou dědického práva. Tímto postupem lze dosáhnout vhodného a dostatečného porovnání možností, které mají dědici v jednotlivých fázích, jak jsou chráněni a jak se potenciálně mohou brát o svá práva. \\ 

Dle mého názoru, který potvrzuje Miloš Kocí v rámci komentáře k úpravě svěřenského fondu, lze hovořit o 3 způsobech vzniku svěřenského fondu. Těmito způsoby jsou, založení svěřenského fondu \textit{inter vivos}, \textit{mortis causa} a potenciální souběch těchto založení ve formě založení svěřenského fondu \textit{inter vivos} s odkládací podmínkou, dle § 584 až 549 OZ, spočívající ve smrti zůstavitele.\footfullcite{svestka_j_obcansky_2014-1} \\

\newpage

\thispagestyle{smallertextinheader}

Vizuálně by se tedy průběh komparace napříč jednotlivými způsoby založení dal vyobrazit následovně, kde střed osy představuje čas úmrtí zůstavitele:

\begin{figure}[h]
\centering
\includegraphics[width=18cm,height=6cm,keepaspectratio]{Vznik_sverenskeho_fondu.png}
\caption{Vznik svěřenského fondu}
\label{fig:komparace}
\end{figure}

---Koment\\
Doplnit další informace.
Popsat dědické právo - co jde do dědictví, co jsou nepominutelní dědicové, jak funguje ědické řízení atp.
Vysvětlit rozdíl mezi majetkem a jměním.\\
---\\

\subsubsection{Ochrana zájmů nepominutelných dědiců zakladatele trustu v Itálii}

Napsat o situaci v Itálii.\footfullcite{lubos_tichy_sverensky_2017}

\subsubsection{Právní úprava v Lichtenštejnsku}

\subsection{Nejasný výklad ve vztahu k vyživovací povinnosti}

\subsubsection{Typy ochrany dědických nároků v rámci Quebec Civil Code}

V Quebeckém právní řádu existují tři instituty, které by se svým účelem a formou daly přirovnat k českému institutu nepominutelného dědice. Ve své pdostatě se jedná o instituty, které mají chránit osobu, které je zůstavitel povinnen poskytnou nějaké plnění před opomenutím této osoby ze strany zůstavitele v rámci závěti, nebo opomenutí této osoby pokud se dědí na základě intestátní posloupnosti. Zákonodárce zároveň upravil vztah těchto institutů k právnímu jednání zůstavitele, které mění výši zůstavitelova majetku již v obodobí před jeho smrtí a s účinností k datu jeho smrti. \\

\vspace{5 mm}

Tři ochrany, které jsou upravené v Quebeckém právním řádu:

\begin{enumerate}
\item Survival of the Obligation to Provide Support
\item Family Patrimony
\item Compensatory Allowance	
\end{enumerate}

\vspace{5 mm}

\underline{\textbf{Survival of The Obligation to Provide Support}}

Prvním způsobem ochrany jistých dědických nároků vůči majetku zachované\-mu zůstavitelem spočívá v Institutu zvaném \textit{The Survival of the Obligation to Provide Support}, který zaručuje příjemci určité finanční podpory vyplácené danému příjemci za zůstavitelova života tuto podporu i po smrti zůstavitele. \\

Důležitou roli pro dokázání argumentu, že Quebecká právní úprava obsahuje v jisté formě ochranu dědiců a upravuje i případy zmenšení zůstavitelova majetku v období před smrtí popřípadě učinné ke dni smrti zůstavitele je odstavec 689, který stanovuje následující: "Where the assets of the succession are insufficient to make full payment of the contributions due to the spouse or to a descendant, as a result of liberalities made by acts inter vivos during the three years preceding the death or having the death as a term, the court may order the liberalities reduced. \\

Liberalities to which the spouse or descendant consented may not be reduced, however, and those he has received shall be imputed to his claim."\footfullcite{noauthor__nodate} Tento odstavec ve zkratce zaručuje osobám s nárokem vůči zůstavitelovi ochranu tohoto nároku v případě nedostatečného jmění v rámci dědictví, které bylo jistým právním jednáním zůstavitele sníženo na tuto nedostatečnou hranici. Soud může v tomto případě omezit právní jednání učiněné zůstavitelem dokonce i za jeho života, co je ale důležité pro založení trustu mortis cause je fakt, že může omezit i právní jednání s účinností ke dni zůstavitelovi smrti. Tímto způsobem tedy může omezit i vyčlenění majetku do svěřenského fondu, respektive Trustu. \\

Dále je potřebné a důležité poznamenat i to, že odstavec 689 ve své druhé části říká, že v případě, že z těchto právních jednání bude mít prospěch právě osoba, která se domáhá svého podílu na zůstavitelově majetku, tak tento prospěch bude započten v rámci potenciálního omezení právního jednání zůstavitele. Z této části je tedy možné dovodit, že uspokojit tento právní nárok je možné i založením trustu, který bude daný prospěch oprávněným osobám poskytovat místo likvidátora dědictví\footnote{V Quebeckém právu je likvidátor osobou, která}.

\underline{\textbf{Family Patrinomy}}

\underline{\textbf{Compensatory Allowance}}

\subsection{Nejasnosti v rámci daňových aspektů svěřenského fondu}

Jisté nejasnosti lze pozorovat i v právní úpravě svěřenského fondu ve vztahu k daním. V souvislosti s přijetím Nového Občanského zákoníku a jeho účinností bylo potřeba novelizovat i další právní předpisy. Jednou z těchto novelizací prošly právní normy upravující zdaňování, především tedy zákon o daních z příjmu. I přes tyto novelizace však shledávám v daňové úpravě vztahující se ke svěřenským fondům jisté nedostatky. Nemyslím si, že s ohledem na téma práce je vhodné hlouběji zabředávat do tématu zdanění svěřenského fondu a jeho výhodnosti z pohledu daňového zatížení, neboť žádná takováto výhodnost neexituje, avšak v tomto rámci se vyskytuje i jeden problém týkající se zániku svěřenského fondu a následného převedení zbylého majetku obmyšlenému, nebo zakladateli, který je dle mého názoru vhodný uvést kvůli potenciálním nejasnostem vztahujícím se k zániku svěřenského fondu.\\

Daňová úprava svěřenských fondů, by se dala nejspíše shrnout následovně.

\subsection{Další systémové nedostatky}

\subsection{Možnosti použití svěřenského fondu k obcházení věřitelů}
\subsection{Rozdílné pojetí vlastnictví}
\subsubsection{Ležící pozůstalost}

\subsection{Zákonné nejasnosti změn svěřenských fondů před a za jejich trvání}

\subsection{Nezávislost správce}

\subsection{Návrhy na změny České právní úpravy}
\subsubsection{Zrušení svěřenského fondu}
\subsubsection{Zrušení nebo omezení institutu nepominutelného dědice}
Se zachováním určité ochrany, inspirovat se můžeme z Kanadské právní úpravy.
\subsubsection{Změna institutu nepominutelného dědice, absolutní pořizovací vůle zůstavitele}
Svěřenský fond jako způsob vydědění, oddělení institutu nepominutelného dědice od institutu svěřenského fondu, ochrana by se tedy nevztahovala na případy, kdy je mezigeneračního převodu majetku docíleno pomocí svěřenských fondů. Tomuto je, dle mého názoru, současná právní úprava nejblíže, je ale nicméně stále potřeba výslovné úpravy.
\subsubsection{Změna institutu svěřenského fondu, ochrana nepominutelných dědiců}
Součást pozůstalosti, nebo ochrana nepominutelných dědiců již v rámci pořízení.

Návrhy jsem vymezil v rámci obrázku na ploše obrazovky.

\subsubsection{Návrhy na změny vedoucí k vyřešení ostatních nejasností svěřenského fondu}

\newpage

\section{Závěr}

Shrnutí toho, co práce obsahovala. Následovat bude shrnutí praktické části práce, tedy že v tato úprava chybí a zároveň není ustálená rozhodovací praxe soudů. Zhodnotit návrhy řešení a nastínit, zda by mohli pomoci, nebo ne.

% Konec hlavní části práce, následují zdroje
\newpage
\thispagestyle{Contents}
\section*{Zdroje}\markright{ZDROJE}
% Counter definition pro přidání čísla ke zdrojům v obsahu práce, možná číslo odeberu
\newcounter{SecZdroje}
\setcounter{SecZdroje}{\thesection}
\addtocounter{SecZdroje}{1}
\addcontentsline{toc}{section}{\theSecZdroje \hspace{1,7 mm} Zdroje}
% Zde je definováno jak budou vypsány zdroje, přesná specifikace je obsažena v mappingu
\printbibliography[type=misc,heading=subbibliography,title={Online zdroje}]
\printbibliography[type=book,heading=subbibliography,title={Knižní zdroje}]
\printbibliography[type=article,heading=subbibliography,title={Články}]
\printbibliography[type=proceedings,heading=subbibliography,title={Zákony}]	
\end{document}