\documentclass{article}

% Language definition
\usepackage[utf8]{inputenc}
%\usepackage[czech]{babel}

% Package definition and folder with images
\usepackage{graphicx}
\graphicspath{ {./Obrazky/} }

% Package difinition
\usepackage{fancyhdr}
\usepackage{xcolor}
\usepackage{titling}
\usepackage{array}
\usepackage{todonotes}
\usepackage{tabularx} % Pro vytvoření tabulky, která bude mít full page width, musím zkontrolovat, zda tento package někde v práci něco nerozhodil

% Base header definition
\setlength\headheight{26pt}
\lhead{\includegraphics[width=3cm,height=\dimexpr \headheight-\dp\strutbox]{newcevro}}
\rhead{\small{\leftmark}}

% Bibliography definition
\usepackage{biblatex}
\addbibresource{Zdroje.bib}

% Footnote package definition
\usepackage{footnote} %Package použitý k tomu, aby byli citace uvnitř float elementů pod čarou
\makesavenoteenv{figure} %Nastavení toho, aby byli citace uvnitř figure pod čarou

% Package definition
\usepackage{xcolor}

% Different language for contents and figures
\renewcommand\contentsname{Obsah}
\renewcommand\listfigurename{Seznam obrázků}
\renewcommand\figurename{Obrázek}

% Special header style definition for sections with long text in header
\fancypagestyle{smallertextinheader}{ % Dokončit styl s menším textem v hlavičce, již je dokončeno
   \fancyhf{}
   \fancyhead[L]{\includegraphics[width=3cm, height=\dimexpr \headheight-\dp\strutbox]{newcevro}}
   \fancyhead[R]{%
   \parbox[b]{\dimexpr \textwidth-3cm-\columnsep}%
   {\small\uppercase\leftmark}}%
   \fancyfoot[C]{\thepage}
}

% Special header for contents only
\fancypagestyle{Contents}{ % Dokončit styl s menším textem v hlavičce, již je dokončeno
   \fancyhf{}
   \fancyhead[LE,LO]{\includegraphics[width=3cm, height=\dimexpr \headheight-\dp\strutbox]{newcevro}}
   \fancyhead[RE,RO]{\small{\uppercase{\rightmark}}}
}

% Basic page style definition
\pagestyle{fancy}

% Pretitle definition
\pretitle{
	\begin{center}
	\LARGE
	\includegraphics[width=10cm,height=3cm,keepaspectratio]{newcevro}
}
\posttitle{\end{center}}

% Document body --------------
\begin{document}

% Title definition for second page
\title{Svěřenský fond jako nástroj transferu majetku mezi generacemi}
\author{Dominik Bálint}
\date{01.12.2019}

% First page of my bachelors work
\pagenumbering{gobble}
  \thispagestyle{empty}
  \begin{center}
  \includegraphics[width=10cm,height=3cm,keepaspectratio]{newcevro} \\
  \end{center}
  \vspace{15mm}
  \begin{center}
  {\Large Vysoká škola Cevro Institut} \\
  \vspace{15mm}
  {\Large \textbf{Svěřenský fond jako mezigenerační nástroj převodu majetku}} \\
  \vspace{15mm}
  {\Large Dominik Bálint} \\
  \vspace{15mm}
  {\Large Bakalářská práce} \\
  \vspace{49mm}
  {\Large \textbf{Praha 2019}} \\
  \end{center}
  
 % Second page information 
\newpage
  \thispagestyle{empty}
  \maketitle
  \begin{center}
  {\Large Katedra veřejného práva} \\	
  \vspace{15mm}
  {\Large \textbf{Studijní program:} Veřejné právo} \\
  {\Large \textbf{Studijní obor:} Právo v obchodních vztazích} \\
  {\Large \textbf{Jméno vedoucího diplomové práce:} } \\
  {\Large Mgr. Ing.  Střeleček  Tomáš LL.M.} \\
  \end{center}
  
% Čestné prohlášení  
\newpage
  \thispagestyle{empty}
  \vspace*{\fill}

\noindent \textbf{Čestné prohlášení} \\

Prohlašuji, že jsem předkládanou práci zpracoval samostatně, uvedl
v ní všechny použité prameny a zdroje, které jsou uvedeny v seznamu použité
literatury, a v textu řádně vyznačil jejich použití. \\

\noindent V Praze dne \today\\%{\selectlanguage{czech}\today} \\ %Změněno na české datum, možná budu muset odebrat, babel package způsobuje z nějakého důvodu error
\vspace{10mm} \\
\begin{tabular}{p{6cm}c}
& ................................................. \\
& Dominik Bálint
\end{tabular}

% Poděkování
\newpage

\thispagestyle{empty}

\vspace*{\fill}
\noindent \textbf{Poděkování} \\

	Na tomto místě bych rád poděkoval svému vedoucímu práce panu Mgr. Ing. 
Tomáši Střelečkovi LL.M. za podporu, trpělivost, spolupráci, podněty ke zlepšení a 
také za čas, který mi věnoval při vedení práce.

\vspace*{\fill}

% Page with contents 
\newpage
  \thispagestyle{Contents}
  \tableofcontents
  \listoffigures

% Part of the page I list appendixes list on
\vspace*{0.5cm}
  
\noindent{\Large{\textbf{Přílohy}}}\\

\normalsize{1\hspace*{1.8em}Vzor smlouvy o založení svěřenského fondu}

% Strana se seznamem použitých zkratek  
\newpage
 \pagenumbering{arabic}
 
\begin{center}
\section{Seznam použitých zkratek}
\end{center}

\vspace{5 mm}

\textbf{Právní předpisy}

\vspace{5 mm}

\begin{tabular}{p{3cm}p{8cm}}
\textbf{OZ} & zákon č. 89/2012 Sb., občanský zákoník	 \\
\textbf{NOZ} & zákon č. 89/2012 Sb., občanský zákoník	 \\
\textbf{CCQ} & Commission de la construction du Québec, neboli Quebecký občanský zákoník \\
\textbf{ABGB} & Všeobecný zákoník občanský	 \\
\textbf{ZŘS} & zákon č. 292/2013 Sb., zákon o zvláštních řízeních soudních	 \\
\end{tabular}

\vspace{5 mm}

\textbf{Zkratky zemí}

\vspace{5 mm}

\begin{tabular}{p{3cm}p{8cm}}
\textbf{JAR} & Jihoafrická republika	 \\
\end{tabular}

% Samotný text práce začíná zde  
\newpage
  
\section{Úvod}

Práce si klade za cíl v historickém a právním kontextu představit institut svěřenského fondu a převzetí jeho právní úpravy z právního řádu provincie Quebec.
\linebreak

\indent Dále se práce bude zabývat představením současné právní úpravy svěřens\-kého fondu, jeho možnostmi a výhodami, ukázanými na příkladu praktického využití – jako nástroj, který může substituovat, nebo být komplementem klasickému dědění a dalším způsobům převodů majetku a jeho porovnání s těmito způsoby. Dále se budu zabývat i potenciálními možnostmi zneužití a jak, a zda, jsou odstraněny obavy o zneužití svěřenských fondů v souvislosti s novelizací občanského zákoníku z roku 2018.
\linebreak

\indent Dalším bodem práce bude faktické porovnání právní úpravy dědického práva s právní úpravou svěřenského fondu. Těžištěm této časti je snaha potvrdit, nebo vyvrátit tvrzení, že právní úprava dědického práva a svěřenského fondu je vyvážená a svěřenským fondem se nedají obcházet práva dědiců. Pokud bych tedy hypotézu refrázoval jako otázku, zněla by následovně: "Dá se svěřenský fond použít k obcházení práv nepominutelných dědiců?"
\linebreak

\indent V tomto rámci se budu zabývat zejména zhodnocením právní úpravy svěřens\-kých fondů a její analýzou v souvislosti s ochranou dědiců, vyložení potenciálních problémů současné právní úpravy v rámci českého právního řádu, tzn. potenciální zneužití, v souvislosti s právem nepominutelných dědiců, které mohou vzniknout v případě souběhu dědického řízení a založení svěřenského fondu a dalšími spatřenýmy nedostatky.
\linebreak

\indent Tato část je nesmírně důležitá, neboť věřím, že s ohledem na nastupující novou generaci, která se již narodila do svobodného kapitalistického tržněeko\-nomického systému, čeká svěřenský fond období, ve kterém o něj bude velký zájem a začne se hromadně využívat, s čím souvisí i potenciální možnost rozšíření důvodů a účelů založení svěřenského fondu a jeho postupné přibližování angloamerické právní úpravě. Další faktorem mnou předvídaného budoucího rostoucího zájmu o tento institut je stárnoucí vrstva populace, která po pádu soci\-alismu naakumulovala určitý majetek a otázka jeho předání nastupující generaci je pro ně důležitá. Je tedy nezbytné ujasnit si určité otázky vyplývající z recepce\ tohoto právního institutu. \\
%\linebreak

\indent V závěru práce budou vyloženy systémové nedostatky, založené na historickém kontextu, praktické části této práce a komparaci s právní úpravou tohoto institutu v jiných zemích, v případě, že je naleznu, a následovat bude návrh řešení nalezených problémů. Přílohu práce tvoří mnou navrhnutá smlouva o zřízení svěřenského fondu za účelem předání rodinného majetku.

%{\color{green}\noindent===================Comment===================} \\
%Přidat hypotézu.\\
%Dále by bylo vhodné poznamenat do úvodu, že věřím, že s ohledem na nastupující novou generaci, která již se narodila do svobodného kapitalistického tržněekonomického systému, čeká svěřenský fond období, ve kterém o něj bude velký zájem a začně se hromadně využívat. Je tedy třeba ujasnit si určité otázky vyplývající z recepce tohoto právního institutu. Dále si i myslím, že by se mohly rozšířit důvody a účely založení trustu respektive svěřenského fondu s ohledem na to, jak o něj poroste zájem.\\
%Nelze převzít institut ale právní úpravu, tedy Komparace České právní úpravy s úpravou ostatních zemí
%Je možné komparovat čskou a quebeckou právní úpravu
%Zjistit úpravu zadání co se týče porovnání anglického Trustu s českou právní úpravou
%Musím komparovat právní úpravu ne institut

%V komparaci již představím právní úpravy, tedy nemusím dále představit momentální právní úpravu svěřenského fondu

%Porovnání dědického práva.

%V tomato rámci se budu zabývat zejména tím, že porovnám právní úpravu s právní úpravou svěřenského fondu, budu se zabývat její analýzou v souvisloti s ochranou dědiců

%Hypotéza: Pokud není jako otázka - ANO nebo NE, pokud bych dal jako otázku, tak bych dopředu nepresumoval jak to je, ale dospěl bych k tomu

%V závěru může krátce popsat rozdííy mezi Anglickou právní úpravou a Českou právní úpravou

%V závěru bych mohl komparovat pro případy návrhů jak zamezit zneužití
%Ještě se podívat na poznámku ohledně bakalářské práce v Notes
%{\color{green}\noindent============================================} \\

%Kombinace mortis causa a unter vivis pujde podle me jako pohledavka k pozustalosti
%Dle paragrafu 311 je dle meho sverensky fond ucastnikem dedickeho rizeni - nepominutelni dedicove tedy budou chraneni paragrafem 1452 a budou se moci domahat sveho dedickeho naroku nejspise zalobou na vydani, toto musim overit, podivam se na jine zaloby uplatnitelne v ramci dedickeho prava
%Doplnit tu Haagskou úmluvu o uznávání svěřenských fondů, myslím, že mi tento bod vyplynul když jsem četl o možnostech zneužití svěřenských fondů v Itálii
%Napíšu, že ustanovení § 300 něco týkající se nadací, na které odkazuje svěřenský fond je platné, tzn majetek vyčleněný do svěřenského fondu je součástí dědického řízení to vyplývá i ze zákona o zvláštních řízeních soudních, ale není součástí pozůstalosti nicméně stále bude platit to, že vyčlenění majetku do svěřenského fondu nesmí omezit povinný díl nepominutelného dědice, pokusit se vnést i pohled, že přesto, že svěřenský fond nemá právní subjektivitu, je na něj částečně nahlíženo například pro fungování daní jako na právnickou osobu, toto by mohlo posloužit analogicky i pro dědické řízení, nicméně účastníci asi nebudou dle http://www.bulletin-advokacie.cz/funkcni-kategorizace-sverenskych-fondu
%Co je ale vlastně právní subjektivita, jaké požadavky musí splňovat dědic, fond může podnikat, takže může uzavírat smlouvy, má tedy částečně právní subjektivitu, proč by tedy nemohl být součátí dědického řízení

\newpage

\section{Svěřenský fond - obecně}

\subsection{Pojem svěřenství, svěřenský fond a struktura práce}
%Zamyslet se nad tím, zda jinak nepojmenovat tuto část. Přejmenováno, zkonzultovat, zda to takto dává smysl.

\indent Dne 22.3.2012 vstoupil v platnost nový občanský zákoník, který nabyl účinnosti prvním dnem roku 2014. Nový občanský zákoník s číslem 89/2012 Sb. rekodifikoval civilní právo a zavedl v českém právu několik nových institutů. Rekodifikace civilního práva se dotka i právní úpravy správy cizího majetku, v jejichž mezích bylo zavedeno několik nových právních institutů. Mezi tyto instituty můžeme zařadit také institut svěřenství, respektive svěřenského nástupnictví, a svěřenského náhradnictví, který předchozí občanský zákoník, až na jeden paragraf\footfullcite[V 40/1964, §859 je upraven zánik všech omezení vyplývajících ze svěřenského náhradnictví]{noauthor_zakon_1964}, neupravoval a také svěřenský fond, na který se práce bude především zaměřovat. \\

\indent Nově zavedená podoba institutu svěřenského fondu a obecné správy cizího majetku je pro nás sice nová, nicméně jedná se o podobný instrument, který v českém právu v minulosti v určitých formách existoval až do 1.1.1951, do zrušení svěřens\-kého náhradnictví, zvané též jako fideikomisární substituce\footfullcite[§565]{noauthor_zakon_nodate}. Omezení vyplývající z institutu svěřenského náhradnictví nicméně trvala až do roku 1964, kdy vstoupil v platnost, a zanedlouho i v účinnost, nový federální občanský zákoník pod číslem 40/1960 Sb., který úplně zrušil svěřenské náhradnictví\footnote{Viz citace 4.}. Mezi roky 1964 a 2014 lze tedy hovořit o faktickém vakuu právní úpravy svěřen\-ských institutů v Československu,České a Slovenské republice a následně v České republice, respektive zákazu používání těchto institutů.\\

\indent Práce je zaměřena na toto období, na širší historii a vznik těchto institutů i na současnou právní úpravu a možnosti z ní vycházejících. Z tohoto důvodu se tedy sestává ze tří dílčích částí, z nichž každá saturuje informační potřebu nezbytnou pro další část práce. První část práce je čistě teoretická, pojednává o historii a pojmových znacích svěřenského fondu a jemu předcházejících institutů a obecnému úvodu do dědického práva, které je nezbytné pro zodpovězení stanovené hypotézy. Druhá část práce je zaměřena na praktické zkou\-mání možností svěřenského fondu zejména s ohledem na možnosti mezigeneračního převodu majetku s ním spojeného a zodpovězení nejasností z této možnosti vyplývajících, především s ohledem na institut nepominutelného dědice. Ve třetí, a tedy konečné, části práce bude následovat můj pohled na tyto problémy a mnou navrhnuté potenciální řešení těchto problémů. 

%Považuji za vhodné tyto jednotlivé části více popsat a činím proto tak na další straně.\\

\newpage

{\Large Teoretická část}\\

S ohledem na výše uvedený fakt, je v první řadě nezbytné obecné představení tohoto nově zavedeného institutu v teoretické rovině, které je detailněji popsáno v následující podkapitole a bude sloužit jako solidní odrazový můstek pro vylož\-ení možností, které nám svěřenský fond přinesl, stejně tak, jako jeho potenciálních problémů a nejasností. \\

Kapitola a jednotlivé podkapitoly následující rovněž ve zkratce představí institut svěřenství od dob Římské říše, následovat budou popis vývoje svěřenství během období monarchie a po vzniku samostatného Československého státu, až po zrušení svěřenství v roce 1964. V poslední části bude popsáno znovuzavedení institutu svěřenství a svěřenského fondu novým občanským zá\-koníkem v roce 2014, s důrazem na převzetí právní úpravy svěřenského fondu z právního řádu provincie Quebec, kterému bude, pro lepší pochopení české právní úpravy, předcházet popis a historie trustu, s důrazem na současnou úpravu v právním řádu provincie Quebec. Tato kapitola bude vycházet především z Komentovaného znění části nového občanského zákoníku upravující problematiku správy cizího majetku od Jaroslava Svejkovského a kolektivu. Takto vyložené informace jsou, s ohledem na jejich přímou návaznost na institut svěřenského fondu, zásadní pro pochopení historie a funkčnosti svěřenských fondů. Prioritně bude práce klást důraz na svěřenský fond jako na institut, který je možné použít k mezigeneračnímu převedení majetku. V návaznosti na znovuzavedení institutu svěřenství bude následovat základní popis novelizace právní úpravy svěřenských fondů z roku 2018, který bude dále podrobněji objasněn v druhé, tedy praktické, části práce. V neposlední řadě bude součástí této části i stručný úvod do českého dědického práva, který je důležitý pro pochopení kolizí, které mohou mezi právní úpravou dědictví a právní úpravou svěřenských fondů vzniknout.\\

%které je možné používat pro velké množství účelů.

{\Large Praktická část}\\

Po historickém úvodu do problematiky bude následovat plynulý přechod k současné právní úpravě svěřenských fondů a bližšímu vysvětlení novelizace Občanského zákoníku z roku 2018, který mimo jiné přinesl povinnou evidenci svěřenských fondů. Poté se posunu k praktickému využítí instututu svěřenského fondu pro výše zmíněné účely mezigeneračního převod majetku. V této části také zhodnotím výhody, respektive cesty, oproti jiným možnostem převodu majetků za stejným účelem. V tomto rámci se také naplno projeví jakási nedotaženost právní úpravy svěřenského fondu v našem právním řádu, pramenící v první řadě z faktu, že svěřenský fond je obdobou institutu trustu, který je především používaný v systémech common law, ve kterých neexistuje mnoho právních institutů vlastní našemu systému civil law. Přes fakt, že institut svěřenského fondu byl přejat z quebeckého občanského zákoníku, který si zachoval charakter systému civil law, tak i tato norma neupravuje mnoho institutů nám známým, stejně tak i na mnoho institutů nahlíží jinak. Z tohoto vyplývá, že i po šesti letech účinnosti nového občanského zákonníku, a tedy i možnosti zakládat svěřenské fondy, přetrvávají nejasnosti ohledně kolize úpravy svěřenského fondu s jinými právními předpisy. Jednu z těchto kolizí představuje i nejasnost mezi založením svěřenského fondu pro případ smrti a právem nepominutelných dědiců. Na tento rozkol existuje mnoho zásadně odlišných názorů a k dnešnímu dni neexistuje dokonce ani žádné precedenční soudní rozhodnutí, které by tuto otázku zodpovědělo. Na toto a další nejasnosti, které vyplývají z recepce institutu vlastního systémům common law do našeho právního řádu, bude klást důraz část následující po části popisující současnou právní úpravu - Systémové nedostatky a návrhy na řešení. Nejasnosti se pokusím, v historickém kontextu, s ohledem na cizí právní úpravu a zhodnocení odborníků, vyložit a nalézt na ně vhodnout odpověď nebo řešení, které bude představeno v následující, tedy závěrečné, části. \\

{\Large Závěrečná část - Návrhy na změny České právní úpravy a návrh smlouvy}\\

%Návrh smlouvy - pokud nebude čas, tak tuto část odstraním

Po analýze a nalezení nedostatků v rámci české právní úpravy je vhodné se nad těmito problémy hlouběji zamyslet a navrhnout jistá řešení. Daná řešení bych chtěl obsáhnout právě v této části práce. Tedy po zanalyzování a popsání jednotlivých nedostatků a vyložení jednotlivých názorových proudů na tyto nedostatky z předchozí části. Přílohou k práci bude v samém závěru mnou navrhnutá smlouva o zřízení svěřenského fondu, který bude zřízen za účelem předání rodinného majetku. Smlouva bude sepsána s ohledem na problémy v této práci vyložené tak, aby pro vznik svěřenského fondu podle této smlouvy nepředstavovaly žádnou překážku.

\newpage

\subsection{Úvod do historických počátků svěřenství a\\ svěřenských fondů}

\indent Aby bylo možné dále hovořit o institutu svěřenství, respektive svěřenského fondu a jeho předlohy ve formě trustu v CCQ, je třeba si nejprve vymezit co institut svěřenství znamená, představuje a k jakému účelu slouží, tedy definovat pojmové znaky a jeho konstrukci.\\

Samotnou definici českým zákonodárcem přijatého trustu dobře shrnuje Bar\-bora Bednaříková, ta institut svěřenstí definuje jako: "...takový vztah, kdy v principu jedna osoba svěří svůj určitý majetek druhé osobě ve prospěch osoby třetí."\footfullcite[Kapitola Úvod, str. IX]{bednarikova_barbora_sverenske_2014} K této definici bych dále doplnil, že takto svěřený majetek nemusí sloužit pouze třetí osobě, ale i zakladatelem určenému veřejněprospěšnému účelu.\\

Tato definice se do značné míry ztotožňuje s definicí čistě common law trustu z knihy Underhill and Hayton, Law of Trusts and Trustees, která trust definuje následovně:\\ % což je, s ohledem na fakt, že trust sloužil jako inspirace, logické. V daném díle je definován následovně: \\

\textit{"A trust is an equitable obligation, binding a person (called a trustee) to deal with property (called trust property) owned by him as a separate fund, distinct from his own private property, for the benefit of persons (called beneficiaries or, in old cases, cestuis que trust), of whom he may himself be one, and any one of whom may enforce the obligations."}\footfullcite{underhill_a_hayton_d_law_2010} \\
%Doplnit zdroj a pod čáru dát poznámku, že jsem daný zdroj objevil v jiné Diplomové práci, překlad udělám sám, protože nevím koho bych měl ozdrojovat.

\textit{"Trust je spravedlivý závazek, zavazující osobu (trustee - obdoba svěřenského správce), aby se starala o majetek, kterýžto vlastní jako o oddělené jmění, odlišné od jmění jím vlastněné, za účelem prospěchu osob (beneficient), kteroužto může rovněž býti i on sám, nebo kdokoliv další kdo má právo dané závazky vymáhat."}\footnote{Inspirováno překladem z práce Martina Skuhrovce, zdroj překladu není patrný.} \textsuperscript{,}  \footfullcite{skuhrovec_michal_sverensky_2018} \\

Tato podobnost není nikterak zvláštní, neboť, mimo samotné litery zákona a obou definicí, i sama důvodová zpráva k NOZ uvádní, že institut svěřenského fondu vychází z trustu, přičemž se přímo inspiroval quebeckou fiducií, tedy specifickou formou trustu.\\

%Z litery zákona, definice České autorky i zahraničních autorů jasně vyplývá to, co stanovuje i důvodová zpráva, a to jasná inspirace trustem, tedy trustem upraveným v rámci CCQ. \\

Účel takovéhoto svěření, jak je výše popsáno, ať již se bavíme o první, či druhé definici, je tedy zajištění určitého prospěchu pro třetí osobu/y. V historickém kontextu tento prospěch spočíval především v poskytování užitků z vyčleněného majetku a/nebo jeho převedení ve prospěch třetí osoby, nicméně v rámci současné právní úpravy lze hovořit o mnoho dalších účelech svěřenství, ke kterým se v průběhu práce také blíže dostanu. Jedná se jak o svěřenské fondy založené za veřejně prospěšným, tak i za soukromým účelem.\\

\indent V kontinentálním právu se instituty, které by se svým obsahem shodovaly s výše popsaným vymezením, začaly používat již v antickém Římě ve formě tak\-zvaného fideikomisu (latinsky \textit{fideicommissum}, pochází ze slov \textit{fīdē} v překladu "důvěra" a \textit{commissum} v překladu "svěřené", dohromady též jako "tvé důvěře svěřuji"\footfullcite{noauthor_fideicommissum_nodate}). V právu anglosaském, které se vyvíjelo odděleně od dob konce římské nadvlády a především od dob heptarchie v 7. století našeho letopočtu\footfullcite{jan_kuklik_dejiny_2011}, poté ve formě trustů (anglicky \textit{trust}, v překladu "důvěra", již zde lze i z čistě lingvistického hlediska pozorovat podobnost mezi trustem a fideicommissem, významy těchto slov jsou stejné: "důvěře svěřuj" a "důvěra"), které jsou ale obdobně ve své podstatě potomky římských fideicommissů, nicméně ke vzniku trustu jako takového došlo v 16.století, jako důsledek Zákona o závětích\footnote{Dějiny angloamerického práva - strana 86.}. Ve většině zemí západního světa, používajících obou právních sys\-témů, je tato forma správy cizího majetku dlouhodobě používána k uchování, převodu, správě, zachování celistvosti, nebo použití k dalším soukromým, či veřejným účelům\footfullcite[Kapitola Úvod, str. IX]{bednarikova_barbora_sverenske_2014}, v různých formách. Některé země tradičně využívají institutu trustu, například Irsko, jiné země právní úpravu trustu známou především ze zemí Common Law\footnote{Země používající anglosaský systém práva.} přizpůsobili kontinentálnímu právnímu myšlení. Mezi tyto země lze zařadit Lichtenštejnsko (Treuhandverhältnis) a Lucembursko (Fiducie). Další země se rozhodly funkci trustu nahradit jiným institutem - Německo (Treuhand) a Rakousko (Privatstftungsgesetz) \footfullcite{trust_2014}. \\

\indent V současné právní úpravě, tedy v Občanském zákoníku, jsou v paragrafech 1448 a 1449 fakticky vymezeny účely svěřenského fondu, těmito účely je účel soukromý a účel veřejný\footfullcite{noauthor_zakon_2012}. Soukromý svěřenský fond slouží ku prospěchu určité osoby, nebo na její památku, z tohoto ustanovení je možné dovodit, že svěřenský fond založený za soukromým účelem slouží k ochraně, uchování, rozmnožení, správě a převodu vyčleněného majetku, práce tedy bude primárně zaměřena na svěřenské fondy založené za tímto účelem, protože právě tyto svěřenské fondy se dají použít k mezigeneračnímu převodu majetku. Co se veřejného účelu týče, hlavním účelem nemůže být dosahování zisku nebo provozování závodu, svěřenský fond založený za veřejným účelem tedy slouží k veřejnému, především socioekonomickému, prospěchu, do kterého důvodová správa k § 1448 - 1452 řadí účely kulturní, vzdělávací, vědecké, náboženské a další obdobné účely.\footfullcite{jaroslav_svejkovsky_sprava_2015} \\

Stejně jako o základních účelech svěřenských fondů, tedy veřejném a soukro\-mém, lze hovořit i o jednotlivých typech svěřenských fondů používaných v praxi. Pro lepší pochopení je vhodné začít s trustem v common law. Trust jakožto velice flexibilní právní institut by měl mít systém, který dovoluje jisté setřídění individuálních trustů dle jejich účelu a dle potřeb, ke kterým jsou tyto jednotlivé skupiny využívané. Tento systém umožňuje rozdílný přístup k individuálním skupinám lidí jak ze strany úřadů, tak i v rámci samotné funkčnosti trustu. Toto rozdělení trustů je také vhodné pro účely zakladatele (settlor), správce (trustee) a nakonec i obmyšleného (beneficiary), neboť jim poskytuje jistou strukturu, která zjednodušuje rozhodovací proces související s volbou vhodného trustu.\\

Inspiraci je vhodné čerpat například v Anglii, kde trusty mají sofistikovaný systém, jehož jednotlivé části jsou vhodné k různým potřebám a kde slouží jako jeden z pěti možných způsobů jak držet majetek\footfullcite{andrew_burrows_english_2013}.\\

\begin{figure}[h]
\centering
\includegraphics[width=15cm,height=5cm,keepaspectratio]{English_UK_Structure.png}
\caption{Struktura trust systému v Anglii}
\label{fig:struktura}
\end{figure}

Krom výše načrtnuté struktury je možné trusty dělit dle dalších kategorií, living, nebo testamentary, revocable, či irrevocable, funded, či unfunded a dále taky je možné rozdělení dle typů trustů, které jsou používány v praxi. Zde se může jednat například o Insurance Trust, A Spendthrift Trust, Blind Trust a podobně. Vzhledem k tématu práce není třeba jít v tomto směru do detailu, představení tohoto rozdělení v rámci anglického trustu je důležité k představení typů svěřenských fondů v českém právu.\\

\newpage

Za zmínku stojí pouze rozdíl mezi Express a Implied trustem. Charakteristickým rysem express trustu je dle Judith Bray to, že přímo zakladatel (settlor) stanoví účel trustu a vyčlení do něj majetek\footfullcite{judith_bray_students_2012}. Naopak Implied Trust není tvořen z rozhodnutí zakladatele, ale ze zákona.\footnote{Tamtéž.}\\

Český zákonodárce nešel v rámci rozdělení trustu takto do hloubky, nýbrž zavedl pouze 4 základní kategorie svěřenských fondů, které jsou již představeny výše. Jedná se tedy o svěřenský fond založený \textit{inter vivos} a \textit{mortis causa}. Druhé členění pak spočívá v účelu, pro který byl svěřenský fond založen, tedy soukromý, nebo veřejný. S dalším rozlišením, mimo zákonné dělení, se setkáv\-áme v praxi, kdy jsou svěřenské fondy členěny podle způsobu použití nebo podle typu\footnote{Vhodný inspirační zdroj pro výčet individuálních typů svěřenského fondu poskytují jednotlivé společnosti zaměřující se na poskytování služeb související se zakládáním a správou svěřenských fondů.}. Základními typy svěřenských fondů mohou být osobní, rodinný, obchodní, charitativní, investiční. Pro účely této práce se budu zaměřovat především na svěřenský fond rodinný, testamentární a ochranný\footnote{Bližší rozdělení poskytuji v kapitole Typy svěřenských fondů.}.\\

Následující kapitoly možná budou působit vzhledem k názvu a zaměření práce poněkud extenzivně, ale považuji za důležité zaměřit se v mé práci i na historii svěřenských institutů a správy cizího majetku, neboť k pochopení, vysvětlení a popsání mnoha sporných otázek, které si dnes mnozí právníci pokládají, je nutné znát historické souvislosti a kontext vývoje právní vědy na našem území, neboť současná právní úprava se těmito tradičními právními předpisy v značné míře inspirovala\footfullcite[str. XIX-XXI]{svestka_j_obcansky_2014}, proto je třeba se k zodpovězení těchto otázek občas vydat až k samotným kořenům tohoto institutu. Toto bude důležité především v částech zaměřujících se na rozdíly v pojetí vlastnictví a v otázkách dědického práva.\\

%s ohledem na svěřenské fondy

\newpage

\section{Správa cizího majetku - historie}

Počátky svěřenských institutů, které dnes známe v podobě trustů, svěřenských fondů, či jiných obdobných institutů\footnote{Především instituty jiných evropských zemí, které pojali institut trustu odlišným způsobem.}, je možné pozorovat již za dob antického Říma. K pochopení, co vedlo ke vzniku těchto institutů a s nimi spojených problémů, jejichž následky můžeme pozorovat i v dnešní době, bude následující kapitola obsahovat popis systému práva a dědického práva antického Říma, následovat bude představení svěřenských institutů, respektive fiduciárních s\-mluv, vzniklých v této době a následně bude popsán jejich vývoj s ohledem na změny právního systému Říma.

\subsection{Antický řím}

\textit{Ius est ars boni et aequi}, tedy právo je umění dobra a slušnosti. Tímto Ulpiá\-novým výrokem by se dalo charakterizovat římské právo\footfullcite[Strana 23]{blaho_peter_haramia_ivan_a_zidlicka_michaela_zaklady_1997}. Na tomto principu stojí i určitý právní vývoj, který proběhl v římském právu ve vztahu k funkci dědictví a postupně dal římským úředníkům podnět a hlavně důvod vytvořit i alternativní způsoby převodu majetku, do kterých by se daly zařadit právě fiduciární smlouvy, které jsme od Římanů s jistými úpravami zdědili ve formě trustů a obdobných trust-like institutů.% K procesu tohoto vývoje je však nejdříve nezbytné jisté obecné představení římského práva a specificky římské dědické právo.

\subsubsection{Praetorské právo a Ius Civile}

Oním důvodem, který dal vzniknou fiduciárním institutům byla svazůjící pravid\-la římského dědického práva stojící na úpravě občanského práva, která jsou, pokud je porovnáme, v přímém rozpopru s principy, na kterých stojí novodobý Občanský zákoník, ten se snaží ponechat co nejširší prostor svobodné iniciativě jednotlivce. Tato pravidla byla značně omezující ve vztahu k vůli zakladatele a stála na teorii projevu, která byla značně formalizovaná\footfullcite{bartosek_milan_encyklopedie_1981}. Postupem času však k vývoji římského dědického práva přispívala činnost praetorů, která jej velmi posunula směrem k našemu modernímu pojetí možností projevu vůle zůstavitele a povinností a možností vztahujících se nejen na něj, ale i na jeho dědice, či další osoby a vtělila do něj dříve právně nevymahatelné instituty jako například \textit{fideicommissum}. Kvůli činnosti praetorů a z jejich činnosti plynoucího \textit{ius honorario}, které společně s {ius civile} ovlivňovalo denodenní život každého Římana je možné na římské právo, a specificky pro účely této práce dědické právo, nahlížet jako na dvojkolejné.\\

K pochopení dvoukolejnosti římského dědického práva je v první řadě třeba vymezit rozdíly mezi \textit{ius civile} a \textit{ius honorarium}, které se v rámci dědického práva lišily způsoby stanovování dědické posloupnosti a přístupu k dědickému řízení. Pro celkové pochopení struktury římského práva soukromého je vhodné zmínit a popsat i další části, kterými jsou \textit{ius gentium} a \textit{ius naturale}. \\

Soukromé právo se sestávalo ze 4 částí, pro jejich vysvětlení je vhodné zkombinovat práci dvou významných římských právníků Ulpiána a Gaiuse. Můžeme přitom vycházet z \textit{Corpus Juris Civilis}, jakožto souhrného občanského zákonníku vydaného ve Východořímské říši za vlády císaře Justiniána a specificky ze sbírky děl římských právnků s názvem \textit{Digesta} vydaného roku 530 našeho letopočtu. \\

%\newpage

Ulpián píše následující: \textit{Ius naturale est, quod natura omnia animalia docuit: nam ius istud non humani generis proprium, sed omnium animalium, quae in terra, quae in mari nascuntur, avium quoque commune est}, což lze volně přeložit jako: "Přírodní řád je řádem, kterůmu příroda učí všem živým tvorům, tento řád nejen že není člověku cizí, ale ovlivňuje všechny tvory ať již pochází ze země, moře, či jsou ptáky."\footfullcite[Dig. 1.1.1.3, Ulpianus 1 inst., vlastní překlad]{noauthor_digest_nodate}\textsuperscript{,}\footfullcite[Použito k překladu]{noauthor_digest_book1} \\

Ulpián dále říká: \textit{privatum ius tripertitum est: collectum etenim est ex naturalibus praeceptis aut gentium aut civilibus}, což lze volně přeložit jako: "Soukromé právo skládá se ze tří částí, neboť odvozeno jest buď z příkazů přírodních, příkazů národů, nebo těch vycházejíc z práva civilního."\footfullcite[Dig. 1.1.1.2, Ulpianus 1 inst., vlastní překlad]{noauthor_digest_nodate}\textsuperscript{,}\footfullcite[Použito k překladu]{noauthor_digest_nodate} \\

Gaius v knize první (Dig. 1.1.9., Gaius 1 inst.) soukromé právo rozděluje obdobě na právo civilní a právo národů: \textit{Omnes populi, qui legibus et moribus reguntur, partim suo proprio, partim communi omnium hominum.}, což lze volně přeložit jako: "Všechny národy, které se řídí zvyky a právem částečně používají práva svého a práva národů", tedy \textit{ius civile} a \textit{ius gentium}. \\

Poslední součást římského soukromého práva tvoří tvz. \textit{ius honorarium}, tedy právo tvořené Praetory, které bylo pro rozvoj římského práva zásadní. \\

\newpage

Nejobecněji řečeno, se tedy římské soukromé právo sestávalo ze 4 základních prvků, respektive práv:
\vspace{5 mm}

\begin{itemize}
\item \textit{ius civile},
\item \textit{ius honorarium},
\item \textit{ius gentium},
\item \textit{ius naturale}.
\end{itemize}

\vspace{5 mm}

%\newpage

Při ponechání \textit{Ius naturale} jako základního práva všeho stranou, by se struktura římského soukromého práva, s ohledem na popis v této kapitole, dala vizuálně vyzobrazit následovně, pro popis fungování dědického práva je důležité především \textit{ius civile} a \textit{ius honorarium}:

\begin{figure}[h]
\centering
\includegraphics[width=15cm,height=5cm,keepaspectratio]{rimskepravostruktura.jpeg}
\caption{Struktura soukromého římského práva}
\label{fig:struktura}
\end{figure}

\underline{\textbf{\textit{Ius civile}}} je právo, jehož subjekty jsou římští občané, původně vycházelo z obyčejů. Nejstarší \textit{ius scriptum}\footnote{Psané právo.}, vykládající ius civile, který je nám znám je \textbf{Zákon \MakeUppercase{\romannumeral 12} desek} (Latinsky \textit{"Lex duodecim tabularum"}) z 5. století před naším letopočtem. Dle několika dochovaných informací potenciálně existovala psaná úprava již na samém počátku republiky ve formě sbírky \textit{leges regiae}\footnote{Zákony vydávané Římskými krály.}, která byla kodifikována do takzvaného \textit{ius civile Papirianum}\footfullcite[str.31]{mousourakis_roman_2014}. \textit{Ius civile bylo} tvořeno pontifikálními interpretacemi\footnote{Zákony tvořeny Pontifiky, kteří vykládali právo.} a komiciálními zákony\footnote{Zákony tvořeny lidovým shromážděním.} a v době císařství císařskými konstitucemi\footnote{Zákony tvořeny císařem.}. Problém \textit{ius civile} však spočíval v jeho nepružnosti, která se projevovala špatným přizpůsobováním změnám ve společnosti a vývoji doby, a jednoduchého a striktního určení některých zásad\footfullcite[kapitola 1, str.3]{bednarikova_barbora_sverenske_2014}. \\

%Dle několika informací, doplnit zdroj

%\newpage

\underline{\textbf{\textit{Ius honorario}}}, neboli praetorské právo, bylo tvořeno vysokým římským úředníkem zvaným \textit{Praetor}. Úřad praetora byl vytvořen roku 367 před naším letopočtem Liciniovým zákonem\footfullcite[str.35]{blaho_peter_haramia_ivan_a_zidlicka_michaela_zaklady_1997}. Praetor řídil soudní procesy a za pomoci \textit{aequitas}\footnote{Spravedlnost a slušnost.} a \textit{bona fide}\footnote{Dobrá víra, dobrý úmysl.} mohl ovlivňovat a řídit soudní řízení a tím i vytvářet, respektive nalézat, \textit{ius honorarium}, jež rozvíjelo neobratné a zastaralé principy civilního práva, ze kterého vycházel a jehož principy vztahoval na skutkové vztahy, které původní \textit{ius civile} neupravovalo. V situacích, kdy se pro svůj nedostatek formy a obsahu z \textit{ius civile} vycházet nedalo, mohl praetor vytvořit hypotézu (\textit{Formula in factum concepta}) skutkové podstaty, která když je naplněna, má být něco vykonáno, například má být žalovaný odsouzen, tyto hypotézy nemusely být založeny na žádných existujících normách. Praetor zároveň v řízení účastníkům umožnil podávat námitku (\textit{exceptio}), na kterou dle civilního práva nebyl brán zřetel, nicméně pokud se v rámci \textit{ius honorario} daná námitka prokázala, brala se v úvahu\footnote{Tamtéž.}. Praetor zároveň vydával na začátku svého jednoročního funkčního období takzvaný edikt, který fakticky sloužil jako soupis honorárního práva a tímto byla odstraněna potřeba sepisovat \textit{Ius honorarium} ve formě zákoníků. \\

%Doplnit další informace. \\

\underline{\textbf{\textit{Ius gentium}}} je obligační majetkové právo\footnote{Například právo rodinné a dědické zůstávalo vyhrazeno zásadně civilnímu, tedy národnímu, právu a právu tvořeného praetory.}, které bylo tvořeno cizineckým praetorem za účelem poskytování právní ochrany při obchodních stycích Římanů s cizinci, případně cizinců s cizinci na území Říma. V překladu právo národů kombinovalo \textit{Ius civile}, zahraniční, především řecké, právo a usance obchodníků. Ze své podstaty se jednalo o právní předpis jednoduchý, který nekladl na právní subjekty zásadnější formální požadavky. Ve svém důsledku tedy, obdobně jako \textit{Ius honorarium} přispěl i rozvoji \textit{Ius civile}, kterému se stalo podnětem k určitému zjednodušování norem, které byli rigidní, zastaralé a svojí existencí překážely\footfullcite{kincl_j_rimske_1995}.\\

\underline{\textbf{\textit{Ius naturale}}} není v rámci římské jurisprudence vždy vykládáno stejně. Dle mého názoru nejosobitější stanovisko zastává Ulpianus, který \textit{Ius naturale} vysvětluje jako jakési nejvyšší právo působící na všechny tvory, tedy jak lidi, tak i zvířata, které jim bylo vnuknuto samotnou přírodou. V úvozovkách se tedy jedná o právo propůjčující možnost svazků mezi opačným pohlavím, plozením potomsta, či jeho výchovu. Pro téma této práce není nikterak důležité, z toho důvodu jsem ho vynechal i ve výše načrtnutém obrázku přibližujícím strukturu římského soukromého práva a dále se jím již ani nebudu zajímat, uvedl jsem ho pouze pro úplnost\footnote{Tamtéž.}.\\

%\newpage

\subsubsection{Dědické právo v dobách antického Říma}

Po vyjasnění struktury římského práva je tedy možné přesunout se přímo k popisu způsobu fungování Římského dědického práva. Tato část je důležitá pro historické pochopení vzniku svěřenských institutů ve starověkém Říme a pokračování dalšího vývoje těchto institutů, které mají svůj prvopočátek právě v Římské říši. \\

%Římské dědické právo bylo prvně zakotveno v rámci v Zákonu desek, jež dávalo římským občanům právo se domáhat svého dědického práva žalobou\footfullcite{kincl_j_rimske_1995}\textsuperscript{,}\footnote{Obdobně Barbora Bednaříková Svěřenské fondy Institut pro uchovaní a převody rodinného majetku}.

Římské dědické právo bylo prvně zakotveno v rámci \textit{Ius civile} v Zákonu \MakeUppercase{\romannumeral 12} desek, jež dávalo římským občanům právo domáhat se svého dědického práva žalobou\footfullcite{kincl_j_rimske_1995}\textsuperscript{,}\footnote{Obdobně Barbora Bednaříková: Svěřenské fondy. Institut pro uchování a převody rodinného majetku.}. Přes takovouto, dnes by se dalo říci, vyspělost tehdejšího římského práva však tento rigidní zákon kladl před zůstavitele v rámci jeho volnosti pořizovat o svém majetku i značné překážky a obdobně značně omezoval i samotné dědice. Z důvodu této rigidnosti římského civilního práva se začalo do popředí dostávat dědické právo, tak jak na něj nahlížel praetor, to ve svém důsledku zapříčinilo jeho dvojkolejnost\footnote{Ta byla zakotvená právě z důvodu výše popsaného systému práva a jejich, pro vývoj dědického práva, dvou nejdůležitějších složek, \textit{Ius civile} a \textit{Ius honorario}.}.\\

Dědické právo v Říme, vycházelo z principu univerzální sukcese\footnote{Do dědictví spadá jmění, tedy jak majetek, tak i závazky.}\textsuperscript{,} \footfullcite{blaho_peter_haramia_ivan_a_zidlicka_michaela_zaklady_1997}, což sehrálo spolu s pasivní dědickou legitimací významnou roli při tvorbě alternativních způsobů odkazu majetku zůstavitele. \\

Do doby samotného převzetí pozůstalosti byl pozůstalostí majetek ležící pozůstalostí. Ten se mohl v průběhu své existence zvětšovat, či zmenšovat. Později byla ležící pozůstalost chápána jako právnická osoba, existující po ome\-zenou dobu, dokud nebyla převzata povolanými dědici. V případě odúmrti byl majetek \textit{bona vacantia} a připadl \textit{fiscusu}, tedy státní pokladně, ta ručila za pohledávky ručila do výše nabytého dědictví.\\

Právní důsledky přijetí dědictví tedy pro dědice znamenaly, že se stal subjektem všech práv a závazků zůstavitele. Přijetím dědictví došlo ke konfuzi (splynutí) majetku zůstavitele a dědice. Dědic tak plně ručil za dluhy zůstavitele a věřitel se tak mohl dožadovat splnění takovýchto dluhů i v případě, že bylo dědictví předlužené\footnote{Tamtéž.}. Toto mohlo mít negativní dopad i na věřitele dědice. Praetor proto dle situace uděloval \textit{beneficium inventarii}, nebo \textit{beneficium separationis}.\\

V případě určení více dědiců, přičemž někteří se dědici nestali, se uvolněné podíly poměrně přirůstaly ostatním dědicům i s právním břemenem v případě jeho existence. Tato akresence uvolněného podílu proběhla i v případě, že si to dotčený dědic nepřál.\\

%\newpage

Po zůstavitelově smrti šlo dědit na základě zákona, nebo na základě testamentu, toto zároveň určovalo i dědickou posloupnost. \\

\vspace{5 mm}

Římské právo tedy rozlišovalo dvě posloupnosti:
\begin{enumerate}
\item \textit{hereditas testamentaria} - dědická posloupnost na základě testamentu,
\item \textit{hereditas legitima} - dědická posloupnost na základě zákonu.
\end{enumerate}

\vspace{5 mm}

%\newpage

\underline{\textbf{\textit{Dědická posloupnost testamentární}}}

\vspace*{5 mm}

Posloupnost zakládající se na testamentu, tedy jednostranného právního dokumentu ustanovujícího zůstavitelovi dědice, měla přednost před posloupností zákonnou a římské dědické právo stálo na principu \textit{Nemo pro parte testatus, pro parte intestatus decedere potest}. Nevyčerpání celého dědictví ze strany hereditas testamentaria, tedy nemohlo založit právní důvod pro delaci\footnote{Povolání.} intestátních dědiců\footnote{Dědicové dle hereditas legitima.}, v takovém případě dědil i zbytek pozůstalosti dědic povolaný na základě testamentu. \\

Přes fakt, že dědění na základě testamentu mělo přednost, znalo římské právo i institut nepominutelného dědice, jednalo se o osoby v blízkém příbuzen\-ství se zůstavitelem. Právo na část pozůstalosti se ve svých prvopočátcích vymáhalo takzvanou kverelou, tedy žalobou před soudem na zrušení testamentu. Až v pozdější době císařské měli nepominutelní dědici při jejich opomenutí zůstavitelem v závěti právo na dorovnání svého zákonného podílu bez nutnosti rušit testament\footfullcite[3. vydání z roku 2012, str.230]{marek_karel_redakcni_rada_casopis_2012}. Povinnost zohlednit nepominutelného dědice se mohl zůstavitel zprostit vyděděním daného dědice. Z počátku nemusel zůstavitel uvádět důvod, toto bylo změněno vydáním Justiniánských reforem, které taxativně vymezili důvod pro vydědění nepominutelného dědice \footfullcite[str.147]{blaho_peter_haramia_ivan_a_zidlicka_michaela_zaklady_1997} . \\

Z hlediska budoucího vývoje římského dědického práva a v návaznosti na to vývoje fiduciárních smluv je důležitá pouze posloupnost testamentární. Zde bylo kladeno před testátora, i přes krok vpřed, který představoval zákon dvanácti desek a obecně i pozdější iterace \textit{Ius civile}, mnoho překážek, které omezovali jeho testovací volnost. Tyto se projevovali například ve smyslu omezení volnosti určení dědiců v testamentu vycházejícího z rigidnosti římského civilního práva, například co do osob blíže neurčených, aneb \textit{personae incertae} a zapříčinilo jejich dědickou nezpůsobilost. Takovéto osoby jsou určeny jen popisem, namísto jejich jména, tedy například \textit{"dědicem budiž ten, kdo přijde první na můj pohřeb}, takovéto vymezení bylo neplatné. Původně se toto omezení vztahovalo obdobně i na pohrobky a právnické osoby\footnote{Původně mohl dědit pouze \textit{fiscus}, tedy císařská pokladna, a až později města a obce.}, tento přístup se nicméně později obrátil v jejich prospěch\footfullcite{blaho_peter_haramia_ivan_a_zidlicka_michaela_zaklady_1997}.\\

Značně formalizované a omezující římské dědické právo vycházející ve svých počátcích výhradně z \textit{Ius civile} značně omezovalo zůstavitele a stavělo jiné občany do pozice pasivní dědické nezpůsobilost (latinsky \textit{testamenti factio pasiva}) i v mnoha dalších ohledech týkajících se například pohlaví, majetku, rodinného stavu nebo typu příbuzenství\footnote{Tamtéž.}.\\

Jisté řešení skýtalo použití jedné ze tří variant dědické substituce. Tato však sloužila pouze k ustanovení náhradního dědice v případě, že se dědic nemohl, nebo nechtěl dědictví ujmout a neposkytovala tak řešení dědické nezpůsobilosti\footfullcite{kincl_j_rimske_1995}.\\

Zde se naplno projevila důležitost práva praetorského a jeho důležitost na vývoj římského práva a v návaznosti na to i dalších právních systémů vzniknuv\-ších po úpadku a následném zániku Římské říše a s tím i spojených fiduciárních institutů.\\

\newpage

\underline{\textbf{\textit{Dědická posloupnost intestátní}}}

\vspace*{5 mm}

V případě úmrtí zůstavitele, který nezanechal testament, popřípadě byl zanechaný testament neplatný, nastupovalo dědictví \textit{ex lege}.\\

%\newpage

Římské právo rozlišovalo právní postavení dědiců ve smylu povinnosti dědit na dvě posloupnosti \footfullcite[str.141]{blaho_peter_haramia_ivan_a_zidlicka_michaela_zaklady_1997}:

\vspace{5 mm}

%Římské právo tedy rozlišovalo dvě posloupnosti \footfullcite[str.141]{blaho_peter_haramia_ivan_a_zidlicka_michaela_zaklady_1997}:
\begin{enumerate}
\item \textit{heredes voluntarii} - dobrovolní dědicové, jsou povoláni k dědictví projevem vůle nebo zachováním se ve smyslu přijetí - tedy například uhrazení pohledávky zůstavitele,
\item \textit{heredes necessarii} - nutní dědicové, jsou povoláni k dědictví smrtí zůstav\-itele bez možnosti odmítnutí dědictví, tato skupina se dále dělí na další dvě podskupiny:
\begin{itemize}
\item \textit{heredes sui et necessarii} - dědici nutní a vlastní, římská rodina byla silně patriarchálního charakteru a vztahy se rozlišovaly na takzvané agnátské a kognátské. Paterfamilias, tedy otec rodiny měl moc nad členy jeho rodiny. Tito členové, kteří v době zůstavitelovi smrti spadali pod jeho moc se stali členy této podskupiny. Do vzniku \textit{beneficium abstinendi} v rámci praetorského práva něměli možnost dědictví odmítnout.
\item \textit{heredes necessarii} - jedná se o otroky, kteří byli v testamentu zároveň osvobozeni, neměli možnost dědictví odmítnout. Pokud byl nicméně v testamentu jako dědic označen otrok v moci jiného římského občana, stal se dědicem onen římský občan.
\end{itemize}
\end{enumerate}

\vspace{5 mm}

%Toto ještě ověřit.

%Pokud byl nicméně v testamentu jako dědic označen otrok v moci jiného římského občana, stal se dědicem onen římský občan. \todo{Dohledat zdroj a najít vhodné místo k doplnění v textu} \\

V souvislosti s dalším vývojem v rámci praetorského práva vyhradil dědicům praetor takzvané beneficium abstinenti, které jim umožňovalo vzdát se dědictví a přesto, že stále byli dědici dle ius civile, zabraňovalo také možnosti podávání žaloby zůstavitelových věřitelů proti těmto dědicům\footnote{Tamtéž.}.\\

S dalším vývojem římského práva se rozvíjela i intestátní dědická posloupnost, kterou vnímalo jinak \textit{ius civile}, pro které bylo rozhodující příbuzenství agnátské a \textit{ius honorarium}, pro které bylo rozhodující příbuzenství kognátské. Rozdíly mezi oběma soustavami byly smazány Justiniánovými novelami, ty založili novou posloupnost, která byla velice podobná naší moderní úpravě dědění ze zákona a jednotlivým třídám.\\

\newpage

\subsubsection{Jiná pořízení pro případ smrti}

V rámci římského práva a zvyklostí bylo možné přenechat určitou část majetku někomu, kdo nebyl dědicem, formou takzvaného legata (v překladu odkazu), jedná se o takzvanou singulární sukcesi. Povinnost převést majetek formou odkazu mohla být uložena pouze testamentárnímu dědici. Tento způsob převodu majetku byl u Římanů velice oblíbený. Smyslem legatu bylo, aby dědic po smrti zůstavitele poskytl nějakou část pozůstalosti jiné osobě. Často se však stávalo, že legatum tvořilo celou pozůstalost, což z hlediska nastavení římského práva nic nepřinášelo dědici, který nejenže neměl žádný prospěch z dědictví, ale stal se fakticky pouhým vykonavatelem závěti s z toho plynoucími obligacemi. Takováto přetěžovaná dědictví pak byla často dědici odmítána\footfullcite{blaho_peter_haramia_ivan_a_zidlicka_michaela_zaklady_1997}. Testament bez dědice však nemohl v římském dědickém právu existovat, takovýto testament se tedy stal nulitním. Další důsledek odmítnutí dědictví zahrnoval užití intestátní posloupnost a neplatnost všech dispozic v počátečním testamentu. Aby se předešlo těmto důsledkům, které byly škodlivé pro všechny zúčastněné strany, začali Římané proti tendenci ukládat dědici příliš mnoho odkazů vydávat zákony\footfullcite{stagl_lex_2018}.\\

%Následně tedy s ohledem na pojetí římského dědického práva, kdy testament nemohl platně existovat bez dědice, se stal nulitním.\\ 

Dále byl tento způsob zatížen formálními požadavky na jeho formu, z tohoto důvodu se začaly vyčleňovat i jiné postupy převodu majetku, které nebyly zatíženy takovou formálností\footfullcite{noauthor_fideicommissum_nodate}. Jeden z těchto postupů byl již výše zmíněný fideikomis, jehož splnění bylo ve svých prvopočátcích svěřeno dědici čistě na základě důvěry. Faktem však nadále zůstávalo, že fideikomis nebyl právně vymahatelný a tím pádem nebylo možné splnění zůstavitelova přání požadovat formou žaloby\footnote{Tamtéž.}.\\

%Doplnit další zdroje dle předmětu římské právo

\subsubsection{Fideikomis v antickém Římě}

V důsledku faktických omezení kladených na \textit{legatum} a obecně na římské dědické právo, ať již dle \textit{Ius civile}, nebo \textit{Ius honorario}, se tedy, již za dob Římské republiky, začal vyvíjet i jiný, zprvu neformální, institut k účelu převodu majetku po smrti zůstavitele, který byl potřebný pro specifické, \textit{Ius civile} přehlížené, situace v rámci dědění ve smyslu dědické nezpůsobilosti, kladené přísnosti na formu testamentu,

\newpage

 či situací, kdy chtěl zůstavitel vyjádřit určité jeho přání ohledně nakládání s majetkem\footnote{Takovouto možnost neposkytovalo ani civilní ani honorární právo, mohlo se jednat například o předání majetku nezpůsobilým dědicům, tedy například cizincům, či použití majetku k nějakému opakujícímu se úkonu, Gaius II, 246.}.\\

 Tímto institutem byly takzvané \textit{fideikomisy}, respektive fiduciární smlouvy. Podstata \textit{fideikomisu}, na který lze nahlížet jako otce současně používaného trustu a obdobných fiduciárních institutů používaných jako suplementy nebo substituty klasického dědění, respektive převodu majetku, spočívala v tom, že zůstavitel v testamentu žádal jím určenou osobu (především univerzálního dědice), aby po jeho smrti naplnil určité zůstavitelovo přání. Toto přání spočívalo nejčastěji v převodu části zůstavitelova majetku na jinou osobu, než na určené dědice. Jak již je poznamenáno v úvodu, slovo \textit{fideicommissum} se dá volně přeložit jako "tvé důvěře svěřuji", toto označení bylo výstižné, protože tato žádast se opravdu zakládala čistě na důvěře mezi zůstavitelem a osobou jím určenou, šlo tedy pouze o formální obligaci. Fideikomisář (rozuměj osoba v jejíž prospěch má být plněno) se tedy nemohl žalobou domáhat svého práva. Tato žádost se obvykle připojovala ke kodicilu, jež tvořil právě s odkazem a fideikomisem ustanovení pro případ smrti\footfullcite{michal_skrejpek_rimske_2011}\textsuperscript{,}\footfullcite{leopold_heyrovsky_dejiny_1910}. \\

Hlavní výhody tohoto institutu spočívaly právě v převedení majetku i oso\-bám nezpůsobilým dědit podle \textit{Ius civile} a dále možnost zřídit ho bez složitějších formálních požadavků\footfullcite{bednarikova_barbora_sverenske_2014}.\\

%Za vlády císaře Augustina byla \textit{fideicomisům} přiznána právní závaznost, Claudius později pro fideikomisi zřídil magistráta a Justinianus ve své podstatě sloučil legatum s fideikomisem\footfullcite{blaho_peter_haramia_ivan_a_zidlicka_michaela_zaklady_1997}.\\

Běžné byly univerzální fideikomisy, kterými se měl převádět celý zůstavitelův majetek. Touto formou šlo, především díky fakticky neexistující regulaci a malým nárokům na formu, obejít principy dědického práva, dědic si při tom nemohl zpočátku strhnout falcidiální quartu tak, jako tomu bylo u odkazu. Tato forma převodu majetku však nebyla možná (uskutečnitelná) bez součinosti dědice, ten musel jako univerzální dědic dědictví přijmout, což bylo v případě, kdyby měl majetek celý převést (obdobně jako u odkazu), pro dědice nevýhodné a opět by se stal pouhým vykonavatelem závěti ale se všemi obligacemi vycháze\-jícími z toho, že dědictví přijmul, dědic tedy ručil za dluhy zůstavitele, přičemž krytí ze strany fideikomisáře bylo pouze obligačního rázu. Je však pravdou, že po dlouhou dobu, po kterou nebyl fideikomis právně zakotven a nebyla mu přiznána právní závaznost\footnote{Ta mu byla přiznána až za vlády Císaře Augustiana.} mohl dědic dědictví v zásadě přijmout a neřídit se zůstavitelovým přáním vyjádřeným v rámci fideikomisu.\\

\newpage

Dle Barbory Bednaříkové je možné hledat prameny takovéto konstrukce v římských \textit{commendationes}, neboli mandátech. Ty spočívaly v předání jisté věci osobě blízké na smrtelném lóži k jejímu uchování obdobně se však jednalo pouze o nevymahatelnou morální obligaci\footfullcite{johnston_d_roman_1988}.\\

%Je však pravda, že později bylo možné v rámci fideikomisu předat i určitý úkol, který měla daná osoba vykonat ve vztahu k tomuto majetku. Je zde tedy určitá paralela mezi fideikomisem a mandátem.

A právě \textit{commendationes} představovali inspiraci ke zrození konstrukce i pozdějšího fideikomisu. Ten se však od mandátu líší v tom, že umožňuje krom předání určitého majetku i předání určitého úkolu, týkajícího se tohoto majetku, fiduciáři u smrtelného lóže, který měl daný úkol vykonat vůči třetí osobě, tedy fideikomisáři. Fiduciář tedy takto označený majetek držel a spravoval a při splnění či nesplnění určité zůstavitelem uložené podmínky měl fiduciář daný nerozdělený\footnote{Snaha uchovat nerozdělený majetek a předat ho další generaci hrála důležitou roli v celém vývoji svěřenských institutů, jak bude dále vidno při popisu historického vývoje svěřenství na našem území.} majetek předat zůstavitelem určené osobě\footfullcite{bednarikova_barbora_sverenske_2014}.\\

Tento typ fideikomisu se však stále velmi lišil od našeho českého svěřenského fondu, či trustu v Québecu, tak jak jej známe dnes. Zde narážím na, dle mnohých zásadní problém přijímání trustu do evropských právních řádů, tím je pojetí vlastnictví (viz kapitola 7.9 Rozdílné pojetí vlastnictví). V době římské se totiž stále nejednalo o účelově určený majetek dle Lepaulla tak jak jej známe u nás, nebo v Québecu. Tehdejší institut by se dal mnohem více připodobnit k úpravě trustu ve Skotsku, americkém státě Louisiana, nebo v Jihoafrické republice, v těch se fiducář stává vlastníkem, avšak se značným omezením, jedná se tedy v zásadě o vlastnictví \textit{sui generis}\footfullcite{lionel_d_smith_re-imagining_2012}.\\

%Fiduciář tedy takto označený majetek držel a spravoval a při splnění či nesplnění určité zůstavitelem uložené podmínky měl fiduciář daný nerozdělený majetek předat zůstavitelem určené osobě - HOTOVO. 

\subsubsection{Legislativní činnost směřující ke změnám v odkazech\\ a fideikomisech}

Používání odkazů bylo u Římanů velice rozšířené a oblíbené. Problémem však bylo, že jimi přetěžovali dědictví, která pak byla odmítána. Bylo proto žádoucí, aby se zřizování odkazů určitým způsobem omezilo. To s ohledem na budoucí vývoj, který směřoval ke sloučení odkazů a fideikomisů ovlivnilo i samotný fideikomis.\\

\noindent\textbf{\textit{Lex Falcidia}}\\

Zákon přijatý v roce 40 př. Kr. směřoval k omezení odkazů. Nově musel dědic obdžet minimálně jednu čtvrtinu aktivní pozůstalosti ve formě takzvané falcidiánské kvarty. Zůstavitel tak mohl za pomoci odkazu odkáza maximálně 3/4 svého majetku. Tento zákon měl vést k tomu, aby dědicové byli motivováni přijímat dědictví právě z důvodu obdžení minimálně jedné čtvrtiny z celého dědictví a zůstavitelova vůle, která byla vyjádřena ve formě odkazu, byla respektována\footfullcite{blaho_peter_haramia_ivan_a_zidlicka_michaela_zaklady_1997}.\\

\noindent\textbf{\textit{SC Trebellianum}}\\

Jak již jsem však popsal výše tento institut byl stále právem neupravený. Toto se však s jeho vývojem zanedlouho změnilo neboť za dob vlády císaře Augusta byla fideikomisům přiznána právní závaznost. Následně za vlády Nera s vydáním normativního právního aktu s názvem Senatus Consultum Trebe\-llianum senátem z roku 56 našeho letopočtu, tedy již z dob císařství, bylo právně upraveno fungování fideikomisu. Tato právní úprava stanovila, že po vydání dědictví fideikomisáři mají žaloby z aktiv a žaloby věřitelů směřovat výhradně vůči němu\footfullcite[Str. 185]{frydek_mirosla_rimske_2009}. Fiduciář tedy získal exception restitutae hereditatis, která právě toto zaručovala\footfullcite[Str. 460]{noauthor_encyclopedic_1968}. Univerzální fideikomisář nesl tíži dokazu, nemohl si však srazit čtvrtinu pro sebe.\\ 

\noindent\textbf{\textit{SC Pegasianum}}\\

%https://books.google.cz/books?id=oR0LAAAAIAAJ&pg=PA460&lpg=PA460&dq=exceptio+restitutae+hereditatis&source=bl&ots=sUQwV-jwfF&sig=ACfU3U2RoXeRycRvU6f90NbZiPpO7aq6rQ&hl=en&sa=X&ved=2ahUKEwijuezSyoHpAhXxt3EKHbaRAj8Q6AEwAHoECAYQAQ#v=onepage&q=exceptio%20restitutae%20hereditatis&f=false - HOTOVO, ozdrojoval jsem výše. 

Stále však v rámci fideikomisu existoval zásadní problém, který spočíval v tom, že dědic dědictví častokrát odmítl, to mu však koneckonců zaručovala samo římské právo.\\ 

Toto časté odmítání, a problém ve smyslu toho, že tíhu odkazu nesl právě dědic, přijetí dědictví ze strany fiducáře vyřešila až Senatus Consultum Pegasiana. Ta rozšiřuje na fideikomisy ustanovení Lex Falcidia, dědic měl tedy nyní možnost strhnout si z dědictví falcidiánskou kvartu, tedy čtvrtinu převáděného majetku\footfullcite{michal_skrejpek_rimske_2011}, jelikož tvůrci chtěli, aby dědic přijímal dědictví dobrovolně. Tvůrci SC tedy zavádějí také povinou 1/4 dědictví pro dědice, s tím, že pokud není zaručena zůstavitelem je možné, aby si ji dědic sám podržel pro sebe. V případě toho, že by se zdráhal dědictví přijmou, je možné k přijetí a následnému vydání celého dědictví donutit, bez nebezpečí placení pozůstalostních dluhů.\\

Fideikomis s takto vymezenými možnostmi začal být v průběhu času v praxi využíván pro rozšíření možností projevení vůle ze strany zůstavitele. Takto vyčleněný majetek se nazýval fideikomisární substituce jeho specifickým případ\-em byl právě rodinný fideikomis, kterýžto je obdobou současného typu rodinného svěřenského fondu v českém Občanském zákoníku, na nějž bude s ohledem na téma práce kladen největší důraz. Ten hrál zásadní roli v případech ochrany majetku zámožných římských rodin. Takovýto nerozdělený majetek je spravován výhradním dědicem, kterého zůstavitel vybral, a nemůže být rozprodám ani rozdělen. Jeho účelem je hmotné zabezpečení římské rodiny pro další generace\footfullcite[Strana 714 až 715]{bonfante_pietro_instituce_1932}.\\

V římském právu však existovaly i další instituty pro převody majetku, výkladem výše jsem tedy nevyčerpal všechny římským právem umožněné mož\-nosti, jednou z těchto možností představuje například darování pro případ smrti, které se také standardně mezi římskými občany používalo. Částečně bych tedy tuto část uzavřel s tím, že i přes dlouhou cestu, kterou musel fideikomis ještě ujít, aby se z něj vyvinuly trust-like instituty dnešní podoby, je zřejmé, že právě zde se začal vyvíjet tento v dnešní době, minimálně v jiných koutech světa, tak často využívaný institut.\\

%Předělat to, dát tu dolní část před tím a říct, že byli další možnosti, ale neni již moc přínosné o nich mluvit - HOTOVO.

%Obsahem tedy bylo takový přesně označený majetek držet a spravovat, případně ho při splnění či naopak nesplnění určité podmínky předat další osobě, nerozdělený (na toto slovo je potřeba klást zvláštní důraz, neb co do dalšího vývoje v budoucnu bude hrát naprosto stěžejní roli v evoluci institutů fideikomisního typu). Gaius pak ve svých Institucích taktéž poskytuje určitý návod, jakým způsobem fideikomis zřídit, kdy za vhodná slova označuje "peto, rogo, volo, fidei commito", která bychom mohli přeložit jako "žádám, prosím, přeji si, věrnosti svěřuji". Takové doporučení pak jasně podtrhuje charakter fideikomisu jakožto GAIUS Učebnice práva ve čtyřech knihách pouhé morální obligace, kterou v této době skutečně byla - HOTOVO, napsáno o stěžejní roli nerozdělování majetku. 


%Toto psát asi nebudu
%Zásadním zlom pro právní vymahatelnost a stavební kámen celého dalšího vývoje představuje případ prokonzula Lucia Lentula, sloužícího v Africe, který si zhruba roku 4 n. l. za svého fiduciáře zvolil samotného císaře Augusta. Císař povinnost svého svěřenského správcovství splnil a tento případ tak nabyl až precedenčního charakteru. Koneckonců, kam sám císař vedl, kdo jemu nižší mohl takovou povinnost nesplnit? Citace 28 Augustus dále rozhodl, že v případech, kdy fiduciář selže a nesplní svou fiduciární povinnost, mohou se osoby takovým ommitere dotčené domáhat spravedlnosti u konzulů. 

%TOTO DOPLNIT
%Kodicil tvořil v této době společně s legatem a fideikomisem 

\newpage

\subsection{Historie svěřenských fondů u nás a ve světě}

Není pochyb, že Římská říše a následnické státy (Západořímská říše a Byzantská říše) měli zásadní vliv na rozvoj dědického práva, fideikomisu a práva jako celku, nejen na svém území, ale i ve státních útvarech zbytku Evropy, do které se římské právo jako poměrně vyspělý konstrukt šířilo, a potažmo i celého světa.\\

 Právní vývoj mnoha zemí byl tedy ovlivněn římským právem přičemž právní řády mnoha z těchto zemí dnes stojí na kontinentálním právním řádu, pro který položilo základ právě římské právo.\\

 Česká republika se řadí do skupiny zemí používající, obdobně jako celý evropský kontinent, kontinentální systém práva, přijetí institutu trustu tak představovalo jistou výzvu. Je tedy otázkou proč se český zákonárce nezaměřil spíše na přijetí institutu například Německého Treuhandu, jež v rámci Německého práva řeší jak institut nepominutelného dědice, tak i problematiku odděleného vlastnictví. Věřím nicméně, že polemika nad co by kdyby nemá pro zbytek práce nikterak zásadní význam, můj pohled na tuto otázku by se tedy dal shrnout následovně: Ve skrze si trufám tvrdit, že důvod pro který se Český zákonodárce rozhodl přijmout trust vycházející z Common Law místo například Německého Treuhandu, jehož historický vývoj se mnohem více blíží Fideicomisu\footfullcite[Piotr Stec Strana 46]{cooke_modern_2003} souvisí v zásadě s Globalizací nesoucí ssebou významnou měrou mezinárodní obchod a s tím související jakousi soutěž mezinárodních právních řádů.\\
 
 Na první pohled by se mohlo zdát, že německý treuhand je s naším právním řádem kompatibilnější než trust, není však koneckoncům tajemství, že Německo používá, stejně jako my, kontinentální systém práva, tato zjevná kompatibilita tak není překvapující. Německé právo s ohledem na treuhand řeší jak samotný pojem vlastnictví, tak i chrání v případě dalších institucí jako nadace, nepominutelného dědice.\\
 
 Je nepochybné, že hospodářsky je trust potřebný, to koneckonců stanovuje sám zákonodárce v důvodové zprávě, neboť nadnárodní propojení obchodů a finančních transakcí vytváří na přijímání těchto institutů, analogických k anglosaskému pojetí trustu, jakýsi tlak. Profesor Rainer Kulms s tímto zmiňuje problematiku německého přístupu k trustu. To dle něho pomíjí koncepci právního vztahu vytvořeného zakladatelem inter vivos či pro případ smrti, v němž je jmění svěřeno pod kontrolu svěřenského správce ve prospěch obmyšleného či za určitým účelem. Německé soudy pak přistupují k otázce odděleného vlastnictví, k němuž má titul svěřenský správce, velmi zdrženlivě\footfullcite[Německo mezi trustem a treuhandem]{trust_2014}. Čl. 1 odst. 2 písm. h Nařízení o právu rozhodném pro smluvní závazkové vztahy (Řím I) dokládá, že trusty a vztah zakladatele, svěřenského správce a obmyšlených nelze jednoduše zařadit do závazkového práva\footnote{Tamtéž}\textsuperscript{,}\footfullcite{noauthor_narizeni_nodate}. Další argumenty pro recepci trustů do civil law systémů poskytuje Irina Gvelesiani ve své práci na téma "Treuhand" and "fiducie" (Terminological Problematics), mezi nimiž například:\\
 
 \begin{itemize}
 \item \textit{"the globalization of the legal practice has exhibited an overall tendency of unification and harmonization of legal systems of the world contries",}	
 \item \textit{"the dominance of U.S. and English law firms in complex transactions including non-trust jurisdictions...",}
 \item \textit{"all mature legal systems have catered in various ways to the same needs that have promoted and informed the developement of the common law trust: division between the economic value of assets and their holding and management..."}\footfullcite{gvelesiani__2016}
 \end{itemize}

Je tedy zřejmé, že s ohledem na soutěž právních řádů a použití k oboum institutům vychází trust lépe a nezbívá nic jiného, než pochválit českého zákonodárce, že při jeho přijímání hleděl směrem k trustu. Obdnobně by se dalo hovořit o faktoru ekonomického přínosu. Samotný inspirační zdroj ve formě Quebecké fiducie už pak představoval jasné završení procesu jeho vybírání pro rekodifikační komise, o čem ostatně budu hovořit v kapitole Recepce právní úpravy Svěřenského fondu z právního řádu provincie Quebec. Ohledně kompatibility obou institutů obdobně nemá treuhand nikterak navrch viz kapitola Nejasný výklad ochrany nepominutelných dědiců.\\

Samotný historický vývoj fiduciárních institutů na našem území je v obdobích dalece předcházející snaze české rekodifikační komise o nalezení vhodného inspiračního zdroje právní úpravy trustu a jeho přijetí také velice důležitý a svědčí o tom, že instituce obdobné svěřenskému fondu nejsou v Česku ničím zas tak převratně novým. Pojďmě si ho tedy ve zkratce představit.

%Samotný hostorický vývoj fiduciárních institutů na našem území v jehož průběhu jsme se dostali až k dnešní uzákoněné formě svěřenského fonduPo krátkém objasnění proč český zákonodárce pohlédl zrovna jako na trust jako inspirační zdroj pro českou právní úpravu
 
 %Napsat že jsem kontinentální řádem a proto je dobré podrobit analýze respektive osvětlit historický vývoj na našem území, aby bylo jasné proč jsme přijali v podstatě trust, který je vlastní právním řádům common law a místo něho jsme nepřijali spíše nějaký systém, který vychází přímo z římského práva a tedy potažmo i z římského fideicomisu.
 
 %Popsat to, že římské právo mělo například vyřešen problém, který nás momentálně s ohledem na svěřenský fond trápí v Českém právu a to vyplacení povinného dílu nepominutelnému dědici. Dědic si totiž mohl strhnout falcidiánskou quartu. 
 
 % a jsou založeny na základě kontinentálního systému právapřičemž mnoho z těchto zemí stojí na kontinentálním právním řádu, pro který položilo základ právě římské právo.\\

%Za vlády císaře Justiniána kdy bylo římské právo a římská právní kultura na svém vrcholu, na naše území začaly přicházet první slovanské kmeny, které poté dali v 9. a 10. století vzniknout základu Českého státu. Celkový historický vývoj na našem území byl, dalo by se říci až do pádu socialismu a několika následujích let, poměrně turbulentní a s tím šel ruku v ruce i vývoj právní kultury a práva jako takového. \\

%Co se našeho území týče, tak

%Co se historie svěřenský fondů týče, tak \footnote[4]{Zde bude nějaká informace}\\
%---Koment\\
%Doplnit úpravu za dob království/monarchie a první republiky a vývoj svěřenských institutů jinde ve světě, držet se struktury popsané v druhém paragrafu v kapitole druhé, to znamená popsat i vznik trustu.
%Dále popsat zrušení svěřenských institutů v období první republiky a úplné zrušení po druhé světové válce.\\
%---\\

\subsubsection{České území}

Za vlády císaře Justiniána kdy bylo římské právo a římská právní kultura na svém vrcholu, na naše území začaly přicházet první slovanské kmeny, které poté daly v 9. a 10. století vzniknout základu Českého státu. Celkový historický vývoj na našem území byl, dalo by se říci až do pádu socialismu a několika následujích let, poměrně turbulentní a s tím šel ruku v ruce i vývoj právní kultury a práva jako takového a s ohledem na to i svěřenských institutů na našem území. \\

Toto období lze také charakterizovat nízkým zájmem o zachování rodinného, respektive rodového, majetku, samotné vlastnictví majetku a jeho udržení vycházelo zásadně z rodových tradic. První zásadnější institut, který přineslo období feudalismu, a který lze přirovnan ke svěřenskému fondu, představoval takzvaný rodinný nedíl, který byl využíván šlechtou a představoval společné a nedílné rodinné vlastnictví. Princip rodinného nedílu spočíval v tom, že po smrti zůstavitele byl majetek v rodinném nedílu převeden na nejstaršího potomka. Vlastník majetku v nedílu měl volné právo s majetkem disponovat, nemohl o něm však pořídit záveť. Jak poukazuje Barbora Bednaříková, lze si tak představit, že tento institut vedl k uchování rodového majetku, takto tomu však nebylo. Kvůli tehdejší úpravě dědického práva, se mohl jakýkoliv člen z nedílu oddělit a měl právo na část majetku v tomto nedílu, samotná existence nedílu tak vedla naopak k rozdrobování majetku na místo toho, aby ho udržovala po hromadě. Koneckonců toto oddělení samo předpokládala tehdejší dědické právo a samo ho aktivně podporovalo\footnote{Z pohledu historiků zřejmě jako úmysl krále oslabit svého hlavního kompetitora, kterýmřto byla šlechta.}. V případě, že nebylo nejstaršího syna, nebo dcery, připad majetek králi.\footfullcite[Strana 24]{bednarikova_barbora_sverenske_2014}.\\

V době pohusitské si pak Česká a Moravská šlechta postupně na králi vydobyla zrušení odúmrti, právo pořídit testament však měla stále pouze s jeho souhlasem. Zrušení odúmrti zapříčinilo to, že zůstala zachována přednost nedílných příbuzných při převodu nedílu. Namísto odúmrti tedy mohli dědictví přijmout dříve oddělení příbuzní. Toto částečně zabránilo většímu dělení majetku, bylo však stále možné se z nedílu oddělit\footnote{Tamtéž strana 26}.\\

Prvním skutečným nástrojem, který by se dal připodobnit k modernímu svěřenskému fondu, v tomto případě rodinnému, byl konstituován pro rod Rožmberků roku 1493. Jeho účelem bylo zabránění problémového drobení majetku tím, že celý majetek přecházel na dědice na základě primogenitury\footnote{Právo založené na senioritě, tedy na pořadí narození.}\textsuperscript{,}\footfullcite[Valentin Urfus Rodinný fideikomis v Čechách, z Sveřenské fondy, Barbora Bednaříková]{vaclav_vojtisek_sbornik_nodate} a ostatní členové měli pouze právo na požitky z tohoto majetku, vyměřené jim podle vůle hlavy rodu\footnote{Tamtéž strana 27}.\\

I rodinný statut rodiny Rožmberků však nebyl bezchybný a existovala oprávněná obava, že by se mohli rodinní členové takovémuto faktickému vydědění vzepřít. Proto šlechta prosadila roku 1498 stavovské usnesení, které bylo zapsáno jako zákon do zemských desk. Tento zákon umožňoval vznik rodinného zřízení, které pokud je stvrzneno králem ve formě rodinného statutu, neumožňuje, mohl být povolán za dědice pouze nejstarší člen rodiny. Takovýto majetek byl nezcizitelný a platil zákaz jeho zadlužování. Zároveň bylo povinností dědice majetek držet a bylo mu znemožněno, aby se proti němu dědic mohl jakkoliv zcizením bránit. Takto se ve své podstatě proměnil rodinný nedíl v 15. a 16. století v institut svěřenství\footnote{Přičemž termín svěřenství byl již za těchto dob nezřídka kdy používán.}, který byl i kodifikován v zemských deskách\footnote{Tamtéž strana 28}.\\ 

Přestože římskoprávní základ není jasně zřetelný lze shrnout, že samotná funkce rodinného nedílu a následně svěřenství přesto vychází z konstrukce a myšlenky římského fideikomisu, tedy že je majetek odevzdán jisté osobě, aby ho spravovala a ta ho v určitý čas nezkráceně předala osobě třetí. Což je jasná paralela k funkci římského fideikomisu.\\

Doba pobělohorská přinesla v rámci Obnoveného zřízení zemského zásadní změny do právního uspořádání svěřenských institutů. Zavedla možnost volného pořizování závětí v rámci dědického práva. Potřeba institutu, který by zajistil rodové udržení majetku však stále byla významná. Tato potřeba se postupně vyvinula v průběhu 17. století a dala vzniknout rodinnému fideikomisu. Rodinný fideikomis šlo zřídit pouze se souhlasem krále. V počátcích bylo možné konstruovat perpetuální rodinné fideikomisy, novela 159 však omezila jejich trvání maximálně na čtyři generace. Majetek vyčleněný do rodinného fideikomisu přecházel na předem omezenou dobu do správy zakladatelem určené osoby, která vykonávala správu v souladu s instrukcemi zakladatele, pro tento vyčleněný majetek byla zaručena nedělitelnost a nezcizitelnost. Rodinný fideikomis sloužil šlechtě, měšťani mohli využít fideikomisární substituce.\\

%Ještě napsat, že se začalo pohlížet na povinný dědický podíl, stránka 49 z Barbory Bednaříkové.

Další vývojový krom přišel s vydáním Všeobecného občanského zákoníku, takzvaného ABGB. Ten obsahoval instituty, které se víceméně blíží současnému pojení svěřenských institutů v Novém občanském zákoníku. Ten v paragrafech 604 až 646 obsahoval ustanovení upravující obecnou substituci (v NOZ je jeho obdobou svěřenské náhradnictví), svěřenské náhradnictví (v NOZ je jeho obdobou svěřenské nástupnictví) a rodinný fideikomis (v NOZ je jeho obdobou svěřenský fond)\footfullcite{noauthor_9461811_nodate}.\\

%Konstituován bych možná měl přepsat na strukturován

%která sama předpokládala vydělení

\subsubsection{Vývoj v Anglii}

Anglie jakožto kolébka angloamerického právního systému dala vzniknout i samotnému institutu trustu. Trust anglického původu byl s jistými koncepčními změnami přejat právě do právního řádu provincie Quebec a ačkoliv je mladším z dvojice, položil společně s římským fideikomisem základy současného trustu a jiných obdobných institutů stojících na jeho základu.\\

Základní exkurz do historického vývoje trustu v Anglii začíná v 11. století našeho letopočtu při počátcích vývoje obecného práva, v angličtině common law.

\subsection{Recepce právní úpravy Svěřenského fondu z právního řádu provincie Quebec}

Trust si postupem času získal na mezinárodní scéně velkou oblibu. Mnoho autorů v souvislosti s tímto tvrdí, že přijetí tohoto fiduciárního insititutu je nezbytné. Stejné stanovisko zaujímá i zákonodárce v důvodové zprávě, kdy hned na začátku říká následující: \textit{"Generální úprava správy cizího majetku je hospodářsky, sociálně i právně velmi potřebná a v našem soukromém právu je její nedostatek citelný."}\footfullcite{Duvodovazprava}.\\

Obdobné stanovisko zastává i profesorka Madeleine Cantin Cumyn, která nicméně vidí benefit trust like institutů v rámci systémů civil law především v jejich business potenciálu. Společně s tím dodává, že implementace těchto intitutů může v Evropských zemích představovat velkou výzvu a být méně atraktivnější, neboť, na rozdíl od zemí používajících systém common law, jsou svázané restrikcemi v rámci svého dědického práva\footfullcite[Madeleine Cantin Cumyn, strana 14]{lionel_d_smith_re-imagining_2012} a z toho důvodu by se nemuseli používat tak výrazně k převodům rodiných majetků\footnote{A to i přes to, že historicky jak trust, tak i kontinentální fiduciární instituce byli používány zejména k mezigeneračním převodům majetku.}, jako tomu bylo většinou v zemích používající systému common law a trustu.\\

%Doplnit, že přes business vhodnost je historicky trust spajt hlavně s mezigeneračním předáním majetku a také uchováním majetku.

I na toto by se dalo pohlížet jako na jeden z důvodů, proč zákonodárce přímo neupravil určité náležitosti ve vztahu svěřenských fondů k dědickému právu. Právě v tomto ohledu zůstalo nejvíce mezer, což je vzhledem k úpravám, kterými prošly ostatní právní předpisy s ohledem na přijetí trustu zarážející\footnote{Například daňové normy.}, nelze však tvrdit, že zákonné mezery existují pouze ve vztahu k dědickému právu, nicméně jsou zde největší. Lze se domnívat, že takováto nedostatečná úprava nahrává zůstavitelově testovací svobodě, neboť zákonodárce nechtěl takto omezit možnost zakladatele pořizovat o svém jmění a znevýhodnit tak typ svěřenského fondu \textit{mortis causa} a omezit obecné použivání svěřenských fondů pouze k business účelům\footnote{Příkladem mohou být fondy kolektivního investování.}.\\

Samotný inspirační zdroj pro převzetí nové právní úpravy si český zákonodárce vybral v Občanském zákoníku provincie Quebec. Jak již napovídá sama důvodová zpráva, bylo tomu tak kvůli tomu, že si Quebecké právo zachovalo svůj výrazný charakter kontinentálního práva, jemuž tento institut common law funkčně přizpůsobilo\footfullcite{Duvodovazprava}.\\

S tímto přizpůsobením jde ruku v ruce i originalita, kterou si Quebecký právní řád přizpůsobil Trust. Zejména se jedná o typ přijatého trustu a způsob pojetí vlastnictví.\\

S přijetím Qubecké právní úpravy jsme zároveň do Českého právního řádu přejali i tuto originální formu pojetí vlastnictví. Qubecké právo, kde z hlediska snahy inkorporace common law institutu do systému práva civil law, docházelo k prolínání či přímo koalescenci mezi oboumi právními systémy a tím pádem se jevil jako vhodnější inspirační zdroj, než právní řády zemí jako je Jihoafrická republika, Skotsko či Americký stát Louisiana, které aplikují problematičtější pojetí vlastnického práva správce k vyčleňenému majetku formou vlastnictví majetku správcem s mnoha souvisejícími omezeními, kdy se tak ve své podstatě nejedná o účelově určený majetek, který vlastní sám sebe (pojetí v CCQ a Čr, tvz. Lepaullovo vlastnictví)\footfullcite[Vít Zvánovec, Založení svěřenského fondu se zlváštním důrazem na oddělení majetku]{trust_2014}, ale jedná se o vlastnictví sui generis\footfullcite{lionel_d_smith_re-imagining_2012}.\\

%Toto prolínání a úspěšná integrace common law institutu do systému práva civil law

Do Quebeckého právního řádu byl trust přejat s mírou adaptace odpovídající rozdílné právní kultuře z klasické konstrukce trustu v anglosaském právu. Do Quebeckého právní řádu tak byl přejat v zásadě pouze Express trust, jiné druhy trustů jejichž vznik stojí zásadně na principech common law jako je Resulting (Výsledné)\footnote{Chybí-li podstatná náležitost zakládacího aktu, majetek je převeden bez právního důvodu.}, nebo Constructive (Sestrojené)\footnote{Bezdůvodné obohacení.} a subsequentně instituční\footnote{Vznik je postaven na základě deklaratorního rozsudku.} a nápravný\footnote{Vznik je postaven na základě konstitutivního rozsudku.} trust nebyli přijaty, neboť se v jejich případě jedná o implicitní vznik (viz. kapitola Úvod do historických počátků svěřenství a svěřenských fondů), který je v systémech civil law vhodně řešet pomocí obligačních institutů\footfullcite[Vít Zvánovec, Založení svěřenského fondu se zlváštním důrazem na oddělení majetku]{trust_2014}\textsuperscript{,}\footnote{Viz komentář k CCQ, k přijetí trustu, specificky strana 375, odstavec 5}.\\

Je pak pouze logické, že český zákonodárce vycházejíc z úpravy trustu v CCQ\footfullcite[Strana 375]{noauthor_report_1978} přijal pouze úpravu Express trustu a jen s malými úpravami ji včlenil do našeho právního řádu v podobě svěřenského fondu.

Tuto část je možné uzavřít se závěrem, že analogickým převzetím pasáží z Québeckého občanského zákoníku, upravující trust, došlo k implementaci institutu, který není svázán podmínkamy uvedenými v dědickém právu. Toto ostatně uznávají i autoři v rámci komentářové literatury jako je například Miloš Kocí\footfullcite[Strana 1996 až 1999]{svestka_j_obcansky_2014-1}, ale najdou se i tací, kteří v takovéto transplantaci spatřují chybu a právní ustanovení svěřenského fondu vykládají ve vztahu k dědickému právu jinak, například Vlastimil Pihera\footfullcite[Strana 1186 až 1189]{spacil_j_a_kolektiv_obcansky_2013}. Nicméně ať se ztotožním s jakýmkoliv z těchto prodů, je nesporné, že svěřenský fond, hlavně tedy testamentární, otevřel spoustu nových dvěří, do kterých bych rád v průběhu práce vstoupil a představil možnosti, které se v nich nacházejí primárně ve vztahu k mezigeneračnímu převodu majetku.\\

%PŘEPSAT
%Quebecké právo si zachovalo svůj výrazný charakter kontinentálního práva, jemuž tento institut common law funkčně přizpůsobilo. Důvodová zpráva

%PŘEPSAT
%Je poměrně logické, že při pátrání po vhodném institutu zákonodárce sledoval 10
%stopy kontinentální právní kultury tam, kde se nejspíše střetávala s common law a docházelo tak k
%určité "amalgamaci". Nicméně další adeptské právní řády, Skotsko, Louisiana a JAR přinášejí
%poněkud problematičtější koncepci vlastnického práva (kdy se svěřenský správce se stává
%vlastníkem, ovšem s takovými omezeními, že se jeho vlastnické právo stává v zásadě vlastnictvím
%sui generis)11. A právě v tomto aspektu se projevuje originalita CCQ, která sahá po tzv. "Lepaullově
%vlastnictví". Smith

---Koment\\
 Následně popsat znovuzavedení těchto institutů v roce 2014, převzetí z Quebecké právní úpravy a popsat i novelu z roku 2018 - na základě stížnosti ministerstev a vrchního státního zastupitelství, odkážu poté na další kapitolu věnující se této novelizaci.
 
 Tyto instituty v mnohém navazují na starší instituty používané již za dob antického říma a později v Česku, Československu, jejich tradice a dále na to dobré, na co se ohlížíme.
 
 Existují i právní úpravy, respektive země, které kontinuálně využívali tohoto institutu.\\
 
 Pro další pochopení takto flexibilního institutu a možnosti z něho vycházejících pro převody rodiného majetku je nejprve nutné vysvětlit dvě věci. První z těchto věcí je samotná úprava svěřenského fondu, kterou obsahuje kapitola následující, věcí druhou je pak samotný základ institutu správy cizího majetku a trustu, tedy jeho struktura a základní premisy a historické souvislosti vedoucí až k podobě tohoto institutu tak, jak jí známe dnes, ať již se jedná o historické souvislosti sahající až k dobám Římské říše, nebo znovupoložení si otázek, které si kladli slavní právníci 20. století a snahu zodpovědět je.\\
 ---\\
 %Tento odstavec podle mě nedává smysl, bude nutné ho mírně upravit, chci v něm říct to, že jsem v předchozích kapitolách vyložil historické souvisloti a v další kapitole popíšu právní úpravu. S touto přípravou je možné zaměřit se na možnosti svěřenského fondu v rámci mezigeneračního majetku a i na mojí výzkumnou otázku.

%\newpage
%\thispagestyle{smallertextinheader}

%\section{Pojem svěřenský fond, úvod do problematiky}

%Text.

%S výše uvedeným je již jistě mnohem pochopitelnější kontrukce CCQ a z něho vycházejícího OZ:

Základní podobnost svěřenského fondu výcházející z konstrukce trustu respektive fiducie v CCQ se dá s ohledem na českou transplantaci shrnout v následujících ustanoveních obsažených v Pododílu 1 části upravující Svěřenský fond s odkazem na odpovídající ustanovení CCQ.\\

% a OZ z nichž je patrná zásadní podobnost obou institutů.

§1260 CCQ : "A trust results from an act whereby a person, the settlor, transfers property from his patrimony to another patrimony constituted by him which he appropriates to a particular purpose and which a trustee undertakes, by his acceptance, to hold and administer."\\

Trust vzniká na základě jenání osoby, zakladatele, který převádí majetek ze svého jmění do jiného jmění jím zřízeného, které vyčlení ke konkrétnímu účelu a kterého se akceptací ujme správce, aby jmění držel a spravoval\footfullcite[Srovnávací tabulka]{jaroslav_svejkovsky_sprava_2015}.\\

§ 1448 pak zní:
(1) Svěřenský fond se vytváří vyčleněním majetku z vlastnictví zakladatele tak, že ten svěří správci majetek k určitému účelu smlouvou nebo pořízením pro případ smrti a svěřenský správce se zaváže tento majetek držet a spravovat.\\

1265. Acceptance of the trust divests the settlor of the property, charges the trustee with seeing to the appropriation of the property and the administration of the trust patrimony and is sufficient to establish the right of the beneficiary with certainty.\\

Akceptací správcem zakladatel pozbývá majetek, správci vzniká povinnost respektovat přivlastnění majetku (účelu) a (povinnost) spravovat jmění trustu a (akceptací) je na jisto postaven vznik práva obmyšleného\footnote{Tamtéž.}.\\

§ 1448 (2):
(2) Vznikem svěřenského fondu vzniká oddělené a nezávislé vlastnictví vyčleněného majetku a svěřenský správce je povinen ujmout se tohoto majetku a jeho správy.\\

1261. The trust patrimony, consisting of the property transferred in trust, constitutes a patrimony by appropriation, autonomous and distinct from that of the settlor, trustee or beneficiary and in which none of them has any real right.\\

Jmění trustu sestávající se z majetku převedeného do trustu, tvoří jmění přivlastněné účelu, autonomní a oddělené od jmění zakladatele, správce nebo obmyšleného a takové (jmění), ke kterému nemá žádný z nich jakékoliv věcné právo\footnote{Tamtéž.}.\\

§ 1448 (3):
(3) Vlastnická práva k majetku ve svěřenském fondu vykonává vlastním jménem na účet fondu svěřenský správce; majetek ve svěřenském fondu však není ani vlastnictvím správce, ani vlastnictvím zakladatele, ani vlastnictvím osoby, které má být ze svěřenského fondu plněno. \\

V článcích 1266-1270 CCQ je pojednáno o typech, které je možné podle CCQ založit, já již jsem tento rozdíl ve zkratce zmínil v úvodu práce. Tímto rozdílem oproti CCQ je účel, za kterým lze založit v česku svěřenský fond, je jím účel veřejný a účel soukromý. V CCQ má však zakladatel možnost vybrat si ze tří účelů založení trustu, respektive fiducie.\\

\textbf{1266.} Trusts are constituted for personal purposes or for purposes of private or social utility.\\

Trusty jsou zřizovány pro osobní účely nebo pro účely soukromé či sociální\footnote{Tamtéž.}.\\

Článek 1266 CCQ stanoví, že v Quebecu je možné fiducii založit za 3 účely, a to za účelem osobním, soukromým a sociálním. Přičemž tyto se od sebe liší následovně:\\

Osobní trust je možné založit za účelem.\\

Soukromý trust je možné založit za účelem.\\

Sociální trust je možné založit za účelem.\\

Zajímavé je zmínit, že v článku 1273 CCQ se explicitně stanovuje, že trusty za soukromým a sociálním účelem mohou být perpetuity, mohou být tedy založeny na neomezeně dlouhou dobu.\\

§ 1449 (1):
(1) Účel svěřenského fondu může být veřejně prospěšný, anebo soukromý.\\

\newpage
\section{Dědické právo v České republice}

Svěřenský fond je třeba v rámci mezigeneračního převodu majetku, respektive práv a závazků, vnímat především, krom jiných, spíše okrajových forem, jako alternativu či suplement klasické možnosti převodu věci, (ať už se jedná o věc nemovitou, movitou, zastupitelnou, hromadnou etc.) tedy převodu v rámci dědického řízení, který potvrzuje soud\footfullcite[§ 1670]{noauthor_zakon_2012}.\\

Zde záleží na optice, kterou nahlédneme na svěřenský fond a na otázku, zda je svěřenský fond považován za dědice, účastní se dědického řízení a převod vyčleněné věci tedy ve svém důsledku potvrzuje soud, nebo stojí mimo toto řízení a věci vyčleněné do svěřenského fondu nejsou součástí pozůstalosti.\\

Takovéhoto klasického převodu lze ze strany zůstavitele dosáhnou různými způsoby, respektive instrumenty, které jsou upravené v rámci Dědického práva. Jedná s primárně o intestátní posloupnost ve chvíli, kdy zůstavitel o svém majetku nepořídil, popřípadě je majetek převeden na dědice v souladu s přáním, které vyjádří zůstavitel, zde se jedná o pořízení pro případ smrti. Na základě tohoto přání, nebo ze zákona, vzniká dědicům dědický titul, přičemž je možné dědit i na základě několika dědických titulů, tzn. kumulace dědických titulů. V dalším případě pak vzniká jiným osobám, v jejichž prospěch zřídil zůstavitel odkaz, pohledávka za dědici.\\

Za dědice je možné zvolit i právnickou osobu, například nadaci, ta je ale nadána právní subjektivitou a dle § 311 Občanského zákoníku se do ní vyčleňuje majetek povoláním za dědice, je tedy nepochybné, že ustanovení o nepominutelném dědici je zde plně uplatnitelné.\\

Se svěřenským fondem je to ovšem složitější a dle výše zmíněné optiky, kterou na svěřenský fond nahlížíme může, ale zároveňe nemusí být účastníkem dědického řízení a s ohledem na to se na něj buď budou, nebu nebudou vztahovat jednotlivé ustanovení dědického práva.\\

Při založení svěřenského fondu \textit{mortis causa} zákonem povolenou cestou, tedy pořízením pro případ smrti je však nutné uvažovat nad právní úpravou vztahující se k tomuto právnímu jednání. To samo o sobě poskytuje v § 1492 ochranu nepominutelným dědicům a je tedy otázka, zda je toto bez dalšího na takovýmto způsobem vyčleněný majetek aplikovatelné.\\

%Jak je to ale se svěřenským fonde? Na to se budu soustředit dále ve své práci.

\newpage

Aby tedy bylo možné hovořit o výhodách, které svěřenský fond oproti klasickému dědění\footnote{Tedy převodu majetku v rámci dědického řízení (jakožto instrumentu převodu majetkových práv v rámci dědického řízení), tohoto lze dosáhnou mnoha způsoby upravenými právě v rámci Dědického práva, především se jedná o pořízení pro případ smrti.} poskytuje a rozebrat jednotlivé výše nastíněné problémy, je třeba čtenáře stručně obeznámit s právní úpravou dědictví a jeho funkcí v rámci mezigeneračního převodu majetku a to zejména v bodech, ve kterých dochází k, právní obcí vnímaným, kolizím\footnote{Rozdílný pohled je představen v části Systémové nedostatky a návrhy na řešení, existují dva názorové proudy, přičem jeden z nich, který zastává především Kocí v komentáři k Občanskému zákoníku kontrastuje s názory jiných autorú, mezi ty je možné zařadit Piheru, Svejkovského, Koláře a Horna.} obou právních úprav.\\

%Doplnit či provést jakoukoliv komparaci.

Tyto kolize spočívají především v úpravě pozůstalosti, respektive dědického řízení, vydědění a s ním související ochranou nepominutelného dědice.\\

%Převod práv a závazků je bez pochyby nejčastější formou mezigeneračního převodu majetku. Dědické právo je obecně upraveno převážně v paragrafech 1475 až 1720, přičemž důležitou součást tvoří Díl 5, který upravuje institut nepominutelného dědice.\\

%, která je upravena v Hlavě III Občanského zákoníku v části Absolutní majetková práva v § 1475 a dále, tedy v rámci Dědického práva\footnote{Tedy převodu majetku v rámci dědického řízení (jakožto instrumentu převodu majetkových práv v rámci dědického řízení), tohoto lze dosáhnou mnoha způsoby upravenými právě v rámci Dědického práva, především se jedná o pořízení pro případ smrti.}. Tyto možnosti jsou bezpochyby nejčastější formou mezigeneračního převodu majetku a je tak nutné je stručně představit, aby bylo možné popsat nejen výhody svěřenského fondu oproti způsobům upraveným Dědickým právem, ale také právní obcí vnímané kolize\footnote{Rozdílný pohled je představen v části Systémové nedostatky a návrhy na řešení, existují dva názorové proudy, přičem jeden z nich, který zastává především Kocí v komentáři k Občanskému zákoníku kontrastuje s názory jiných autorú, mezi ty je možné zařadit Piheru, Svejkovského, Koláře a Horna.} mezi oboumi právními úpravami.\\

%Popsat dědický titul.

%Je svěřenský fond věcí hromadnou? Popsat rozdělení věcí podle občanského zákoníku.

%S ohledem na to, že svěřenský fond představuje alternativu či suplement k této možnosti převodu majetku, je třeba čtenáře stručne obeznámit s právní úpravou dědického práva a to především v bodech, ve kterých dochází ke kolizi obou právních úprav. Toto je důležité (nejen) především k definici a popsání těchto kolizí, ale i k vyložení výhod, které svěřenský fond poskytuje oproti formám upraveným v dědickém právu.

%Tyto kolize spočívají především v úpravě pozůstalosti, respektive dědického řízení, vydědění a s ním související ochranou nepominutelného dědice.

%Pro komparaci právní úpravy svěřenského fondu s právní úpravou dědického práva je v první řadě důležité popsat jakým způsobem v České republice funguje dědické právo. Dědické právo je obecně upraveno převážně v paragrafech 1475 až 1720, přičemž důležitou součást tvoří Díl 5, který upravuje institut nepominutelného dědice. \\

\subsection{Stručný úvod do právní úpravy dědictví}

Převod práv a závazků v rámci dědictví je bez pochyby nejčastější formou mezigeneračního převodu majetku. Dědické právo je upraveno převážně v paragrafech 1475 až 1720, přičemž důležitou součást tvoří Díl 5, který upravuje institut nepominutelného dědice.\\

V rámci české právní úpravy vzniká dědické právo smrtí zůstavitele a je právem na pozůstalost nebo na poměrný podíl z ní, ten kdo zemře společně se zůstavitelem, nebo před ním, není způsobilý dědit\footfullcite[§ 1479]{noauthor_zakon_2012}. S ohledem na toto je vhodné poukázat na formální požadavky právní úpravy na zůstavitele a dědice. Nejobecnějším formálním požadavkem kladeným na zůstavitele, je pořizovací způsobilost, respektive nezpůsobilost. Obecně zákon neklade žádné překážky tomu, aby bylo jmění zůstavitele součástí dědického řízení, a aby tak mohlo přejít na jeho dědice. Je nicméně omezen způsob, který může zůstavitel použít k vyjádření své vůle o tom, jakým způsobem má toto převedení proběhnout. To znamená, že zůstavitel je omezen ve způsobech, kterými může pořizovat o svém majetku pokud byl nezletilcem, který nedovršil věku patnácti let, popřípadě pokud by byl omezen ve svéprávnosti\footfullcite[§ 1525 až § 1528]{noauthor_zakon_2012}. Tyto ustanovení dopadají pouze na pořízení pro případ smrti, dědění na základě dědického titulu ze zákona tímto není dotčeno. \\

Je vhodné stručně popsat všechny způsoby vyloučení dědice z dědického práva, neboť dle mnohých autorů je i samotný svěřenský fond způsob vyloučení dědice z dědického práva. Toto může a nemusí být pravda, pojďmě tedy toto tvrzení podrobit analýze. K tomu je potřeba osvětlení základních principů českého dědického práva a jeho hmotných a procesněprávních náležitostí.\\

Množina požadavků kladených na jednotlivé dědice, jinak řečeno tedy náležitosti, které dědicové musí splňovat, aby mohli býti dědicky způsobilí, je z rozhodnutí zákonodárce širší. Ve chvíli, kdy tyto požadavky nejsou naplněny nebo je porušena nějaká povinnost kladená na dědice, není umožněno danému dědici nabýt celou pozůstalost, nebo její část. \\

Z tohoto vyplývá, že aby byl dědic způsobilý dědit, musí splňovat jak pasivní, tak i aktivní podmínky na něj zákonem kladené. Aktivní podmínky, které musí dědic splňovat jsou vymezeny v §1481-1482 Občanského zákoníku (Dědická nezpůsobilost), aby dědici nesvědčila dědická způsobilost, nesmí se tedy dopustit ani jednoho z vymezených jednání. Pokud se ho dopustí, může být ještě způsobilý dědit, pokud mu zůstavitel dané jednání výslovně odpustí. A podmínky pasivní, které ve své podstatě spočívají pouze v tom, že dědic musí být buď právnická, nebo fyzická osoba. Tato pasivní podmínka je nesmírně důležitá ve vztahu ke svěřenským fondům, neboď svěřenský fond není považován ani za jednu z těchto osob.\\

\newpage

Jako jistou podskupinu k aktivním podmínkám bych dále zařadil i ostatní skutečnosti vylučující dědice z dědického práva, které mohou nastat na straně dědice, tedy již uvedená dědická nezpůsobilost, či další jednání na straně dědice, tak i na straně zůstavitele. Co se tedy tohoto vyloučení dědice z dědického práva v rámci aktivní podskupiny týče, Občanský zákoník taxativně vymezuje následující důvody k němu vedoucí:

\begin{figure}[h]
\centering
\includegraphics[width=12cm,height=4cm,keepaspectratio]{Dedicka_nezpusobilost.png}
\caption[Vyloučení dědice z dědického práva]{Vyloučení dědice z dědického práva\footfullcite{noauthor_informacni_nodate}}
\label{fig:komparace}
\end{figure}

Vyloučení z dědického práva se obecně dá rozdělit na důvody závislé na jednání dědice a nezávislé na jednání dědice. Do první skupiny patří ve své podstatě dědická nezpůsobilost, zřeknutí, odmutnutí, vzdání se a zcizení dědictví. Do druhé skupiny patří vydědění. \\

\noindent\textbf{Jednotlivé důvody vylučující dědice z dědického práva:} \\

Dědická nezpůsobilost je upravena v § 1481 až § 1483, jedná o případy, ve kterých se dědic dopustil úmyslného trestného činu proti zůstavitelovi, nebo jeho blízkým, popřípadě pokud se dopustil zavrženíhodného činu proti zůstavitelově poslední vůli, spočívající v lstivém svedení k projevu poslední vůle, popřípadě donucení k tomuto projevu, překažení projevu, zatajení, falšování a podvrhnutí nebo zničení pořízení. Dále pokud se dědic dopustil činu uvedeného v § 1482 OZ. Dědic se i přes toto stává způsobilým, pokud mu zůstavitel výslovně tento čin prominul. \\

Zřeknutí se dědického práva je upraveno v § 1484 a stanovuje, že se dědic může předem zříci dědického práva, nebo práva na povinný díl smlouvou se zůstavitelem. Tato smlouva může působit jak proti samotnému dědici, tak i proti jeho potomkům, jedná se o dispozitivní ustanovení občanského zákona a záleží tedy na vůli smluvních stran, aby si mezi sebou toto upravili. Smlouva musí být uzavřena formou veřejné listiny. \\

Odmítnutí dědictví je upraveno v § 1485 OZ, v případě, že to nevylučuje dědická smlouva, může dědic dědictví po smrti zůstavitele, výslovný prohlášením vůči soudu, odmítnout. Nepominutelný dědic může dědictví odmítnout s výhradou povinného dílu. V případě odmítnutí se pak na dědice pohlíží, jako by dědictví nikdy nenabyl.\\

Vzdání se dědictví je upraveno v § 1490, dědic se po smrti zůstavitele může vzdát dědictví ve prospěch jiného dědice. Pokud byl však tento dědic obtížen příkazem, nařízením odkazu, nebo jiným opatřením, které podle zůstavitelovi vůle může a má splnit osobně, pak se této povinnosti vzdáním se dědictví nezbavuje.\\

Pomocí prohlášení o vydědění lze vydědit nepominutelného dědice. Toto prohlášení způsobuje vyloučení dědice z dědění po zůstavitelovi. Takto vyděděný titul nenabívá práva založeného na dědickém titulu, ani nenastupuje po zůstavitelově smrti do jeho povinností kvůli ztrátě práva na svůj zákonnem stanovený díl. Lze vydědit pouze v zákonech stanovených případech (viz kapitola Vydědění). S ohledem na toto omezení je možné nahlížet na svěřenský fond jako na možnost obejití tohoto omezení.\\

Dědic může po smrti zůstavitele své dědické právo rovněž bezúplatně, nebo úplatně zcizit dle § 1714 OZ. Nabyvatel tak vstupuje do práv a povinností náležejícím k pozůstalosti. Smlouva vyžaduje formu veřejné listiny.\\

\subsection{Nepominutelný dědic}

Dědicem jako takovým může být nejen člověk, ale i právnická osoba, pokud vznikne maximálně do jednoho roku od zůstavitelova úmrtí. Nepominutelným dědicem však může být pouze osoba označená v § 1643 OZ, tou je dítě zůstavitele, nebo jeho potomci.\\

V dědickém právu nová úprava směřuje především k poznání a respektování skutečné vůle a pohnutek zůstavitele, které jej vedly k určitému právnímu jednání, přes toto však zákonná úprava dědického práva staví před zůstavitele i určité podmínky a požadavky. Jednou z těchto podmínek je institut nepominutelného dědice.

\subsection{Vydědění}

Nepominutelného dědice lze vydědit z jednoho ze čtyř stanovených důvodů. Zůstavitel může vydědit potomka pokud:

\begin{enumerate}
\item mu neposkytl potřebnout pomoc v nouzi,
\item o zůstavitele neprojevu opravdový zájem, jaký by projevovat měl,
\item byl odsouzen pro trestný čin spáchaný za okolností svědčící o jeho zvrhlé povaze, nebo
\item vede trvale nezřízený život.
\end{enumerate}

\subsection{Pozůstalost}

Laická veřejnost nezřídka kdy plete pojmy dědictví a pozůstalost. Rozdíl spočívá v tom, že jako pozůstalost je označováno veškeré jmění zůstavitele k okamžiku jeho smrti, které je způsobilé přejít na právní nástupce. Do pozůstalosti nespadají práva a povinnosti, které jsou výhradně vázána na osobu zůstavitele, ledaže byla jako dluh uznána u orgánu veřejné moci, zde se jedná například o bolestné či satisfakci v penězích. Naproti tomu dědictví je skutečná část pozůstalosti, která přejde na zůstavitelovi dědice.\\

NOZ zachovalo koncepti dědického práva a nevrací se k římskému pojetí tvz. ležící pozůstalosti\footnote{Tak jako tomu bylo před rokem 1950}.

\subsection{Procesní aspekty dědického práva}

Jak již jsem zmínil výše, účelem této části není podání komplexního přehledu o fungování dědického práva a jeho vysvětlení, ale pouze představení jeho nezbytných částí, které jsou podstatné pro následnou komparaci s právní úpravou svěřenského fondu. Z tohoto důvodu neshledávám jako důležité vyložit celý průběh dědického (nově pozůstalostního) řízení, ale pouze ve zkratce vyložit jeho existenci a funkci, neboť součástí analýzy kolizí dědického práva a práva svěřenských fondů bude otázka spojená s průběhem pozůstalostního řízení a v souvislosti s ním také krátky exkurz do Zákona o zvláštních řízeních soudních a do Občanského soudního řádu.\\

Řízení se standardně zahajuje ve chvíli, kdy matriční úřad oznámí soudu úmtí zůstavitele. Řízení lze také zahájit na návrh dědice.\\

Průběh řízení je složene ve své podstatě ze dvou fází, první začíná předběžným řízením u notáře a končí nařízením jednání u soudního komisaře, v oněž moment začíná druhá fáze pozůstalostního řízení. V první fázi se zjišťuje jmění zůstavitele, okruh dědiců a z Evidence o právních jednání pro případ smrti se zjišťuje, zda po sobě zůstavitel nezanechal listiny důležité v rámci řízení o pozůstalosti, tedy například závěť či dovětek.\\

V rámci jednání u soudního komisaře, jímž je soudem pověřený notář, se následně pozůstalost rozdělí mezi jednotlivé dědice, podle dohody dědiců, podle zákonných dědických podílů, nebo podle přání zůstavitele. Notář následně vydá usnesení, kterým se dědické řízení končí.\\

%Co když se pořízení pro případ smrti, nebo závěť učiněná před svědky neshoduje se závětí, dle které byl vyčleněn majetek do svěřenského fondu. Podle mě soud zhodnotí, kdo má slabší nárok respektive důkazy a toho pak odkáže, aby podal žalobu v občanskoprávním řízení.

\newpage
\thispagestyle{smallertextinheader}

\section{Svěřenský fond jako nástroj mezigeneračního transferu majetku}

V rámci důvodové zprávy k Novému Občanskému zákoníku uvedl zákonodárce velice zajímavou myšlenku: \textit{Návrh vychází z ideje, že funkční určení soukromého práva je sloužit člověku jako prostředek k prosazování jeho svobody. Účel občanského kodexu je umožnit i garantovat svobodné utváření soukromého života, a ponechat tedy co nejširší prostor svobodné iniciativě jednotlivce. Proto také osnova klade zásadní důraz na hledisko autonomie vůle.}\footfullcite[Strana 20]{Duvodovazprava}. Tato idea se zásadně projevila v oblasti správy majetku, kde představuje základní východisko pro nově zavedený institut svěřenského fondu, a dalších obdobných forem správy majetku, a staví na základním stvebním kamenu pro rekodifikaci soukromého práva. Tímto stavebním kamenem není nic jiného, než pohled na soukromé právo jako na právo, které má v soukromé svéře sloužit subjektům práva jako prostředek prosazování svobody a vytváření svobodného prostoru a iniciativy k vlastním řešením určitých problémů.\\

Velkou součást této části práva tvoří právo majetkové relativní a absolutní. Existence takovéto úpravy je naprosto zásadní pro lidi, kteří v průběhu svého života kumulovali majetek a otázka předání, udržení či rozmnožení, ať už vůči sobě, nebo jiným osobám, takto naakumulovaného majetku je pro ně tedy zásadní.\\

Je nepochybné, že v případě, kdy se vlastník tohoto majetku, respektive jmění, rozhodne svůj majetek svěřit jinému subjektu do správy, za jakýmkoliv účelem, tedy rozmnožením, zařízením převodu, správy a podobně, bude primárně vybírat z možností jemu představených zákonem. Těchto možností je mnoho, majetek bývá pravidelně spravován na základně smlouvy, standardně příkazního typu, osobami určenými v této smlouvě, standardně jimi může být investiční fond, záložna, družstvo, jiný k tomu pověřený správce či banka.\\

Nově lze krom výše zmíněných druhů správy majetku a jiných, pro běžné občany poněkud exotičtějších způsobů správy majetku, jako je vložení majetku do ústavu\footnote{V tomto případě zásadně za účelem sledujícím veřejný prospěch.}, obchodní korporace či fundace\footnote{Nadační fond a nadace, kdy musí být obdobně sledování společensky nebo hospodářsky prospěšný účel.} nebo využítí i různých dalších nástrojů jako jsou různá spoření s investiční složkou či investičních fondů\footfullcite{noauthor_zakon_nodate_investicni_spol_a_investicni_fondy}, využít i jednotlivé formy svěřenského fondu.\\

Zajímavé je zmínit i možnost využití zahraničních trust-like struktur, je ale nutné vzít na vědomí, že Česká republika není signatářem Haagské konvence o právu uplatnitelném na trusty a jejich uznávání a funkčnost těchto institutů, v praxi České právní úpravy tak může být, oproti zemím, které jsou signatáři a mají vyřešené kolizní normy, jako je například Itálie, problematický\footfullcite{noauthor_hcch_nodate}\textsuperscript{,}\footnote{Příklady těchto institutů uvádím v kapitole Úvod do historických počátků svěřenství a svěřenských fondů.}.\\



--Koment\\
Popsat jak je v momentální právní úpravě vyložen institut svěřenského fondu a jaké možnosti poskytuje. \\
--\\

Česká právní úprava zakládá mnoho možností, které lze použít k převedení majetku. Těmito mohou býti klasické aktivní způsoby běžně využívané na denní bázi všemi z nás, zahrnující například koupi, darování či výměnu. Lze nicméně použít i méně častých možností jako vydržení, nebo i zakládání právnických osob sloužících za účelem převodu majetku jako je například nadace. Troufám si však tvrdit, že nejčastější jsou způsoby upravené v rámci dědického práva počínajících od nabytí majetku na základě dědického titulu vyplývajícího ze zákona, tedy v případech kdy zůstavitel zůstal za svého života nečinný a neupravil způsob, jakým má být s jeho jměním naloženo po jeho smrti, nebo na základě dědických titulů vycházejících z pořízení pro případ smrti, tedy z právního jednání aktivně učiněného za života zůstavitele. \\

Způsoby, kterými lze dosáhnout převedení majetku jsou tedy dva, aktivní jednání zůstavitele, nebo ponechání převodu majetku na zákoně a na tom jak jsou dědicové schopni se mezi sebou dohodnout, tedy určitá pasivnost zůstavitele. Dá se říci, že s ohledem na převáděné jmění a na budoucí nakládání s takto převedeným jměním, je více než vhodné, aby zůstavitel upravil tyto záležitosti již za svého života. Může tím totiž i nadále v uvozovkách vykonávat správu nad svým jměním, respektive může určit jak s jeho jměním má být po jeho smrti naloženo, takovéto jednání je vhodné zejména s ohledem na mezilidské vtahy v rodině, ochránění jmění před neuvážlivým utrácením ze strany dědiců, ochraně před rozdrobením majetku a mnoha dalšími potenciálními nepříznivými dopady převedení majetku. S ohledem na toto primárně přichází v úvahu způsoby, které jsou upravené v rámci dědického práva, v očích běžného občana tímto bude především závěť, v očích těch znalejších to může být i dědická smlouva, dovětek, nebo odkaz. Tyto způsoby však mají i svá specifická omezení a ve svém důsledku spočívají v konečném převedení jmění na jednotlivé dědice, nebo odkazovníky. Zůstavitel také nemá tak širokou možnost stanovit svým dědicům růzce podmínky a to i přes existenci dovětku. Z tohoto důvodu nemusí být vhodné jejich použití ve všech případech. \\

%Porovnat a popsat možnosti, které má zůstavitel v dovětku a ve svěřenském fondu.

\newpage
\thispagestyle{smallertextinheader}

Jako jistá alternativa k těmto klasickým způsobům byla do české právní úpravy transplantována úprava trustu z Quebeckého občanského zákoníku, tento poskytuje zůstavitelům další možnost, jak naložit se svým jměním již za svého života (založení svěřenského fondu \textit{inter vivos}), popřípadě po své smrti (založení svěřernského fondu \textit{mortis causa}). Institut svěřenského fondu přináší mnoho možností, které zůstavitel před jeho zavedením neměl, ale zároveň do českého práva vnáší i spoustu otázek, které je potřeba pro jeho správné používání a ochranu zůčastněných stran zodpovědět. V následujících kapitolách proto představím způsoby, kterými lze svěřenský fond použít pro mezigenerační převod majetku, představím jeho přínosy a benefity, ale zároveň se budu snažit zodpovědět otázky týkající se nejistého výkladu právní úpravy svěřenského fondu, jeho potenciálních problémů a možností zneužití. \\

\newpage

\thispagestyle{smallertextinheader}

\subsection{Svěřenský fond v České republice}

%Za použití zcela shodných argumentů jako v kapitole 5.2.2 lze opět dojít k závěru, že majetek ve svěřenském fondu, jakožto autonomní vlastnictví, není součástí pozůstalosti a tedy ani není součástí dědického řízení. Nemůže k němu vznikat dědické právo a tím z něj nikdo ani nemůže být vyloučen. Pro takový případ nelze než doporučit, aby budoucí zůstavitelé-zakladatelé ponechali určitý prostor pro správce, aby ten mohl v případech podobných (jako předpokládá OZ u dědické nezpůsobilosti) rozhodnout o plnění obmyšleným podle obecných zásad spravedlnosti a dobrých mravů.

--- Toto asi odstraním\\
Pro další pochopení svěřenského fondu a jeho možností, problémů, výhod, nevýhod a nejasností je v samém úvodu této kapitoly nutné čtenáře obeznámit s právní úpravou svěřenského fondu v rámci českého práva. \\
---\\

Definovat správně svěřenský fond, jakožto velice flexibilní právní institut, je poměrně složité a až doposud, tedy i relativně dlouhou dobu po jeho zavedení, existuje spousta protichůdných názorů (například v rámci lingvistického výkladu jednotlivých ustanovení, definicí majetku nebo koexistenci právního institutu přejatého z právního systému Common Law s instituty vlastní kontinentálním právním systémům, jako je například institut nepominutelného dědice) na tento institut, který v době svého přijetí do naší, tedy kontinentální, právní úpravy vyvolal poměrně mnoho emocí, od mnohých právních expertů se dočkal dílem chladného přijetí a od mnohých dílem přímé kritiky. Ať již nahlížíme na svěřenský fond jakkoliv, je zřejmé, že pro jednotlivé subjekty práva a pro jednotlivé účely, ke kterým je možné jej užít, skýtá i velké výhody. S ohledem na název bakalářské práce bude hlavní těžiště této práce v rámci praktické části spočívat především v představení těchto nových možností a výhod v procesu převádění rodinného majetku na další generace a jeho uchování jako alternativní, nebo rozšiřující prostředky ke stávajícím možnostem mezigeneračního převodu majetku, nebo převodu majetku obecně. \\
% Dopsat někde de lege lata
% Dopsat alterace a modifikace testamentu
% Nevystupuje jako subjekt ale spíše jako objekt právních vztahů
% Jelikož nelze svěřenský fond zařadit mezi právnické osoby, není možné na něj aplikovat ustanovení o právnických osobách v procesních, daňových, trestních a dalších předpisech. Zákonodárce proto musel tyto právní předpisy novelizovat s ohledem na úpravu svěřenských fondů. Důsledky zařazení svěřenského fondu do českého právního řádu tak lze pozorovat v právních předpisech napříč celým právním systémem

%Nově lze využít i formy svěřenského fondu, jehož základ tvoří osamostatněný majetek (resp. jmění), vyčleněný jeho zakladatelem k určitému účelu, svěřený do správy svěřenskému správci.


V rámci současné právní úpravy je svěřenský fond upravený především v části třetí OZ, hlavě \MakeUppercase{{\romannumeral 2}} - věcná práva v § 1448 až § 1474. \\

%Doplnit pojem svěřenský fond, inspirace z práce Lukáš Kolenský Obchodní korporace v.svěřenský fondz pohledu věřitele

Věci ve svěřenském fondu náležejí svému účelu\footfullcite{svejkovsky_jaroslav_sverenske_2018}.\\

Úprava institutu svěřenského fondu a obecná úprava správy cizího majetku vstoupila v účinnost 1. ledna 2014 jako součást Nového Občanského zákoníku. Svěřenský fond je speciálně upraven v rámci ustanovení paragrafů 1448 až 1474, které jsou zařazeny do oddílu 4, jež je součástí dílu 6, Správa cizího majetku, spadající do části třetí, hlavy druhé, Věcná práva. \\

Úprava svěřenského fondu jest úpravou speciální, která doplňuje úpravu generální obsaženou v ustanoveních 1400 až 1447, Správa cizího majetku. Český zákonodárce se inspiropval, jak již je vysvětleno v kapitole Historie \footnote{Správa cizího majetku - historie; Recepce právní úpravy svěřenských fondů}, Občanským zákoníkem z Quebecu ("CCQ"), z kterého ve své podstatě transplantoval právní úpravu svěřenského fondu, která, přes určité výkladové změny \footnote{například v pojmech majetek a jmění, viz Svěřenský fond a trust - Jejich fungování v mezinárodním srovnání od Luboše Tichého}, do značné míry zachovala původní znění úpravy trustu tak, jak jej známe z CCQ, jak dokládá Svejkovský, Marek a kol. v díle Správa cizího majetku v novém občanském zákoníku v části Srovnání právní úpravy ObčZ a CCQ se stručným popisem rozdílů. 

\newpage
\thispagestyle{smallertextinheader}

Důvod, pro který si Český zákonodárce vybral jako zdroj inspirace pro zavedení trustu do našeho právního řádu práve CCQ je ten, že Quebecký Občanský zákoník si zachoval výrazný charakter kontinentálního práva\footfullcite{Duvodovazprava}, ačkoli je obklopena státy s právním systémem Common law\footnote{Výjimku představují například DOPLNIT}.\\

Výhodou je i vysoká flexibilita svěřenského fondu. Ve statutu lze do značné míry podle potřeby upravit vnitřní fungování svěřenského fondu podle představ zakladatele - např. počet svěřenských správců a obmyšlených, a způsob jejich jmenování, pravidla výplaty plnění obmyšleným, pravidla správy apod.

%Tento paragraf bude ještě potřeba přepsat, aby pasoval do této kapitoly.

Svěřenský fond je tedy jakousi quazi právní entitou bez právní subjektivity, sestávající se z autonomního majetku bez vlastníka vyčleněného svému účelu\footnote{Jedná se o něco, co není v českém právu běžné, z toho důvodu existuje i mnoho výkladových rozporů.}, je spravován svěřenským správcem podle instrukcí zakladatele, které jsou součástí zakladatelského právního jednání, respektive statutu.\\

Důležité je zmínit dobu, na kterou svěřenský fond může být zřízen. Pro svěřenské fondy založené za soukromým účelem je tato doba omezená. Pro svěřenské fondy založené za veřejným účelem nicméně není stanovena maximální doba. Při porovnání obou forem založení tak můžeme \textit{a contrario} dospět k tomu, že svěřenský fond založený za veřejným účelem není omezen maximální dobou trvání a může tak být založen jako takzvaná perpetuita.\\

%Doplnit zdroj z OZ, tedy paragraf.

%PŘEPSAT
%Nabízí se otázka, co lze rozumět pod pojmem majetek. Občanský zákoník chápe majetek jako souhrn všeho, co osobě patří. Přitom musí být pro účely vyčlenění do svěřenského fondu předmětem právního obchodu. Tato definice nevylučuje ani pohledávky a neexistuje důvod, proč by nemohly být předmětem vyčleňovaného majetku. To se týká také pohledávek zakladatele vůči fondu. Zakladatel proto může vyčlenit majetek tak, že zřídí ve prospěch fondu určité právo, jímž se třeba sám zavazuje. Naproti tomu dluhy zakladatele rozhodně mezi majetek počítat nelze. Existuje však možnost, že bude vyčleněný majetek zatížen nějakými závazky. V tom případě se mohou společně s majetkem přesunout do svěřenského fondu také dluhy. Vyčleněným majetem tedy mohou být hmotné věci movité (pila) i nemovité (pozemek) i nehmotné (pohledávka). ZDROJ: https://books.google.cz/books?id=aeq4DwAAQBAJ&pg=PT164&lpg=PT164&dq=sv%C4%9B%C5%99ensk%C3%BD+fond+vy%C4%8Dlen%C4%9Bn%C3%AD+dluh%C5%AF&source=bl&ots=ah6SoELD-z&sig=ACfU3U1vKrdxRriyDfwSDEsHE9npgYfwpg&hl=en&sa=X&ved=2ahUKEwjul--0wvnpAhX2BGMBHeXQC74Q6AEwBHoECAoQAQ#v=onepage&q&f=false

\subsubsection{Svěřenský fond a právní subjektivita}

% Dobrý zdroj je následující: https://advokatnidenik.cz/2019/05/23/sverenske-fondy-nejsou-obchodni-korporace-fikce-podle-zakona-o-skutecnych-majitelich/

Důležitou součástí definice svěřenského fondu je fakt, že svěřenský fond je koncipován jako oddělený majetek, který je svěřen svému účelu, jež nemá vlastníka a nemá ani právní subjektivitu. Přes jasnou definici, kterou obsahuje Občanský zákoník, důvodová zpráva a komentář k občanskému zákoníku a i samotný zdroj inspirace právní úpravy svěřenského fondu, tedy CCQ, není však naškodu podrobit svěřenský fond jisté komparaci s právnickou osobou. Není totiž tajemstvím, že určité právní normy na svěřenský fond pohlížejí jako na právnickou osobu\footfullcite[§17]{noauthor_zakon_zdp}\textsuperscript{,}\footfullcite[§4b]{noauthor_zakon_zdph} a právě tato tendence k subjektivizaci svěřenského fondu by se mohla projevit i v jiných aspektech svěřenského fondu. DOPLNIT

Brilantní náhled na toto téma již vyslovil Michal Skuhrovec ve své diplomové práci, která z tohoto hlediska poskytuje vhodný náhled na tuto otázku a také patřičný úvod do této problematiky.

%Doplnit do začátku práce, že jsem se touto prací hodně inspiroval, ale myslím si, že neposkytuje odpovědi na všechny otázky a v určitých částech není aktuální, nebo odpovídající, doplnil bych jí tedy o svůj pohled, nicméně některé pohledy autora jsou výtečné a já se s nimi jen ztotožňuji, stejně tak jako se s nimi ztotožňují i pohledy právních expertů. Je nicméně potřeba jeho pohled rozšířit i o nové poznatky v souvislosti s institutem svěřenského fondu

%Na tomto místě je vhodné uvést dvě základní premisy související se svěřenským fondem, které budou v bližším detailu rozvedeny dále. Povaha svěřenského fondu předně vylučuje, aby měl svěřenský fond právní osobnost (svěřenský fond tedy není právnickou osobou) a stejně tak vylučuje, aby byl svěřenský fond věcí v právním slova smyslu. Uvedenému přirozeně neodporují závěry některých autorů, kteří nevylučují, aby měl majetek vložený do svěřenského fondu povahu věci hromadné. 2 HORN, K.: Podrobněji k svěřenskému fondu. Ad notam, č. 6, ročník 2014, s. 1.

%Lze nicméně uvést, že svěřenský fond představuje naprosto ojedinělou a zvláštní entitu, která se skládá z (bývalého) majetku zakladatele a v souvislosti s kterou je prostřednictvím činnosti správce svěřenského fondu naplňován zakladatelem určený cíl. Přitom základním cílem společným pro všechny svěřenské fondy je nepochybně ochrana majetku zejména před nežádoucími dispozicemi 260 260 AKONTinfo, Trust jako nejúčinnější nástroj ochrany majetku. [online]. 2010. On-line zdroj: http://www.akont.cz/cz/375.trust-jako-nejucinnejsi-nastroj-ochrany-majetku.

PŘEPSAT
Pihera v tomto konextu vytahuje na povrch diskuzi pocházející z germánského prostoru, kdy Jhering kritizoval koncepci samostatného jmění jako contradictio in adiecto, kdy svou kritiku zakládá na "subjektlose subjektive Rechte", neboli existenci subjektivního práva bez subjektu. Takové by mělo představovat nebezpečí popření ústředního postavení člověka v centru soukromého práva, jak dovozuje Becker. Nicméně, lze oběma autorům přisvědčit, že jeho uznání by při racionálním zhodnocení všech okolností nemělo bez dalšího vést k dotváření "Rechtlosen subjekten". PIHERA NEJPODIVNĚJŠÍ ZVÍŘE V LESE Komentář k OZ pak zároveň dodává, že určité samostatné majetky již v právním řádu 14 známe a vyrovnáváme se s nimi již tradičně, například ležící pozůstalost (hereditas iacens) v ABGB. Dalším právem bez subjektu jsou například práva nenarozeného dítěte.

PŘEPSAT
Rozdíl mezi úpravou CCQ a OZ pak tkví dále v samotné definici majetku a je otázkou, zda zde český zákonodárce udělal quebeckému trustu dobrou službu co do přípravy "podhoubí". CCQ totiž v § 915 stanoví: "Property belongs to persons or to the State or, in certain cases, is appropriated to a purpose." Majetek patří buď osobám, státu a nebo, v určitých případech, je vyčleněn účelu. Oproti tomu §495 ani §1011 nic takového nestanoví (viz výše). Je tak nutno vycházet z toho, že §1448 OZ je ustanovením speciálním a tvoří základnu teoretické koncepce svěřenského fondu. PIHERA NEJPODIVNĚJŠÍ ZVÍŘE V LESE Lze tak uzavřít, že svěřenský fond je entitou bez právní subjektivity, která je majetkem se samostatným účelovým určením. S ohledem na výše uvedené, podpořeno argumenty zaznívající v diskuzi odborné veřejnosti, se lze přiklonit na stranu zastánců přijaté koncepce v duchu "patrimoine d`affectation, neb zjevně skutečně představuje vhodný most mezi dvěma právními kulturami, podepřený zkušenostmi z Quebecu. Smith

\subsection{Novelizace právní úpravy - zápis svěřenských fondů do rejstříku}

Majetek ve svěřenském fondu tedy není ve vlastníctví nikoho, před novelizací Občanského zákoníku pak ve veřejných seznamech nebo evidencích mohla být dohledána pouze osoba svěřenského správce. Samotné svěřenské fondy se pak neměli povinnost se jakkoliv registrovat nebo evidovat, v souvislosti s tím pak nebyl veřejně dostupný ani statut svěřenského fondu, který přeci musel mít formu veřejné listiny, ale v souvislosti s absencí povinnosti evidence svěřenských fondů nebyl nikde zveřejněn. Absence jakékoliv evidence, vedla k anonimitě zakladatele, obmyšlených a přispívala k celkové netransparantnosti vlastnické struktury a majetku vyčleněného do svěřenského fondu.\\

V souvislosti s tímto panoval oprávněný strach, že svěřenské fondy by mohli být využity k legalizaci výnosů z trestné činnosti, daňovým únikům nebo být jinak zneužity. V souvislosti s touto obavou hned tři instituce, kterými byli Ministerstvo financí, Ministerstvo vnitra a Vrchní státní zastupitelství volali po novelizace právní úpravy svěřenského fondu a povinné evidenci. Důvodová zpráva k novele pak stanovuje přímo následující: \textit{"V reakci na naléhavé připomínky Ministerstva financí, Ministerstva vnitra a Vrchního státního zastupitelství v Praze se navrhuje zavedení evidence svěřenských fondů z důvodu jejich netransparentní vlastnické struktury a vysokého potenciálu zneužití především k legalizaci výnosů z trestné činnosti."}\footfullcite{noauthor_4602016_nodate}.\\

%Zákon č. 33/2020 Sb. novela, se svěřenských fondů téměř vůbec nedotýká, pro tuto práci není důležitá, neboť se měni jen určité ustanovení ve vztahu k notářskému zápisu.

%Rozepsaná novelizace svěřenský fondů z roku 2018, zhodnotit, zda přispělo ke zmírnění možností využití svěřenských fondů jako prostředků k legalizaci výnosu z trestných činností.\\

Poměrně zásaní změny v souvislosti se zakládáním svěřenských fondů tak v souvislosti s tímto přinesla novelizace č.z. 460/2006, která s sebou přinesla nutnost zápisu svěřenského fondu do evidence s tím, že tento zápis je konstitutivní. Svěřenský fond tedy vzniká tímto zápisem. Toto se nicméně dotklo pouze svěřenského fondu založeného mezi živými. Pro svěřenský fond založený pro případ smrti zákonodárce stanovil v § 1451 odstavci 3 následující:\\

\textit{"Byl-li však svěřesnký fond zřízen pořízením pro případ smrti, vznikne smrtí zůstavitele. Do evidence svěřenských fondů se zapíše po svém vzniku."}\\

Důvod, pro který zákonodárce dopřál toto, do jisté míry zvýhodňující, postavení způsobu založení svěřenského fondu \textit{mortis causa}, je v důvodové zprávě objasněno následovně:\\

\textit{"Odlišný princip deklaratorního zápisu se má uplatnit u svěřenských fondů, které byly zřízeny pořízením pro případ smrti. Tento princip lépe vyhovuje fundamentu testovací volnosti a respektu k vůli zůstavitele."}

\newpage
\thispagestyle{smallertextinheader}

\subsection{Založení svěřenského fondu}

S ohledem na téma práce, tedy mezigenerační převod majetku, je otázka založení svěřenského fondu, ať již mezi živými, tedy \textit{inter vivos}, či pořízením pro případ smrti \textit{mortis causa}, nebo velice specifickou možností založení svěřenského fondu \textit{inter vivos} s odkládací podmínkou smrti zakladatele a dalších naležitostí s tímto spojených, integrální součástí této práce a je nezbytné k dalšímu zkoumání jednotlivých možností, které nám s přijetím tohoto institutu Český zákonodárce vložil do rukou a obecně i celé problematiky spojené se svěřenským fondem. \\

Ať již se budeme bavit o jakémkoliv způsobu založení svěřenského fondu, tak zákon klade na zakladatele v paragrafech 1448 - 1474 Občanského zákoníku následující čtyři povinnosti:

\begin{enumerate}
\item Označení a vyčlenění majetku k určitému účelu,
\item statut\footnote{Ve formě veřejné listiny},
\item přijetí správy svěřenského fondu ze strany svěřenského správce,
\item zápis svěřenského fondu do evidence.
\end{enumerate}

Co se však bodu 4 týče, jde nutné rozlišovat mezi založením svěřenského fondu \textit{inter vivos} a \textit{mortis causa}, neboť v obou případech se bod číslo 4 liší. V případě založení \textit{inter vivos} je zápis svěřenského fondu konstitutivní, v případě založení \textit{mortis causa} je pouze deklaratorní, svěřenský fond totiž vznikne smrtí zůstavitele, o tomto však blíže u jednotlivých typů svěřenských fondů.\\

Obecné vysvětlení k pojmu a založení svěřenského, tedy k paragrafům 1448 až 1452 Občanského zákoníku, fondu je obsaženo již v důvodové zprávě, respektive v její části vtahující se ke svěřenskýn fondům.\\

\textit{"Svěřenský fond může být zřízen jednak bezúplatně rozhodnutím zakladatele věnovat část svého majetku určitému účelu, zřízení svěřenského fondu však může být sjednáno i za úplatu nebo jiné protiplnění (např. od osoby, která se má stát obmyšleným, ale i od osoby jiné). Podle toho se pak odvíjí i právní postavení obmyšleného, jak to reflektují následující ustanovení osnovy. Osnova rozlišuje svěřenské fondy podle účelového určení na veřejně prospěšné (zřízené k naplňování účelů kulturních, vzdělávacích, vědeckých, náboženských nebo podobných) a soukromé. Důvodem vzniku svěřenského fondu může být zákon, pravidelně však smlouva nebo ustanovení závěti. Protože svěřenskému fondu musí být ustaven svěřenský správce, vyžaduje § 1448 souhlas osoby označené za svěřenského správce se svým ustavením. I to je tedy právní podmínka vzniku svěřenského fondu (§ 1451). Svěřenský fond se spravuje statutem obsaženým ve smlouvě, závěti či vydaným samostatně."}\footfullcite{prof_dr_judr_karel_elias_a_kolektiv_novy_2012}\\

S touto částí důvodové zprávy pak přímo souvisí §1448, ve kterém zákonodárce uvádí následující:\\

\textit{"§ 1448
(1) Svěřenský fond se vytváří vyčleněním majetku z vlastnictví zakladatele tak, že ten svěří správci majetek k určitému účelu smlouvou nebo pořízením pro případ smrti a svěřenský správce se zaváže tento majetek držet a spravovat.\\
(2) Vznikem svěřenského fondu vzniká oddělené a nezávislé vlastnictví vyčleněného majetku a svěřenský správce je povinen ujmout se tohoto majetku a jeho správy.\\
(3) Vlastnická práva k majetku ve svěřenském fondu vykonává vlastním jménem na účet fondu svěřenský správce; majetek ve svěřenském fondu však není ani vlastnictvím správce, ani vlastnictvím zakladatele, ani vlastnictvím osoby, které má být ze svěřenského fondu plněno."}\\

V uvedeném paragrafu tedy zákonodárce umožňuje pouze explicitní založení svěřenského fondu dvěma způsoby, to znamená, že svěřenský fond nemůže být založen rozhodnutím soudu tak, jako tomu je v jiných právních řádech. Jedním z těchto právních řádů je i CCQ, který umožňuje i takzvané implicitní založení svěřenského fondu, tedy založení na základě rozhodnutí soudu, nebo i ze zákona, pokud existuje zákonné zmocnění, například advokátní a notářské úschovy\footfullcite{claxton_john_b_studies_2005}. Další možností, která není explicitně uvedena v tomto paragrafu, ale je možné ji využít k založení svěřenského fondu, je založení svěřenského fondu mezi živými s odkládací podmínkou.\\

Co se samotného průběhu založení a vzniku svěřenského fondu týče, přikláním se stejně, jako komentářová literatura k multifázovému pojetí vzniku svěřenského fondu.

%Svěřenský fond vzniká vyčleněním majetku z vlastnictví zakladatele. Obecným východiskem zde je, že v zásadě nemůže existovat svěřenský fond bez majetku nebo bez svěřenského správce. Za majetek se přitom považuje i pohledávka. Tedy asi když vzniká svěřenský fond mortis causa smrtí zůstavitele, tak do něho již musí být vložen majetek. Není tedy asi možné, aby byl majetek součástí pozůstalosti a svěřenskému fondu ho přiznal teprve až soud. To že však majetek není součástí pozůstalosti neznamená, že svěřenský fond není účastníkem dědického řízení. Dle zákona o zvláštních řízeních soudních jím totiž je, zastupuje ho zde svěřenský správce, který ale sám o sobě dědicem být nemůže, protože za dědice nebyl povolán. Svěřenský fond tedy v tomto případě má jekési quazi postavení právní subjektivity, tak jako tomu je například třeba i v případě daňového práva.

Účelem se rozumí veřejný nebo soukromý.

\subsubsection{Vznik svěřenského fondu \textit{inter vivos}}

S ohledem na historické konotace svěřenských fondů popsané výše a velkou flexibilitu tohoto institutu skutečně není divu, že Český zákonodárce umožnil založení svěřenského fondu jak mezi živými, tak i pro případ smrti. Začnu s popisem založení svěřenského fondu mezi živými, neboť tento způsob založení je méně problematický než založení svěřenského fondu pro případ smrti a přináší obdobné možnosti mezigeneračního předání majetku.\\

V tomto rámci je podstatné zmínit, že se rozlišuje mezi vytvořením a vznikem svěřenského fondu, celkový proces konstituování svěřenského fondu, tedy od okamžiku samotného rozhodnutí vytvořit svěřenský fond, až k samotnému vzniku, má tedy více fází. Vytvořením se rozumí jednání jednoho, nebo i více zakladatelů\footfullcite{spacil_j_a_kolektiv_obcansky_2013}\textsuperscript{,}\footnote{S ohledem na fakt, že samotná právní úprava svěřenského fondu v §1452 odkazuje na založení nadace, je vhodné i jiné ustanovení, která se týkají nadací, využít analogicky ve vztahu ke svěřenským fondům. Jedním z těchto paragrafů je paragraf  309 odstavec 3, který stanoví, že pokud na straně zakladatelů stojí více osob, považují se obdobně za jednoho zakladatele.}, kteří vyčlení část svého majetku, popřípadě celý svůj majetek a tento poté převedou do jiného jmění, tedy svěřenského fondu, který je zřízen za určitým účelem. Toto jednání zahrnuje v první řadě řádnou nabídku, tedy ofertu, vůči jiné osobě, která by se měla stát potenciálním správcem\footfullcite{jaroslav_svejkovsky_sprava_2015}.\\

%Doplnit nějaké zdroje k založení svěřenského fondu ve více fázích

Jak již je z podstaty svěřenského fondu založeného za života zakladatele zřejmé, jedná se o minimálně o dvoustranné právní jednání, přičemž samotná akceptace této oferty\footnote{Návrh na uzavření smlouvy je dále upraven v §1731 a dalších v Občanském zákoníku} vede dle §1448 a souvisejícího §1451 Občanského zákoníku jednáním vedoucím ke zřízení, tedy vytvoření, svěřenského fondu na základě takto uzavřené smlouvy, kterou je možné uzavřít jak s, tak i bez protiplnění. Tato oferta a akceptace zahrnující uzavření smlouvy o správě by se dala považovat za první fázi založení svěřenského fondu. Druhá fáze spočívá v samotném zápisu svěřenského fondu do evidece dle §1451 Občanského zákoníku a §65a Zákona o veřejných resjtřících právnických a fyzických osob, který je konstitutivní, svěřenský fond tedy formálně vzniká až tímto zápisem.\\

Je zřejmé, že v případě založení svěřenského fondu \textit{inter vivos} je v zásadě bezpředmětné bavit se o možnosti ochrany nepominutelných dědiců. V rámci tohoto způsobu založení svěřenského fondu je nutné poznamenat, že ačkoli je možné, aby zakladatel vyčlenil svůj majetek do svěřenského fondu ještě během svého života s tím, že by tímto jednáním sledoval zájem poměrně ponížit o vyčleněný majetek hodnotu pozůstalosti a tím snížit hodnotu vyplacenou jednotlivým nepominutelným dědicům v rámci jejich povinného dědického podílu, není možné se tomuto jednání ze strany jednotlivých nepominutelných dědiců bránit ve smyslu zkrácení povinného podílu. Toto tvrzení lze doložit hned několika ustanoveními, předně Čl. 11 Listiny základních práv a svobod, která stanoví, že \textit{"Každý má právo vlastnit majetek...."}\footfullcite[čl. 11]{noauthor_zakon_lzps} a v návaznosti na toto poté Občanský zákoník v §996 \textit{"Poctiví držitel smí v mezích právního řádu věc držet a užívat ji, ba ji i zničit nebo s ní jinak nakládat..."} a v §1012 \textit{"Vlastník má právo se svým vlastnictvím v mezích právního řádu libovolně nakládat a jiné osoby z toho vyloučit...."}, z tohoto vyplývá, že vlastník může za svého života se svým majetkem jakkoliv nakládat, tedy má možnost ho zcizit. Ustanovení § 1475, které stanoví, že pozůstalost tvoří celé jmění zůstavitele už dále jen potvrzuje, že takto zcizené věci nejsou a nemohou být součástí pozůstalosti. S ohledem na fakt, že dle definice svěřenského fondu v § 1448 vzniká oddělené a nezávislé vlastnictví vyčleněného majetku a dále již tento vyčleněný majetek není ve vlastnictví zůstavitele a dále vzhledem k tomu, že toto právní jednání bylo učiněno ještě za života zakladatele, lze důvodně předpokládat, že tento majetek již nebude \textit{apriori} předmětem pozůstalosti a dědického řízení, tím pádem nemůže být ani připočten k jmění zakladatele v době smrti a tím pádem z tohoto majetku nemůže být počítát povinný díl nepominutelných dědiců.\\

%Dopsat ještě ústavu, že je chráněno právo na to, aby si majitel se svým majetkem nakládal podle své vůle.

%Doplnit konstitutivní a deklaratorní

Jako jednoduchá analogie k založení svěřenského \textit{inter vivos} a dopadu tohoto jednání na dědický podíl se může jevit jiná forma zcizení majetku, například darování, koupě, směna a další smluvní či mocenské formy zcizení věci, které působí na snížení, či zvýšení majetku zůstavitele. \\

---Koment\\
Doplnit další informace.\\
---\\
Je nicméně vhodné říci, že toto neznamená, že by dědici neměli možnost brát se o své právo jinou formou.\\
---Koment\\
Doplnit možnosti žaloby ve smyslu zneplatnění právního jednání pokud by bylo učiněno například pod tlakem, relativní a absolutní neúčinnost.\\
---\\

\newpage
\thispagestyle{smallertextinheader}

\subsubsection{Vznik svěřenského fondu \textit{mortis causa}}

Na rozdíl od svěřenského fondu založeného mezi živými, dochází ke vzniku v svěřenského fondu založeného pro případ smrti již samotnou smrtí zakladatele, nikoli tedy zápisem do evidence. Tento fakt je důležitý k pozdějšímu zkoumání, zda tento majetek spadá do pozůstalosti, nebo nikoliv.\\

Další možnost pro převod majetku představuje založení svěřenského fondu \textit{morti causa}, která vedle dědického řízení slouží jako další prostředek, který zůstavitel může, krom pořízení pro případ smrti, užít ke kontrole způsobu nakládání se svým jměním po jeho smrti. Potenciální problém se založením svěřenského fondu \textit{mortis causa} může nicméně nastat, pokud by toto založení mělo směřovat k úmyslnému zmenšení jednotlivých povinných dílů náležejících nepominutelným dědicům. V aktuální právní úpravě totiž není jasně stanoveno jakým způsobem se majetek při smrti zůstavitele, který svěřenský fond zřídil pořízením pro případ smrti do tohoto fondu vyčlení. Nejasný výklad ustanovení svěřenského fondu ve vztahu k dědickému právu představuje právní nejistotu, nejen pro dědice, která není žádoucí. \\

S ohledem na popis dědického práva vyložený v předchozí kapitole je třeba s ohledem na založení svěřenského fondu \textit{mortis causa} nejdříve vyřešit jakým způsobem probíhá jeho založení pořízením pro případ smrti. S ohledem na § 1452, odstavec 1 je po výkladové stránce zřejmé, že zákonodráce zamýšlel způsob vyčlenění majetku do svěřenského fondu ve smyslu § 311, který upravuje vyčlenění majetku do nadace, a který zároveň satnovuje v ostavci 1 následující: "(1) Při založení nadace pořízením pro případ smrti se do nadace vnáší vklad povoláním nadace za dědice nebo nařízením odkazu. V takovém případě nabývá založení nadace účinnosti smrtí zůstavitele."\footfullcite[§ 311, odst. 1]{noauthor_zakon_2012} Z tohoto ustanovení, které je možné analogicky použít na svěřenský fond, je možné dovodit úmysl zákonodárce pohlížet na majetek, který má být vyčleněn do svěřenského fondu, jako na součást pozůstalosti. Toto tvrzení svým komentářem potvrzuje i Vlastimil Pihera, který ve svém komentáři k § 1448 dovozuje, že při zřízení svěřenského fondu pořízením proo případ smrti vyčlení zakladatel (rozuměj zůstavitel) majetek ve prospěch svěřenského fondu tak, že v jeho prospěch vyhradí dědické právo, nebo nařídí odkaz\footfullcite[komentář k § 1448]{spacil_j_a_kolektiv_obcansky_2013}. \\

%Na svěřenský fond zřízený \textit{mortis causa} je v rámci občanského zákoníku kladený velký důraz, je nicméně vhodné poznamenat, ať již se jedná o svěřenský fond založený za veřejným a nebo soukromým účelem.

V rámci pohledu na právní úpravy trustů ze zahraničí je možné se setkat s dalšími druhy trustů a mnohými jinými možnosti, za kterýchžto účelem může být svěřenský fond založený, tyto se však Český zákonodárce dále rozhodl v našem občanském zákoníku neupravovat.

\newpage
\thispagestyle{smallertextinheader}

Tento majetek je dle Vlastimila Pihery a i dle mého výkladu a pochopení OZ tedy součástí pozůstalosti, tím pádem, dle § ...., který stanovuje, že pozůstalost tvoří veškeré jmění v držení zůstavitele ke dni jeho smrti, se z výše tohoto jmění spolu se započtením výše jmění, které není vyčleněno do svěřenského fondu počítají jednotlivé dědické podíly, které je správce pozůstalosti povinen převést jednotlivým nepominutelným dědicům. S ohledem na to, že další úprava upravující tetnto souběh v OZ chybí a zároveň k tomuto problému neexistuje žádná judikatura a není k němu ustálená soudní praxe.....

a výklad komentáře.

Popsat, že existují dva různé pohledy, já se ztotožňuji s druhým pohledem.

\subsubsection[Vznik svěřenského fondu \textit{inter vivos} s odkládacé podmínkou]{Vznik svěřenského fondu \textit{inter vivos} s odkládací\\ podmínkou}

Zajímavý je dovození, že lze vyčlenit majetek svěřenského fondu smlouvou s odkládací podmínkou ve smyslu ustanovení § 548 až 549 OZ, jak činí Miloš Kocí232. Pokud by tou podmínkou byla smrt zakladatele, šlo by o svěřenský fond mortis causa, ale založený způsobem předpokládaným u svěřenských fondů inter vivos, tedy dvoufázově (případně jednofázově, pokud takové přijetí pověření ke správě je obsaženo ve smlouvě). Pokud nepřijme svěřenský správce pověření ke správě ke smrti zakladatele, takový svěřenský fond vůbec nevznikne.

V souvislosti s ustanovením § 548, zejména odstavce (2), a 549 občanského zákoníku lze (obdobně) uvažovat o další formě vzniku svěřenského fondu. Tento vznik by byl vázán na určitou odkládací podmínku. Zakladatel může stanovit mnoho různých druhů odkládacích podmínek, na jejichž splnění se váže vznik práv, může se tak jednat například o podmínku dokončení studia na vysoké škole, po jejímž splnění\footnote{Je samozřejmě nutné splnění dalších podmínek, které na založení svěřenského fondu klade zákon, tedy například přijetí svěřenským správce, srovnání viz výše.} vznikne svěřenský fond a obmyšlený, v tomto případě daný student vysoké školy, bude mít bez dalšího právo na vyplácení prostředků z tohoto svěřenského fondu.\\

Zajímavá je ovšem situace, která může v tomto případě nastat. Ta souvisí, se samotným stavem zakladatele svěřenského fondu. Pokud by zakladatel zemřel a odkládací podmínka by nebyla vázána na jeho smrt ale na splnění jiné podmínky, hledělo by se na takto založený fond jako na \textit{inter vivos}.\\

%(Pokud je mrtev jedná se stále o založení svěřenského fondu \textit{inter vivos}? Já si myslím, že ano}.

Dalším zajímavým příkladem je možnost stanovit smrt zakladatele jako odkládací podmínku, protože pak by se jednalo ve své podstatě, samotnou právní skutečností, smrtí zůstavitele, o obdobu založení svěřenského fondu \textit{mortis causa}, přitom však věřím, že by se na toto založení pohlíželo jako na založení svěřenského fodnu {inter vivos}. Kdy by tedy zakladatel ještě za svého života určil ve smlouvě majetek, který má být z jeho vůle svěřenským správcen držen a spravován za vymezeným účelem v rámci první fáze vzniku a následně by v rámci druhé fáze vzniku přijal toto pověření určený svěřenský správce. Založení by také mohlo být jednofázové, pokud by přijetí takovéto správy bylo již vedené jako součást smlouvy mezi zakladatelem a svěřenským správcem. Přestože by samotným důvodem ke vzniku svěřenského fondu byla smrt zakladatele, nejednalo by se dle mého názoru o založení svěřenského fondu \textit{mortis causa}, ale o založení svěřenského fondu \textit{inter vivos} s tím, že by svěřenský fond dle novely č. z. 460/2016 Sb. vznikl až zápisem do evidence svěřenských fondů, tento zápis by tedy byl konstitutivní.\\

S ohledem na fakt, že takovéto založení svěřenského fondu stále spadá do kategorie \textit{inter vivos}, v rámci kterého takto vyčleněný majetek není součástí dědického řízení, respektive pozůstalosti po zesnulém\footnote{Viz výše.}. Je dle mého názoru možné využít tuto formu založení svěřenského fondu místo klasického založení \textit{mortis causa}, u kterého existuje právní nejistota, zda je takto vyčleněný majetek součástí pozůstalosti.\\

Tato forma založení svěřenského fondu ssebou ale nese i jisté nevýhody oproti formě založení pro případ smrti. Těmi jsou především.

Pokud by tedy zakladatel chtěl docílit mezigeneračního převodu majetku avšak chtěl by ad libidum fakticky vydědit jakéhokoliv svého dědice, nebo se vyhnout dědickému řízení, je cesta buď založení svěřenského fondu \textit{inter vivos}, nebo v případě, že by chtěl zakladatel se svým majetkem v mezích disponoval až do doby své smrti, pak je s ohledem na nejasnosti vztahující se k založení svěřenského fondu \textit{mortis causa} vhodné, vydat se cestou založení svěřenského fondu \textit{inter vivos} s odkládací podmínkou smrti zakladatele.

%V nějaké zemi je možné žádat o to, aby bylo dědicům vyplaceno část dědictví i z toho, co dal zakladatel komukoliv 3 roky před jeho smrtí, najít kde a doplnit to sem a poznamenat, že u nás toto neplatí.

\newpage
\thispagestyle{smallertextinheader}

\subsection{Typy svěřenských fondů}

%http://www.bulletin-advokacie.cz/funkcni-kategorizace-sverenskych-fondu#ftn10

%Podívat se ještě na to, zda existují v českém právu odvolatelné a neodvolatelné svěřenské fondy, a toto také doplnit do těla textu, respektive do této kapitoly

Trust je založen na velké, řekl bych téměř neomezené, variabilitě účelů. Z toho důvodu je možné očekávat, že v rámci české obdoby trustu, jeho využití k velké škále předem nedefinovaných účelů. Z tohoto důvodu není, krom zákonného vymezení, možné poskytnout celkový a komplexní list zahrnující seznam všech potenciálních účelů svěřenského fondu, nebo oblastí jejich využití. V rámci publikací zaměřujících se na toto téma se však lze setkat s určitou kategorizací v rámci jejich účelu a využití, s přihlédnutím k liteře zákona a záměru zákonodárce a zahraničního využití svěřenských fondů. S ohledem na toto tedy odborná a komentářová literatura nabízí následující kategorizaci oblastí využití svěřenských fondů:

\begin{enumerate}
\item obchodní svěřenský fond
\item rodinný svěřenský fond
\item diskreční svěřenský fond
\item testamentární svěřenský fond
\item bezúplatný svěřenský fond mezi živými
\item zajišťovací svěřenský fond
\item ochranný svěřenský fond
\item přidružený fond nadace.
\end{enumerate}

\subsection{Ostatní formy správy majetku, rozhodující faktory pro vhodný výběr a z toho vycházející problémy, doporučení a výhody}

Vhodný výčet rozhodujících faktorů, který je přímo aplikovatelný na výběr, pro zakladatele, vhodné formy správy majetku dobře shrnuje ve svém příspěvku do Bulletinu advokacie Kateřina Ronovská\footfullcite[Vydání 7-8 2016]{ronovska_bulletin_vyber_formy_spravy_majetku}.\\

S těmito faktory se osobně ztotožňuji, mé potenciální doporučení a upozornění plynou zejména z aplikování těchto faktorů na právní úpravu svěřenských fondů.\\

Docentka Ronovská uvádní následující faktory, které aplikuje na větší množství forem správy majetku:

\begin{itemize}
	\item Faktor 1: stabilita právního prostředí, důvěra v právní řád jako celek, předvídatelnost soudního rozhodování,
	\item Faktor 2: vhodně a proporcionálně nastavený právní rámec; v oblasti práva soukromého zejména ponechání dostatečného prostoru pro realizaci představ vlastníka majetku (při vymezení účelu, způsobu správy, zakotvení práv osob beneficientů), jakož i efekt (ochrana majetku) vůči třetím osobám,
	\item Faktor 3: míra diskrétnosti, resp. ochrany soukromí,
	\item Faktor 4: způsob a adekvátnost dohledu nad správou (důvěra v osobu správce majetku a eliminace možnosti jeho selhání),
	\item Faktor 5: nákladnost zřízení určitého instrumentu, jakož i následné "provozní" náklady spojené se správou,
	\item Faktor 6: "příznivý" nebo alespoň "neutrální", resp. "nedemotivující" daňový režim.
\end{itemize}

%Do takto vymezených faktorů lze zařadit všechny mnou nalezené nedostatky.

% Možná přidám nákladnost/nenákladnost řízení do poslední části práce, možná taky přidám zavedení evidence svěřenských fondů.

\newpage
\thispagestyle{smallertextinheader}

\noindent\begin{tabularx}{\textwidth}{X|X|X}
\textbf{Faktor} & \textbf{Problém} & \textbf{Výhoda} \\
\hline
Faktor 1 & 
\begin{itemize}
\item Nejasná regulace a chybějící judikatura 
\item Rozdílné pojetí vlastnictví
\item Zákonné nejasnosti změn svěřenských fondů před a za jejich trvání
\end{itemize}
& \\
\hline
Faktor 2 & 
\begin{itemize} 
\item Systémové nedostatky vztahující se k dědickému právu,
\item Započtení na dědický podíl,
\item Nejasný výklad ochrany nepominutelných dědiců,
\item Nejasný výklad ve vztahu k vyživovací povinnosti
\item Možnosti použití svěřenského fondu k obcházení věřitelů
\end{itemize} 
& \\
\hline
Faktor 3 & 
\begin{itemize}
\item Zavedení evidence svěřenských fondů
\end{itemize}
& \\
\hline
\end{tabularx}

\newpage
\thispagestyle{smallertextinheader}

\noindent\begin{tabularx}{\textwidth}{X|X|X}
\textbf{Faktor} & \textbf{Problém} & \textbf{Výhoda} \\
\hline
Faktor 4 & 
\begin{itemize}
\item Nezávislost správce, nezávislost správce, který je zároveň obmyšlený
\end{itemize}
& \\
\hline
Faktor 5 & 
\begin{itemize}
\item Nákladnost řízení
\end{itemize}
& Nenákladnost řízení\\
\hline
Faktor 6 & 
\begin{itemize}
\item Nejasnosti v rámci daňových aspektů svěřenského fondu
\end{itemize}
& \\
\hline
\end{tabularx}

\subsection{Potenciální možnosti použití v rámci převodu majetku a jeho přínos oproti klasickým způsobům pořízení pro případ smrti a jiným způsobům převodu práv}

%Je prokázáno, že pokud je rodina hodně bohatá, tak její našetřené peníze v průměru utratí již 6 generace, nicméně asi to nebude výhoda ve vztahu ke svěřenskému fondu s holedem na maximální vyplácení po 100 letech a kvůli tomu, že svěřenský fond nemůže zakládat další svěřenský fond. Ale raději zjistím, mohl bych totiž potenciálně přednést jako další výhodu a benefit.

%Možná toto pojmu s ohledem na jakékoliv jiné formy převodu majetku, ne jenom dědění, částečně to tak již mam, ale mohl bych to i rozšířit. Musel bych tedy tuto část rozšířit i o porovnání s těmito způsoby převodu majetku.

%Měl bych zde vyjít i z Barbory bednaříkové, dále doplnit zdroje od Lucie Joskové ze Správy Cizího Majetku.

Nový občanský zákoník, který vstoupil v účinnost prvního ledna roku 2014 okruh nástrojů sloužících pro převody a správu majetku nadále rozšířil zavedením obecné úpravy správy cizího majetku a možnosti zřizování svěřenských fondů. S ohledem na téma práce je důležité zmínit, že tato nová právní úprava občanského práva přispěla k rozšíření možností, které lze využít při mezigeneračním převodu majetku, tedy především v případech, které byli dříve řešeny výhradně v rámci dědického řízení popřípadě pomocí darování.\\

Tyto mezigenerační převody majetku spočívají nejčastěji v situacích, kdy takovéto nakládání s majetkem je činěno ve vztahu k mladší generaci v jedné rodině, tedy především k potomkům a prapotomkům. %Lingvistický výklad slova mezigenerační v sobě nicméně nezahrnuje pouze pokrevně či právně zpřízněny generace, ale i různé demografické ročníky. Je tak klidně možné tento převod učinit 
Tento převod majetku samozřejmě nemusí být jenom mezigenerační, jedná se i o případy, kdy byl majetek převáděn manželovi, partnerovi, rodinému příslušníkovi, či kterémukoliv dalšímu příbuznému. Není nicméně vyloučeno, aby takovéto jednání spočívalo v převedení majetku jiným blízkým osobám, jako jsou například přátelé, nebo klidně i osobám zůstaviteli naprosto cizím.\\

%či k jiným blízkým osobám - přátelé, není nicméně vyloučeno, aby takovéto jednání spočívalo ve snaze převést majetek na osobu, která je zůstavitelovi zcela cizí. \\

%Toto budu asi muset ještě upravit a lépe navázat na další odstavec.

Převod majetku z účelem jeho mezigeneračního předání lze samozřejmě dosáhnout i jinými způsoby. Tyto způsoby zahrnují klasické způsoby převodu ve smyslu dědění ale i jiné způsoby nabytí vlastnických práv. Způsobů nabytí vlastnických práv upravuje Občanský zákoník v části třetí, Absolutní majetková práva, hlavě II., věcná práva, dílu 3, oddílu 2 hned několik. V rámci mezigeneračního převodu je však použitelný výhradně pododdíl 6, převod vlastnického práva, na kterém stojí i všechny odlišné způsoby mezigeneračního převodu majetku.

%Doplnit všechny odlišené způsoby převodu majetku a okomentovat je v souvislosti s výhodami mezigeneračního převodu majetku za použití svěřenského fondu

S ohledem na to, že institut svěřenského fondu představuje v české právní úpravě stále poměrně nový přírůstek, tak nejsou zásadněji představeny, natož používány, možnosti, kterých lze prostřednictvím založení svěřenského fondu dosáhnout v rámci mezigeneračního převodu majetku. S ohledem na výše zmíněné a na, i díky událostem poslední doby, větší zájem široké veřejnosti o toto téma, se tato kapitola bude soustředit na možnosti a výhody svěřenského fondu právě v oblasti převodu majetku mezi generacemi a zároveň s ohledem na tyto body, které do značné míry vychází z pojetí svěřenského fondu z pohledu Kocího, načne tedy i potenciální problémy, především s ohledem na kogentní úpravu dědického práva a absenci právní úpravy svěřenských fondů, která by upravovala kolizi mezi kogentní úpravou dědického práva a úpravou svěřenských fondů, které mohou vzniknout při zvolení této cesty společně s nevýhodami, které může tato forma převodu majetku pro všechny zůčastněné strany mít. Část textu se bude věnovat i čistě jursiprudenčnímu pohledu na recepci právní úpravy trustu, spočívající především v pohledu na jednotlivé otázky vyvstávající s přijetím institutu běžnému právní úpravě common law do systému civil law, například ve smyslu rozdílného pojetí vlastnického práva, právní osobnosti.
%Poslední větu bude asi nejspíše třeba upravit. V rámci právní osobnosti myslím právnickou osobu. V případě potřeby upravit tento odstavec.
%Hlavně bude tuto poslední část potřeba přesunout do vhodné sekce v rámci mé bakalářské práce.

\subsubsection{Výhody svěřenského fondu jako alternativního nástroje dědické sukcese}

%Doplnit porovnání i s ostatními prostředky převodu majetku mimo dědictví, to je v další kapitole, doplnit tedy tak například, darování, založení nadace, či jiné právnické osoby a tak dále.

%S ohledem na hlavní téma této práce, tedy svěřenský fond jako mezigenerační nástroj převodu majetku, představují způsoby založení svěřenského fondu a jejich specifika vhodný startovní bod pro tuto část práce a obecně pro celou zkoumanou problematiku. \\

%Základním faktem týkajícím se založení svěřenského fondu je, i s ohledem na popsanou flexibilut instutut trustu, ze kterého český zákonodárce vycházel, je nepřekvapivě možnost založení jak mezi živými (\textit{inter vivos}), tak i pro případ smrti (\textit{mortis causa}).

Pokud se podíváme na internet a položíme jednomu z mnoha internetových vyhledávačů otázku "Jaké výhody má založení svěřenského fondu", tak nás ohromí nepřeberné množstí odkazů na jednotlivé stránky, které jen překypují popsanými benefity. Při mé první exkurzi do světa svěřenských fondů po novelizaci Občanského zákoníku mě toto nepřeberné množství výhod opravdu překvapilo. Od té doby jsem samozřejmě vystřízlivěl a na jednotlivé body již nahlížím mnohem kritičtějším pohledem. Jedna věc se ovšem musí svěřenským fondům nechat, možnosti, které jejich založení přináší zakladateli ve smyslu možností, jak ovlivňovat nakládání se svým majetkem i po své smrti jsou opravdu praktické.\\

Přes toto vše není založení svěřenského fondu pro všechny, neboť má i jisté nevýhody a vyplácí se jen v určitých situacích. Dále nepovažuji zakotvení této právní úpravy v Občanském zákoníku za konečné a troufám si tvrdit, že s ohledem na různé výkladové nejasnosti, je velice pravděpodobné, že dojde ke změnám právní úpravy. Tyto změny, jejichž příklady uvádím v poslední kapitole, jsou nutné k tomu, aby tyto alternativní nástroje mohli zcela plnit svůj účel jako alternativy, nebo suplementy ke klasickým způsobům převodu majetku zejména v oblasti dědického práva. Z tohoto hlediska je možné v tomto směru pochybovat o právní jistotě, neboť se dle mého názoru dočkáme jak hmotněprávních, tak i procesněprávních změn.\\

 Problematické části svěřenských fondů budou nicméně zhodnoceny níže. Tato část je věnována, jak již název napovídá, výhodám tohoto alternativního způsobu předání majetku. 
 
 Pokud se budeme bavit o mezigeneračním převodu majetku, tak se nám jistě primárně vybaví rodinný svěřenský fond. Jak již jsem výše poznamenal, obmyšlený může být samozřejmě kdokoliv, i rodině naprosto cizí člověk. Věřím však, že svěřenský fond jako nástroj mezigeneračního převodu majetku bude sloužit především převodu v rámci rodiny, a z tohoto důvodu se budu dále soustředit výhradně na tento typ svěřenského fondu a na ukázku příkladů k jednotlivým situacím, které je výhodné řešit právě pomocí založení svěřenského fondu.\\
 
 %Popsat i ostatní typy svěřenských fondů
 
 Rodina dozajista představuje největší radost v životě a dá se považovat i za jistý cíl naší existence. Je proto logické, že pro staší generaci představuje rodina motivaci, kvůli které se snaží finančně zajistit jak sebe, tak i svoji rodinu a společně pak pro všechny členy najít bezpečí po všech směrech a ve smyslu všech lidských potřeb. Finanční zajištění saturuje mnoho těchto potřeb a pro mnoho rodičů tedy představuje zachování a předání jistého odkazu ve formě dědictví pro další generace téměř životní poslání. Předání tohoto odkazu může ale mít i potenciální negativní konsekvence, jak říká i známe české přísloví \textit{"Rychle nabyl, rychle pozbyl"}, velkou neznámou tedy může představovat fakt, zda by takto rychle nabyté dědictví mladší generace zvládla. Řešení nejen tohoto problému může představovat právě založení svěřenského fondu, ať už za života zakladatele, či pro případ jeh smrti.\\
 
 Věřím, že nejlepší pojetí této kapitoly představuje kompilace mnoha různých výhod svěřenského fondu a k nim uvedeným příkladům, proto jsem se rozhodl zvolit obdobnou formu jako Barbora Bednaříková ve svém díle Institut pro uchování a převody rodinného majetku. Začal bych tedy, z mého pohledu, největší výhodou svěřenského fondu. Zde se bez jakýchkoliv výhrad ztotožňuji s pohledem Lucie Joskové a Lukáše Pěsny, kteří tvrdí, že základní výhodou svěřenského fondu je samostatné vyčlenění části majetku\footfullcite{lucie_joskova_sprava_2017}. Tato výhoda tedy spočívá ve faktu, že daný majetek je v podstatě majetkem \textit{sui generis}, který nepatří nikomu, ale pouze je spravován svěřenským správcem. Je tedy chráněn před počínáním obmyšlených a jedná se mimo jiné o jistou formu ochrany dědictví. Jedná se zde tedy nejen o základní výhodu samotného svěřenského fondu, která svojí použitelností přesahuje i do mnoho dalších právních odvětví, ale specificky ve vztahu k dědickému právu z této obecné výhody vyplývá i, dle mého názoru, nejzásadnější výhoda a důvod, pro které je vhodné využít svěřenský fond k mezigeneračnímu předání majetku. Touto výhodou je ochrana dědictví jako takového.\\
 
 %Tato ochrana je dle m=ho názoru základní výhodou svěřenského fondu, pokud ho speciálně vztáhneme na tuto funkic. 
 
 %mezigeneračního převodu majetku za pomoci svěřenského fondu, kterou jsem již nakousl výše, tedy jako jistou ochranu dědictví.
 
 %Další výhoda, možnost dát si podmínku, aby se o zakladatele obmyšlení starali v rámci statutu svěřenského fondu, samozřejmě pro svěřenské fondy inter vivos, porovnat s dovětkem
 
 %Ochrana nejen před věřiteli, ale i před třetími osobami, které by mohli chtít obmyšlené, nebo zakladatele využít
 
 %Hodně těchto benefitů lze dosáhnou i klasickými smlouvami, ale všechny jednotlivé benefity svěřenských fondů působí společně s těmi dvoumi nebo třemi hlavními benefity a těmi jsou oddělené vlastnictví, ochrana před věřiteli a anonimizace vlastnictví, pokud by byla tedy zvolena cesta pouze smluvní, tyto benefity by vedle těchto smluvně ujednaných náležitostí, které poskytují tyto benefity dále nepůsobily.\\
 
 \begin{enumerate}
 {\Large\item Ochrana dědictví před rozmařilostí, nezkušeností a nezodpovědností ze strany dědiců}
 \item[1.] Mezigenerační transfer a uchování majetku
 \end{enumerate}
 
 Přes zákonné nejasnosti\footnote{ V první řadě je důležité zmínit, že následující výklad nemusí být kvůli nedostatečnostem jak lingvistickým, tak i věcným v právní úpravě svěřenských fondů absolutně přesný a je možné, že díky potenciálním problémům vytyčeným níže se může zákonná úprava změnit a s tím společně i tento bod. S ohledem na pohled právních expertů na problematiku svěřenských fondů a mého názoru k datu publikace této práce je vhodné tento bod zařadit v podobě, která se s ohledem na názorovou disputaci a pohledem na procesněprávní náležitosti svěřenského fondu kloní spíše k názoru Miloše Kocího. Tedy, že majetek vyčleněný do svěřenského fondu není součástí pozůstalosti, další informace o tomto problému jsou popsány v další kapitole.} v úpravě svěřenských fondů, se možnost založení svěřenského fondu za účelem ochrany dědictví jeví jako vhodná v případech, kdy se jedná o větší majetek.\\
 
 Nestává se zřídka, že rodiče se strachují, zda jejich potomci bezpečně zvládnou po nich nabyté dědictví. Založení svěřenského fondu tak v tomto případě může pomoci nejen s ohledem na fakt, že potomci nedostanou jednorázově velký finanční obnos, ale i v mnoha dalších ohledech.\\
 
 Nezvládnutí takto nově nabytého dědictví se může projevovat mnoha způsoby, které zahrnují; neuvážlivé a rozmařilé utrácení peněz, ztrátu motivace k vlastnímu výdělku, zanedbávání povinností, špatná finanční rozhodnutí a nerozvíjení finanční gramotnosti, nepoznání hodnoty peněz, práce a času. Tyto rizika lze mitigovat právě založením svěřenského fondu, díky kterému nejen, že dědicové nedostanou jednorázově všechno jmění\footnote{Srovnání další bod, ochrana před věřitely}, ale v rámci statutu svěřenského fondu jim lze stanovit podmínky, které musí pro plnění ze svěřenského fondu splňovat a plnit\footfullcite[§1452, odst. 2, bod d)]{noauthor_zakon_2012}. Těmito mohou být například požadavek studia, podmínka dosáhnutí určitého věku k vyplácení vyšší části a podobně.\\
 
 Dále lze jako součást procesu založení svěřenského fondu definovat i jisté rodinné poslání, které může sloužit jako opora při stanovení společných hodnot a cílů. Založení svěřenského fondu tak ve svém důsledku nemusí chránit jednotlivce před majetkem a mejetek před jedinci, ale může sloužit jako ochrana pro celou rodinu. Díky svěřenskému fondu lze sledovat společné cíle všech členů rodiny a vytvořit tak díky tomuto společný mezigenerační plán. Správa rodinného majetku je tak svěřena všem jejím členům, kteří se na ní podílejí a sdílí mezi sebou odpovědnost za rozhodnutí učiněná při péči o vyčleněn\-ý majetek. Tento proces pozitivně formuje potomky a v konečném důsledku je připraví na zodpovědnou správu rodinného jmění.\\
 
 Zde lze opět pochválit zákonodárce, za přijetí svěřenského fondu, neboť absence úpravy ochrany, před marnotratnou osobou před přijetím nového Občanského zákoníku, byla opravdu velkou chybou, neboť tato absence může vést k pauperizaci jednotlivých státem nechráněných osob.\\
 
 Je zde samozřejmě na místě diskuze, do jaké míry by měl stát chránit občany a dělat jim faktického opatrovníka a do jaké by měl nechat jejich počínání čistě na jejich vůli. Nicméně právní stát založený na vzájemné úctě a solidaritě by se měl snažit své občany před chudnutím chránit, a to zejména v případech, kdy by tato marnotratnost mohla dopadnout i na jiné osoby, což by v případě dědického práva do značné míry určitě mohla. Již v antickém římě byl majetek před tímto počínáním chráněn ve formě opatrovnictví pro marnotratné\footfullcite[Strana 54]{michaela_zidlicka_personae_2015}, kteréžto se svým obsahem nejblíže rovná spendthrift trustu. Zavedení takovéto právní úpravy tedy vede k facilializaci ochrany majetku před marnotratností a posílení ochrany zůstavitelovi vůle vůči tomuto majetku i po jeho smrti a ochrany dalších dědiců, zejména marnotratníkových, nebo obmyšlených.\\
 
 %cura furiosi, cura debilium a cura prodigy
 
 Je třeba zmínit, že tento bod představuje opravdový benefit pouze v případě, faktrické možnosti vydědění pomocí svěřenského fondu, pokud bychom na problematiku nepominutelných dědiců pohlíželi jinou optikou, a to takovou, že nepominutelný dědic má právo i z majetku vyčleněného do svěřenského fondu, tento benefit by pak nebyl tak zřetelný, i v takovém případě by však mohl chránit majetek více, než pokud by zůstavitel ponechal převod majetku na dědickém právu bez pořízení pro případ smrti, tedy za pomoci intestátní posloupnosti.\\
 
 Věřím, že tento účel může představovat jeden z hlavních důvodů, pro který by mohli rodiče zakládat rodinný svěřenský fond \textit{inter vivos} i \textit{mortis causa}, neboť v českém právním řádu nebyli před přijetím NOZ umožněny podobné konstrukce, respektive právní uspořádání, které by takto majetek chránili. Velkou oblibu si tento typ fondu získal například ve Spojených státech Amerických pod pojmem "spendthrift" trust ZDROJ Barbora Bednaříková.\\
 
 Jistou alternativu spatřuji v použití dovětku, ve kterém zůstavitel může stanovit určitou podmínku, lze ale jasně říci, že použití dovětku nelze docílit takové univerzálnosti a všestrannosti, kterou poskytuje svěřenský fond ZDROJ OZ, další zdroj zde: https://www.nkcr.cz/sluzby/dedicke-pravo/dovetek. Obdobně lze spatřit možnost využití paragrafů 724 a 1647 Občanského zákoníku, opět se ale jedná o použití ve specifických situacích, přičemž u druhého z případů musí dojít k vydědění, kdy zůstavitel takovéto marnotratné počínání ze strany svého dědice musí spatřovat již za svého života a nemůže ho použít pouze jako prevenci potenciálního marnotratného počínání.\\
 
 \begin{enumerate}
 \item[2.] Ochrana dědictví před jinými dědici
 \end{enumerate}
 
 Přes obdobné zákonné nejasnosti\footnote{Při tomto bodu lze také namítat nejasnost právní úpravy svěřenských fondů, srovnej DOPLNIT ODKAZ NA PROBLÉM VZTAHUJÍCÍ SE K TOMUTO BODU} lze jako na potenciální benefit použití svěřen\-ského fondu k předání majetku nahlížet i na možnost zvolení beneficienta dle uvážení zakladatele. Přes kogentní ustanovení dědického práva je tedy možné stanovit obmyšlené, kteří nespadají do kategorie nepominutelných dědiců, či dokonce obmyšlených, kteří nejsou v žádném příbuzenském vztahu se zakladatelem, a tímto způsobem tak omezit právo těchto dědiců na podíl z pozůstalosti, bez jejich faktického vydědění\footnote{Srovnej s kapitolou Dědické právo v České republice, důvody k vydědění}\textsuperscript{,}\footnote{Toto lze samozřejmě považovat i za nevýhodu, pokud se na tuto možnost koukáme z pohledu dědice, srovnej DOPLNIT ODKAZ}. Tento způsob se jeví při současném výkladu právní úpravy svěřenského fondu, s porovnáním s právní úpravou dědického práva, jaho vhodnější oproti vydědění. Dědické právo nenabízí mnoho důvodů vydědění a není výjimkou, že i když takovéto důvody existují, je složité je prokázat. Tato možnost je vhodná v případech, kdy chce zakladatel zamezit dědickému nároku osob, u kterých neshledává, že by měli mít na část pozůstalosti nárok. Na majetek vložený do svěřenského fondu by tedy tyto dědicové neměli již nárok, neboť takto vylčeněný majetek přestává být ve vlastnictví zakldadatele a stává se výlučným vlastnictvím svěřenského fondu a plody a užitky z tohoto majetku náleží výhradně obmyšlenému.
 
 %Hodnota peněz a práce a času
 
 \newpage
 \thispagestyle{smallertextinheader}
  \begin{enumerate}
 {\Large\item[2.] Ochrana před věřitely}
 \end{enumerate}
 
 Další důležitou výhodu uvádí ve svém díle Správa cizího majetku (dále jen Správa cizího majetku) Lucie Joková a Lukáš Pěšna. Tento bod úzce souvisí s výhodou, kterou jsem již nastínil v předchozím bodu, tedy s převedením majetku na jiný kvazi subjekt a tím subsekventní oddělení daného majetku od majetku zakladatele, svěřenského správce i obmyšleného. Věřitelé koholiv z výše vymezených osob, v rámci tohoto bodu je jako tato osoba přednostně myšlena osoba zakladatele svěřenského fondu, a tedy nemohou v rámci svých pohledávek za jakoukoliv z těchto osob uspokojovat z majetku, který byl vyčleněn do svěřenského fondu, toto platí i v případě úpadku těchto osob. Je nicméně důležité stanovit, že plnění z tohoto fondu ve prospěch obmyšlených může sloužit k uspokojení věřitelů\footfullcite{lucie_joskova_sprava_2017} dle exekučního řádu\footfullcite[§59]{noauthor_zakon_2001}.\\
 
 Výborný příklad je uveden v tomtéž díle: \textit{"Zakladatel se zymýšlí pustit do rizikového podnikání. Vyčlení proto část svého majetku do svěřenského fondu a jako obmyšlené určí sebe a svou rodinu. Tím majetek vyjme z dosahu svých potenciálních budoucích věřitelů. V případě podnikatelského neúspěchu se věřitelé budou moci uspokojit z majetku zakladatele, nikoli však z majetku vyčleěného do svěřenského fondu. To nevylučuje právo věřitelů domáhat se uspokojení z jednotlivých nároků, které zakladateli ze svěřenskému fondu plynou jako obmyšlenému."}\\
 
 Jak dále uvádí autoři, je nicméně nutné podotknout, že si pod založením svěřenského fondu nelze představovat jakýsi nástroj ochrany zakladatelova majetku pro každou situaci. Obecně tak například nelze do svěřenského fondu vyčlenit majetek po vzniku pohledávky, neboť takto vyčleněný majetek by mohl zhoršit dobyvatelnost pohledávky a toto vyčlenění by tak bylo zatíženo relativní neúčinností dle § 589 až 599 obč. zák., popřípadě by mohl být věřiteli odporován dle paragrafů insolvenčního práva, §235 až §243 insolvenčního zákona.\\
 
 Tomuto přisvědčuje nejen Kocí v Komentáři k Občanskému zákoníku jak uvádí správně autoři, ale i Svejkovský a Marek ve svém díle Správa cizího majetku v novém občanském zákoníku v komentáři k paragrafu 1467\footfullcite{jaroslav_svejkovsky_sprava_2015}.\\
 
 \newpage
 \thispagestyle{smallertextinheader}
 
  \begin{enumerate}
 {\Large\item[3.] Anonimizace či diskrétnost}
 \end{enumerate}
 
 V případech snahy o zachování jisté diskrétnosti, pro jednoduchost uveďmě, převodce a nabyvatele určitých práv, poskytuje svěřenský fond v tomto ohledu značnou výhodu oproti klasickým způsobům převodu, kdy je ve veřejně přístupných seznamech možné vidět obě osoby ať již přímo, či ve sbírce listin. Novela Občanského zákoníku a Zákona o veřejných rejstřících právnických a fyzických osob z roku 2018 tento bod poměrně zkomplikovala povinným zápisem svěřenských fondů do evidence. Měla ovšem i pozitivní přínost v jiných ohledech, zejména pokud se jednalo o oprávněné obavy, že by svěřenský fond mohl být použit k legalizaci výnosu z trestné činnosti, či dalších nelegálních aktivit. V mnohém tedy tato novela zbourala představu absolutní anonimizace zakladatelé a obmyšlených, kteří jsou teď povinně zapisováni do evidence. Není tomu ovšem tak, je sice pravda, že osoba zakladatele a jednotliví obmyšlení jsou zapsání do evidence, ale data z této evidence jsou poskytnuty jiné osobě pouze pokud prokáže právní zájem\footfullcite[§65e odst. 3]{noauthor_zakon_vr}. Na první pohled je ve všech relevantních seznamech uvedena pouze osoba svěřenského správce.\\
 
 Tato dodatečná bariéra, mezi daty a subjekty, které by dané informace chtěli získat slouží jako další ochrana před věřiteli, či jinými osobami, kteřé by tyto informace mohli využít ke svému prospěchu a poškození zakladatele či obmyšlených, slouží tedy jako firewall, který zajišťuje zakladateli a obmyšleným větší bezpečí a pohodu. Kvůli nečitelnosti vlastnictví je tedy ztíženo zjistit informace týkající se převodu majetku, distribuci výnosů z majetku ať již vně rodiného kruhu, či mimo rodiný kruh.\\
 
 % \begin{enumerate}
 %{\Large\item[4.] Mezigenerační transfer a uchování majetku}
 %\end{enumerate}
 
 %Vypuštěno, neboť si myslím, že jsem tento bod zahrnul pod prvním bodem
 
 \newpage
 \thispagestyle{smallertextinheader}
 
  \begin{enumerate}
 {\Large\item[4.] Udržení celistvosti a ochrana majetku}
 \end{enumerate}
 
 Založení rodinného svěřenského fondu má také pozitivní vliv na udržení celistvosti a tedy i na ochranu majetku. Toto je důležité zejména v případech, kdy existuje více dědiců a pozůstalost by mělo tvořit jmění, u kterého si lze jen ztěží představit možnost jeho rozdělení mezi dědice, popřípadě by takové rozdělení představovalo například značnou konkurenční nevýhodu, popřípadě přenesení odpovědnosti na všechny dědice, aby se zabránilo nezodpovědnému počínání jen některých z nich, ovšem bez nutnosti ponižovat podíly náležející těmto obmyšleným. Takovéto jmění může představovat například rodinný závod, nemovitý majetek či různé investiční fondy a podobně. Je samozřejmě možné tento potenciální problém vyřešit dědickou smlouvou, závětí, či darováním. S ohledem na bod 1 si však myslím, že vložení tohoto majetku do svěřenského fondu mu zajistí ochranu a v souvilosti s bodem 1 může obmyšlené i v uvozovkách vzdělat v uvědomělé správě majetku. Vyčlenění majetku do svěřenského fondu také může sloužit k zajištění řádné správy tohoto majetku v případě, kdy by větší problém, než potenciální rozdrobení majetku mohla představovat například potenciální budoucí nemožnost nebo další neochota ze strany zakladatele se o majetek dále starat nebo odůvodněné obavy, že právní nástupci nejsou, ať už z důvodu nedostatku zkušeností nebo nízkého věku, připraveni na povinnosti spojené se zodpovědnou správou majetku.\\
 
S tímto souvisí i v podstatě téměř neomezené možnosti na straně zůstavitele spočívající v obecné výhodě svěřenského fondu a to té, že zakladatel může nastavit správu svěřenského fondu podle svého uvážení, samotná škála variant, které může zvolit je opravdu nepřeberná a záleží pouze na zakladatelově vůli jakou z nich, při uvážení svých potřeb, ochrany majetku a předpokládaných schopností dědice, či obmyšleného, zvolí, respektive nastaví ve formě výplaty ze svěřenského fondu, převádění majetku, zániku či volnosti uvážení ohledně těchto činností ze strany správce.\\

%Diskrece.
 
 %Podívat se jak probíha dědění rodinného závodu a jak rodinný závod funguje ve svěřenském fondu.
 
 Příklad uvedený v díle Správa cizího majetku: \textit{"Zakladatel může  do svěřenského fondu vyčlenit závod na výrobu automobilů a potomkům přiznat právo podílet se rovným dílem na zisku dosaženým tímto závodem."} Tento závod může býti přitom spravován profesionálem, potřeba jisté odbornosti dědiců ve směru řízení podniku tak nemusí být nikterak vysoká, je jim pouze přiznáno právo na plnění, aby byly zajištěny jejich potřeby. A dále uvádím pro druhý dílčí přínos zvolení předání majetku ve formě svěřenského fondu tento příklad:\textit{"Otec vyčlení část majetku - akcií a určí jako obmyšlené své nezletilé děti. Správou majetku pověří profesionála zběhlého ve správě cenných papírů a děti budou až do dosažení vysokoškolského vzdělání oprávněni pouze k výnosu z akcií."} V tomto příkladu bude zabráněno rozdrobení majetku vlivem rozdílný názorů a ekonomických zájmů jednotlivých dědiců, či jak již je výše zmíněno nepřipravenosti\footfullcite{lucie_joskova_sprava_2017}.\\
 
 %S tímto samozřejmě do značné míry souvisí i obecný výhoda svěřenského fondu spočívající v tom, že zakladatel může nastavit správu svěřenského fondu pdole svého uvážení, samotná škála variant, které může zvolit je opravdu nepřeberná a záleží pouze na zakladatelově vůli jakou z nich zvolí, respektive nastaví. 
 
 \newpage
 \thispagestyle{smallertextinheader}
 
 Svěřenský fond také dále může být praktickým nástrojem pro správu velkého majetku, který lze rozdělit a toto rozdělení by potenciálně ani nepředstavovalo vážnou konkurenční nevýhodu, ale je praktičtější majetek držet pohromadě než ho drobit mezi dědice. Dobrým příkladem tak může být syndikovaný úvěr\footnote{Syndikovaný úvěr je takový úvěr, kdy je úvěr poskytnut syndikátem, tedy nějakou skupinou věřitelů, tyto úvěry jsou poskytovány v případech, kdy je třeba vysokého objemu zapůjčených finančních prostředků, například při fúzi, akvizici nebo expanzi korporace, refinancování stávajících závazků, nebo realizaci projektového financování jak správně uvádí na svých stránkách například Komerční banka,https://www.kb.cz/cs/firmy-a-instituce/produkty/uvery-a-financovani/strukturovane-financovani/syndikovany-uver}, kdy majetek slouží ku prospěchu velkého počtu osob, jež jsou spojeny společným zájmem, v těchto případech je se správou práv a povinností spojena velká administrativní zátěž, která může být vytvořením svěřenského fondu zmírněna jak pro věřitele, tak i pro dlužníky\footfullcite{lucie_joskova_sprava_2017}. Další příklad uvádí Barbora Bednaříková u rodiny Rockefellerů. Jedná se o velice rozsáhlou rodinu, která svým počtem převyšuje 100 členů. I přes fakt, že mnoho majetku bylo rozdrobeno do menších fondů, je obecně majetek rodiny držen v celku místo osobního vlastnictví jednotlivých členů. Společně s takto značným majetkem, který je centralizovaně spravován, a správnou investiční strategií, dosahují výnosy z takovýchto fondů nemalé částky, z nichž jednotliví členové mohou pohodlně žít. I přes fakt, že by tento majetek tedy potenciálně mohl být rozdělen mezi členy rodiny, tak pokud je soustředěn v celku, výnosy z něho budou mnohem větší a mohou sloužit rodiným členům k jejich potřebám lépe, než kdyby se majetek rozdělil\footfullcite{bednarikova_barbora_sverenske_2014}.
 
 %v případě, že zakladatel již nechce nebo nemůže řádně majetek spravovat, stejně tak pokud by měl pochybnosti o tom, zdaby jeho právní nástupci například pro nedostatek zkušeností, či nízký věk tuto zprávu zvládli, m, 
 
 \newpage
 \thispagestyle{smallertextinheader}
 
  \begin{enumerate}
 {\Large\item[5.] Péče o členy rodiny}
 \end{enumerate}
 
 Další výhodu svěřenského fondu spatřuji v možnosti jeho nastavení tak, aby se dal využít pro péči o jednotlivé členy rodiny, kteří danou péči potřebují, ať již díky fyzickému nebo mentálnímu postižení nebo kvůli pokročilému, nebo naopak nízkému věku, a tedy zajištění jejich hmotného zabezpečení. Těmito mohou býti jak zakladatelé, tak jiní členové rodiny, či dokonce osoby mimo rodinný kruh. Tato výhoda je velice posílena flexibilitou svěřenského fondu, kterou vložil zákonodárce do rukou jeho zakladateli a ten prostřednictvím statutu do rukou svěřenského správce, který poté může vykonávat profesionální správu svěřenského fondu bez toho, aby se jakýkoliv člen rodiny musel angažovat. Díky flexibilitě svěřenského fondu, je možné přesně popsat a vymezit situace a podmínky, za kterých mají jednotliví členové rodiny právo na plnění ze svěřenského fondu. Vhodným nastavením je možné docílit hmotného zabezpečení a péči o osoby, které to nejvíce potřebují po celý jejich život.\\
 
 Toto využití je tak například vhodné v případech, kdy je zakladatel stižen vážnou chorobou, která v budoucnu zapříčiní znemožnění vlastního racionálního úsudku. Založením svěřenského fondu tak může zakladatel docílit toho, že potomci budou mít motivaci se o něj starat a prostředky z majetku takto vyčleněného budou distribuovány výhradně osobám, které o zakladatele projevují zájem a starají se o něj v jakékoliv nelehké situaci a podíl na majetku nebudou mít osoby, které si takovéto plnění nezaslouží\footnote{Tamtéž strana 113}\textsuperscript{,}\footfullcite[Kocí M. Strana 1208]{svestka_j_obcansky_2014-1}. Vedle tohoto benefitu také samozřejmě stojí i výhody popsané v bodech 1, 4 a 6. Toto myslím tak, že například s ohledem na výhody uvedené v bodě 1 se může zakladatel takto rozhodnout, aby uchránil majetek před dědici, či jinými osobami, které by chtěli využít jeho ztížené situace, kdy by dle platných předpisů nebyla ještě učiněna žádná podpůrná opatření při narušení schopnosti jednat\footnote{§38-65 Občanského zákoníku} a byl by tak plně způsobilý jednat po právní stránce, nicméně po stránce mentální by tohoto již nemusel být schopen.\\
 
 Obdobně je možné svěřenského fondu využít jako zajištění pro postižené dítě pro smrti rodičů spolčně s přiznámím odměny za tuto péči jednotlivým osobám.
 
 %Je praktické vyčlenit určitou část do svěřenského fondu a jako obmyšlené určit potomky. Tento způsob\\
 
 %Dědic je v rámci postavení k ostatním dědicům v postavení věřitele, je možné se bránit jako dědic jednání zůstavitele za jeho života, v podstatě to nejde, jen pokud by takovéto jednání bylo učiněno pod nátlakem, nebo pokud bychom se bavili o kolaci, tzn. započtení na dědický podíl díl?
 
 %Pokud by byli dva dědicové a zůstavitel by založil svěřenský fond jen pro jednoho dědice, mohl by druhý dědic požadovat, aby se tento majetek započetl na dědický podíl?
 
 %V první řadě je důležité zmínit, že následující výklad nemusí být kvůli nedostatečnostem jak lingvistickým, tak i věcným v právní úpravě svěřenských fondů absolutně přesný a je možné, že díky potenciálním problémům vytyčeným níže se může zákonná úprava změnit a s tím společně i tento bod. S ohledem na pohled právních expertů na problematiku svěřenských fondů a mého názoru k datu publikace této práce je vhodné tento bod zařadit v podobě, která se s ohledem na názorovou disputaci a pohledem na procesněprávní náležitosti svěřenského fondu kloní spíše k názoru Miloše Kocího. Tedy, že majetek vyčleněný do svěřenského fondu není součástí pozůstalosti, další informace o tomto problému jsou popsány v další kapitole.\\ 
 %DOPLNIT

%Ukázat příklady, ke kterým se hodí svěřenský fond, v souvislosti s tématem práce by zde bylo vhodné popsat institut svěřenského fondu jako nástroj k mezigeneračnímu převodu majetku.

%S ohledem na hlavní téma této práce, tedy svěřenský fond jako mezigenerační nástroj převodu majetku, představují způsoby založení svěřenského fondu a jejich specifika vhodný startovní bod pro tuto část práce a obecně pro celou zkoumanou problematiku. \\

%Základním faktem týkajícím se založení svěřenského fondu je, i s ohledem na popsanou flexibilut instutut trustu, ze kterého český zákonodárce vycházel, je nepřekvapivě možnost založení jak mezi živými (\textit{inter vivos}), tak i pro případ smrti (\textit{mortis causa}).

  \begin{enumerate}
 {\Large\item[6.] Přiznání různých práv oprávněným osobám}
 \end{enumerate}
 
 S ohledem na široké možnosti účelu svěřenského fondu a podmínky plnění, které je možné stanovit ve statutu, dále svěřenský fond umožňuje efektivní využívání majetku. Toto efektivní využívání majetku spočívá ve své podstatě v zahrnutí možností poskytnutých dědickým právem v §1507-1524 Občanského zákoníku, tedy možností svěřenského náhradnictví a svěřenského nástupnictví. V rámci statutu je možné, aby zakladatel přiznal jednotlivým obmyšleným různá práva jednorázově i postupně\footfullcite{lucie_joskova_sprava_2017}. Toto se dá využít například tak, že zakladatel přizná určité právo na věci vložené do svěřenského fondu jedné osobě po dobu jejího života a po její smrti či naplnění jiného zakladatelem stanoveného kritéria přejde toto právo na osobu jinou. Pokud by se tak například zůstavitel nemohl spolehnout na svého potomka, kvůli jakémokoliv důvodu, že se nebude schopen o dědictví vhodně postarat, ale chtěl by aby dané dědictví následně přešlo na potomka tohoto potomka, přičemž by ale chtěl zabezpečit i svého potomka, tak použití institutu svěřenského nástupnictví není vhodné ani v případech, kdy by podle §1521 nebo §1522 Občanského zákoníku nebylo svěřerno dědici s dědictvím volně nakládat. V takovéto obdobné situaci je nicméně zřízení svěřenského fondu ve prospěch potomka se stanovením podmínky vyplacení celé podstaty svěřenského fondu po jeho smrti jeho potomkovi\footnote{Toto je plně v souladu s §1460 Občanského zákoníku} velice praktické, neboť majetek vložený do svěřenského fondu může spravovat profesionální svěřenský správce a v souladu s bodem 1 - Ochrana majetku, bude majetek zachován pro další generace. V souladu s důrazem na pořizovací volnost tedy takto bude zajištěno, že zakladatel svěřenského fondu bude mít nepřímý vliv na svůj majetek i po své smrti a o tento majetek bude postaráno dle jeho vůle.\\
 
 Vhodný příklad opět uvádí autoři v díle Správa cizího majetku, pro ilustraci výše napsaného si však dovolím tuto ukázku trochu pozměnit: \textit{"Po smrti zakladatele má vyčleněný majetek (bytovou jednotku) využívat jeho syn s jeho vnukem. Po smrti jeho syna má majetek získat vnuk zakladatele."}
 
   \begin{enumerate}
 {\Large\item[6.] Možnost převedení majetku na jinak nezpůsobilé dědice}
 \end{enumerate}
 
Tato možnost využítí svěřenského fondu, respektive účel, za kterým by mohl být založen, souvisí významně s bodem 5. Nejprve je asi vhodné poznamenat, že tuto výhodu spatřuji zejména ve vztahu založení svěřenského fondu \textit{inter vivos} u založení svěřenského fondu \textit{mortis causa} ji považuji spíše za okrajovou.\\

Toto činím především z důvodů, které spatřuji jako potenciální důvod pro založení svěřenského fondu za tímto účelem pro případ smrti. Zde lze považovat za ony důvody především určitou snahu zůstavitele "morálně" vzdělat dědice po své smrti v případech, kdy se oni dědicové za zůstavitelova života k němu zachovali nepatřičně, nebo přímo učinili něco, co by je z dědění po zůstaviteli, bez jeho výslovného odpuštění, přímo vylučovalo, pro tento bod připomeňme kapitolu Dědické právo v České republice a k tomu specifikum dědické nezpůsobilosti.\\

Pokud bychom tedy uvažovali, že dědic byl zákonně vyloučen z dědického práva a zůstavitel mu výslovně toto jednání neprominul, mohl by zůstavitel založit svěřenský fond jako jakousi formu vzdělání dědice, tedy obmyšleného, po své smrti a nastavit možnost výplaty tak, aby správce mohl, při splnění určitých podmínek, takovémuto dědici vyplácet plnění ze svěřenského fondu. Nastavení těchto podmínek je pak plně na vůli zakladatele a od toho se odvíjí i diskrece správce zhodnotit plnění takto nastavených podmínek a přiznávat výplatu ze svěřenského fondu.\\

Tímto způsobem a pro tento účel pak lze obejít dědickou nezpůsobilost bez výslovného prominutí činu vedoucího apriori k dědické nezpůsobilosti ze strany zůstavitele, převést na něj majetek a čistě teoreticky dědice i po tomto směru, například tedy po morální stránce, vzdělat.\\

Větší výhody tohoto bodu však spatřuji při založení svěřenského fondu \textit{inter vivos}, zde tedy narážím na podobnost s bodem 5. Příkladem budiž případ, kdy je zakladatel stížen vážnou chorobou a uvědomuje si, že se o sebe nebude schopný v budoucnu postarat. V takovém případě může použít svěřenský fond jako motivaci pro své dědice, aby se o něj postarali. Za oplátku pak při splnění jím stanovených podmínek nechá na správci posouzení, zda byly podmínky opravdu splněny a pokud ano, tak bude dědicům, tedy obmyšleným, vypláceno plnění ze svěřenského fondu i ve chvíli, kdy by podle dědického práva jinak na dědictví neměli nárok. Tento svěřenský fond tedy může sloužit jako jakási šance pro dědice ukázat se v lepším světle a odčinit skutky, které napáchali vůči zůstaviteli a za oplátku získat něco z původního zůstavitelova majetku, na který by jinak v případě pozůstalostního řízení neměli nárok.\\

Při porovnání se samotným prominutím tohoto jinak vylučujícího jednání z dědického práva tak docházíme k závěru, že při použití svěřenského fondu lze vhodně působit na osobnost daného dědice a lze nad ním uvažovat jako nad určitou motivací k polepšení a je z tohoto hlediska i společensky prospěšný. Předpokladem je, že nejspíše nebude mnoho fondů založených za tímto účelem a mnoho potenciálních zakladatelů ani nad podobným účelem, pro který založit svěřenský fond, nebude přemýšlet, tento bod je proto pojmut pouze jako jistá inspirace pro lidi čelící této situaci.\\

\newpage

\thispagestyle{smallertextinheader}
Data.
\newpage

\section{Systémové nedostatky a návrhy na řešení}

%Zneužití svěřenského fondu viz Babiš, nebo k obcházení věřitelů, dále rozdílné pojetí vlastnictví, započtení, obcházení nepominutelného dědice

%Právní jednání za svěřenský fond, úvěr, závazky, vložení závoud do svěřenského fondu, změna statutu za trvání svěřenského fondu

%Je možné převést do svěřenských fondů například majetek, na kterém vázne zástavní právo? Je tedy možné převést jmění? Tedy nejanom aktiva ale i pasiva

%Pokud je zakladatel i správce, tak je neovlivnitelnost druhého správce, kterého musí povinně mít velkou otázkou.

%Daně nejsou nevýhodou, v rámci příbuzenských vztahů je plnění bezúplatné, problém by byl při plnění ze zisku, nebo pokud by obmyšlenou byla cizí osoba. Při zániku svěřenského fondu a převedení majetku na obmyšleného by se žádná daň platit neměla, tohle ale musím ještě ověřit. Dle článku(portál pohoda), který jsem našel by stejné principy, které platí na plnění ze svěřenského fondu měli platit i na zánik a tedy převedení majetku ze svěřenského fondu obmyšlenému, nebo zakladateli. Jedná se o bezúplatný příjem na základě Zákona o dani z příjmu.

%Co se daní týče tak zde je to hezky shrnuto v oblasti daní z pozemků a nemovitostí

%Daňové aspekty

%Daňová problematika týkající se svěřenských fondů je rozsáhlá a vzhledem k absenci relevantní judikatury či ustálené praxe ne vždy jednoznačná. S ohledem na zaměření tohoto článku se proto omezíme na daňové povinnosti svěřenského fondu u daně z pozemků a ze staveb a daně z nabytí nemovitých věcí.

%Svěřenský fond je poplatníkem daně z nabytí nemovitosti dle ust. § 1 odst. 1 a 2 zákonného opatření Senátu č. 340/2013 Sb. o dani z nabytí nemovitých věcí („zákon o dani z nabytí nemovitosti“) a poplatníkem daně z pozemků a staveb dle ust. § 3 odst. 2 písm. b) a § 8 odst. 2 písm. b) zákona č. 338/1992 Sb., o dani z nemovitých věcí.

%Co se týče daně z pozemků a staveb je svěřenský fond standardním poplatníkem a platí pro něj srovnatelná pravidla jako pro ostatní daňové poplatníky.

%Často řešenou otázkou je však vklad nemovitosti do svěřenského fondu z pohledu daně z nabytí nemovité věci. Jak je uvedeno výše, svěřenský fond je poplatníkem této daně. Předmětem daně je úplatné nabytí vlastnického práva k nemovitosti.

%Vzhledem k tomu, že se při vyčlenění nemovitosti z majetku zakladatele či zvýšení majetku svěřenského fondu o nemovitost dle ust. § 1468 občanského zákoníku nejedná o úplatné nabytí vlastnického práva, není tento převod ani zatížen daní z nemovitých věcí. Nelze zde analogicky uplatňovat úpravu pro nepeněžité vklady do obchodních korporací, neboť zakladatel/vkladatel nezískává podíl na svěřenském fondu ani obdobné protiplnění.[18] K tomuto závěru dochází i odborná literatura.[19]

%Závěrem

%Využití svěřenského fondu pro řešení rozličných situací již nevyvolává údiv (ani úlek), nýbrž je akceptovaným a efektivním východiskem mnohých situací. Výše nastíněný scénář má za cíl ukázat, že i z praktického hlediska se nejedná o nic složitého.

%[19] SVEJKOVSKÝ, Jaroslav a Ivan KOVÁŘ. Svěřenské fondy: příležitosti a rizika. V Praze: C. H. Beck, 2018. Právní praxe. ISBN 978-80-7400-726-2, str. 56.

%Další nejasnost, jak naložit s majetkem ve chvíli, kdy všechni obmyšlení zemřeli? Vrátí se následně do dědictví a dědicům bude náležet povinný díl? Popsaáno v práci Ivany Vladyková, ochrana nepominutelných dědiců a jejich vydědění - Dle https://www.altaxo.cz/poradna/sverenecke-fondy/zanik-a-zruseni-sverenskeho-fondu připadne státu, možná se ještě podívat, že pokud není možné dosahovat účelu svěřenského fondu, tak soud může pozměnit statut a tím pádem i účel nebo majetek převést nějakému subjektu, který má podobný účel jako měl ten svěřenský fond, tak jestli se nestane toto v případě že nebude ani obmyšleného a ani zůstavitele místo toho aby takovýto majetek připadl státu. Jinak by to ale smysl dávalo.

%---Koment\\
%Zhodnocení faktu, že by taková právní úprava měla existovat, poznámka, že tento problém zatím u nás není řešen soudní praxí. Pokusit se navrhnout legislativní změny, vedoucí ke zlepšení postavení nepominutelných dědiců v rámci dědického řízení za předpokladu, že zároveň vznikne i svěřenský fond mortis causa, do kterého se vyčlení majetek, který by jinak byl součástí dědického řízení. Zkusit navrhnout řešení na základě inspirace jinými právními systémy, ve kterých existuje jak institut trustu, tak i ins  titut nepominutelného dědice, mohlo by se jednat například o provincii Quebec, nebo stát Louisiana, pokusit se tyto změny navrhnout na základě studia zákonů a dále se pokusit najít soudní rozhodnutí, v Angličtině se institut nepominutelného dědice nazývá forced heir. \\
%---\\

Je nepochybé, že s přijetím institutu svěřenského fondu vzniklo mnoho otázek, na které nám zákonodárce neposkytl patřičné odpovědi. Zajímavé ovšem je, že zákonodárci v rámci procesu přejímání právní úpravy svěřenského fondu, a její implementací do českého právního řádu, muselo, u mnoha z níže popsaných problémů, být jasné, že takovéto vzniknou. Otázka tedy je, proč byl svěřenský fond takto upraven a proč jeho přijetí nevedlo k odpovídajícím změnám v rámci českého práva, které by takovéto problémy řešily. Věřím, že pomocí analýzy těchto problémů a zodpovězením otázek z nich plynoucích, budu schopen v závěrečné kapitole navrhnou patřičná opatření a změny, které by mohly vést k vyjasnění, či přímému odstranění těchto problémů české právní úpravy.\\

Následující část se zabývá a analyzuje primárně problematiku spojenou s vlastnictvím vyčleněného majetku do svěřenského fondu.\\

\newpage

%Ověřit si, že jsem nezapomněl na žádné nedostatky, nebo nejasnosti

\subsection{Nejasná regulace a chybějící judikatura}

\subsection{Systémové nedostatky vztahující se k dědickému právu}

%Zde je můj pohled obecně následující: V případě svěřenského fondu založeného inter vivos se není o čem bavit, zde je to jasné. V případě založení svěřenského fondu inter vivos s odkládací podmínkou smrti zakladatele, respektive zůstavitele, je to složitější, zde musím zjistit, zda je majetek takto vložený do svěřenského fondu součástí pozůstalosti.

%V případě založení svěřenského fondu mortis causa nemá dle mého názoru dle výkladu občanského zákoníku nemá dědic právo na povinný díl a zakladatel se takto efektivně vyhne institutu nepominutelného dědice. Jak je to však s nezletilým dědicem? Majetek je vložen do svěřenského fondu okamžikem smrti zůstavitele, protože svěřenský fond vzniká okamžikem smrti zůstavitele a nemůže existovat bez vloženého majetku. Dědic by se mohl dle mého názoru bránit pouze v souvislosti s neplatností právního jednání vedoucího k založení svěřenského fondu vztahujícímu se k například donucení zůstavitele takovýto fond založit. Dědic by se možná mohl domáhat na správci vydání svého povinného dílu v souvislosti s ustanovením paragrafu 1492 o zákazu zkrácení nepominutelného dědice, pokud by však dostal částku minimálně rovnající se jeho dílu, který by mu jinak dle zákona minimálně připadnul, myslím si, že by se nemohl bránit. Obdobně si myslím, že plnění ze svěřenského fondu se ani nezapočítává na dědický podíl, toto však musím ještě zjistit. Paragraf §1452 přece stanoví, že se použije ustanovení paragrafu 311, ale podle mě v souvislosti s paragrafem 311 proběhne jenom jakýsi quazi stejný proces jako u dědického řízení, tzn. majetek je vložen v rámci dědického řízení, ale není součástí pozůstalosti a fond, protože fond nemá právní subjektivitu, zastupuje při tomto jednání svěřenský správce. Takto vyčleněný majetek ale dle mého názoru stále není součástí pozůstalosti a svěřenský správce plní v řízení jenom jakousi informační povinnosti a o tomto majetku se ani nevede dědické řízení, takže je součástí jenom jakoby, ale nejedná se o něj.

%Důležitý je taky okamžik smrti, poněvadž se majetek vylčení do svěřenského fondu ve stejnou cvhíli jako vznikne pozůstalost. Dobrá je argumentace Michala Skuhrovce, ale zapomněl na paragrafy 311 a 1492 a 1452.

Věřím, že na problematiku spojenou s kolizí norem dědického práva a svěřenského fondu lze nahlížet dvěma způsoby, podle kterých záleží, na jakou stranu názorového spektra se v souvislosti se zodpovězením mé hypotézy přikloním.\\

První pohled bych nazval "výkladový", pomocí jeho optiky nahlížím na kolizní problematiku z pohledu formálního výkladu jednotlivých norem a jejich vzájemných vztahů s ohledem na základní principy práva.\\

%Pohled jak je to napsané v tom samotném předpisu a jak tomu lze po výkladové stránce v souvislosti s porovnáním kolizních norem porozumnět. Dle tohoto pohledu je například chybný paragraf 311, protože není v souladu s tím, že svěřenský fond není jako takový účasten dědického řízení a vyčlenění majetku do něj nepotvrzuje soud, majetek takto vyčleněný tedy ani není součástí pozůstalosti.

%Pokud bychom tak porovnávali dva paragrafy, kdy jeden by nedával smysl, tak bychom ho při vyložení hypotézy nebrali v potaz.

Druhý pohled bych nazval "úmyslový", který je vykládán jako mé pochopení potenciálního úmyslu zákonodárce a jeho směřování při implementaci tohoto nového právního institutu, krom samotného výkladu smyslu jednotlivých právních ustanovení, se zde zaměřuji i na logicky vyvozené korelace mezi jednotlivými zákonnými ustanoveními a z toho plynoucí kauzalitu, kterou spatřuji v několika zákonných ustanoveních, jejichž výklad nelze ztotožňovat s výkladovým pohledem, neboď s ohledem na transplantaci právní úpravy svěřenského fondu je vyloučený. V jeho rámci se zaměřím spíče na teoretickou rozpravu s ohledem na základní principy dědického práva a svěřenského fondu v souvislosti s materiálními prameny práva.\\

%Pohled jak to myslel zákonodárce, zmínil paragraf 311, proč by to dělal, pokud by nechtěl, aby byl aplikovatelný. S ohledem na to že dle zákona o zvláštních řízeních soudních je například svěřenský správce účastníkem dědického řízení a plní tedy takovou quazi informační a informativní funkci, je možné, že by mohl zastupovat samotný svěřenský fond a ten by potom majetek nabýval potvrzením soudu.

%Tohle budu muset možná trochu ještě přepsat.

%a smysl respektive základní principy svěřenského fondu, pokud budeme připodobňovat svěřenský fond k právnické osobě kdykoliv budeme chtít, popíráme tím základní principy svěřenského fondu a lepaullova vlastnický na jehož základě svěřenský fond stojí a na jehož základě jako účelově nezávislé jmění jsme ho přijali. Pokud bychom mu přiřazovali quazi právní subjektivitu kdykoliv, kdy by se nám to hodilo, tak proč bychom ho tedy vůbec měli, mohli bycho mu buď přiřadit vlastní právní subjektivitu, která by vyřešila mnoho problémů, ale proč bychom to potom nazývali svěřenským fondem, ale místo toho jsme jen neuvolnili podmínky, které je nutné splnit při zakládání nadace. A proč bychom tedy svěřenský fond takto přijímali pokud bychom jeho výklad směřovali k tomu, že by měl právní osobnost.

%Soud by tak mohl rozhodovat i s ohledem na neplatné paragrafy kvůli veřejnému pořádku.

%a budu prezentovat svůj názor

\subsection{Započtení na dědický podíl}

%Způsobilost být obmyšleným

%3. Zákon blíže nespecifikuje, kdo má způsobilost být obmyšleným, neboť se vychází z toho, že je zcela na uvážení zakladatele, koho bude považovat za způsobilého a koho nikoliv. Může se jednat o člověka i o právnickou osobu včetně osob nenarozených nebo právnických osob ještě nezaložených (§ 1464), kdy hovoříme o tzv. eventuálních obmyšlených (v Québecu bénnéficiaires éventuels).

%4. V této souvislosti je třeba uvést, že v zájmu spravedlivého uspořádání vztahů vzniklých založením svěřenského fondu důvodová zpráva proponuje, že „obmyšleným svěřenského fondu zřízeného bezúplatně může být jen osoba, která je způsobilá po zakladateli v den vzniku svěřenského fondu dědit“. Zde důvodová zpráva evidentně vychází z reality CCQ, který ve svém čl. 1279 stanoví, že „beneficientem bezúplatného svěřenského fondu může být pouze osoba, která má v době vzniku svého práva vlastnosti potřebné pro obdržení daru nebo dědictví“. Toto ustanovení však v účinném znění občanského zákoníku obsaženo není, a proto by s ohledem na zásadu „vše je dovoleno, co není zákonem výslovně zakázáno“ mělo být zakladateli umožněno, aby učinil obmyšleným z testamentárního svěřenského fondu i osobu, která není způsobilá být dědicem. Obmyšlený z testamentárního svěřenského fondu totiž nutně dědicem být nemusí, a proto není důvodu, aby byl podroben stejným standardům jako dědic jen proto, že bylo ke zřízení svěřenského fondu mortis causa využito závěti, tj. institutu dědického práva (viz § 1448 odst. 1). Tento závěr by měl platit tím spíše, že majetek vyčleněný do svěřenského fondu je vyloučen z dědického řízení, neboť již samotnou smrtí zůstavitele je vyčleněn do svěřenského fondu, a není tedy součástí pozůstalosti.

%5. Pokud by s majetkem ve svěřenském fondu mělo být zacházeno jako se součástí pozůstalosti, bylo by případně nutné provést zejména započtení plnění poskytnutého obmyšleným na dědický podíl (§ 1662 až 1664), ačkoliv tito nemají k plnění ze svěřenského fondu vlastnické právo (§ 1448 odst. 3) a u diskrečních fondů navíc ani nemusí být jisté, zda jim právo na plnění vůbec vznikne, popřípadě v jaké výši (viz níže sub II). To by v rámci dědického řízení vedlo ke značným komplikacím a průtahům. Zakladatel by způsobilost obmyšleného k plnění ze svěřenského fondu teoreticky mohl vázat např. i na mnohem přísnější podmínky, než jsou podmínky předvídané zákonodárcem pro dědickou nezpůsobilost (§ 1481 až 1483). Pokud tak učinit z nějakého důvodu nechce, měla by být jeho vůle respektována, stejně jako jeho rozhodnutí nepodrobit obmyšleného žádným podmínkám. Konkrétně se může jednat např. o situaci, kdy chce zakladatel obmyšlenému kompenzovat zřeknutí se dědického práva (§ 1484), které je důvodem nezpůsobilosti k dědění v širším slova smyslu a které má v konkrétní situaci např. umožnit hladší mezigenerační přechod majetku zakladatele v duchu rodinných tradic, anebo o situaci, kdy se zakladatel obává, že by jeho potenciální dědic mohl být z dědického práva vyloučen z důvodu spočívajícího v dědické nezpůsobilosti v užším slova smyslu (§ 1481 až 1483), o němž zatím nemá vědomost, a chce tak tvrdému právnímu následku předejít tím, že pro takový případ ze svého dědice učiní zároveň i obmyšleného ze svěřenského fondu. Je třeba zmínit i skutečnost, že občanský zákoník nikde nepočítá s tím, že by svěřenský správce nebo obmyšlený měli být v jakékoliv roli účastníky dědického řízení. Zřízení svěřenského fondu přitom v žádném případě nelze považovat za svěřenské nástupnictví nebo jeho druh (§ 1512).

%6. Konečně dodejme, že u svěřenských fondů založených za soukromým účelem představuje určitou časovou limitaci způsobilosti obmyšleného ustanovení § 1460 (viz níže).

\subsection{Nejasný výklad ochrany nepominutelných dědiců}

% První bod popsat ty moje dva rozdílné pohledy, druhý bod pak posat ty jednotlivé pohledy autorů, 4 pohledy, pak napsat, že si nemyslím, že je takový problém znevýhodnění svěřenského fondu mortis causa oproti svěřenskému fondu inter vivos. Viz poznámky, které jsem si poslal na mail a napsal na druhém počítači. Pak také popsat to, že již nebude neplatné celé pořízení pro případ smrti, ale pouze vznikne opominutému dědici pohledávka za jinými dědici (tady je další problém, je svěřenský fond dědicem?) tohle je v souvislosti s paragrafem upravujícím pořízení pro případ smrti, a který zároveň stanovuje nutnost respektovat povinný díl nepominutelného dědice.

V České republice, obdobně jako ve většině zemí používajících kontinentální systém práva, právní řád chrání nepominutelné dědice před opomenutím ze strany zůstavitele a přiznává jim nárok minimálně na jejich povinný díl v zákonné výši. Ovšem jak již je zmíněno v kapitole výše, v rámci české právní úpravy bohužel chybí explicitní úprava funkce svěřenského fondu, pokud se nachází ve střetu s institutem nepominutelného dědice. Pro úplnost připomínám kapitolu 5.2 Nepominutelný dědic.\\

Inkorporace svěřenského fondu, dle mnohých, pak přinesla možnost zřídit obecný nelimitovaný fiduciární institut. Tato možnost společně na straně jedné s vůlí zákonodárce chránit nepominutelného dědice na straně druhé logicky nejsou slučitelné\footfullcite[Strana 333]{jaroslav_svejkovsky_sprava_2015}. Diskuze ohledem aplikace právních předpisů úpravujících povinný díl a jejich souvislost se založením svěřenského fondu \textit{mortis causa} je tedy na místě.\\

Jak již jsem nastínil výše, dle mého názoru je na tuto problematiku možné nahlédnout dvojí optikou. V prvé řadě bych však chtěl poznamenat, že se v této části budu věnovat čistě svěřenskému fondu založenému pro případ smrti, neboť jak již jsem zmínil v kapitolách 6.3.1 a 6.3.3 omezení založená právní úpravou dědického práva se na svěřenský fond \textit{inter vivos} nevztahují.\\

Dobrý postoj k tomuto problému zaujal například Michal Skuhrovec ve své diplomové práci, která pojednává o podobném tématu jako má práce\footfullcite[Strana 54]{skuhrovec_michal_sverensky_2018}. Nemyslím si však, že názor jím vyslovený je komplektí a zahrnuje širší problematiku kolize právní úpravy nepominutelného dědice a svěřenského fondu. Z tohoto důvodu bych chtěl názor vyslovený v dané práci konfrontovat a zoršířit o můj pohled na problematiku, na níž jsem pohlédl z o něco širší perspektivy.\\

Skuhrovec na začátku hovoří o svázanosti evropských právních úprav dědickým právem, toto je do značné míry pravdivé a je faktem, že je třeba nalézt jistou harmonii mezi common law institutem trustu a kontinentálním dědickým právem. Dále poukazuje na to, že quebecký Občanský zákoník neobsahuje žádnou ochranu nepominutelného dědice, tím pádem ani neobsahuje ustanovení řešící případy touto kolizí vznikající, tato úprava tedy nemohla být součástí transplantace, nebo sloužit jako inspirační zdroj, či zdroj pro komparaci a obdobně ani jiné common law systémy tuto úpravu neobsahují a nemohou tak sloužit jako inspirační zdroj pro českého zákonodárce\footnote{Tamtéž.}. Ačkoli s první částí o svázanosti evropských právních systémů dědickým právem se dá autorovi přisvědčit, druhá část o neexistenci ustanovení sloužících jako pojistky řešení kolize trustu a jiných nároků připodobnitelných k nárokům nepominutelných dědiců se již dle mého podledu nezakládá zcela na pravdě. V této souvislosti bych chtěl upozornit na kapitolu 7.5.1, kde popisuji ochranu nároků, které lze s trochou představivosti připodobnit k institutu nepominutelného dědice v Občanském zákoníku provincie Quebec.\\

Zcela souhlasím s dalším tvrzením, že český zákonodárce si této potenciální kolize musel být vědom a přesto neučinil žádná opatření ve formě ustanovení, která by tyto kolize řešila. Ta tedy bohužel v celé části Občanského zákoníku týkající se správy cizího majetku z, pro mě nepochopitelných důvodů, absentují a zapříčiňují značnou právní nejistotu při zakládání testamentárního svěřenského fondu.\\

Není nicméně vhodné hned zpočátku tvrdit, že ze součásné právní úpravy nelze v žádném případě vyčíst řešení této situace, jen že nemusí být, a také není, na první pohled zřejmé. Ná následujících řádích je tedy vhodné podrobit tuto premisu hlubší analýze za pomocí dvou mnou výše vytyčených optik.\\

   \begin{enumerate}
 {\Large\item[1.] Ochrana nepominutelných dědiců s ohledem na výklad jednotlivých ustanovení}
 \end{enumerate}
 
 Než přejdu k samotnému pohledu na věc za pomoci této optiky, je vhodné nejprve představit jednotlivé názorové proudy.\\
 
 Vhodný vhled do problematiky svěřenského fondu a nepominutelného dědice poskytují čtyři autoři, z jejichž názorů také budu čerpat a postavím na nich svojí analýzu. Těmito autory jsou Miloš Kocí, Vlastmil Pihera, Kryštof Horn a Ivan Kolář.\\
 
 Z prvu pojďme podrobit analýze pohledy Miloše Kocího z komentáře k Občanskému zákoníku od Wolters Kluwer a Vlastimila Pihery z komentáře k Občanskému zákoníku od vydavatele C.H.Beck.\\
 
 Kocí se profiluje jakožto podporovatel názorového proudu, který nepovažuje majetek vyčleněný do svěřenského fondu za součást pozůstalosti a tím pádem zavrhuje možnost jeho ponížení o povinný díl nepominutelného dědice, přičemž svojí argumentaci staví na dvou základních bodech.\\
 
 Prvním bodem pak jest konstatování, že majetek vložený do svěřenského fondu nemůže být součástí pozůstalosti, neboť přestává být k okamžiku smrti zůstavitele v jeho vlastnictví: \textit{"16. Lze pochybovat o tom, že svěřenský správce může být považován za dědice nebo za osobu, která by v dědickém řízení vykonávala jeho práva, a to již proto, že takový majetek žádného dědice nemá. Jestliže tedy svěřenský fond zřizovaný pořízením pro případ smrti vzniká k okamžiku smrti zůstavitele bez ohledu na to, zda svěřenský správce přijal pověření k jeho správě či nikoliv, pak není pojmově možné, aby byl tento oddělený a nezávislý majetek předmětem řízení, jehož účelem je potvrzení vlastníka. V opačném případě by de facto potvrzoval vznik svěřenského fondu soud (§ 1670) a správu svěřenského fondu by do té doby vykonával správce pozůstalosti nebo vykonavatel závěti, pokud by byli zůstavitelem ke správě pozůstalosti povoláni (§ 1677). Ani s jednou z těchto variant však právní úprava svěřenského fondu nepočítá. Na svěřenský fond by tak neměly přecházet ani dluhy zůstavitele, což je v souladu s odst. 1 komentovaného ustanovení, podle něhož lze vyčlenit pouze „majetek“, tedy aktiva ve vlastnictví zakladatele (§ 495). V tomto ohledu není důvodu, proč by se na pořízení pro případ smrti mělo nahlížet jinak než na zřízení svěřenského fondu inter vivos, kde je zakladateli rovněž umožněno, aby vyčlenil pouze svoje aktiva bez ohledu na dluhy. Takové pořízení pro případ smrti, které by věřitele zůstavitele poškozovalo na jejich právech, by však mohlo být napadeno jako neplatné pro rozpor s dobrými mravy (§ 580 odst. 1), resp. by mohlo být neplatné částečně (§ 576), přičemž za určitých okolností by k neplatnosti takového právního jednání soud přihlédl i bez návrhu (§ 588)."}\footfullcite[Komentář k § 1448]{svestka_j_obcansky_2014-1}\\
 
 Tento argument s ohledem na ustanovení paragrafu 1475 Občanského zákoníku, pohlížející na pozůstalost jako na celé jmění zůstavitele ke dni jeho smrti, tak Kocího pohledu přisvědčuje\footnote{Viz kapitola 5.4 Pozůstalost.}. Zároveň však zcela opomíjí existenci § 1452 OZ stanovujícího v bodě 1 následující: \textit{"Každý svěřenský fond musí mít statut. Statut svěřenského fondu vydává zakladatel. Zřizuje-li se svěřenský fond pořízením pro případ smrti, použije se přiměřeně § 311."}, při pohledu na paragraf § 311 OZ pak zákonodárce stanovuje následující: \textit{"Při založení nadace pořízením pro případ smrti se do nadace vnáší vklad povoláním nadace za dědice nebo nařízením odkazu. V takovém případě nabývá založení nadace účinnosti smrtí zůstavitele."}. Toto ustanovení by tedy analogicky mělo být použito při vyčleňování majetku do svěřenského fondu. Zde se však v rámci pohledu na svěřenský fond první optikou dostáváme do částečného rozporu s jednou z základních premis, na které je vystavěn svěřenský fond v České republice, touto premisou je samotná forma svěřenského fondu, v rámci které rozhodně nelze na svěřenský fond pohlížet jako na osobu mající právní subjektivitu, jak by tedy mohl svěřenský fond být dědicem, účastnit se dědického řízení a tím pádem majetek do něj vyčleněný spadat do pozůstalosti? \\
 
 Odpovědí by mohla být osoba svěřenského správce, který by svěřenský fond v pozůstalostním řízení zastupoval a ustanovení § 311 by tak ve své podstatě mohlo být pouze jakýmsi pojmovým obratem. Problémem však zůstává, že ani svěřenskému správci vlastnické právo k majetku vyčleněnému do svěřenského fondu nesvědčí\footfullcite[POPOVICI A. Trust québeckého a českého práva:autonomní vlastnictví?]{lubos_tichy_sverensky_2017}. Dále s ohledem na úpravu řízení o pozůstalosti v ZŘS, oddílu 4, účastníci řízení je možné dovodit, že samotný svěřenský správce by neměl být ani účastníkem řízení, tím by ve své podstatě měl být svěřenský fond, jehož by měl svěřenský správce zastupovat. Kvůli nedostatku právní osobnosti svěřenského fondu však zase dospíváme k závěru, že samotný svěřenský fond nemůže být dědicem a tedy se ani řízení účastnit.\\
 
 S ohledem na výše zmíněné je třeba rozlišit, zda je svěřenský správce sám dědicem \textit{sui generis} kvůli nedostatku právní osobnosti svěřenského fondu, nebo zda je svěřenský fond pro tuto příložitost nadán právní subjektivitou, respektive tedy v uvozovkách považován za právnickou osobou, tak jako tomu zákonodárce činí  v daňových předpisech\footfullcite[§ 4b (2)]{noauthor_zakon_zdph} a tedy i možností účastnit se tohoto řízení a svěřenský správce ho pouze zastupuje (tak jak by zastupoval statutární orgán jinou právnickou osobu v případě, že by byla určena dědicem).\\
 
 OZ ani ZŘS toto nikde explicitně neupravují, naopak sám OZ v § 1448 hovoří odděleném a nezávislém vlastnictví, tedy o entitě bez právní subjektivity, znovu tedy vzniká problém spočívající v neexistenci právní subjektivity svěřenského fondu a tedy nemožnosti být účastníkem jakéhokoliv řízení.\\
 
 Nejenom k tomuto problému se vyjádřil Kryštof Horn: \textit{"Dle názoru autora je potřeba vycházet z toho, že svěřenský fond mortis causa se zřizuje pořízením pro případ smrti. Půjde tedy o závěť či dědickou smlouvu (popř. dovětek), které v sobě budou zřejmě muset obsahovat náležitosti statutu a současně budou muset splňovat formální požadavky na ně kladené. Pokud tedy majetek součástí pozůstalosti je, lze mít za to, že zřizovací instrument (tj. pořízení pro případ smrti) musí respektovat ustanovení o povinném dílu, a to zřejmě i v případě, kdy by zůstavitel zamýšlel zřídit svěřenský fond ohledně veškerého svého majetku ve prospěch nepominutelného dědice. Předně proto, že povinný díl může být zůstaven jen v podobě dědického podílu nebo odkazu a především musí být zcela nezatížen, ale také proto, že samotným vytvořením fondu se nepominutelnému dědici ničeho nezůstavuje, a tak se dědic dle názoru autora nemůže při použití § 1644 odst. 3 OZ ani rozhodnout pro přijetí takto omezeného „dědictví“ namísto povinného dílu. Proto autor nesouhlasí s názorem, podle kterého by bylo nutné započtení plnění poskytnutých obmyšlenému ze svěřenského fondu na jeho souběžný dědický podíl podle § 1662– 1664 OZ, neboť postavení dědice sui generis zde má svěřenský správce a nikoliv obmyšlený."}\footfullcite{sro_httpwwwpraguebestcz_prakticke_nodate-1}.\\
 
 Dle Horna má tedy fakticky dědit svěřenský správce ve prospěch svěřenského fondu, což je do značné míry vhodnější výklad, než opět pohlédnout na svěřenský fond jako na právnickou osobu, neboť konstantní subjektivizace svěřenského fondu nesvědčí jeho účelu.\\
 
 S Hornem souhlásí i Pihera, který se řadí do opačného názorová spektra něž Kocí, kdy tvrdí následující: \textit{"Poněkud odlišnou povahu má vyčlenění majetku do svěřenského fondu zřizovaného pořízením pro případ smrti (mortis causa). V těchto případech vyčlení zakladatel majetek ve prospěch svěřenského fondu tak, že v jeho prospěch vyhradí dědické právo nebo nařídí odkaz (§ 1452 odst. 1 ve spojení s § 311 odst. 1). Správce svěřenského fondu mortis causa pak vykonává práva dědice majetku vyčleněného do svěřenského fondu nebo práva odkazovníka. Majetek vyčleněný zakladatelem (zůstavitelem) do svěřenského fondu mortis causa je součástí pozůstalosti a nabytí tohoto majetku do svěřenského fondu buď potvrzuje soud, nebo je převeden ve prospěch svěřenského fondu dědicem obtíženým odkazem. Z hlediska dědického práva je na postavení svěřenského fondu nutné pohlížet „jako na dědice“ nebo „jako na odkazovníka“, a to se vším všudy, co se s tímto postavením spojuje (např. odpovědnost za dluhy zůstavitele)."}\footfullcite[Komentář k § 1448]{spacil_j_a_kolektiv_obcansky_2013}\\
 
 Při výkladovém pohledu je však nutné podrobit tento názor hlubší analýze. Při pohledu na výše uvedené tři argumenty se lze ztotožnit s vysvětlením Michaela Skuhrovce proč se zdá Kocího pohled ve vztahu k základní konstrukci svěřenského fondu vhodněji aplikovatelnější: \textit{"Vyjdeme-li však zásadně z předpokladu, že ve smyslu § 1448 OZ se svěřenský fond vytváří vyčleněním majetku, v případě testamentárního svěřenského fondu pak tedy pořízením pro případ smrti, a vzniká smrtí zůstavitele, opora pro Hornovu argumentaci se nachází pouze s těžkostmi. Především je totiž nutné důsledně aplikovat esenciální obsah konstrukce Lepaullova vlastnictví. Jak Horn sám správně uvádí, ve smyslu §1475 odst. 2 OZ pozůstalost tvoří celé jmění zůstavitele. Součástí tohoto jmění je pak bezpochyby i majetek, který je vyčleněn k určitému účelu a tvoří tak svěřenský fond (v případě testamentu pak jakýsi „spící” svěřenský fond, neb vznikne až smrtí zůstavitele, která představuje dies certus an, incertus quando, a vzhledem k možnostem dispozice s majetkem je nejasný i jaký majetek bude součástí svěřenského fondu). Pokud je zároveň postaveno najisto, že svěřenský fond vznikne smrtí zůstavitele (na rozdíl od CCQ, které sice neobsahuje institut nepominutelného dědice, ale stále váže vznik svěřenského fondu na přijetí správce, byť s retroaktivním účinkem), nelze považovat tento majetek dále za vlastnictví zůstavitele, nýbrž za entitu bez právní subjektivity a se samostatným účelovým určením. Ve spojení s § 1479, který ohledně dědického nápadu stanoví „Dědické právo vzniká smrtí zůstavitele.”, pak dostáváme určitou konkurenci norem OZ sui generis. To ovšem pokud na ni tak budeme chtít nahlížet. Bezpochyby nezle tvrdit, že v tomto jinak právně uceleném okamžiku (smrti) by šlo předřadit vznik jednoho práva před druhým. Rozumnější a konformnější by bylo nahlížet na vznik obou konformně a ponechat je koexistovat, kde se v ten samý okamžik majetek vyčleněný do svěřenského fondu stane majetkem samostatným a „zbytek” (je-li takového) pak zůstává zůstaviteli, ergo jen ten se může stát předmětem dědického řízení. Výkladem zákona také nelze rozumně dojít k závěru, že by mohl být obmyšlený či správce účastníkem dědického řízení, neb ani jednomu vlastnické právo nesvědčí, ke svěřenskému fondu (trustu dle CCQ) totiž nemá majetková práva nikdo."}\\
 
 Je tedy ?zřejmé?, že majetek vložený do svěřenského fondu nejspíše nebude součástí pozůstalosti, stejně tak jako svěřenský fond nebude i přes velmi nešťastně ustanovení § 311 dědic. Mohlo by se tedy zdát, že svěřenský fond \textit{mortis causa} může být úspěšně použit k faktickému vydědění nepominutelného dědice a má hypotéze je tedy zodpovězena. I v případě neúspěchu ochrany dědice ve vztahu k pozůstalostnímu řízení je však stále třeba zaměřit se i na samotný instrument použití k založení svěřenského fondu a omezení na něj kladená.
 
 Skuhrovec je ve svém výkladu velmi konzistentní a svojí argumentaci zakládá na pevných základech, já s ním po všech směrech souhlasím. Opomněl však zohledit následující část Hornova argumentu: \textit{"lze mít za to, že zřizovací instrument (tj. pořízení pro případ smrti) musí respektovat ustanovení o povinném dílu"}, Horn k tomu v jedné větě ovšem současně dodáva, že je tak třeba učinit v případě, že vyčleněný majetek je součástí pozůstalosti. Já si však myslím, že i v případě, kdy bysme uplatnili jednotlivá ustnanovení v zákoném postupu, jak navrhuje autor v díle Svěřenské fondy, Příležitosti a rizika, v časové linii je samotné zůstavitelovo pořízení na prvním místě, bez ohledu na následný průběh pozůstalostí řízení by tak mělo splňovat nároky na něj kladené bez jakých koliv výjimek, pokud je ovšem zákonodárce v jiné části nestanoví, v tomto případě tak neučinil. Lze tedy dovodit, že podmínky omezují zůstavitelovu dispozici v § 1492 OZ o nemožnost zkrácení povinného dílu nepominutelných dědiců musí být dodržena a aplikována. K tomuto poskytuje dobré stanovisko Kolář: \textit{"Při kolizi svěřenského fondu zřizovaného pro případ smrti s nároky nepominutelných dědiců je třeba příslušená ustanovení, to je § 1451 odst. 3 ObčZ (svěřenský fond vzniká smrtí zůstavitele) a § 1479 ObčZ (dědické právo vzniká smrtí zůstavitele) vykládat tak, jak je upraveno v zákoně v logickém postupu. Postavení nepominutelných dědiců je upraveno v § 1642 a násl. ObčZ s tím, že nepominutelnému dědici náleží z pozůstalosti povinný díl. Nepominutelný dědic proto bude mít nárok proti správci svěřenského fondu a bude uspokojen ze svěřenského fondu právě na tento povinný díl, nikoli však na věci vyčleněné do svěřenského fondu zřizovaného pro případ smrti, který vznikne smrtí zůstavitele, čímž se naplní § 1448 odst. 2, resp. 3 ObčZ, podle kterého majetek ve svěřenském fondu není ani vlastnictvím správce, ani vlastnictvím osoby, ani vlastnictvím obmyšleného."}\footfullcite[Strana 82]{svejkovsky_jaroslav_sverenske_2018}.\\
 
 Ochrana nepominutelných dědiců by tak měla být zajištěna před samotným pozůstalostním řízením faktickým omezením možných dispozic s majetkem pořízením pro případ smrti, respektive omezením možnosti zůstavitele pořídit o veškerém svém majetku bez zohledněním nepominutelného dědice, respektive jeho omezením nebo opomenutím. Samotné opomenutí nepominutelného dědice však nezakládá neplatnost pořízení pro případ smrti tak jako tomu bylo za účinnosti OZ z roku 1964\footfullcite{sro_httpwwwpraguebestcz_postaveni_nodate}, i v případě opomenutí nepominutelného dědice není založen důvod, pro který by pořízení pro případ smrti nemělo být platné a z toho titulu by neměl vzniknout svěřenský fond. Je zřejmé, že v případě neplatnosti pořízení pro případ smrti by svěřenský fond nevznikl, jak ostatně i sám Kocí uznává ZDROJ. Toto dovození ale není v obdobném případě uplatnitelné.
 
 
  V NOZ vzniká dědici při jeho opomenutí v pořízení pro případ smrti ze strany zůstavitele pouze pohledávka za ostatními dědici, ten se tak stává jejich věřitelem. V případě toho, že by tedy svěřenský fond a ani svěřenský správce nebyli účastni pozůstalostního řízení jako dědicové a nebyli by za ně tedy ani povážování, je k diskuzi, zda by takováta pohledávka, jdoucí oproti ostatním dědicům, obstála. Kovář tvrdí ve svém příspěvku, že ano, opět ale nárážíme na to, že explicitní úprava této kolize chybí, stejně tak jako soudní rozhodnutí tento problém řešící, tento názor je tedy jednoduše zpochybnitelný.\\
  
  Je samozřejmě možné uvažovat nad tím, že s ohledem na veřejný pořádek by soud v případném soudním sporu mohl tuto pohedávku za svěřenským fondem postavit na roveň jiným pohledávkám za jinými dědici a z toho důvodu by opominutý dědic mohl mít právo dožadovat se svého povinného dílu po svěřenském fondu. Je to ale také obdobně diskutabilní.
 
 
 %Napsat o tom, že při takovémto opomenutí může být majetek pozůstalosti (pokud by nezahrnoval majetek vložený do svěřenského fondu absolutně nedostatečný k samotnému otevření pozůstalostního řízení, opravné prostředky použitelné v jeho průběhu by tak nebylo možné uplatnit, krom ustanoveních, o kterých zákonodárce říká, že jdou uplatnit v každém případě.
 
 
  tak nespočívá na základě pozůstalostního řízení (není z titulu pozůstalostního řízení tzn. dědic proti dědici), ale z titulu samotné ochrany nepominutelnému dědici poskytované už v rámci možných dispozic s majetem v rámci pořízení pro případ smrti ze strany zůstavitele.
  
  Možnost tak při tomto pohledu, kdy neuvažujeme, že samotná pohledávka za svěřenským fondem by obstála představuje pouze pouze vyvolaný spor o dědické právo, kdy by dědic popřel pořízení pro případ smrti.
 
 %Dále napsat co myslí Kovář tím, že pohledávka půjde za svěřenským fondem, ne za věc vyčleňovanou v rámci pozůstalostního řízení (nebo i mimo ně), toto souvisí s tím, že se již nevyhlašuje celé pořízení pro případ smrti neplatné, ale pouze opomenutému dědici vzniká pohledávka za ostatními dědici. Pořízení pro případ smrti je § 1492.
 
 %Rozdíl mezi opomenutím a omezením, opomenutí neuvedu dědice v závěti a 9 milionů z 10 odkážu svěřenskému fondu, omezení - omezím povinný díl nepominutelného dědice tak, že mu například odkážu pouze jeden milion a zbytek odkážu svěřenskému fondu (§ 1644)
 
 %Pokud by dědic napadl platnost závěti mohl by soud závěť i zrušit v tom případě by jak říká Kocí svěřenský fond nevznikl, dále k tomuto napříkl https://is.muni.cz/th/c82g3/Veronika_Kubatova_TEXT_RIGOROZNI_PRACE.pdf.
 
 V případě výkladu těchto ustanovení tak, že svěřenský fond nemůže být dědicem a obdobně svěřenský správce nemůže být dědicem \textit{sui generis} dědící ve prospěch svěřenského fondu, omezila by se účast svěřenského správce podle § 164 ZŘS pouze na informační a informativní.\\
 
 V jiných řízeních ve věcech svěřenského fondu je pak svěřenský správce plnohodnotným účastníkem, ve věcech pozůstalostního řízení je však jeho účast diskutabilní. Je pak pouze logické, že v případě, že by svěřenský fond nebyl účastníkem řízení o pozůstalosti, pak by takový vyčleněný majetek nebyl ani součástí pozůstalosti, nabytí jehož části ve formě dědictví by následně nepotvrzoval soud a z toho důvodu by nepominutelní dědici neměli nárok na svůj povinný díl ani v případě, že byl do svěřenského fondu vyčleněn veškerý zůstavitelův majetek.\\
 
 %na kterou rozhodně nělze pohlížet jao na subjekt
 
 Druhým budem Kocího argumentace je pak jakási nutná rovnost mezi oboumi formami založení svěřenského fondu: \textit{"17. Pokud by měl být majetek vyčleněný do svěřenského fondu součástí pozů­stalosti, mohl by být teoreticky snížen o povinný díl nepominutelného dědice (§ 1642). To by znamenalo, že by zůstavitel pořízením pro případ smrti při existenci nepominutelného dědice nemohl do svěřenského fondu vyčlenit všechen svůj majetek, což by testamentární svěřenský fond značně znevýhodňovalo ve srovnání s ostatními typy svěřenských fondů (viz komentář k § 1449), a to bez rozumného důvodu a jakékoliv opory v ustanoveních o jeho vzniku, která na obě formy zakladatelského aktu, tj. na smlouvu a pořízení pro případ smrti, nahlížejí jako na rovnocenné."}\footnote{Tamtéž.}\\
 
 Přes fakt, že při pohledu na úpravu svěřenského fondu a její konkurenci s právní úpravou dědického práva touto optikou docházím k obdobnému závěru jako Michal Skuhrovec ve své práci, nemohu se v žádném případě ztotožnit s tímto bodem. V případě ochrany nepominutelného dědice i ve chvíli vyčlenění majetku do svěřenského fondu \textit{mortis causa} ať už dovozené ze současné litery zákona, nebo potenciálně v budoucnu implementované formou novely si nemyslím, že dochází k zásadnímu znevýhodnění formy založení \textit{inter vivos} oproti \textit{mortis causa}. Neboť obě dvě formy, přestože se jedná o stejný isntitut, jsou založeny jiným způsobem a mohou být založeny často za značně jiným účelem (například ochrana majetku před věřiteli nebo za účelem businessu vs mezigenerační převod majetku). Obě tyto formy lze paralelně připodobnit k možnostem převedení majetku za života zůstavitele, jako je například darování, směna, prodej (neuvažuji daňové zatížení, pouze ve vztahu k institutu nepominutelného dědice), nebo obdobným, podmínkamy dědického práva nezatíženým, formám převodu, jakými jsou například darování pro případ smrti\footnote{Viz kapitola 6.3.3 Vznik svěřenského fondu \textit{inter vivos} s odkládací podmínkou} a převedení majetku za pomoci dědického práva, tedy v rámci pozůstalostního řízení, po smrti zůstavitele (kde pokud exituje nepominutelný dědic a není platně vyděděn, má nárok na svůj nepominutelný díl).\\
 
 Přestože se tedy jedná o stejný institut, lze předpokládat, že účelem založení svěřenského fondu pořízením pro případ smrti bude s největší pravděpodobností mezigenerační převod majetku, zatímco u založení svěřenského fondu smlouvou mezi živými může být množina účelů širší. Toto samozřejmě nevylučuje možnost založení svěřenského fondu pořízením pro případ smrti za stejnými účely jako u \textit{inter vivos}, tyto jsou však mnohem výhodnější za zůstavitelova života, už jen zkrátka proto, že je zakladatel naživu. Stejně tak i pokud bychom se bavili o jedné výhodě, o kterou by tak testamentární svěřenský fond při ochraně nepominutelných dědiců přišel, tedy o faktickou možnost jeho pomocí vydědit, stále má tato forma mnoho výhod oproti svěřenskému fondu založeného \textit{inter vivos}, o tomto ostatně hovořím v kapitole 6.3 u jednotlivých forem založení svěřenského fondu.\\
 
 %Nárážím zde na to, že účely, které může zakladatel sledovat založením svěřenského fondu \textit{inter vivos}
 
 Pihera se pak řadí do opačného názorová spektra, kdy tvrdí následující: \textit{"Poněkud odlišnou povahu má vyčlenění majetku do svěřenského fondu zřizovaného pořízením pro případ smrti (mortis causa). V těchto případech vyčlení zakladatel majetek ve prospěch svěřenského fondu tak, že v jeho prospěch vyhradí dědické právo nebo nařídí odkaz (§ 1452 odst. 1 ve spojení s § 311 odst. 1). Správce svěřenského fondu mortis causa pak vykonává práva dědice majetku vyčleněného do svěřenského fondu nebo práva odkazovníka. Majetek vyčleněný zakladatelem (zůstavitelem) do svěřenského fondu mortis causa je součástí pozůstalosti a nabytí tohoto majetku do svěřenského fondu buď potvrzuje soud, nebo je převeden ve prospěch svěřenského fondu dědicem obtíženým odkazem. Z hlediska dědického práva je na postavení svěřenského fondu nutné pohlížet „jako na dědice“ nebo „jako na odkazovníka“, a to se vším všudy, co se s tímto postavením spojuje (např. odpovědnost za dluhy zůstavitele)."}\footfullcite[Komentář k § 1448]{spacil_j_a_kolektiv_obcansky_2013}\\
 
 Základním předpokladem, který musí být splněn je tedy nutnost považovat svěřenský fond za účastníka pozůstalostího řízení, přičemž za dědice \textit{sui generis} bude považován svěřenský správce, k tomu ostatně § 311 OZ přímo vybízí.
 
 %Je třeba rozlišit, zda je svěřenský správce sám dědicem sui generis kvůli nedostatku právní osobnosti svěřenského fondu, nebo zda je svěřenský fond pro tuto příložitost nadán právní subjektivitou, respektive tedy v uvozovkách považován za právnickou osobou, tak jako tomu činí zákonodárce v daňovýc předpisech (Uvést ZDROJ) a tedy i možností účastnit se tohoto řízení a svěřenský správce ho pouze zastupuje (tak jak by zastupoval statutární orgán jinou právnickou osobu v případě, že by byla určena dědicem.
 
 \textit{"Svěřenský fond může být zřízen též pořízením pro případ smrti, tedy závětí nebo dědickou smlouvou; jeho podmínky mohou být moderovány též dovětkem (§ 1492). Závěť nebo dědická smlouva v takovém případě musí obsahovat náležitosti statutu svěřenského fondu (§ 1452 odst. 1, 2) a musí splňovat formální požadavky kladené na ně dědickým právem (§ 1533–1593). To platí též o vedlejších doložkách v závěti. Správce takto zřízeného svěřenského fondu pak vykonává práva dědice nebo odkazovníka, aniž by však sám byl osobním dědicem či odkazovníkem (dále srov. komentář k § 1451 a § 1452 odst. 1)."}\footnote{Tamtéž.}\\
 
    \begin{enumerate}
 {\Large\item[2.] Ochrana nepominutelných dědiců s ohledem na úmysl specifických ustanovení}
 \end{enumerate}
 
 Přes chybějící explicitní ustanovení přiznávající svěřenskému fondu právní subjektivitu ve vztahu k pozůstalostnímu řízení je a dalších ustanovení upravující přímo kolize mezi právní úpravou svěřenského fondu a dědického práva, je i dle mnohých autorů, tuto dovodit a je tak možné hovořit o tom, že majetek vyčleňovaný do svěřenského fondu je součástí pozůstalosti a může tak teoreticky být ponížen o povinný díl nepominutelného dědice.\\
 
 Nicméně samotný fakt, že takováto explicitní úprava chybí a je možné jí dovodit jen s notnou dávkou imaginace a i přes to si člověk nemůže být svým výkladem tohoto problému jistý jasně dokládá potřebu, že je tyto kolize nezbytné explicitně upravit. Já tento samotný návrh činím v kapitole 7.12.4.

%je proto logické, že v současném Občanském zákoníku tyto ustanovení ochybý, tedy nemohla být součástíV této souvislosti bych však chtěl upozornit na kapitolu 7.5.1, kde popisuji ochranu nároků, které lze připodobnit k institutu nepominutelného dědice v Občanském zákoníku provincie Quebec. Tím pádem nelze zcela souhlasit s druhou částí

%kdy bych chtěl jeho názor konfrontovat a rozšířit o můj názor, nicméně z jednoho pohledu s ním však souhlasím.

%Neboť s ním minimálně v rámci jednoho z mých pohledů souhlasím.

Snahu o zodpovězení tohoto problému neulehčuje ani fakt, že standardně používáné komentáře si při pohledu na tuto problematiku vzájemně odporují.

%upravuje právní řádve svém právním řádu 

Je tedy zřejmé, že pokud bychom důsledně aplikovali konstrukci svěřenského fondu, při pohledu první optikou by tak dědic neměl téměř žádnou možnost dožadovat se svého povinného dílu. Přes toto všechno je však možné si povšimnout jisté tendence zákonodárce za pomoci všech výše zmíněných ustanovení nepominutelného dědice chránit i v takovém případě, to že tato ochrana není ve vztahu ke konstrukci svěřenského fondu moc dobře aplikovatelná je věc jiná. Dle mého pohledu je úmysl zákonodárce nějak dědice chránit i v tomto případě z těctho ustanovení jasný a tedy i při úmyslovém pohledu na věc si myslím, že dědice chránit chtěl. Je tedy obdobně možné, že by soud rozhodl ve prospěch takovéhoto opomenutého dědice právě na základě výše zmíněných ustanovení i přes samotný ohled na konstrukci svěřenského fondu.

Pokud by mu tedy přiznal jakousi právní subjektivitu, tak jak tomu jednou udělal v daňových zákonech, bylo by zřejmé, že tato ochrana by dědici byla poskytnuta. S ohledem, že tak zákonodárce jednou i učinil, není absolutně vyloučeno, aby tak udělal i ve vztahu k dědickému právu, ačkoli by to bylo skrytě a mlčky.










Obdobné stanovisko zaujímá například i Kryštof Horn ve svém příspěvku do Časopisu Ad Notam. Ve svém příspěvku říká následující: \textit{"Příkladem takové nejasnosti může být vztah fondu k institutu nepominutelného dědice, který je neznámý jak v zemích common law, kde má své kořeny trust, tak bohužel i v Québecu, odkud zákonodárce přejal úpravu svěřenských fondů. Výslovná ustanovení o vztahu svěřenského fondu mortis causa a dědického práva totiž chybí."}. \footfullcite{sro_httpwwwpraguebestcz_prakticke_nodate-1} \\

Tento příspěvek se při pohledu na úpravu svěřenského fondu zakládá na pravdě a autor v něm potvrzuje fakt, že takováto právní úprava v českém právu chybí a bylo by vhodné jí prostřednictvím novelizace doplnit. S výrokem nicméně nemohu souhlasit jako s celkem, protože přesto, že je pravda, že institut nepominutelného dědice jako takový v Quebeckém právu neexistuje a Quebecká právní úprava se dále vyznačuje téměr úplnou testamentální volností, tak Quebec Civil Code připouští určitou ochranu dědiců před opomenutím v závěti. Abych byl přesný, tak Quebecký právní řád upravuje tři případy ochrany dědiců, které by se daly přirovnat k institutu nepominutelného dědice v našem právním řádu. Je tedy možné říci, že, přestože tak zákonodárce neučinil, mohl se u Quebecká právní úpravy inspirovat i v rámci ochrany nepominutelných dědiců, protože Quebecké právní úprava, narozdíl od té České, upravuje i případy, ve kterých by tyto nároky nemohly být splněny kvůli právnímu jednání učiněnému zůstavitelem, mezi toto právní jednání je právě možné zařadit i založení trustu.\footfullcite{leclercq_inheritance_2014} \\

Obdobně je také možné hledat inspiraci pro vyřešení této nejastnosti ve vztahu k dědickému právu i v případech z Itálie, či v právní úpravě institutu obdobného svěřenskému fondu v Lichtenštejnsku, tento fakt a jeho možné aplikace na českou právní úpravu jsou zhodnoceny v kapitolách. \\

Z výše uvedeného tedy vychází, že při kolizi práva nepominutelného dědice a vytvoření svěřenského fondu \textit{mortis causa} není zřejmé, zda je právo na povinný díl chráněno, nebo nikoli. Při pohledu mojí první definovanou optikou se naprosto ztotožňuji s výkladem Skuhrovce a to do té míry, že nepominutelný dědic není nikterak chráněn. Není ale možné přehlížet ustanovení, která by přes svojí zjevnou nekompatibilitu s konstrukcí svěřenského fondu přesto měla být aplikována, z těchto vyvozuji potenciální úmysl zákonodárce chránit nepominutelné dědice i v případě založení svěřenského fondu \textit{mortis causa} pořízením pro případ smrti a při užítí druhé definované optiky je tak možné dojít i protichůdného názoru a tvrdit, že nepominutelný dědic chráněn je. Zůstává tedy otázkou, zda by soud v případě sporu prioritizoval naprostou testovací volnost zakladatele, nebo důkladně aplikova ustanovení o nepominutelném dědici i přes jejich zjevnou nedotaženost a nekompatibilitu ve světle úpravy svěřenského fondu. Samotná absence norem, které by toto explicitně upravovali je alarmující, výklad ve prospěch nepominutelného dědice tedy při důsledné aplikaci konstrukce svěřenského fondu nemůže obstát, bylo by ale nutné naprosto ignorovat jiná ustanovení omezující konání zůstavitele jako je § 1492 OZ, či ustanovení uznačující svěřenský fond jako dědice, tedy § 311 OZ. Minimálně ustanovení § 1492 OZ by však mělo být bez dalšího zůstavitelem respektováno v případě, že nepominutelného dědice platně nevydědil, lze tedy obdobně tvridit, že toto ustanovení samo o sobě ochranu nepominutelnému dědici poskytuje a je do jisté míry kompatibilní i s absencí právní osobnosti svěřenského fondu, v takovém případě je však nutné pohlížet na svěřenský fond jako na dědice (entitu) \textit{sui generis}, vůči které může uplatnit opomenutý dědic svoji pohledávku a být tak jeho věřitelem, při takovémto pohledu tak nemůže váhu argumentů ustát ani první tvrzení. Nezbývá však než nechat řešení této otázky na zákonodárci, či na výkladu soudu, ani jeden výklad totiž není bez dalšího superiorní a nelze považovat za jediné správné řešení.\\

\subsubsection{Komparace právní úpravy svěřenského fondu s právní úpravou dědictví}

%Tuto kapitolu možná přejmenuji jako na dílčí závěr problematiky nepominutelného dědice zkoumané v této části.

---Koment\\
Zde proběhne komparace, musím zde napsat, že neexistuje právní úpravy, která by upravovala situaci, ve které bude založen svěřenský fond mortis causa a zároveň bude probíhat dědické řízení, ve kterém budou figurovat nepominutelní dědici, na které v souvislosti se založením svěřenského fondu nevyjde jejich zákonný podíl. Konstatovat, že těmto problémům se dá vyhnout tím, že se svěřenský fond vyčlení už za života, dokázat toto tvrzení na základě jiných institutů převodu majetku za zůstavitelova života a poukázat na to, že tento majetek následně není předmětem dědického řízení a tudíž ani nemůže připadnout jakýmkoliv dědicům, neboť pokud budeme mít na časové ose bod x, ve kterém zůstavitel zemře, a svěřenský fond vznikne v bodě x-y, tak v době zůstavitelova úmrtí již není součástí jmění a tudíž nemůže být součástí dědického řízení. Pokud by nicméně zůstavitel zemřel v bodě x a svěřenský fond by také vznikl v bodě x, tak lze polemizovat o tom, zda by dědici nemohli mít právo na snahu obsáhnout i tento vyčleněný majetek v soupisu dědického řízení.\\
---\\

---Koment\\
Dále se pokusit obhájit můj názor, že pokud budou obmyšlenými svěřenského fondu i nepominutelní dědicové, tak plněním ze svěřenského fondu právě tímto dědicům dojde k neformálnímu doplnění jejich povinného podílu ze strany zůstavitele - pokusit se najít nějakou právní zásadu, která by něco takového říkala. Pokud by však z fondu mělo být plněno nikomu jinému, tak se pokusit obhájit fakt, že tato situace není zákonně obhájena a není na ní ani žádná judikatura či ustálená soudní praxe, a potenciálně by se tedy dědicové mohli bránit soudní cestou. Pokusit se stanovit, jak by takovýto spor mohl v rámci našich existujících právních norem dopadnout, pokud to půjde tak se zkusit podívat, zda toto neni řešeno na mezinárodní úrovni například v rámci mezinárodního práva soukromého. \\
---\\

Pro tuto kapitolu, a následné návrhy na změny, je v první řadě podstatné vymezit,  v jakých fázích vzniká svěřenský fond, a tyto následně porovnat s úpravou dědického práva. Tímto postupem lze dosáhnout vhodného a dostatečného porovnání možností, které mají dědici v jednotlivých fázích, jak jsou chráněni a jak se potenciálně mohou brát o svá práva. \\ 

Dle mého názoru, který potvrzuje Miloš Kocí v rámci komentáře k úpravě svěřenského fondu, lze hovořit o 3 způsobech vzniku svěřenského fondu. Těmito způsoby jsou, založení svěřenského fondu \textit{inter vivos}, \textit{mortis causa} a potenciální souběch těchto založení ve formě založení svěřenského fondu \textit{inter vivos} s odkládací podmínkou, dle § 584 až 549 OZ, spočívající ve smrti zůstavitele.\footfullcite{svestka_j_obcansky_2014-1} \\

\newpage

\thispagestyle{smallertextinheader}

Vizuálně by se tedy průběh komparace napříč jednotlivými způsoby založení dal vyobrazit následovně, kde střed osy představuje čas úmrtí zůstavitele:

\begin{figure}[h]
\centering
\includegraphics[width=18cm,height=6cm,keepaspectratio]{Vznik_sverenskeho_fondu.png}
\caption{Vznik svěřenského fondu}
\label{fig:komparace}
\end{figure}

---Koment\\
Doplnit další informace.
Popsat dědické právo - co jde do dědictví, co jsou nepominutelní dědicové, jak funguje ědické řízení atp.
Vysvětlit rozdíl mezi majetkem a jměním.\\
---\\

V případě založení svěřenského fondu mezi živými s odkládací podmínkou smrti zakladatele se ve své podstatě jedná o založení \textit{mortis causa} formou \textit{inter vivos}, s ohledem, že se nejedná o založení svěřenského fondu pořízením pro případ smrti je zřejmé, že v tomto případě nelze uvažovat ani nad ochranou dle §1492, pro ilustraci je možné porovnat tento způsob založení s další formou mezigeneračního předání majetku a to darováním pro případ smrti. Podklad pro toto tvrzení poskytují například autoři článku Darování pro případ smrti v Časopisu pro právní vědu a praxi Adam Talanda a Iveta Talandová. Ti přímo říkají: \textit{"Darování pro případ smrti je při splnění všech zákonných podmínek obligačním vztahem mezi dárcem a obdarovaným a neřídí se ustanoveními dědického práva. Zákon tak umožňuje užitím jednoho zákonného institutu obcházet soubor jiných zákonných institutů. Především se jedná o obcházení ustanovení omezujících zůstavitele v nakládání s jeho majetkem, která slouží k ochraně nepominutelných dědiců."}\footfullcite{noauthor_casopis_2015}.

%Dědic by se s ohledem na paragraf 1492 dal brát jako věřitel zůstavitele. Musím na to nahlížet z hlediska veřejného pořádku a základních morálních principů.

%Je darování vždy oboustraný vztah, co když mu daruji například peníze jednostranně, bude se pak také jednan o obligatorní právní vztah založený na smlouvě? Nebo poté bude předmět takovéhoto darování součástí pozůstalosti a bude se na něj například nahlížet jako na odkaz? Nikoliv je to dvoustranné právní jednání, jedná se tedy o obligatorní vztah.

\subsubsection{Ochrana zájmů nepominutelných dědiců zakladatele trustu v Itálii}

Napsat o situaci v Itálii.\footfullcite{lubos_tichy_sverensky_2017}

\subsubsection{Právní úprava v Lichtenštejnsku}

\subsection{Nejasný výklad ve vztahu k vyživovací povinnosti}

\subsubsection{Typy ochrany dědických nároků v rámci Quebec Civil Code}

V Quebeckém právní řádu existují tři instituty, které by se svým účelem a formou daly přirovnat k českému institutu nepominutelného dědice. Ve své pdostatě se jedná o instituty, které mají chránit osobu, které je zůstavitel povinnen poskytnou nějaké plnění před opomenutím této osoby ze strany zůstavitele v rámci závěti, nebo opomenutí této osoby pokud se dědí na základě intestátní posloupnosti. Zákonodárce zároveň upravil vztah těchto institutů k právnímu jednání zůstavitele, které mění výši zůstavitelova majetku již v obodobí před jeho smrtí a s účinností k datu jeho smrti. \\

\vspace{5 mm}

Tři ochrany, které jsou upravené v Quebeckém právním řádu:

\begin{enumerate}
\item Survival of the Obligation to Provide Support
\item Family Patrimony
\item Compensatory Allowance	
\end{enumerate}

\vspace{5 mm}

\underline{\textbf{Survival of The Obligation to Provide Support}}

Prvním způsobem ochrany jistých dědických nároků vůči majetku zachované\-mu zůstavitelem spočívá v Institutu zvaném \textit{The Survival of the Obligation to Provide Support}, který zaručuje příjemci určité finanční podpory vyplácené danému příjemci za zůstavitelova života tuto podporu i po smrti zůstavitele. \\

Důležitou roli pro dokázání argumentu, že Quebecká právní úprava obsahuje v jisté formě ochranu dědiců a upravuje i případy zmenšení zůstavitelova majetku v období před smrtí popřípadě učinné ke dni smrti zůstavitele je odstavec 689, který stanovuje následující: "Where the assets of the succession are insufficient to make full payment of the contributions due to the spouse or to a descendant, as a result of liberalities made by acts inter vivos during the three years preceding the death or having the death as a term, the court may order the liberalities reduced. \\

Liberalities to which the spouse or descendant consented may not be reduced, however, and those he has received shall be imputed to his claim."\footfullcite{noauthor__nodate} Tento odstavec ve zkratce zaručuje osobám s nárokem vůči zůstavitelovi ochranu tohoto nároku v případě nedostatečného jmění v rámci dědictví, které bylo jistým právním jednáním zůstavitele sníženo na tuto nedostatečnou hranici. Soud může v tomto případě omezit právní jednání učiněné zůstavitelem dokonce i za jeho života, co je ale důležité pro založení trustu mortis cause je fakt, že může omezit i právní jednání s účinností ke dni zůstavitelovi smrti. Tímto způsobem tedy může omezit i vyčlenění majetku do svěřenského fondu, respektive Trustu. \\

Dále je potřebné a důležité poznamenat i to, že odstavec 689 ve své druhé části říká, že v případě, že z těchto právních jednání bude mít prospěch právě osoba, která se domáhá svého podílu na zůstavitelově majetku, tak tento prospěch bude započten v rámci potenciálního omezení právního jednání zůstavitele. Z této části je tedy možné dovodit, že uspokojit tento právní nárok je možné i založením trustu, který bude daný prospěch oprávněným osobám poskytovat místo likvidátora dědictví\footnote{V Quebeckém právu je likvidátor osobou, která}. Dá se tak tedy fakticky hovořit o obdobě započtení na dědický podíl v českém právu.

\underline{\textbf{Family Patrinomy}}

Pro účely family patrimony je důležité ustanovení 416: \textit{"In the event of separation from bed and board, or the dissolution or nullity of a marriage, the value of the family patrimony of the spouses, after deducting the debts contracted for the acquisition, improvement, maintenance or preservation of the property composing it, is equally divided between the spouses or between the surviving spouse and the heirs, as the case may be."}\\

\underline{\textbf{Compensatory Allowance}}

\subsection{Nejasnosti v rámci daňových aspektů svěřenského fondu}

Jisté nejasnosti lze pozorovat i v právní úpravě svěřenského fondu ve vztahu k daním. V souvislosti s přijetím Nového Občanského zákoníku a jeho účinností bylo potřeba novelizovat i další právní předpisy. Jednou z těchto novelizací prošly právní normy upravující zdaňování, především tedy zákon o daních z příjmu. I přes tyto novelizace však shledávám v daňové úpravě vztahující se ke svěřenským fondům jisté nedostatky. Nemyslím si, že s ohledem na téma práce je vhodné hlouběji zabředávat do tématu zdanění svěřenského fondu a jeho výhodnosti z pohledu daňového zatížení, neboť žádná takováto výhodnost neexituje, avšak v tomto rámci se vyskytuje i jeden problém týkající se zániku svěřenského fondu a následného převedení zbylého majetku obmyšlenému, nebo zakladateli, který je dle mého názoru vhodný uvést kvůli potenciálním nejasnostem vztahujícím se k zániku svěřenského fondu.\\

Daňová úprava svěřenských fondů, by se dala nejspíše shrnout následovně.

Problém dle mého názoru nastává při zániku svěřenského fondu, kdy není zákonně zcela jasně upraven postup.

Obdobně problém vzniká v případě, kdy za existence svěřenského fondu bylo vypláceno s daní/bez daně, nově fond tedy zaknikne a trápí mě neexistující úprava stanovující jakým způsobem má být zdaněna finální výplata ze svěřenského fondu, má to být stejně jako za existence svěřenského fondu, já se domnívám, že pokud aplikujeme § 24 daňového řádu, tak ano(protože v jistých případech výplata ze svěřenského fondu zdaněna není, v jiných případech zdaněna je), ale dle mého názoru by bylo vhodné toto jasně a srozumitelně popsat, neboť i tento nedostatek může přispět k nerozhodnosti zda je vhodné svěřenský fond založit a nebo ne ze strany zakladatele, a s ohledem na to, že tvrdíme, že svěřenský fond chceme umožnit zakládat, abychom tim rozšířili možnosti projevení vůle ze strany zakladatele a fakticky mu chceme umožnit, aby mohl ovládat svůj majetek, nebo respektive, aby s jeho majetkem po jeho smrti bylo nakládáno podle jeho vůle, tak tady to nelze jasně aplikovat, protože ve chvíli kdy by tato vůle směřovala například k tomu, že se svěřenský fond má po nějaké době zrušti, že je tedy založen jen na určitou dobu, může zakladatel při založení takovéhoto svěřenského fondu jasně předpokládat jak na následné převedení dopadne daňové zatížení? Já si myslím, že ne a to tedy ve svém konečném důsledku omezuje možnosti zakladatele plánovat jak má být s jeho majetkem po jeho smrti naloženo. Právě proto by tedy bylo dobré tyto nedostatky a nejasnosti napravit a objasnit.

Mohlo by se namítat, že problémy se zdaňením převáděného majetku ze svěřenského fondu obmyšleným nebo zůstavitelům již nejsou problémem zakladatele, neboť ten již bude při založen ísvěřenského fondu mortis causa nebo inter vivos s odkládací podmínkou smrti po smrti, ale je vhodné si uvědomit, že tento problém dopadá i na svěřenské fondy založené inter vivos, kdy zakladatel může být stále na živu a majetek tak může připadnou po zániku svěřenského fondu jemu. Je tedy v zájmu zakladatele o tomto problému přemýšlet a je vhodné, aby toto bylo tedy zákonnem patřičně upraveno.

Obdobně může zakladatel o tomto problému přemýšlet i ve vztahu k založení mortis causa či s odkládací podmínkou smrti ve vztahu k obmyšleným svěřenského fondu (tedy těm v jejichž prospěch se mezigenerační majetek převádí a zda daňové zatížení v takovémto případě nebude horší než v případě převedení majetku pomocí klasického dědického řízení, kdy se pokud se majetek převádí osobám v kategorii, které jsou osvobozeni od placení daně a paltí tak pouze notáři).

Samotné daňové zatížení v rámci funkčního svěřenského fondu není horší než například pokud by se majetek předával pomocí založení právnické osoby, jak například stanovuje tady v té práci, ale při zániku mohou vznikat nejistoty a potenciální problémy a vyšší zdanění, než které by vzniklo například při předání majetku v klasickém dědickém řízení.

Nakonec je problém i v likvidaci svěřenského fondu a tedy ochranou věřitelů (toto je obecný zájem, ne zájem zakladatele, obmyšlených, nebo svěřenského správce), protože likvidaci, ačkoli by mohla nějakým způsobem být vložena do statutu, jak říká například následující práce, tak standardně u svěřenských fondů neprobíhá.

\subsubsection{Ochrana v Anglii}

%file:///Users/dominikbalint/Downloads/RPTX_2011_1__0_370222_0_120694.pdf

\subsection{Další systémové nedostatky}

\subsection{Možnosti použití svěřenského fondu k obcházení věřitelů}
\subsection{Rozdílné pojetí vlastnictví}

Další otázku, kterou si mnozí právní experti položili je samotné pojetí vlastnictví v systému civil law, které je dle mnohých jen těžko kompatibilní s děleným vlastnictvím, které je inherentně spojeno se systémy common law.\\
%které je vlastní právním systémům common law. 
Jedná se o velice dobrou otázku, neboť na svěřenský fond by se dalo nahlížet jako na dělené vlastnictví. Tento fakt byl dlouhou dobu povážován za překážku vstupu trustu do kontinentálních právních řádů, neboť kontinentální právní řády jsou založené na konceptu nedělitelnosti vlastnictví\footfullcite{lucie_joskova_sprava_2017}, zejména od doby, kdy se na Evropském kontinentu opustil koncept feudálních vztahů\footfullcite{michael_milo_trusts_2000}.\\

Pojem dělené vlastnictví by se dal jednoduše popsat na příkladu reálných vlastnických vztahů několika rodin, bydlící ve vícegeneračním domu. V takovémto případě by tyto rodiny fungovali na principu děleného vlastnictví, dalo by se řící tedy i na principu rodinného fideikomisu, kdy o nemovitosti, jako o větší majetkové hodnotě, nerozhoduje pouze formální vlastník, který je zapsán v katastru nemovitostí, ale rodina jako celek\footfullcite{ondrej_horak_vd_2010}.\\

Samotná Francouzská úprava, z níž z historického hlediska vychází i právo Kanadské provincie Quebec, nepoužívá dělené vlastnictví ale nahlíží na něj jako na obdobu českého jmění. České právo tento koncept obdobně nevyužívá a samotný občanský zákoník hovoří jen o majetku, jmění \textit{§ 495 Souhrn všeho, co osobě patří, tvoří její majetek. Jmění osoby tvoří souhrn jejího majetku a jejích dluhů.} a vlastnictví \textit{§ 1011 Vše, co někomu patří, všechny jeho věci hmotné i nehmotné, je jeho vlastnictvím.}, kde zásadnější rozdíl představuje pouze majetek a jmění, přičemž majetek a vlastnictví nemají mezi sebou téměř žádný rozdíl.\\

Není však absolutní pravdou, že by bylo dělené vlastnitví kontinentálním řádům cizí úplně, například Rakouský občanský zákoník ABGB obsahoval koncept děleného vlastnictví\footnote{Narozdíl však třeba od Francouzského Code civil, které naopak instituty založené na děleném vlastnictví až do Napoleonovi prohry rušilo, po konci Napoleonovi vlády se však i ve Francii přistoupilo o obnovení těchto institutů. Ve větším množství se například fideikomisy začaly rušit až po první světové válce například ve Výmarské ústavě, nebo v Československu po vyhlášení samostatného státu}, kdy k jednomu majetku existovalo právo vrchní a právo užitkové, při hlubším zamyšlení lze dojít k tomu, že právě takovýmto způsobem je posataven trust, kdy právo vrchní, tedy formální, zapsané, má svěřenský správce, ale právo užitkové má obmyšlený\footnote{Tamtéž.}. V CCQ byl tento problém vyřešen poměrně netradičně, a to takovým způsobem, že vyčleněný majetek není nikým vlastněn, ale že se jedná o autonomní vlastnictví, tedy respektive majetek svěřen vlastnímu účelu\footfullcite[POPOVICI A. Trust québeckého a českého práva:autonomní vlastnictví?]{lubos_tichy_sverensky_2017}. Tento koncept je založen na \textit{patrimoine d'affectation}, v CCQ patrimony by appropriation, v občanském zákoníku účelově určený majetek\footnote{Tento definuje Miloš Kocí v komentáří k Občanskému zákoníku od Wolters Kluwer jako "majetek bez vlastníka vyčleněný zakladatelem k naplňování určitého účelu a spravovaný svěřenským správcem podle instrukcí vtělených zakladatelem do zakladatelského právního jednání, resp. do zvláštního statutu"}, od Francouzského právníka Pierra Lepaullea\footfullcite{skuhravy_jan_trust_2010}.\\ 

V rámci tohoto pojetí vlastnictví spočívá Québecký trust na obligačních vztazích mezi správcem a beneficientem, nikoli na principu odděleného vlastnictví. Podle CCQ, které se v tomto přímo inspirovalo Lepaullem, nenáleži ani zakladateli, ani správci a ani obmyšlenému\footfullcite[Článek 1261]{noauthor__nodate}, ale existuje samo o sobě společně se závazky, které k němu vznikly v souvislosti s plněním účelu trustu\footfullcite{skuhravy_jan_trust_2010}.\\

Trust je tedy v CCQ vnímán jako samostatná a oddělená samosprávná quasi entita, která obsahuje jak majetek, tak závazky s ním spojené\footfullcite[Tomáš Richter: Mezi smlouvou, vlastnictvím a korporací: právní úprava trustu v návrhu nového občanského zákoníku]{stenglova_ivana_pocta_2006}, tedy obsahuje vlastní jmění vyhrazené svému vlastnímu účelu. Tam, kde majetek samozřejmě nemá vlastníka je vhodné zamyšlení nad jeho samotným odlišením od právnické osoby\footfullcite[Kenneth G.C. Reid: Patrimony not Equity: the Trust in Scotland]{michael_milo_trusts_2000}\textsuperscript{,}\footnote{Srovnání s kapitolou Svěřenský fond a právní subjektivita}, přes fakt, že zákonodárce nezavedl pro trust klasifikaci právnické osoby, tak trust tímto způsobem funguje\footfullcite[Madeleine Cantin-Cumyn: The Quebec Trust]{michael_milo_trusts_2000}, neboť je sám vlastníkem majetku a závazků v něm obsažených. Z formálního hlediska je však správce zapsán v registrech a vykonává správu nad majetkem ve svěřenském fondu\footfullcite[Článek 1260]{noauthor__nodate}\textsuperscript{,}\footfullcite{skuhravy_jan_trust_2010}.\\

Takto byl do jisté míry vyřešen problém děleného vlastnictví a mnoho tedy již, i přes existující teorie, které problém děleného vlastnictví vyvrací\footfullcite{lucie_joskova_sprava_2017}, nebránilo tomu, aby mohl být tento institut přijat do našeho kontinentálního právního řádu s obdobným pohledem na majetek takto do něj vložený.


PŘEPSAT
Lepaulle, ve snaze přiblížit trust kontinentální právní kultuře a jejím právníkům, šel "až na dřeň" a vymezil základní prvek trustu. Argumentuje, že po založení trustu není takový vůbec závislý na svém zakladateli, ani na správci, který může být vyměněn (srov. existence "bare" trustu, viz. kap. 3.4.). Obmyšleného pak jako esenciální prvek odmítá úplně. "Jedině, jak se dá k této otázce přistoupit, je hledat ty prvky, které jsou pro existenci trustu nezbytné", říká, a definuje trust jako "právní institut sestávající se z majetku nezávislého na právním subjektu, jehož jednotnost je definována učelovým určením v mezích zákona a veřejného pořádku". POPOVICI ALEXANDRA Nutným závěrem je pak vznik koncepce "patrimoine d`affectation", nebo též "patrimony by appropriation", konečně pak v OZ "účelově určený majetek", nebo též jak rozvádí Kocí v komentáři k OZ "majetek bez vlastníka vyčleněný zakladatelem k naplňování určitého účelu a spravovaný svěřenským správcem podle instrukcí vtělených zakladatelem do zakladatelského právního jednání, resp. do zvláštního statutu". KOCÍ KOMENTÁŘ 

%Doplnit reivindikační žalobu, dle RPTX_2011_1__0_370222_0_120694 staženého souboru na stránce 89

\subsubsection{Ležící pozůstalost}

\subsection{Zákonné nejasnosti změn svěřenských fondů před a za jejich trvání}

\subsection{Nezávislost správce}

%Kontrola správce, dohled nad správcem, odpovědnost správce, potenciální problémy, možná dopsat pokud zbyde čas a prostor.

%Diskrece posiluje korupční motivaci správce.

\subsection{Návrhy na změny České právní úpravy}

Jako nejzásadnější otázku, která je zároveň hypotézou, považuji nedostaky právní úpravy zaměřující se na řešení otázek vznikajících při kolizi dědického práva a práva svěřenských fondů, blíže se tedy budu věnovat návrhu řešení tohoto problému. V samotném závěru této části práce také uvedu svůj pohled na ostatní mnou výše popsané nedostatky, které spatřuji v právní úpravě svěřenských fondů a sním souvisejících institutů.\\

Z jednotlivých ustanovení dědického práva a svěřenského fondu je nopochybná vůle zákonodárce chránit na jedné straně nepominutelné dědice a na straně druhé zřídit fiduciární institut, který umožní zůstavitelovi téměř neomezeně nakládat se svým majetkem. Záleží samozřejmě na optice, kterou pohlédneme na jednotlivá ustanovení svěřenského fondu a pořízení pro případ smrti, úmyslně zmiňuji jen pořízení pro případ smrti a s tím související založení svěřenského fondu pro případ smrti, neboť ostatní formy založení nejsou a ani nemohou být omezovány dědickým právem, je však nezpochybnitelné, že zavedení neomezeného fiduciárního institutu je neslučitelné s právem nepominutelných dědiců. Způsobů k vyřešení tohoto nedostatku právní úpravy se nabízí hned několik.\\

\subsubsection{Zrušení svěřenského fondu}

První možností jak řešit nedostatky spojené se svěřenským fondem je v zásadě legislativně, nikoli však politicky, nejjednodušší. Zrušení svěřenského fondu by dozajista vyřešilo všechny problémy s ním spojené, zapříčinilo by však mnoho jiných problémů, které by se objevili při zániku jednotlivých svěřenských fondů.\\

Nemyslím si však, že takovýto krok je vhodný, už jen kvůli tomu, že institut svěřenského fondu byl přijat relativně nedávno a jeho odstranění by nepřispělo k již tak, dle mého názoru relitivně nestabilní, právní jistotě občanů Českého státu.\\

S návrhem na zrušení svěřenského fondu tak například přišla Pirátská strana, jedná se o návrh již z roku 2014, jednotlivé problematické body v něm zmíněné tak byly odstraněny novelou Občanského zákoníku účinné od 1.1.2018. Při snaze o objektivní posuzování dalších jednotlivých bodů se však nelze ubránit pocitu, že jsou věcné a mají, minimálně částečně, pravdu. S ohledem na ochranu věřitelů se dá souhlasit s bodem učiněným v závěru: "Do té doby lze očekávat obrovské problémy s vyhýbáním se daňové povinnosti, zneužíváním při zastření vlastnictví a absenci právní jistoty při obchodním styku s účastníky svěřenského fondu."\footfullcite{noauthor_stanovisko_nodate}. Z této citace je dle mého názoru aktuální její poslední část a to nejistota při obchodním styku s účastníky svěřenského fondu, zejména tedy se svěřenským správcem. Tato nejistota působí zejména ve chvíli zániku svěřenského fondu kvůli abscenci likvidace svěřenského fondu. Uspokojení věřitelů tak může být problematické, viz kapitola Možnost použití svěřenského fondu k obcházení věřitelů.\\ 

%Napsat i o politických problémech, které by se tímto vyřešili?

\subsubsection{Zrušení nebo omezení institutu nepominutelného dědice}
Se zachováním určité ochrany, inspirovat se můžeme z Kanadské právní úpravy.
\subsubsection{Změna institutu nepominutelného dědice, absolutní pořizovací vůle zůstavitele}
Svěřenský fond jako způsob vydědění, oddělení institutu nepominutelného dědice od institutu svěřenského fondu, ochrana by se tedy nevztahovala na případy, kdy je mezigeneračního převodu majetku docíleno pomocí svěřenských fondů. Tomuto je, dle mého názoru, současná právní úprava nejblíže, je ale nicméně stále potřeba výslovné úpravy.
\subsubsection{Změna institutu svěřenského fondu, ochrana nepominutelných dědiců}
Součást pozůstalosti, nebo ochrana nepominutelných dědiců již v rámci pořízení.

Návrhy jsem vymezil v rámci obrázku na ploše obrazovky.

\subsubsection{Návrhy na změny vedoucí k vyřešení ostatních nejasností svěřenského fondu}

\newpage

\section{Závěr}

%Shrnutí toho, co práce obsahovala. Následovat bude shrnutí praktické části práce, tedy že v tato úprava chybí a zároveň není ustálená rozhodovací praxe soudů. Zhodnotit návrhy řešení a nastínit, zda by mohli pomoci, nebo ne.

Základní myšlenka dědického práva v novém občanském zákoníku je, že je třeba co nejvíce respektovat vůli zůstavitele. Tato myšlenka však není omezena pouze na dědické právo, ale je obecně reflektována v celém občanském zákoníku. Ke snaze o vytvoření co největší volnosti občana, ve smyslu mezigeneračního převodu majetku, nově přispívá i možnost použití svěřenského fondu k tomuto účelu. Problémem však i nadále zůstává, že tato myšlenka mnohdy potlačuje do pozadí oprávněné zájmy nepominutelných dědiců.\\

Za základní problém u svěřenských fondů považuji právě nedostatečnou úpravu ve vztahu k dědickému právu, a ať již pohlížíme na institut svěřenského fondu jako na posílení možností pořízení o svém majetku ze strany zůstavitele, je nepochybné, že by tuto možnost měla "podpírat" solidní a jasná právní úprava. Ta žal ke dni vyhotovení této práce chybí a zapříčiňuje tak ve svém důsledku právní nejistotu občana, který by mohl mít zájem o založení svěřenského fondu a v souvislosti s tímto i obmyšlených a dědiců.\\

Samotný fakt, že na svěřenský fond lze pohlédnout z mnoha perspektiv a dle toho dospět k patřičnému závěru ve vztahu k tomu, jak bude vyřešena kolize s dědickým právem ukazuje, že právní úprava svěřenského fondu není finalizována a že je naopak třeba značných změn, aby byla právní nejistota všech zůčastněných stran odstraněna. Je tedy zřejmé, že k plnému pochopení a integrování tohoto institutu a jeho běžnějšímu využívání tak povede ještě dlouhá a dost možná trnitá cesta.\\

Není pochyb, že zavedení institutu svěřenského fondu do našeho právního řádu je jevem pozitivním a vkládá zůstaviteli do rukou možnosti, kterými lze dosáhnout mnoha rozmanitých uspořádání. Je tedy nepochybné, že OZ klade důraz nejen na posílení testovací vůle zakladatele, ale i na možnost dosáhnout uspořádání svého majetku i jinými způsoby, mezi které lze zařadit právě svěřenský fond.\\

Jeho samotné zavedení a formy založení mezi živými a mezi živými s odkládací podmínkou smrti nepředstavují žádný zásadní rozpor mezi právem dědickým a právem zůstavitele nakládat se svým majetkem. Při aplikování zásady autonomie vůle na založení svěřenského fondu pro případ smrti však dochází k zásadní kolizi mezi vůlí zakladatele o uspořádání vlastního majetku a rigidní úpravou dědického práva chránící zájmy nepominutelných dědiců. Respektování vůle zůstavitele je velice důležité, neboť její její podcenění by mohlo vést k popření základní funkce občanského práva a nového občanského zákoníku, kterou zákonodárce popsal v důvodové zprávě. Tato funkce umožňuje svobodné rozvíjení soukromého života a soukromoprávních vztahů a zahrnuje respektování osobnosti člověka jako svobodného individua způsobilého žít podle svého a rozhodovat o svých soukromých věcech samostatně dle vlastní svobodné vůle. Rozhodování o svých soukromých věcech zahrnuje i určení toho, co se stane se jměním osoby po její smrti. Je však nepochybné, že v případě aplikování ustanovení dědického práva na toto rozhodování je zůstavitel povinen tyto ustanovení respektovat.\\

Zde vyvstává základní problém úpravy svěřenského fondu založeného pro případ smrti, kdy není ze zákona jasné, jakým způsobem je zajištěno zachování práva nepominutelného dědice. K tomuto přispívá i neexistence soudních rozhodnutí, které by se tomuto tématu věnovali. Samotná právní obec je v pohledu na tento problém rozdělena a lze se setkat s různými pohledy na problematiku potenciálního obcházení práva nepominutelných dědiců pomocí svěřenského fondu.\\

S ohledem na momentální stav právní úpravy a názorů jednotlivých autorů, z jejichž prací jsem v průběhu své bakalářské práce čerpal nelze se stoprocentní jistotou zodpovědět moje hypotéza stanovená v úvodu práce a dochází tak k právní nejistotě u zakladatelů svěřenských fondů zda mohou použít svěřenský fond k faktickému vydědění jejich nepominutelných dědiců a obdobně dochází k nejistotě na straně dědiců, kterým neposkytuje momentální právní úprava jasné stanovisko, zda by se měli o své právo na povinný díl brát, nebo nikoli. Lze tak pouze doporučit, aby se zákonodárce na tento problém co nejdříve zaměřil a provedl patřičné novelizace, které tyto otázky zodpoví a vyřeší problémy, ke kterým neodvratně dojde, pokud tak neučinní.\\

Jak již jsem nastínil v poslední kapitole Systémové nedostatky a návrhy na řešení, problémy spojené s přijetí právní úpravy trustu do našeho Českého právního řádu ve formě svěřenského fondu přineslo krom nejasností ve vztahu k dědickému právu mnoho dalších problémů a krom toho tedy zůstává otevřeno mnoho otázek, jejichž řešení může pomocí novelizace momentální nedostačující právní úpravy zodpovědět pouze zákonodárce.\\

Lze tedy uzavřít s tím, že aplikace právní úpravy trustu byla v našem právním řádu a právním prostředí potřebná, nelze se však ztotožnit a spokojit s momentální právní úpravou, která není sto schopná unést otázky na ní padající ve vztahu k tomuto nově implementovanému institutu.\\

Svěřenský fond jako takový tedy přináší příjemné osvěžení k možnostem převodu rodinného majetku, přináší ale zároveň spoustu potenciálních problému, na které momentální právní úprava nenabízí řešení. Pokub by tedy otázka stála ve smylsu: "Je vhodné užití svěřenského fondu na mezigenerační převody majetku?", tak bych odpověděl že s ohledem na momentální právní úpravu pouze ve formě založení mezi živými. Založení pro případ smrti ssebou nese tolik potenciálních problémů, které budou muset zůstavitelovi dědici, obmyšlení a svěřenský správce řešit, že dle mého názoru není jeho použití vhodné a tímto bych ho také ani nedoporučil.\\

Přes všechna pozitiva, kterých je mnoho, tak lze pouze doufat, že problémy spojené s právní úpravou svěřenských fondů budou co nejdříve odstraněny, aby česká společnost mohla tento velice užitečný institut začít požívat plnými doušky a se zajištěnou právní jistotou. Doufejme tedy, že s dopřáním více času se zákonodárci povede svěřenský fond opravit a dostat na výsluní tak, aby byl atraktivní a použitelný i pro běžné lidi. Pak se možná zvýší i počet založených svěřenských fondů, kterému je ke dnešnímu dni jen něco okolo dvou tísíc\footfullcite[Zde lze zjistit momentální číslo, evidence svěřenských fondů]{noauthor_isesf_nodate}.\\

%Jak je to v případě, že zůstavitel napíše dědickou smlouvu a majetek, který je součástí takovézo smlouvy pak za svého života zcizí? Je to v pořádku, nebo je to zakázané, musím se podívat do úpravy dědické smlouvy. V případě neshod mezi dědickou smlouvou a závětí má přednost dědická smlouva, takže tady o tomto není pochyb. Co však kdyby proti sobě stálo několik dědických nároků, například pokud by zůstavitel pořídil o větším majetku, než který by poté byl souřástí dědického řízení? Co když pořídí více závětí, je pak platná ta naposledny pořízená závěť?

%Podívat se ještě na google na tokové ty články týkající se svěřenského fondu, často ještě pojednávali o dalších problémech, které bych tady v závěru mohl zmínit a podpořit jím tak stanovisko, že použití svěřenského fondu pro případ smrti není vhodné a rozhodně bych ho nedoporučoval.

% Konec hlavní části práce, následují zdroje
\newpage
\thispagestyle{Contents}
\section*{Zdroje}\markright{ZDROJE}
% Counter definition pro přidání čísla ke zdrojům v obsahu práce, možná číslo odeberu
\newcounter{SecZdroje}
\setcounter{SecZdroje}{\thesection}
\addtocounter{SecZdroje}{1}
\addcontentsline{toc}{section}{\theSecZdroje \hspace{1,7 mm} Zdroje}
% Zde je definováno jak budou vypsány zdroje, přesná specifikace je obsažena v mappingu
\printbibliography[type=misc,heading=subbibliography,title={Online zdroje}]
\printbibliography[type=book,heading=subbibliography,title={Knižní zdroje}]
\printbibliography[type=article,heading=subbibliography,title={Články}]
\printbibliography[type=proceedings,heading=subbibliography,title={Zákony}]	
\end{document}
